% Options for packages loaded elsewhere
\PassOptionsToPackage{unicode}{hyperref}
\PassOptionsToPackage{hyphens}{url}
%
\documentclass[
  12pt,
  letterpaper,
]{book}

\usepackage{amsmath,amssymb}
\usepackage{setspace}
\usepackage{iftex}
\ifPDFTeX
  \usepackage[T1]{fontenc}
  \usepackage[utf8]{inputenc}
  \usepackage{textcomp} % provide euro and other symbols
\else % if luatex or xetex
  \usepackage{unicode-math}
  \defaultfontfeatures{Scale=MatchLowercase}
  \defaultfontfeatures[\rmfamily]{Ligatures=TeX,Scale=1}
\fi
\usepackage[]{mathptmx}
\ifPDFTeX\else  
    % xetex/luatex font selection
\fi
% Use upquote if available, for straight quotes in verbatim environments
\IfFileExists{upquote.sty}{\usepackage{upquote}}{}
\IfFileExists{microtype.sty}{% use microtype if available
  \usepackage[]{microtype}
  \UseMicrotypeSet[protrusion]{basicmath} % disable protrusion for tt fonts
}{}
\usepackage{xcolor}
\usepackage[top=3cm,bottom=2cm,left=3cm,right=2cm]{geometry}
\setlength{\emergencystretch}{3em} % prevent overfull lines
\setcounter{secnumdepth}{5}
% Make \paragraph and \subparagraph free-standing
\makeatletter
\ifx\paragraph\undefined\else
  \let\oldparagraph\paragraph
  \renewcommand{\paragraph}{
    \@ifstar
      \xxxParagraphStar
      \xxxParagraphNoStar
  }
  \newcommand{\xxxParagraphStar}[1]{\oldparagraph*{#1}\mbox{}}
  \newcommand{\xxxParagraphNoStar}[1]{\oldparagraph{#1}\mbox{}}
\fi
\ifx\subparagraph\undefined\else
  \let\oldsubparagraph\subparagraph
  \renewcommand{\subparagraph}{
    \@ifstar
      \xxxSubParagraphStar
      \xxxSubParagraphNoStar
  }
  \newcommand{\xxxSubParagraphStar}[1]{\oldsubparagraph*{#1}\mbox{}}
  \newcommand{\xxxSubParagraphNoStar}[1]{\oldsubparagraph{#1}\mbox{}}
\fi
\makeatother

\usepackage{color}
\usepackage{fancyvrb}
\newcommand{\VerbBar}{|}
\newcommand{\VERB}{\Verb[commandchars=\\\{\}]}
\DefineVerbatimEnvironment{Highlighting}{Verbatim}{commandchars=\\\{\}}
% Add ',fontsize=\small' for more characters per line
\usepackage{framed}
\definecolor{shadecolor}{RGB}{241,243,245}
\newenvironment{Shaded}{\begin{snugshade}}{\end{snugshade}}
\newcommand{\AlertTok}[1]{\textcolor[rgb]{0.68,0.00,0.00}{#1}}
\newcommand{\AnnotationTok}[1]{\textcolor[rgb]{0.37,0.37,0.37}{#1}}
\newcommand{\AttributeTok}[1]{\textcolor[rgb]{0.40,0.45,0.13}{#1}}
\newcommand{\BaseNTok}[1]{\textcolor[rgb]{0.68,0.00,0.00}{#1}}
\newcommand{\BuiltInTok}[1]{\textcolor[rgb]{0.00,0.23,0.31}{#1}}
\newcommand{\CharTok}[1]{\textcolor[rgb]{0.13,0.47,0.30}{#1}}
\newcommand{\CommentTok}[1]{\textcolor[rgb]{0.37,0.37,0.37}{#1}}
\newcommand{\CommentVarTok}[1]{\textcolor[rgb]{0.37,0.37,0.37}{\textit{#1}}}
\newcommand{\ConstantTok}[1]{\textcolor[rgb]{0.56,0.35,0.01}{#1}}
\newcommand{\ControlFlowTok}[1]{\textcolor[rgb]{0.00,0.23,0.31}{\textbf{#1}}}
\newcommand{\DataTypeTok}[1]{\textcolor[rgb]{0.68,0.00,0.00}{#1}}
\newcommand{\DecValTok}[1]{\textcolor[rgb]{0.68,0.00,0.00}{#1}}
\newcommand{\DocumentationTok}[1]{\textcolor[rgb]{0.37,0.37,0.37}{\textit{#1}}}
\newcommand{\ErrorTok}[1]{\textcolor[rgb]{0.68,0.00,0.00}{#1}}
\newcommand{\ExtensionTok}[1]{\textcolor[rgb]{0.00,0.23,0.31}{#1}}
\newcommand{\FloatTok}[1]{\textcolor[rgb]{0.68,0.00,0.00}{#1}}
\newcommand{\FunctionTok}[1]{\textcolor[rgb]{0.28,0.35,0.67}{#1}}
\newcommand{\ImportTok}[1]{\textcolor[rgb]{0.00,0.46,0.62}{#1}}
\newcommand{\InformationTok}[1]{\textcolor[rgb]{0.37,0.37,0.37}{#1}}
\newcommand{\KeywordTok}[1]{\textcolor[rgb]{0.00,0.23,0.31}{\textbf{#1}}}
\newcommand{\NormalTok}[1]{\textcolor[rgb]{0.00,0.23,0.31}{#1}}
\newcommand{\OperatorTok}[1]{\textcolor[rgb]{0.37,0.37,0.37}{#1}}
\newcommand{\OtherTok}[1]{\textcolor[rgb]{0.00,0.23,0.31}{#1}}
\newcommand{\PreprocessorTok}[1]{\textcolor[rgb]{0.68,0.00,0.00}{#1}}
\newcommand{\RegionMarkerTok}[1]{\textcolor[rgb]{0.00,0.23,0.31}{#1}}
\newcommand{\SpecialCharTok}[1]{\textcolor[rgb]{0.37,0.37,0.37}{#1}}
\newcommand{\SpecialStringTok}[1]{\textcolor[rgb]{0.13,0.47,0.30}{#1}}
\newcommand{\StringTok}[1]{\textcolor[rgb]{0.13,0.47,0.30}{#1}}
\newcommand{\VariableTok}[1]{\textcolor[rgb]{0.07,0.07,0.07}{#1}}
\newcommand{\VerbatimStringTok}[1]{\textcolor[rgb]{0.13,0.47,0.30}{#1}}
\newcommand{\WarningTok}[1]{\textcolor[rgb]{0.37,0.37,0.37}{\textit{#1}}}

\providecommand{\tightlist}{%
  \setlength{\itemsep}{0pt}\setlength{\parskip}{0pt}}\usepackage{longtable,booktabs,array}
\usepackage{calc} % for calculating minipage widths
% Correct order of tables after \paragraph or \subparagraph
\usepackage{etoolbox}
\makeatletter
\patchcmd\longtable{\par}{\if@noskipsec\mbox{}\fi\par}{}{}
\makeatother
% Allow footnotes in longtable head/foot
\IfFileExists{footnotehyper.sty}{\usepackage{footnotehyper}}{\usepackage{footnote}}
\makesavenoteenv{longtable}
\usepackage{graphicx}
\makeatletter
\newsavebox\pandoc@box
\newcommand*\pandocbounded[1]{% scales image to fit in text height/width
  \sbox\pandoc@box{#1}%
  \Gscale@div\@tempa{\textheight}{\dimexpr\ht\pandoc@box+\dp\pandoc@box\relax}%
  \Gscale@div\@tempb{\linewidth}{\wd\pandoc@box}%
  \ifdim\@tempb\p@<\@tempa\p@\let\@tempa\@tempb\fi% select the smaller of both
  \ifdim\@tempa\p@<\p@\scalebox{\@tempa}{\usebox\pandoc@box}%
  \else\usebox{\pandoc@box}%
  \fi%
}
% Set default figure placement to htbp
\def\fps@figure{htbp}
\makeatother
% definitions for citeproc citations
\NewDocumentCommand\citeproctext{}{}
\NewDocumentCommand\citeproc{mm}{%
  \begingroup\def\citeproctext{#2}\cite{#1}\endgroup}
\makeatletter
 % allow citations to break across lines
 \let\@cite@ofmt\@firstofone
 % avoid brackets around text for \cite:
 \def\@biblabel#1{}
 \def\@cite#1#2{{#1\if@tempswa , #2\fi}}
\makeatother
\newlength{\cslhangindent}
\setlength{\cslhangindent}{1.5em}
\newlength{\csllabelwidth}
\setlength{\csllabelwidth}{3em}
\newenvironment{CSLReferences}[2] % #1 hanging-indent, #2 entry-spacing
 {\begin{list}{}{%
  \setlength{\itemindent}{0pt}
  \setlength{\leftmargin}{0pt}
  \setlength{\parsep}{0pt}
  % turn on hanging indent if param 1 is 1
  \ifodd #1
   \setlength{\leftmargin}{\cslhangindent}
   \setlength{\itemindent}{-1\cslhangindent}
  \fi
  % set entry spacing
  \setlength{\itemsep}{#2\baselineskip}}}
 {\end{list}}
\usepackage{calc}
\newcommand{\CSLBlock}[1]{\hfill\break\parbox[t]{\linewidth}{\strut\ignorespaces#1\strut}}
\newcommand{\CSLLeftMargin}[1]{\parbox[t]{\csllabelwidth}{\strut#1\strut}}
\newcommand{\CSLRightInline}[1]{\parbox[t]{\linewidth - \csllabelwidth}{\strut#1\strut}}
\newcommand{\CSLIndent}[1]{\hspace{\cslhangindent}#1}

\usepackage{booktabs}
\usepackage{longtable}
\usepackage{array}
\usepackage{multirow}
\usepackage{wrapfig}
\usepackage{float}
\usepackage{colortbl}
\usepackage{pdflscape}
\usepackage{tabu}
\usepackage{threeparttable}
\usepackage{threeparttablex}
\usepackage[normalem]{ulem}
\usepackage{makecell}
\usepackage{xcolor}
\makeatletter
\@ifpackageloaded{tcolorbox}{}{\usepackage[skins,breakable]{tcolorbox}}
\@ifpackageloaded{fontawesome5}{}{\usepackage{fontawesome5}}
\definecolor{quarto-callout-color}{HTML}{909090}
\definecolor{quarto-callout-note-color}{HTML}{0758E5}
\definecolor{quarto-callout-important-color}{HTML}{CC1914}
\definecolor{quarto-callout-warning-color}{HTML}{EB9113}
\definecolor{quarto-callout-tip-color}{HTML}{00A047}
\definecolor{quarto-callout-caution-color}{HTML}{FC5300}
\definecolor{quarto-callout-color-frame}{HTML}{acacac}
\definecolor{quarto-callout-note-color-frame}{HTML}{4582ec}
\definecolor{quarto-callout-important-color-frame}{HTML}{d9534f}
\definecolor{quarto-callout-warning-color-frame}{HTML}{f0ad4e}
\definecolor{quarto-callout-tip-color-frame}{HTML}{02b875}
\definecolor{quarto-callout-caution-color-frame}{HTML}{fd7e14}
\makeatother
\makeatletter
\@ifpackageloaded{bookmark}{}{\usepackage{bookmark}}
\makeatother
\makeatletter
\@ifpackageloaded{caption}{}{\usepackage{caption}}
\AtBeginDocument{%
\ifdefined\contentsname
  \renewcommand*\contentsname{Índice}
\else
  \newcommand\contentsname{Índice}
\fi
\ifdefined\listfigurename
  \renewcommand*\listfigurename{Lista de Figuras}
\else
  \newcommand\listfigurename{Lista de Figuras}
\fi
\ifdefined\listtablename
  \renewcommand*\listtablename{Lista de Tabelas}
\else
  \newcommand\listtablename{Lista de Tabelas}
\fi
\ifdefined\figurename
  \renewcommand*\figurename{Figura}
\else
  \newcommand\figurename{Figura}
\fi
\ifdefined\tablename
  \renewcommand*\tablename{Tabela}
\else
  \newcommand\tablename{Tabela}
\fi
}
\@ifpackageloaded{float}{}{\usepackage{float}}
\floatstyle{ruled}
\@ifundefined{c@chapter}{\newfloat{codelisting}{h}{lop}}{\newfloat{codelisting}{h}{lop}[chapter]}
\floatname{codelisting}{Listagem}
\newcommand*\listoflistings{\listof{codelisting}{Lista de Listagens}}
\makeatother
\makeatletter
\makeatother
\makeatletter
\@ifpackageloaded{caption}{}{\usepackage{caption}}
\@ifpackageloaded{subcaption}{}{\usepackage{subcaption}}
\makeatother

\ifLuaTeX
\usepackage[bidi=basic]{babel}
\else
\usepackage[bidi=default]{babel}
\fi
\babelprovide[main,import]{brazilian}
% get rid of language-specific shorthands (see #6817):
\let\LanguageShortHands\languageshorthands
\def\languageshorthands#1{}
\usepackage{bookmark}

\IfFileExists{xurl.sty}{\usepackage{xurl}}{} % add URL line breaks if available
\urlstyle{same} % disable monospaced font for URLs
\hypersetup{
  pdftitle={Confiabilidade e Manutenção: Aplicações em R},
  pdfauthor={Rafael da Silva Fernandes},
  pdflang={pt-BR},
  hidelinks,
  pdfcreator={LaTeX via pandoc}}


\title{Confiabilidade e Manutenção: Aplicações em R}
\usepackage{etoolbox}
\makeatletter
\providecommand{\subtitle}[1]{% add subtitle to \maketitle
  \apptocmd{\@title}{\par {\large #1 \par}}{}{}
}
\makeatother
\subtitle{Análise de Dados de Falha, Modelagem de Vida Útil e
Estratégias de Manutenção}
\author{Rafael da Silva Fernandes}
\date{2025-11-22}

\begin{document}
\frontmatter
\maketitle

\renewcommand*\contentsname{Índice}
{
\setcounter{tocdepth}{2}
\tableofcontents
}

\setstretch{1.5}
\mainmatter
\bookmarksetup{startatroot}

\chapter*{Bem-vindo}\label{bem-vindo}
\addcontentsline{toc}{chapter}{Bem-vindo}

\markboth{Bem-vindo}{Bem-vindo}

Bem-vindo ao livro \textbf{``Confiabilidade e Manutenção: Aplicações em
R''}!

\begin{tcolorbox}[enhanced jigsaw, arc=.35mm, toprule=.15mm, opacityback=0, colback=white, breakable, left=2mm, bottomrule=.15mm, leftrule=.75mm, colframe=quarto-callout-note-color-frame, rightrule=.15mm]
\begin{minipage}[t]{5.5mm}
\textcolor{quarto-callout-note-color}{\faInfo}
\end{minipage}%
\begin{minipage}[t]{\textwidth - 5.5mm}

\vspace{-3mm}\textbf{Sobre este Livro}\vspace{3mm}

Este livro fornece uma introdução \textbf{prática e aplicada} aos
conceitos de confiabilidade, análise de dados de falha e modelagem de
vida útil, com ênfase no uso da linguagem \textbf{R} como ferramenta
principal.

\end{minipage}%
\end{tcolorbox}

\section*{📚 O que você vai aprender}\label{o-que-vocuxea-vai-aprender}
\addcontentsline{toc}{section}{📚 O que você vai aprender}

\markright{📚 O que você vai aprender}

\begin{itemize}
\tightlist
\item
  \textbf{Fundamentos de confiabilidade} --- Conceitos, métricas e
  distribuições de vida útil
\item
  \textbf{Análise de dados de falha} --- Técnicas estatísticas para
  dados censurados
\item
  \textbf{Modelagem Weibull} --- Aplicação prática da distribuição mais
  usada em engenharia
\item
  \textbf{Análise de sobrevivência} --- Modelos paramétricos e
  semiparamétricos
\item
  \textbf{Estratégias de manutenção} --- Otimização de políticas
  preventivas e preditivas
\item
  \textbf{Machine learning} --- Predição de falhas com dados de sensores
\item
  \textbf{Simulação Monte Carlo} --- Avaliação de políticas sob
  incerteza
\item
  \textbf{Boas práticas} --- Governança de dados e reprodutibilidade
\end{itemize}

\section*{🎯 Para quem é este livro}\label{para-quem-uxe9-este-livro}
\addcontentsline{toc}{section}{🎯 Para quem é este livro}

\markright{🎯 Para quem é este livro}

\begin{itemize}
\tightlist
\item
  \textbf{Engenheiros de confiabilidade e manutenção}
\item
  \textbf{Analistas de dados} em contextos industriais
\item
  \textbf{Estudantes e pesquisadores} em engenharia e estatística
\item
  \textbf{Gestores técnicos} que tomam decisões baseadas em dados
\end{itemize}

\section*{🚀 Como usar este livro}\label{como-usar-este-livro}
\addcontentsline{toc}{section}{🚀 Como usar este livro}

\markright{🚀 Como usar este livro}

\begin{enumerate}
\def\labelenumi{\arabic{enumi}.}
\tightlist
\item
  \textbf{Leia sequencialmente} --- Os capítulos são progressivos
\item
  \textbf{Execute os exemplos} --- Todos os códigos R são reproduzíveis
\item
  \textbf{Faça os exercícios} --- Pratique para consolidar o aprendizado
\item
  \textbf{Use como referência} --- Consulte os apêndices quando
  necessário
\end{enumerate}

\section*{📖 Estrutura}\label{estrutura}
\addcontentsline{toc}{section}{📖 Estrutura}

\markright{📖 Estrutura}

O livro está organizado em quatro partes:

\subsection*{Parte I: Fundamentos e
Teoria}\label{parte-i-fundamentos-e-teoria}
\addcontentsline{toc}{subsection}{Parte I: Fundamentos e Teoria}

Conceitos básicos, distribuições de probabilidade e métodos de
estimação.

\subsection*{Parte II: Análise e Modelagem
Prática}\label{parte-ii-anuxe1lise-e-modelagem-pruxe1tica}
\addcontentsline{toc}{subsection}{Parte II: Análise e Modelagem Prática}

Análise de dados de falha, modelagem Weibull e análise de sobrevivência.

\subsection*{Parte III: Estratégias de
Manutenção}\label{parte-iii-estratuxe9gias-de-manutenuxe7uxe3o}
\addcontentsline{toc}{subsection}{Parte III: Estratégias de Manutenção}

Manutenção preventiva, preditiva e modelagem de sistemas complexos.

\subsection*{Parte IV: Métodos Avançados e
Aplicações}\label{parte-iv-muxe9todos-avanuxe7ados-e-aplicauxe7uxf5es}
\addcontentsline{toc}{subsection}{Parte IV: Métodos Avançados e
Aplicações}

Simulação, estudos de caso industriais e governança de dados.

\section*{💻 Requisitos}\label{requisitos}
\addcontentsline{toc}{section}{💻 Requisitos}

\markright{💻 Requisitos}

\begin{itemize}
\tightlist
\item
  \textbf{R} (versão ≥ 4.0)
\item
  \textbf{RStudio} ou VS Code
\item
  \textbf{Pacotes R} --- \texttt{tidyverse}, \texttt{survival},
  \texttt{flexsurv}, \texttt{ggplot2}, entre outros
\end{itemize}

Veja instruções detalhadas de instalação no capítulo
\href{setup.qmd}{Instalação e Ambiente}.

\section*{🔗 Recursos Adicionais}\label{recursos-adicionais}
\addcontentsline{toc}{section}{🔗 Recursos Adicionais}

\markright{🔗 Recursos Adicionais}

\begin{itemize}
\tightlist
\item
  \textbf{Código fonte:}
  \href{https://github.com/seu-usuario/QuartoBook}{GitHub}
\item
  \textbf{Dados de exemplo:} Disponíveis em \texttt{resources/data/}
\item
  \textbf{Discussões:}
  \href{https://github.com/seu-usuario/QuartoBook/issues}{Issues no
  GitHub}
\end{itemize}

\begin{center}\rule{0.5\linewidth}{0.5pt}\end{center}

\begin{tcolorbox}[enhanced jigsaw, title=\textcolor{quarto-callout-tip-color}{\faLightbulb}\hspace{0.5em}{Pronto para começar?}, bottomrule=.15mm, bottomtitle=1mm, titlerule=0mm, arc=.35mm, toptitle=1mm, rightrule=.15mm, colbacktitle=quarto-callout-tip-color!10!white, opacitybacktitle=0.6, opacityback=0, colback=white, breakable, toprule=.15mm, leftrule=.75mm, left=2mm, colframe=quarto-callout-tip-color-frame, coltitle=black]

Inicie pelo \href{preface.qmd}{Prefácio} para entender melhor o contexto
e objetivos deste livro, ou vá direto para \href{requirements.qmd}{Como
Usar este Livro} para configurar seu ambiente.

\end{tcolorbox}

\begin{center}\rule{0.5\linewidth}{0.5pt}\end{center}

\textbf{Boa leitura e bons estudos!} 📊✨

\bookmarksetup{startatroot}

\chapter*{Prefácio}\label{prefuxe1cio}
\addcontentsline{toc}{chapter}{Prefácio}

\markboth{Prefácio}{Prefácio}

\section*{Sobre este Livro}\label{sobre-este-livro-1}
\addcontentsline{toc}{section}{Sobre este Livro}

\markright{Sobre este Livro}

A \textbf{confiabilidade} e a \textbf{manutenção} de equipamentos e
sistemas são pilares fundamentais para a operação eficiente e segura de
indústrias em todo o mundo. Com o avanço da coleta de dados, da
computação estatística e do aprendizado de máquina, a análise
quantitativa de falhas e a otimização de estratégias de manutenção
tornaram-se mais acessíveis e poderosas do que nunca.

Este livro tem como objetivo fornecer uma introdução \textbf{prática e
aplicada} aos conceitos de confiabilidade, análise de dados de falha e
modelagem de vida útil, com ênfase no uso da linguagem \textbf{R} como
ferramenta principal. Ele é direcionado a engenheiros, analistas de
dados, gestores de manutenção, estudantes e profissionais que desejam
aplicar métodos estatísticos modernos para melhorar a disponibilidade,
reduzir custos e otimizar políticas de manutenção.

\section*{Público-Alvo}\label{puxfablico-alvo}
\addcontentsline{toc}{section}{Público-Alvo}

\markright{Público-Alvo}

\begin{itemize}
\tightlist
\item
  \textbf{Engenheiros de confiabilidade e manutenção} que buscam
  ferramentas práticas para análise de dados de falha e otimização de
  políticas.
\item
  \textbf{Analistas de dados} interessados em aplicar modelos
  estatísticos e de machine learning em contextos industriais.
\item
  \textbf{Estudantes e pesquisadores} que desejam aprender técnicas de
  análise de sobrevivência e modelagem de confiabilidade com exemplos
  reais.
\item
  \textbf{Gestores técnicos} que precisam tomar decisões baseadas em
  dados para melhorar a performance de ativos.
\end{itemize}

\section*{Objetivos}\label{objetivos}
\addcontentsline{toc}{section}{Objetivos}

\markright{Objetivos}

Ao final deste livro, o leitor será capaz de:

\begin{itemize}
\tightlist
\item
  Compreender os \textbf{conceitos fundamentais} de confiabilidade,
  disponibilidade e manutenção.
\item
  Ajustar e interpretar \textbf{distribuições de vida útil} (Weibull,
  exponencial, lognormal) a dados reais.
\item
  Realizar \textbf{análise de sobrevivência} com dados censurados usando
  técnicas paramétricas e semiparamétricas.
\item
  Implementar \textbf{estratégias de manutenção preventiva e preditiva}
  com base em dados de sensores e histórico de falhas.
\item
  Utilizar \textbf{simulação Monte Carlo} para avaliar políticas de
  manutenção sob incerteza.
\item
  Aplicar boas práticas de \textbf{gestão de dados} e reprodutibilidade
  em projetos de confiabilidade.
\end{itemize}

\section*{Como Usar este Livro}\label{como-usar-este-livro-1}
\addcontentsline{toc}{section}{Como Usar este Livro}

\markright{Como Usar este Livro}

Este livro é \textbf{totalmente reproduzível}. Todos os exemplos de
código R estão embutidos nos capítulos e podem ser executados
diretamente. Recomendamos que o leitor:

\begin{enumerate}
\def\labelenumi{\arabic{enumi}.}
\tightlist
\item
  \textbf{Instale o R e o RStudio} (ou use o VS Code com extensões de R)
  seguindo as instruções no capítulo de instalação.
\item
  \textbf{Execute os exemplos} à medida que avança pelos capítulos para
  consolidar o aprendizado.
\item
  \textbf{Explore os exercícios} propostos ao final de cada capítulo
  para praticar os conceitos.
\item
  \textbf{Utilize os dados de exemplo} fornecidos na pasta
  \texttt{resources/data/} para replicar as análises.
\end{enumerate}

Os capítulos são organizados de forma \textbf{progressiva}, começando
pelos fundamentos teóricos e avançando para aplicações práticas e
estudos de caso. Você pode seguir a ordem proposta ou pular para
capítulos específicos de acordo com suas necessidades.

\section*{Ferramentas e Recursos}\label{ferramentas-e-recursos}
\addcontentsline{toc}{section}{Ferramentas e Recursos}

\markright{Ferramentas e Recursos}

Este livro foi desenvolvido com:

\begin{itemize}
\tightlist
\item
  \textbf{Quarto} --- Sistema de publicação científica e técnica que
  integra código, texto e visualizações.
\item
  \textbf{R} --- Linguagem de programação estatística open-source com
  vasto ecossistema de pacotes.
\item
  \textbf{Pacotes R principais} --- \texttt{survival},
  \texttt{flexsurv}, \texttt{WeibullR}, \texttt{reliability},
  \texttt{tidyverse}, \texttt{ggplot2}, entre outros.
\end{itemize}

Todos os materiais suplementares, incluindo dados, scripts e recursos
adicionais, estão disponíveis no diretório \texttt{resources/} do
projeto.

\section*{Convenções}\label{convenuxe7uxf5es}
\addcontentsline{toc}{section}{Convenções}

\markright{Convenções}

\begin{itemize}
\tightlist
\item
  \textbf{Código R} é apresentado em blocos destacados e pode ser
  copiado diretamente.
\item
  \textbf{Figuras e tabelas} são numeradas e referenciadas ao longo do
  texto.
\item
  \textbf{Exemplos práticos} são marcados com 💡 e \textbf{exercícios}
  com 📝.
\item
  \textbf{Dicas importantes} são destacadas com 🔔.
\end{itemize}

\section*{Agradecimentos}\label{agradecimentos}
\addcontentsline{toc}{section}{Agradecimentos}

\markright{Agradecimentos}

Este livro é fruto de anos de experiência prática em engenharia de
confiabilidade, análise de dados e ensino. Agradeço à comunidade R, aos
desenvolvedores de pacotes open-source e a todos que contribuíram com
feedback e sugestões ao longo do desenvolvimento deste material.

\begin{center}\rule{0.5\linewidth}{0.5pt}\end{center}

\textbf{Rafael da Silva Fernandes} \emph{Novembro de 2025}

\bookmarksetup{startatroot}

\chapter*{Como Usar este Livro}\label{como-usar-este-livro-2}
\addcontentsline{toc}{chapter}{Como Usar este Livro}

\markboth{Como Usar este Livro}{Como Usar este Livro}

\section*{Convenções e
Organização}\label{convenuxe7uxf5es-e-organizauxe7uxe3o}
\addcontentsline{toc}{section}{Convenções e Organização}

\markright{Convenções e Organização}

Este livro foi desenvolvido para ser \textbf{prático, reproduzível e
didático}. Todos os exemplos de código R podem ser executados
diretamente, e os dados utilizados estão disponíveis no diretório
\texttt{resources/data/}.

\subsection*{Organização dos
Capítulos}\label{organizauxe7uxe3o-dos-capuxedtulos}
\addcontentsline{toc}{subsection}{Organização dos Capítulos}

O livro está dividido em \textbf{quatro partes principais}:

\begin{enumerate}
\def\labelenumi{\arabic{enumi}.}
\tightlist
\item
  \textbf{Fundamentos e Teoria} --- Conceitos básicos de confiabilidade,
  distribuições de vida útil e métodos de estimação.
\item
  \textbf{Análise e Modelagem Prática} --- Técnicas de análise de dados
  de falha, modelagem Weibull e análise de sobrevivência.
\item
  \textbf{Estratégias de Manutenção} --- Manutenção preventiva,
  preditiva e modelagem de sistemas complexos.
\item
  \textbf{Métodos Avançados e Aplicações} --- Simulação Monte Carlo,
  estudos de caso industriais e governança de dados.
\end{enumerate}

Cada capítulo inclui:

\begin{itemize}
\tightlist
\item
  \textbf{Teoria concisa} --- Explicação dos conceitos fundamentais.
\item
  \textbf{Exemplos práticos com R} --- Código reproduzível e comentado.
\item
  \textbf{Visualizações e tabelas} --- Gráficos interpretativos e
  resumos estatísticos.
\item
  \textbf{Exercícios} 📝 --- Atividades para consolidar o aprendizado.
\end{itemize}

\subsection*{Convenções de Código}\label{convenuxe7uxf5es-de-cuxf3digo}
\addcontentsline{toc}{subsection}{Convenções de Código}

Todo o código R é apresentado em blocos destacados e pode ser copiado
diretamente:

\begin{Shaded}
\begin{Highlighting}[]
\CommentTok{\# Exemplo de código R}
\FunctionTok{library}\NormalTok{(dplyr)}
\FunctionTok{library}\NormalTok{(ggplot2)}

\CommentTok{\# Carregar dados de exemplo}
\NormalTok{data }\OtherTok{\textless{}{-}} \FunctionTok{read.csv}\NormalTok{(}\StringTok{"resources/data/falhas.csv"}\NormalTok{)}

\CommentTok{\# Análise exploratória}
\FunctionTok{summary}\NormalTok{(data)}
\end{Highlighting}
\end{Shaded}

\begin{itemize}
\tightlist
\item
  \textbf{\texttt{\#\textbar{}\ eval:\ false}} indica que o código não
  será executado automaticamente (apenas exibido).
\item
  \textbf{\texttt{\#\textbar{}\ echo:\ true}} indica que o código será
  mostrado no output final.
\end{itemize}

\subsection*{Símbolos e Ícones}\label{suxedmbolos-e-uxedcones}
\addcontentsline{toc}{subsection}{Símbolos e Ícones}

\begin{itemize}
\tightlist
\item
  💡 \textbf{Exemplo prático} --- Demonstração de aplicação real.
\item
  📝 \textbf{Exercício} --- Atividade proposta para o leitor.
\item
  🔔 \textbf{Dica importante} --- Informação relevante ou ponto de
  atenção.
\item
  ⚠️ \textbf{Atenção} --- Cuidado ou limitação a considerar.
\end{itemize}

\subsection*{Referências Cruzadas}\label{referuxeancias-cruzadas}
\addcontentsline{toc}{subsection}{Referências Cruzadas}

Figuras, tabelas e equações são numeradas e referenciadas ao longo do
texto:

\begin{itemize}
\tightlist
\item
  \textbf{Figuras:} Figura~\ref{fig-weibull}
\item
  \textbf{Tabelas:} Tabela~\ref{tbl-mtbf}
\item
  \textbf{Equações:} Equação~\ref{eq-confiabilidade}
\end{itemize}

\begin{figure}

\centering{

\pandocbounded{\includegraphics[keepaspectratio]{requirements_files/figure-pdf/fig-weibull-1.pdf}}

}

\caption{\label{fig-weibull}Distribuição Weibull ajustada aos tempos de
falha de equipamentos}

\end{figure}%

\begin{longtable}[]{@{}lcc@{}}

\caption{\label{tbl-mtbf}Cálculo de MTBF por tipo de equipamento}

\tabularnewline

\toprule\noalign{}
Tipo de Equipamento & Nº de Falhas & MTBF (horas) \\
\midrule\noalign{}
\endhead
\bottomrule\noalign{}
\endlastfoot
Britador & 5 & 1340.0 \\
Caminhao & 12 & 299.4 \\
Escavadeira & 7 & 625.6 \\

\end{longtable}

\section*{Requisitos de Software}\label{requisitos-de-software}
\addcontentsline{toc}{section}{Requisitos de Software}

\markright{Requisitos de Software}

Para reproduzir os exemplos deste livro, você precisará de:

\subsection*{R (versão ≥ 4.0)}\label{r-versuxe3o-4.0}
\addcontentsline{toc}{subsection}{R (versão ≥ 4.0)}

R é a linguagem de programação principal utilizada. Faça o download em:
\url{https://www.r-project.org/}

\subsection*{RStudio Desktop ou VS
Code}\label{rstudio-desktop-ou-vs-code}
\addcontentsline{toc}{subsection}{RStudio Desktop ou VS Code}

\textbf{RStudio} é o IDE recomendado para desenvolvimento em R:
\url{https://posit.co/download/rstudio-desktop/}

Alternativamente, você pode usar \textbf{VS Code} com as extensões: - R
(reditorsupport.r) - Quarto (quarto.quarto)

\subsection*{Quarto CLI}\label{quarto-cli}
\addcontentsline{toc}{subsection}{Quarto CLI}

Este livro foi desenvolvido com \textbf{Quarto}, um sistema de
publicação científica. Instale em:
\url{https://quarto.org/docs/get-started/}

\subsection*{Pacotes R Necessários}\label{pacotes-r-necessuxe1rios}
\addcontentsline{toc}{subsection}{Pacotes R Necessários}

Os seguintes pacotes R são utilizados ao longo do livro:

\begin{Shaded}
\begin{Highlighting}[]
\CommentTok{\# Manipulação e visualização de dados}
\FunctionTok{install.packages}\NormalTok{(}\FunctionTok{c}\NormalTok{(}\StringTok{"tidyverse"}\NormalTok{, }\StringTok{"dplyr"}\NormalTok{, }\StringTok{"tidyr"}\NormalTok{, }\StringTok{"readr"}\NormalTok{, }\StringTok{"ggplot2"}\NormalTok{, }\StringTok{"plotly"}\NormalTok{))}

\CommentTok{\# Análise de confiabilidade e sobrevivência}
\FunctionTok{install.packages}\NormalTok{(}\FunctionTok{c}\NormalTok{(}\StringTok{"survival"}\NormalTok{, }\StringTok{"flexsurv"}\NormalTok{, }\StringTok{"WeibullR"}\NormalTok{, }\StringTok{"reliability"}\NormalTok{))}

\CommentTok{\# Tabelas e relatórios}
\FunctionTok{install.packages}\NormalTok{(}\FunctionTok{c}\NormalTok{(}\StringTok{"knitr"}\NormalTok{, }\StringTok{"kableExtra"}\NormalTok{, }\StringTok{"DT"}\NormalTok{, }\StringTok{"broom"}\NormalTok{))}

\CommentTok{\# Machine learning e manutenção preditiva}
\FunctionTok{install.packages}\NormalTok{(}\FunctionTok{c}\NormalTok{(}\StringTok{"caret"}\NormalTok{, }\StringTok{"randomForest"}\NormalTok{, }\StringTok{"prophet"}\NormalTok{, }\StringTok{"anomalize"}\NormalTok{))}

\CommentTok{\# Simulação}
\FunctionTok{install.packages}\NormalTok{(}\FunctionTok{c}\NormalTok{(}\StringTok{"simmer"}\NormalTok{, }\StringTok{"MASS"}\NormalTok{))}
\end{Highlighting}
\end{Shaded}

Para instalar todos os pacotes de uma vez:

\begin{Shaded}
\begin{Highlighting}[]
\FunctionTok{source}\NormalTok{(}\StringTok{"resources/install\_packages.R"}\NormalTok{)}
\end{Highlighting}
\end{Shaded}

\section*{Estrutura de Arquivos e
Dados}\label{estrutura-de-arquivos-e-dados}
\addcontentsline{toc}{section}{Estrutura de Arquivos e Dados}

\markright{Estrutura de Arquivos e Dados}

O projeto deste livro está organizado da seguinte forma:

\begin{verbatim}
QuartoBook/
├── _quarto.yml              # Configuração do livro
├── index.qmd                # Página inicial
├── preface.qmd              # Prefácio
├── requirements.qmd         # Este arquivo
├── setup.qmd                # Instruções de instalação
├── chapters/                # Capítulos principais
│   ├── 01-fundamentos.qmd
│   ├── 02-distribuicoes.qmd
│   └── ...
├── appendices/              # Apêndices
│   ├── A-pacotes.qmd
│   └── ...
├── resources/               # Recursos adicionais
│   ├── data/                # Dados de exemplo (CSV, RDS)
│   ├── scripts/             # Scripts R auxiliares
│   └── images/              # Imagens e figuras
├── docs/                    # Output renderizado (HTML)
└── references.bib           # Bibliografia
\end{verbatim}

\subsection*{Dados de Exemplo}\label{dados-de-exemplo}
\addcontentsline{toc}{subsection}{Dados de Exemplo}

Todos os datasets utilizados nos exemplos estão disponíveis em
\texttt{resources/data/}:

\begin{itemize}
\tightlist
\item
  \texttt{falhas\_bombas.csv} --- Dados de falha de bombas industriais.
\item
  \texttt{sensores\_compressor.csv} --- Dados de sensores para
  manutenção preditiva.
\item
  \texttt{historico\_motores.rds} --- Histórico de manutenção de motores
  elétricos.
\end{itemize}

\section*{Como Executar os Exemplos}\label{como-executar-os-exemplos}
\addcontentsline{toc}{section}{Como Executar os Exemplos}

\markright{Como Executar os Exemplos}

\subsection*{No RStudio}\label{no-rstudio}
\addcontentsline{toc}{subsection}{No RStudio}

\begin{enumerate}
\def\labelenumi{\arabic{enumi}.}
\tightlist
\item
  Abra o arquivo \texttt{.qmd} desejado.
\item
  Clique em \textbf{``Run''} para executar os chunks de código
  individualmente.
\item
  Use \textbf{Ctrl+Shift+K} para renderizar o documento completo.
\end{enumerate}

\subsection*{No VS Code}\label{no-vs-code}
\addcontentsline{toc}{subsection}{No VS Code}

\begin{enumerate}
\def\labelenumi{\arabic{enumi}.}
\tightlist
\item
  Instale as extensões Quarto e R.
\item
  Abra o arquivo \texttt{.qmd}.
\item
  Use \textbf{Ctrl+Enter} para executar linhas de código.
\item
  Execute \texttt{quarto\ preview} no terminal para visualizar o livro
  completo.
\end{enumerate}

\subsection*{Via Terminal}\label{via-terminal}
\addcontentsline{toc}{subsection}{Via Terminal}

Para renderizar o livro completo:

\begin{Shaded}
\begin{Highlighting}[]
\ExtensionTok{quarto}\NormalTok{ render}
\end{Highlighting}
\end{Shaded}

Para visualizar com hot reload:

\begin{Shaded}
\begin{Highlighting}[]
\ExtensionTok{quarto}\NormalTok{ preview}
\end{Highlighting}
\end{Shaded}

\section*{Reprodutibilidade}\label{reprodutibilidade}
\addcontentsline{toc}{section}{Reprodutibilidade}

\markright{Reprodutibilidade}

Este livro foi desenvolvido com práticas de \textbf{ciência de dados
reproduzível}:

\begin{itemize}
\tightlist
\item
  Todos os dados de exemplo estão versionados.
\item
  Códigos são testados e validados.
\item
  Sessões R são gerenciadas com \texttt{.Rprofile}.
\item
  Outputs são congelados com \texttt{freeze:\ auto} no Quarto.
\end{itemize}

Para garantir reprodutibilidade, recomendamos:

\begin{Shaded}
\begin{Highlighting}[]
\CommentTok{\# Verificar versão do R}
\NormalTok{R.version.string}

\CommentTok{\# Listar pacotes instalados}
\FunctionTok{installed.packages}\NormalTok{()[, }\FunctionTok{c}\NormalTok{(}\StringTok{"Package"}\NormalTok{, }\StringTok{"Version"}\NormalTok{)]}
\end{Highlighting}
\end{Shaded}

\section*{Dúvidas e Suporte}\label{duxfavidas-e-suporte}
\addcontentsline{toc}{section}{Dúvidas e Suporte}

\markright{Dúvidas e Suporte}

Para dúvidas, sugestões ou relato de erros:

\begin{itemize}
\tightlist
\item
  \textbf{GitHub Issues:}
  \url{https://github.com/SEU_USER/QuartoBook/issues}
\item
  \textbf{Email:} rafael@example.com
\end{itemize}

\begin{center}\rule{0.5\linewidth}{0.5pt}\end{center}

\textbf{Boa leitura e bons estudos!} 🚀

\bookmarksetup{startatroot}

\chapter*{Instalação e Ambiente}\label{instalauxe7uxe3o-e-ambiente}
\addcontentsline{toc}{chapter}{Instalação e Ambiente}

\markboth{Instalação e Ambiente}{Instalação e Ambiente}

Este capítulo fornece instruções detalhadas para configurar o ambiente
de trabalho necessário para reproduzir os exemplos deste livro.

\section*{Instalação do R}\label{instalauxe7uxe3o-do-r}
\addcontentsline{toc}{section}{Instalação do R}

\markright{Instalação do R}

\subsection*{Windows}\label{windows}
\addcontentsline{toc}{subsection}{Windows}

\begin{enumerate}
\def\labelenumi{\arabic{enumi}.}
\tightlist
\item
  Acesse \url{https://cran.r-project.org/bin/windows/base/}
\item
  Baixe o instalador mais recente (ex: \texttt{R-4.5.1-win.exe})
\item
  Execute o instalador e siga as instruções padrão
\end{enumerate}

\subsection*{macOS}\label{macos}
\addcontentsline{toc}{subsection}{macOS}

\begin{enumerate}
\def\labelenumi{\arabic{enumi}.}
\tightlist
\item
  Acesse \url{https://cran.r-project.org/bin/macosx/}
\item
  Baixe o arquivo \texttt{.pkg} apropriado para sua versão do macOS
\item
  Execute o instalador
\end{enumerate}

\subsection*{Linux (Ubuntu/Debian)}\label{linux-ubuntudebian}
\addcontentsline{toc}{subsection}{Linux (Ubuntu/Debian)}

\begin{Shaded}
\begin{Highlighting}[]
\CommentTok{\# Adicionar repositório CRAN}
\FunctionTok{sudo}\NormalTok{ apt update}
\FunctionTok{sudo}\NormalTok{ apt install }\AttributeTok{{-}y}\NormalTok{ software{-}properties{-}common dirmngr}
\FunctionTok{wget} \AttributeTok{{-}qO{-}}\NormalTok{ https://cloud.r{-}project.org/bin/linux/ubuntu/marutter\_pubkey.asc }\KeywordTok{|} \FunctionTok{sudo}\NormalTok{ tee }\AttributeTok{{-}a}\NormalTok{ /etc/apt/trusted.gpg.d/cran\_ubuntu\_key.asc}

\CommentTok{\# Adicionar repositório}
\FunctionTok{sudo}\NormalTok{ add{-}apt{-}repository }\StringTok{"deb https://cloud.r{-}project.org/bin/linux/ubuntu }\VariableTok{$(}\ExtensionTok{lsb\_release} \AttributeTok{{-}cs}\VariableTok{)}\StringTok{{-}cran40/"}

\CommentTok{\# Instalar R}
\FunctionTok{sudo}\NormalTok{ apt update}
\FunctionTok{sudo}\NormalTok{ apt install }\AttributeTok{{-}y}\NormalTok{ r{-}base r{-}base{-}dev}
\end{Highlighting}
\end{Shaded}

\subsection*{Verificar Instalação}\label{verificar-instalauxe7uxe3o}
\addcontentsline{toc}{subsection}{Verificar Instalação}

Abra um terminal ou prompt de comando e execute:

\begin{Shaded}
\begin{Highlighting}[]
\ExtensionTok{R} \AttributeTok{{-}{-}version}
\end{Highlighting}
\end{Shaded}

Você deve ver algo como:

\begin{verbatim}
R version 4.5.1 (2024-06-14) -- "Beagle Scouts"
\end{verbatim}

\section*{Instalação do RStudio
Desktop}\label{instalauxe7uxe3o-do-rstudio-desktop}
\addcontentsline{toc}{section}{Instalação do RStudio Desktop}

\markright{Instalação do RStudio Desktop}

RStudio é o IDE (Ambiente de Desenvolvimento Integrado) mais popular
para R.

\begin{enumerate}
\def\labelenumi{\arabic{enumi}.}
\tightlist
\item
  Acesse \url{https://posit.co/download/rstudio-desktop/}
\item
  Baixe o instalador para seu sistema operacional
\item
  Execute o instalador
\end{enumerate}

\section*{Instalação do Quarto}\label{instalauxe7uxe3o-do-quarto}
\addcontentsline{toc}{section}{Instalação do Quarto}

\markright{Instalação do Quarto}

Quarto é o sistema de publicação utilizado para criar este livro.

\subsection*{Windows}\label{windows-1}
\addcontentsline{toc}{subsection}{Windows}

\begin{enumerate}
\def\labelenumi{\arabic{enumi}.}
\tightlist
\item
  Acesse \url{https://quarto.org/docs/get-started/}
\item
  Baixe o instalador \texttt{.msi} para Windows
\item
  Execute o instalador
\end{enumerate}

\subsection*{macOS}\label{macos-1}
\addcontentsline{toc}{subsection}{macOS}

\begin{Shaded}
\begin{Highlighting}[]
\CommentTok{\# Via Homebrew}
\ExtensionTok{brew}\NormalTok{ install }\AttributeTok{{-}{-}cask}\NormalTok{ quarto}
\end{Highlighting}
\end{Shaded}

Ou baixe o instalador \texttt{.pkg} do site oficial.

\subsection*{Linux}\label{linux}
\addcontentsline{toc}{subsection}{Linux}

\begin{Shaded}
\begin{Highlighting}[]
\CommentTok{\# Ubuntu/Debian}
\FunctionTok{wget}\NormalTok{ https://github.com/quarto{-}dev/quarto{-}cli/releases/download/v1.6.39/quarto{-}1.6.39{-}linux{-}amd64.deb}
\FunctionTok{sudo}\NormalTok{ dpkg }\AttributeTok{{-}i}\NormalTok{ quarto{-}1.6.39{-}linux{-}amd64.deb}
\end{Highlighting}
\end{Shaded}

\subsection*{Verificar Instalação}\label{verificar-instalauxe7uxe3o-1}
\addcontentsline{toc}{subsection}{Verificar Instalação}

\begin{Shaded}
\begin{Highlighting}[]
\ExtensionTok{quarto} \AttributeTok{{-}{-}version}
\end{Highlighting}
\end{Shaded}

\section*{Instalação de Pacotes R}\label{instalauxe7uxe3o-de-pacotes-r}
\addcontentsline{toc}{section}{Instalação de Pacotes R}

\markright{Instalação de Pacotes R}

\subsection*{Pacotes Essenciais}\label{pacotes-essenciais}
\addcontentsline{toc}{subsection}{Pacotes Essenciais}

Abra o R ou RStudio e execute:

\begin{Shaded}
\begin{Highlighting}[]
\CommentTok{\# Manipulação e visualização de dados}
\FunctionTok{install.packages}\NormalTok{(}\StringTok{"tidyverse"}\NormalTok{)  }\CommentTok{\# Inclui dplyr, tidyr, ggplot2, readr, etc.}

\CommentTok{\# Análise de confiabilidade e sobrevivência}
\FunctionTok{install.packages}\NormalTok{(}\StringTok{"survival"}\NormalTok{)}
\FunctionTok{install.packages}\NormalTok{(}\StringTok{"flexsurv"}\NormalTok{)}
\FunctionTok{install.packages}\NormalTok{(}\StringTok{"WeibullR"}\NormalTok{)}

\CommentTok{\# Tabelas e relatórios}
\FunctionTok{install.packages}\NormalTok{(}\StringTok{"knitr"}\NormalTok{)}
\FunctionTok{install.packages}\NormalTok{(}\StringTok{"kableExtra"}\NormalTok{)}
\FunctionTok{install.packages}\NormalTok{(}\StringTok{"DT"}\NormalTok{)}
\FunctionTok{install.packages}\NormalTok{(}\StringTok{"broom"}\NormalTok{)}

\CommentTok{\# Visualização avançada}
\FunctionTok{install.packages}\NormalTok{(}\StringTok{"plotly"}\NormalTok{)}
\FunctionTok{install.packages}\NormalTok{(}\StringTok{"patchwork"}\NormalTok{)}
\end{Highlighting}
\end{Shaded}

\subsection*{Pacotes para Manutenção
Preditiva}\label{pacotes-para-manutenuxe7uxe3o-preditiva}
\addcontentsline{toc}{subsection}{Pacotes para Manutenção Preditiva}

\begin{Shaded}
\begin{Highlighting}[]
\FunctionTok{install.packages}\NormalTok{(}\StringTok{"caret"}\NormalTok{)        }\CommentTok{\# Machine learning}
\FunctionTok{install.packages}\NormalTok{(}\StringTok{"randomForest"}\NormalTok{) }\CommentTok{\# Random Forest}
\FunctionTok{install.packages}\NormalTok{(}\StringTok{"prophet"}\NormalTok{)      }\CommentTok{\# Séries temporais (Facebook Prophet)}
\FunctionTok{install.packages}\NormalTok{(}\StringTok{"anomalize"}\NormalTok{)    }\CommentTok{\# Detecção de anomalias}
\end{Highlighting}
\end{Shaded}

\subsection*{Instalação via Script}\label{instalauxe7uxe3o-via-script}
\addcontentsline{toc}{subsection}{Instalação via Script}

Para instalar todos os pacotes de uma vez, crie o arquivo
\texttt{resources/install\_packages.R}:

\begin{Shaded}
\begin{Highlighting}[]
\CommentTok{\# Script de instalação completa}
\NormalTok{packages }\OtherTok{\textless{}{-}} \FunctionTok{c}\NormalTok{(}
  \CommentTok{\# Manipulação de dados}
  \StringTok{"tidyverse"}\NormalTok{, }\StringTok{"dplyr"}\NormalTok{, }\StringTok{"tidyr"}\NormalTok{, }\StringTok{"readr"}\NormalTok{, }\StringTok{"lubridate"}\NormalTok{,}

  \CommentTok{\# Visualização}
  \StringTok{"ggplot2"}\NormalTok{, }\StringTok{"plotly"}\NormalTok{, }\StringTok{"patchwork"}\NormalTok{, }\StringTok{"ggthemes"}\NormalTok{,}

  \CommentTok{\# Confiabilidade e sobrevivência}
  \StringTok{"survival"}\NormalTok{, }\StringTok{"flexsurv"}\NormalTok{, }\StringTok{"WeibullR"}\NormalTok{, }\StringTok{"reliability"}\NormalTok{,}

  \CommentTok{\# Tabelas e reports}
  \StringTok{"knitr"}\NormalTok{, }\StringTok{"kableExtra"}\NormalTok{, }\StringTok{"DT"}\NormalTok{, }\StringTok{"broom"}\NormalTok{, }\StringTok{"gt"}\NormalTok{,}

  \CommentTok{\# Machine learning}
  \StringTok{"caret"}\NormalTok{, }\StringTok{"randomForest"}\NormalTok{, }\StringTok{"xgboost"}\NormalTok{,}

  \CommentTok{\# Séries temporais}
  \StringTok{"prophet"}\NormalTok{, }\StringTok{"anomalize"}\NormalTok{, }\StringTok{"forecast"}\NormalTok{,}

  \CommentTok{\# Simulação}
  \StringTok{"simmer"}\NormalTok{, }\StringTok{"MASS"}
\NormalTok{)}

\CommentTok{\# Verificar e instalar apenas os que faltam}
\NormalTok{new\_packages }\OtherTok{\textless{}{-}}\NormalTok{ packages[}\SpecialCharTok{!}\NormalTok{(packages }\SpecialCharTok{\%in\%} \FunctionTok{installed.packages}\NormalTok{()[,}\StringTok{"Package"}\NormalTok{])]}
\ControlFlowTok{if}\NormalTok{(}\FunctionTok{length}\NormalTok{(new\_packages)) }\FunctionTok{install.packages}\NormalTok{(new\_packages)}

\FunctionTok{cat}\NormalTok{(}\StringTok{"Pacotes instalados com sucesso!}\SpecialCharTok{\textbackslash{}n}\StringTok{"}\NormalTok{)}
\end{Highlighting}
\end{Shaded}

Execute no R:

\begin{Shaded}
\begin{Highlighting}[]
\FunctionTok{source}\NormalTok{(}\StringTok{"resources/install\_packages.R"}\NormalTok{)}
\end{Highlighting}
\end{Shaded}

\section*{Configuração do
Ambiente}\label{configurauxe7uxe3o-do-ambiente}
\addcontentsline{toc}{section}{Configuração do Ambiente}

\markright{Configuração do Ambiente}

\subsection*{\texorpdfstring{Arquivo
\texttt{.Rprofile}}{Arquivo .Rprofile}}\label{arquivo-.rprofile}
\addcontentsline{toc}{subsection}{Arquivo \texttt{.Rprofile}}

Crie um arquivo \texttt{.Rprofile} na raiz do projeto para configurações
globais:

\begin{Shaded}
\begin{Highlighting}[]
\CommentTok{\# .Rprofile}
\FunctionTok{options}\NormalTok{(}
  \AttributeTok{repos =} \FunctionTok{c}\NormalTok{(}\AttributeTok{CRAN =} \StringTok{"https://cloud.r{-}project.org/"}\NormalTok{),}
  \AttributeTok{scipen =} \DecValTok{999}\NormalTok{,           }\CommentTok{\# Desabilitar notação científica}
  \AttributeTok{stringsAsFactors =} \ConstantTok{FALSE}\NormalTok{,}
  \AttributeTok{max.print =} \DecValTok{100}
\NormalTok{)}

\CommentTok{\# Carregar pacotes frequentes automaticamente}
\ControlFlowTok{if}\NormalTok{ (}\FunctionTok{interactive}\NormalTok{()) \{}
  \FunctionTok{suppressMessages}\NormalTok{(\{}
    \FunctionTok{library}\NormalTok{(tidyverse)}
    \FunctionTok{library}\NormalTok{(here)}
\NormalTok{  \})}
  \FunctionTok{cat}\NormalTok{(}\StringTok{"}\SpecialCharTok{\textbackslash{}n}\StringTok{ Ambiente R configurado! }\SpecialCharTok{\textbackslash{}n\textbackslash{}n}\StringTok{"}\NormalTok{)}
\NormalTok{\}}
\end{Highlighting}
\end{Shaded}

\subsection*{\texorpdfstring{Arquivo
\texttt{.Renviron}}{Arquivo .Renviron}}\label{arquivo-.renviron}
\addcontentsline{toc}{subsection}{Arquivo \texttt{.Renviron}}

Para variáveis de ambiente (ex: chaves de API):

\begin{Shaded}
\begin{Highlighting}[]
\CommentTok{\# .Renviron}
\VariableTok{R\_MAX\_NUM\_DLLS}\OperatorTok{=}\NormalTok{300}
\end{Highlighting}
\end{Shaded}

\section*{VS Code (Alternativa ao
RStudio)}\label{vs-code-alternativa-ao-rstudio}
\addcontentsline{toc}{section}{VS Code (Alternativa ao RStudio)}

\markright{VS Code (Alternativa ao RStudio)}

Se preferir usar VS Code:

\subsection*{Extensões Necessárias}\label{extensuxf5es-necessuxe1rias}
\addcontentsline{toc}{subsection}{Extensões Necessárias}

\begin{enumerate}
\def\labelenumi{\arabic{enumi}.}
\tightlist
\item
  \textbf{R} (REditorSupport.r)
\item
  \textbf{Quarto} (quarto.quarto)
\item
  \textbf{R Debugger} (RDebugger.r-debugger)
\end{enumerate}

\subsection*{Configuração}\label{configurauxe7uxe3o}
\addcontentsline{toc}{subsection}{Configuração}

Crie \texttt{.vscode/settings.json}:

\begin{Shaded}
\begin{Highlighting}[]
\FunctionTok{\{}
  \DataTypeTok{"r.rterm.windows"}\FunctionTok{:} \StringTok{"C:}\CharTok{\textbackslash{}\textbackslash{}}\StringTok{Program Files}\CharTok{\textbackslash{}\textbackslash{}}\StringTok{R}\CharTok{\textbackslash{}\textbackslash{}}\StringTok{R{-}4.5.1}\CharTok{\textbackslash{}\textbackslash{}}\StringTok{bin}\CharTok{\textbackslash{}\textbackslash{}}\StringTok{R.exe"}\FunctionTok{,}
  \DataTypeTok{"r.rterm.linux"}\FunctionTok{:} \StringTok{"/usr/bin/R"}\FunctionTok{,}
  \DataTypeTok{"r.rterm.mac"}\FunctionTok{:} \StringTok{"/usr/local/bin/R"}\FunctionTok{,}
  \DataTypeTok{"r.bracketedPaste"}\FunctionTok{:} \KeywordTok{true}\FunctionTok{,}
  \DataTypeTok{"r.sessionWatcher"}\FunctionTok{:} \KeywordTok{true}\FunctionTok{,}
  \DataTypeTok{"quarto.render.renderOnSave"}\FunctionTok{:} \KeywordTok{false}
\FunctionTok{\}}
\end{Highlighting}
\end{Shaded}

\section*{Verificação Final}\label{verificauxe7uxe3o-final}
\addcontentsline{toc}{section}{Verificação Final}

\markright{Verificação Final}

Execute o seguinte script para verificar se tudo está configurado:

\begin{Shaded}
\begin{Highlighting}[]
\CommentTok{\# Verificar versão do R}
\FunctionTok{cat}\NormalTok{(}\StringTok{"R version:"}\NormalTok{, R.version.string, }\StringTok{"}\SpecialCharTok{\textbackslash{}n}\StringTok{"}\NormalTok{)}

\CommentTok{\# Verificar pacotes principais}
\NormalTok{required\_packages }\OtherTok{\textless{}{-}} \FunctionTok{c}\NormalTok{(}\StringTok{"tidyverse"}\NormalTok{, }\StringTok{"survival"}\NormalTok{, }\StringTok{"ggplot2"}\NormalTok{, }\StringTok{"knitr"}\NormalTok{)}
\NormalTok{missing }\OtherTok{\textless{}{-}}\NormalTok{ required\_packages[}\SpecialCharTok{!}\FunctionTok{sapply}\NormalTok{(required\_packages, requireNamespace, }\AttributeTok{quietly =} \ConstantTok{TRUE}\NormalTok{)]}

\ControlFlowTok{if}\NormalTok{ (}\FunctionTok{length}\NormalTok{(missing) }\SpecialCharTok{==} \DecValTok{0}\NormalTok{) \{}
  \FunctionTok{cat}\NormalTok{(}\StringTok{" Todos os pacotes necessários estão instalados!}\SpecialCharTok{\textbackslash{}n}\StringTok{"}\NormalTok{)}
\NormalTok{\} }\ControlFlowTok{else}\NormalTok{ \{}
  \FunctionTok{cat}\NormalTok{(}\StringTok{" Pacotes faltando:"}\NormalTok{, }\FunctionTok{paste}\NormalTok{(missing, }\AttributeTok{collapse =} \StringTok{", "}\NormalTok{), }\StringTok{"}\SpecialCharTok{\textbackslash{}n}\StringTok{"}\NormalTok{)}
\NormalTok{\}}

\CommentTok{\# Verificar Quarto (via terminal)}
\FunctionTok{system}\NormalTok{(}\StringTok{"quarto {-}{-}version"}\NormalTok{)}
\end{Highlighting}
\end{Shaded}

\section*{Estrutura de Projeto
Recomendada}\label{estrutura-de-projeto-recomendada}
\addcontentsline{toc}{section}{Estrutura de Projeto Recomendada}

\markright{Estrutura de Projeto Recomendada}

Organize seus arquivos assim:

\begin{verbatim}
QuartoBook/
├── .Rprofile              # Configurações R
├── .Renviron              # Variáveis de ambiente
├── _quarto.yml            # Configuração Quarto
├── chapters/              # Capítulos
├── appendices/            # Apêndices
├── resources/
│   ├── data/              # Dados (.csv, .rds)
│   ├── scripts/           # Scripts R auxiliares
│   └── images/            # Figuras
└── docs/                  # Output renderizado
\end{verbatim}

\section*{Próximos Passos}\label{pruxf3ximos-passos}
\addcontentsline{toc}{section}{Próximos Passos}

\markright{Próximos Passos}

Agora que seu ambiente está configurado:

\begin{enumerate}
\def\labelenumi{\arabic{enumi}.}
\tightlist
\item
  Clone ou baixe os materiais do livro
\item
  Abra o projeto no RStudio ou VS Code
\item
  Execute \texttt{quarto\ preview} para visualizar o livro
\item
  Comece pelo Capítulo 1: Fundamentos de Confiabilidade
\end{enumerate}

\begin{center}\rule{0.5\linewidth}{0.5pt}\end{center}

🔔 \textbf{Dica:} Mantenha seus pacotes R atualizados com:

\begin{Shaded}
\begin{Highlighting}[]
\FunctionTok{update.packages}\NormalTok{(}\AttributeTok{ask =} \ConstantTok{FALSE}\NormalTok{)}
\end{Highlighting}
\end{Shaded}

✅ \textbf{Ambiente configurado! Você está pronto para começar!}

\part{Contexto e Aplicação em Mineração}

\chapter{Introdução}\label{sec-introducao}

Durante muito tempo, as empresas focaram suas atenções na capacidade de
projetar produtos que incorporem características e atributos otimizados,
para atender a necessidades e desejos dos clientes {[}1{]}. Entretanto,
segundo {[}2{]}, com vistas ao processo de produção na indústria da
mineração, a qualidade do produto está relacionada ao teor do minério
explorado, tornando o aumento da produção e a redução de custos um dos
principais objetivos estratégicos. Desta forma, o conjunto das
excelências em cada processo, representa a qualidade da organização
sendo, este, o grande objetivo das mineradoras {[}3{]}, {[}4{]}.

No contexto brasileiro, a exploração de reserva de minerais,
principalmente de minério de Ferro, é predominantemente realizada pelo
método de lavra a céu aberto e envolve atividades de prospecção,
exploração, desenvolvimento e explotação e, após a retirada do minério,
este é transportado ao britador primário em que se inicia a etapa de
beneficiamento {[}5{]}, {[}6{]}, {[}7{]}.

Em particular, as atividades de exploração iniciam-se com a preparação
da área a ser lavrada para que ela possa ser perfurada e detonada. Então
a escavação e o carregamento, são feitos por equipamentos de carga que
ficam alocados na frente da lavra. Estes retiram o material e o carregam
nos equipamentos de transporte, caminhões, correias transportadoras,
vagões, entre outros. Os equipamentos de transporte levam o material até
um determinado ponto de descarga, podendo estes serem britadores, pilha
estéril ou pilha pulmão, onde o ciclo da operação recomeça {[}8{]},
{[}9{]}.

A empresa em estudo, trata-se de uma mineradora de grande porte
localizada no sudeste do Pará em que estabelece, como responsabilidade
para a equipe de Engenharia da Confiabilidade, identificar e gerenciar
riscos associados aos seus ativos críticos, classificados assim como os
que podem causar um evento material indesejado (MUE -- \emph{Material
Unwanted Event}) {[}10{]}. Desta forma, um \textbf{incidente} é um
evento não planejado ou uma ocorrência que resulta em dano ou outra
perda {[}11{]}, e, por sua vez, a \textbf{falha} é um incidente que
interrompe o funcionamento e acarreta a indisponibilidade do equipamento
{[}12{]}.

Nesse sentido, a gestão de ativos deve guiar a forma como as
organizações buscam um desempenho operacional excelente, integrando de
forma eficiente o Planejamento e Controle Operacional (PCO) e o
Planejamento e Controle da Manutenção (PCM) {[}13{]}, {[}14{]}.

No escopo de gestão de processos, a função primordial da operação é
garantir a realização dos planos de gestão sem impactar as metas dos
planos de produção {[}15{]}. De outra forma, a operação desempenha um
papel crucial na consecução dos resultados de produção, e uma falha pode
impactar diretamente nos resultados esperados. Portanto, garantir a
confiabilidade dos ativos físicos, gerir os riscos e minimizar as falhas
são atividades integradas entre PCO e PCM {[}16{]}.

Complementarmente, para exercer um papel estratégico, a manutenção
precisa estar alinhada com os resultados empresariais da organização,
principalmente, ao assegurar a disponibilidade dos equipamentos para a
operação, reduzindo a probabilidade de paradas de produção não
planejadas, minimizando a necessidade de manutenção corretiva (MC) e
priorizando as manutenções preventivas (MP).

A manutenção preventiva é uma estratégia importante para melhorar a
disponibilidade física (DF) de um sistema ou equipamento, reduzir o
tempo de inatividade e prolongar a vida útil dos equipamentos. No
entanto, embora esse tipo de manutenção seja uma tática importante para
reduzir a frequência de ocorrência de falhas no sistema, pode haver
casos em que a MP possa interferir ou prejudicar a disponibilidade dos
equipamentos, onde o tempo de execução da manutenção é maior que o tempo
planejado no plano de preventiva.

A implementação de uma política ótima de manutenção, ótima no sentido de
redução do custo total esperado da manutenção, é um problema que tem
sido abordado por vários autores e baseia-se em sistemas que operam sob
uma estratégia de manutenção, na qual MPs são realizadas em tempos
pré-determinados e MCs na ocorrência de falhas entre as manutenções
preventivas {[}17{]}, {[}18{]}, {[}19{]}.

Uma segunda abordagem, é que o planejamento da PM não implica
necessariamente em tempos determinísticos, ou seja, a periodicidade da
PM pode ser alterada devido a um plano de manutenção centrada em
confiabilidade (RCM -- \emph{Reliability Centered Maintenance}) ou pela
adoção de uma política de manutenção com abordagem dinâmica {[}20{]}. Se
a análise concluir que o sistema não é suficientemente confiável, a
periodicidade da PM, fixada antes do tempo de operação inicial, deve ser
reduzida {[}21{]}.

Ocorre que a manutenção preventiva muitas vezes vista como periódica e
determinística, pode ser entendida como aleatória, pois há alterações no
sistema não conhecidos ou não controlados. Desta forma, é importante
fazer investigação desse problema e, para os propósitos deste trabalho,
o objetivo é analisar as principais causas que influenciam no desempenho
dos indicadores de performance da manutenção dos caminhões, em
particular, com foco na mensuração dos impactos causados pela estratégia
de manutenção preventiva adotada nos caminhões \emph{off-road} na
empresa em estudo.

\section{Questões de Pesquisa}\label{sec-questoes-pesquisa}

Neste contexto, este trabalho formaliza o objetivo em quatro questões de
pesquisa (RQs):

\textbf{RQ1:} Quão aderente são as manutenções preventivas em relação às
quantidades previstas e aos tempos pré-determinados pelas pautas de
manutenção?

\textbf{RQ2:} Quantas falhas podem ocorrer devido à falta de aderência
das manutenções preventivas?

\textbf{RQ3:} Qual é o fator de impacto da manutenção preventiva?

\textbf{RQ4:} Qual é o impacto total na disponibilidade física do
sistema ou equipamento?

\section{Estrutura do Trabalho}\label{estrutura-do-trabalho}

Este trabalho está dividido da seguinte forma:

\begin{itemize}
\tightlist
\item
  Na \textbf{seção 2}, é apresentado a descrição do ciclo de processos
  de carregamento e transporte, bem como os indicadores chaves de
  desempenho utilizados no contexto da mineração e seus elementos de
  suporte.
\item
  Na \textbf{seção 3}, são descritos os fatores moderados da manutenção,
  abordando os principais conceitos e indicadores relacionados à
  manutenção, além de apresentar uma proposta de novos indicadores de
  análise.
\item
  Na \textbf{seção 4}, são apresentados os materiais, as informações da
  frota e os dados utilizados para as análises dos resultados.
\item
  Já os resultados e discussões, são apresentados na \textbf{seção 5}.
\item
  Por fim, a \textbf{seção 6} traz as conclusões deste trabalho.
\end{itemize}

\begin{center}\rule{0.5\linewidth}{0.5pt}\end{center}

\textbf{Próximo capítulo:} Medidas de Performance Operacional na
Mineração

\chapter{Medidas de Performance Operação na
Mineração}\label{sec-performance-mineracao}

\section{Ciclo de Processos de Carregamento e
Transporte}\label{sec-ciclo-processos}

O ciclo de processos de carregamento e transporte, pertencente à etapa
de lavra na mineração, ocorre apenas quando o caminhão se encontra com
status de ``apto para operação'' em que as métricas de análise e
indicadores próprios são estabelecidos para otimização do processo. A
Figura~\ref{fig-ciclo-processos} mostra que o ciclo inicia no
direcionamento do caminhão, que percorre vazio até o ponto de carga (TVV
-- tempo de viagem vazio), aguarda na fila para carga (TFC -- tempo de
fila para carga), e realiza a manobra (TMC -- tempo de manobra para
carregar) para o carregamento pela escavadeira (TC -- tempo de
carregamento). Com o caminhão cheio, ocorre o transporte (TVC -- tempo
de viagem cheio), aguarda na fila para basculamento (TFB -- tempo de
fila para bascular) e, finalmente, o basculamento (TB-- tempo de
basculamento). Para um único turno de trabalho da operação, um
determinado número de ciclos (NC) é realizado.

\begin{figure}

\centering{

\pandocbounded{\includegraphics[keepaspectratio]{chapters/../images/figura-01.png}}

}

\caption{\label{fig-ciclo-processos}Ciclo de processos de carregamento e
transporte}

\end{figure}%

\begin{tcolorbox}[enhanced jigsaw, title=\textcolor{quarto-callout-note-color}{\faInfo}\hspace{0.5em}{Nota}, bottomrule=.15mm, bottomtitle=1mm, titlerule=0mm, arc=.35mm, toptitle=1mm, rightrule=.15mm, colbacktitle=quarto-callout-note-color!10!white, opacitybacktitle=0.6, opacityback=0, colback=white, breakable, toprule=.15mm, leftrule=.75mm, left=2mm, colframe=quarto-callout-note-color-frame, coltitle=black]

O tempo total do ciclo influencia diretamente a produtividade da
operação e é afetado por fatores como distância de transporte, condições
da pista, eficiência do equipamento de carga e organização do tráfego.

\end{tcolorbox}

\section{Status do Equipamento}\label{sec-status-equipamento}

Como a medição é feita baseado em durações de tempo, os apontamentos são
realizados sempre que o caminhão alterar o seu status de operação,
estes, que são subdivididos em \textbf{4 classes}: apto para operação,
liberado pela manutenção, em manutenção ou em parada operacional.

Estas classes exercem um importante direcionamento das responsabilidades
de acompanhamento dos indicadores chaves de desempenho (ICDs) e possuem
problemas operacionais distintos. Por exemplo, quando o caminhão está:

\begin{itemize}
\item
  \textbf{``apto para operação'':} equipe de operação responsável e
  focada no ciclo de carregamento e transporte. Principal desafio:
  seleção e dimensionamento dos equipamentos. Segundo Mohtasham et
  al.~(2021), relaciona-se à escolha adequada dos equipamentos para
  manuseio ideal de materiais, resultando em melhoria da produtividade.
\item
  \textbf{``liberado pela manutenção'':} momento imediatamente posterior
  à execução da manutenção; transição de responsabilidade da manutenção
  para operação. Um acordo de nível de serviço (ANS) deve ser
  pré-estabelecido para eficiência na transição.
\item
  \textbf{``em manutenção'':} responsabilidade da equipe de manutenção,
  envolvendo planejamento, inspeção, programação e confiabilidade.
\item
  \textbf{``parada operacional'':} grandes paradas planejadas e
  gerenciadas considerando contingências; embora pertença à operação, o
  planejamento é conduzido pela gerência considerando toda a cadeia
  produtiva.
\end{itemize}

\section{Categoria de Horas}\label{sec-elementos-suporte}

Categoria de horas corresponde à divisão do tempo em classes de serviços
ou condição dos equipamentos, detalhando as horas na execução de
atividades de operação e manutenção no equipamento. Para melhor cálculo
dos indicadores, os tempos de funcionamento são agrupados em horas
dependendo da ocorrência dos eventos e do conjunto de equipamentos que
compõe o sistema. No ciclo de processos de transporte, os dados
coletados em caminhões ``off-highway'' são obtidos dos horímetros
instalados no sistema de medição e denominados tempos operacionais. A
partir daí, estas são apontadas no banco de dados em categorias de
horas, que são as: horas efetivas (HEF), horas de atraso operacional
(HAO), horas trabalhadas diversas (HTD), horas trabalhadas de
infraestrutura (HTI); horas ociosas interna (HOI); horas ociosas externa
(HOE); horas de manutenção corretiva (HMC), horas de acidente (HAC);
horas preventivas sistemática (MPS) e horas preventivas não sistemática
(MPNS). Iremos denominar estes elementos com o tempo de primeiro nível.

A Figura~\ref{fig-categorias-horas} ilustra o fluxo de relacionamento
entre estas categorias:

\begin{figure}

\centering{

\pandocbounded{\includegraphics[keepaspectratio]{chapters/../images/figura-categorias-horas.png}}

}

\caption{\label{fig-categorias-horas}Fluxo de relacionamento entre
categorias de horas}

\end{figure}%

Para o cálculo de indicadores, é comum realizar a síntese de algumas
categorias de horas ou, de outra forma, a junção de horas. Assim, os
\textbf{elementos de tempo de segundo nível} são:

\begin{itemize}
\tightlist
\item
  \textbf{Horas de manutenção corretiva total} (HMCT = HMC + HAC)
\item
  \textbf{Horas de manutenção preventiva} (HMP = MPS)
\item
  \textbf{Horas trabalhadas não produtivas} (HNTP = HTI + HTD)
\item
  \textbf{Horas ociosas} (HO = HOI + HOE)
\item
  \textbf{Horas trabalhadas produtivas} (HTP = HEF + HAO)
\end{itemize}

Por fim, para o \textbf{terceiro nível} temos:

\begin{itemize}
\tightlist
\item
  \textbf{Horas trabalhadas} (HT = HEF + HAO + HTD + HTI)
\item
  \textbf{Horas em manutenção} (HM = HMC + HAC + MPS + MPNS)
\item
  \textbf{Horas ociosas} (HO)
\end{itemize}

Ao agrupar as categorias de horas em dia, semana, mês ou ano, o valor
nominal será dado pela soma da duração de todos os apontamentos, cuja
razão esteja configurada para contabilizar esta categoria. Em todos os
níveis a soma das categorias representam as \textbf{horas calendário}
(HC), por exemplo, temos que:

\[
HC = HT + HM + HO
\]

\begin{tcolorbox}[enhanced jigsaw, title=\textcolor{quarto-callout-note-color}{\faInfo}\hspace{0.5em}{Legenda das Categorias de Horas}, bottomrule=.15mm, bottomtitle=1mm, titlerule=0mm, arc=.35mm, toptitle=1mm, rightrule=.15mm, colbacktitle=quarto-callout-note-color!10!white, opacitybacktitle=0.6, opacityback=0, colback=white, breakable, toprule=.15mm, leftrule=.75mm, left=2mm, colframe=quarto-callout-note-color-frame, coltitle=black]

\begin{itemize}
\tightlist
\item
  \textbf{HMC}: Horas de Manutenção Corretiva
\item
  \textbf{HAC}: Horas de Aguardo Corretiva
\item
  \textbf{MPS}: Manutenção Preventiva Sistemática
\item
  \textbf{MPNS}: Manutenção Preventiva Não Sistemática
\item
  \textbf{HTI}: Horas de Trabalho Improdutivo
\item
  \textbf{HTD}: Horas de Trabalho em Deslocamento
\item
  \textbf{HOI}: Horas Ociosas Internas
\item
  \textbf{HOE}: Horas Ociosas Externas
\item
  \textbf{HEF}: Horas Efetivas
\item
  \textbf{HAO}: Horas de Apoio à Operação
\end{itemize}

\end{tcolorbox}

\section{Indicadores Chave de Desempenho (ICD)}\label{sec-icd}

O principal produto do ciclo de transporte é a \textbf{massa
transportada (MT)}, este configura-se como elemento quantitativo.

Os ICDs são derivados de elementos de medição, logo, como um elemento
pode ser usado nas definições de vários ICDs, é improvável que eles
sejam independentes entre si. De acordo com {[}22{]}, há dois tipos
básicos de relacionamento, o primeiro dado pela relação de identidade
dos ICDs, baseado em suas definições e, o outro, é a relevância com
elementos de suporte compartilhados que podem ser obtidos por comparação
aos pares e/ou por agrupamentos em níveis.

Cada ICD revela um aspecto de desempenho para uma unidade de trabalho ou
sistema, derivado de dados monitorados de elementos de suporte. Os ICDs
podem ser agrupados por aqueles que representam um grupo de aspectos com
atributos semelhantes.

\subsection{Disponibilidade Física
(DF)}\label{disponibilidade-fuxedsica-df}

O ICD de \textbf{DF} indica a capacidade de um sistema estar em
condições de executar uma certa função, e sua indisponibilidade é devido
às horas de manutenção (HM), isto é, o indicador aponta para
oportunidades de melhoria relacionados à manutenção, e associados ao
status de ``em manutenção''.

\[
DF = \frac{HD}{HC} = \frac{HC - HM}{HC} \times 100\%
\]

Onde:

\begin{itemize}
\tightlist
\item
  \textbf{HD}: Horas Disponíveis
\item
  \textbf{HC}: Horas Calendário
\item
  \textbf{HM}: Horas de Manutenção
\end{itemize}

\subsection{Utilização Física (UF)}\label{utilizauxe7uxe3o-fuxedsica-uf}

De forma diferente, a \textbf{taxa de utilização física (UF)} leva em
consideração aspectos operacionais, sobre os quais a equipe de
manutenção não tem influência, ou seja, indica o quanto da
disponibilidade (HD) é utilizada efetivamente para operação (HT), logo,
o tempo não utilizado para operação corresponde à ociosidade do
equipamento (HO). Desta forma, temos que a ociosidade ocorre quando o
equipamento está em ``liberado pela manutenção'' ou ``parada
operacional''.

\[
UF = \frac{HT}{HD} \times 100\%
\]

\subsection{Produtividade (PR)}\label{produtividade-pr}

As horas trabalhadas (HT), são resultantes das ações de melhoria nos
demais indicadores e utilizadas para calcular tanto UF quanto o
indicador de \textbf{Produtividade (PR)}, este último, dado pelo
resultado da massa total transportada (MT) dividida por horas
trabalhadas (HT). Por conseguinte, a PR aponta para o ciclo dos
processos de carregamento e transporte, e está associado ao status
``apto para operação''.

\[
PR = \frac{MT}{HT}
\]

Onde \textbf{MT} é a massa total transportada (em toneladas).

\subsection{Perda de Produção por Indicadores}\label{sec-perda-producao}

Por fim, para o caso em que um sistema produtivo não cumpre as metas de
produção, é possível ainda calcular a \textbf{perda de produção por
indicadores de desempenho}, verificando a representatividade em
toneladas para cada indicador.

As equações de cálculo da perda de produção para cada indicador são
descritas a seguir:

\textbf{Perda de Produção por Disponibilidade Física:}

\begin{equation}\phantomsection\label{eq-perda-df}{
PP_{DF} = (HD_{programado} - HD_{realizado}) \times PR_{realizado}
}\end{equation}

\textbf{Perda de Produção por Utilização:}

\begin{equation}\phantomsection\label{eq-perda-uf}{
PP_{UF} = (HO_{realizado} - HO_{programado}) \times PR_{programado}
}\end{equation}

\textbf{Perda de Produção por Produtividade:}

\begin{equation}\phantomsection\label{eq-perda-pr}{
PP_{PR} = (PR_{programado} - PR_{realizado}) \times (HD_{programado} - HO_{realizado})
}\end{equation}

\begin{Shaded}
\begin{Highlighting}[]
\FunctionTok{library}\NormalTok{(dplyr)}
\FunctionTok{library}\NormalTok{(kableExtra)}

\CommentTok{\# Dados programados vs realizados}
\NormalTok{dados }\OtherTok{\textless{}{-}} \FunctionTok{data.frame}\NormalTok{(}
  \AttributeTok{Indicador =} \FunctionTok{c}\NormalTok{(}
    \StringTok{"Horas Calendário (HC)"}\NormalTok{, }\StringTok{"Horas Manutenção (HM)"}\NormalTok{,}
    \StringTok{"Horas Disponíveis (HD)"}\NormalTok{, }\StringTok{"Horas Ociosas (HO)"}\NormalTok{,}
    \StringTok{"Horas Trabalhadas (HT)"}\NormalTok{, }\StringTok{"Massa Transportada (MT)"}
\NormalTok{  ),}
  \AttributeTok{Programado =} \FunctionTok{c}\NormalTok{(}\DecValTok{720}\NormalTok{, }\DecValTok{72}\NormalTok{, }\DecValTok{648}\NormalTok{, }\DecValTok{65}\NormalTok{, }\DecValTok{583}\NormalTok{, }\DecValTok{140000}\NormalTok{),}
  \AttributeTok{Realizado =} \FunctionTok{c}\NormalTok{(}\DecValTok{720}\NormalTok{, }\DecValTok{95}\NormalTok{, }\DecValTok{625}\NormalTok{, }\DecValTok{88}\NormalTok{, }\DecValTok{537}\NormalTok{, }\DecValTok{125000}\NormalTok{),}
  \AttributeTok{Unidade =} \FunctionTok{c}\NormalTok{(}\StringTok{"h"}\NormalTok{, }\StringTok{"h"}\NormalTok{, }\StringTok{"h"}\NormalTok{, }\StringTok{"h"}\NormalTok{, }\StringTok{"h"}\NormalTok{, }\StringTok{"t"}\NormalTok{)}
\NormalTok{)}

\CommentTok{\# Calcular ICDs}
\NormalTok{icds }\OtherTok{\textless{}{-}} \FunctionTok{data.frame}\NormalTok{(}
  \AttributeTok{Indicador =} \FunctionTok{c}\NormalTok{(}
    \StringTok{"Disponibilidade Física (DF)"}\NormalTok{,}
    \StringTok{"Utilização Física (UF)"}\NormalTok{,}
    \StringTok{"Produtividade (PR)"}
\NormalTok{  ),}
  \AttributeTok{Programado =} \FunctionTok{c}\NormalTok{(}
    \FunctionTok{round}\NormalTok{((}\DecValTok{648} \SpecialCharTok{/} \DecValTok{720}\NormalTok{) }\SpecialCharTok{*} \DecValTok{100}\NormalTok{, }\DecValTok{1}\NormalTok{),}
    \FunctionTok{round}\NormalTok{((}\DecValTok{583} \SpecialCharTok{/} \DecValTok{648}\NormalTok{) }\SpecialCharTok{*} \DecValTok{100}\NormalTok{, }\DecValTok{1}\NormalTok{),}
    \FunctionTok{round}\NormalTok{(}\DecValTok{140000} \SpecialCharTok{/} \DecValTok{583}\NormalTok{, }\DecValTok{1}\NormalTok{)}
\NormalTok{  ),}
  \AttributeTok{Realizado =} \FunctionTok{c}\NormalTok{(}
    \FunctionTok{round}\NormalTok{((}\DecValTok{625} \SpecialCharTok{/} \DecValTok{720}\NormalTok{) }\SpecialCharTok{*} \DecValTok{100}\NormalTok{, }\DecValTok{1}\NormalTok{),}
    \FunctionTok{round}\NormalTok{((}\DecValTok{537} \SpecialCharTok{/} \DecValTok{625}\NormalTok{) }\SpecialCharTok{*} \DecValTok{100}\NormalTok{, }\DecValTok{1}\NormalTok{),}
    \FunctionTok{round}\NormalTok{(}\DecValTok{125000} \SpecialCharTok{/} \DecValTok{537}\NormalTok{, }\DecValTok{1}\NormalTok{)}
\NormalTok{  ),}
  \AttributeTok{Unidade =} \FunctionTok{c}\NormalTok{(}\StringTok{"\%"}\NormalTok{, }\StringTok{"\%"}\NormalTok{, }\StringTok{"t/h"}\NormalTok{)}
\NormalTok{)}

\CommentTok{\# Calcular perdas}
\NormalTok{PR\_real }\OtherTok{\textless{}{-}} \DecValTok{125000} \SpecialCharTok{/} \DecValTok{537}
\NormalTok{perdas }\OtherTok{\textless{}{-}} \FunctionTok{data.frame}\NormalTok{(}
  \AttributeTok{Tipo\_Perda =} \FunctionTok{c}\NormalTok{(}\StringTok{"Perda por DF"}\NormalTok{, }\StringTok{"Perda por UF"}\NormalTok{, }\StringTok{"Perda por PR"}\NormalTok{),}
  \AttributeTok{Calculo =} \FunctionTok{c}\NormalTok{(}
    \FunctionTok{paste0}\NormalTok{(}\StringTok{"("}\NormalTok{, }\DecValTok{648}\NormalTok{, }\StringTok{" {-} "}\NormalTok{, }\DecValTok{625}\NormalTok{, }\StringTok{") × "}\NormalTok{, }\FunctionTok{round}\NormalTok{(PR\_real, }\DecValTok{1}\NormalTok{)),}
    \FunctionTok{paste0}\NormalTok{(}\StringTok{"("}\NormalTok{, }\DecValTok{88}\NormalTok{, }\StringTok{" {-} "}\NormalTok{, }\DecValTok{65}\NormalTok{, }\StringTok{") × "}\NormalTok{, }\FunctionTok{round}\NormalTok{(PR\_real, }\DecValTok{1}\NormalTok{)),}
    \FunctionTok{paste0}\NormalTok{(}\StringTok{"("}\NormalTok{, }\FunctionTok{round}\NormalTok{(}\DecValTok{140000} \SpecialCharTok{/} \DecValTok{583}\NormalTok{, }\DecValTok{1}\NormalTok{), }\StringTok{" {-} "}\NormalTok{, }\FunctionTok{round}\NormalTok{(PR\_real, }\DecValTok{1}\NormalTok{), }\StringTok{") × "}\NormalTok{, }\DecValTok{583}\NormalTok{)}
\NormalTok{  ),}
  \AttributeTok{Perda\_ton =} \FunctionTok{c}\NormalTok{(}
    \FunctionTok{round}\NormalTok{((}\DecValTok{648} \SpecialCharTok{{-}} \DecValTok{625}\NormalTok{) }\SpecialCharTok{*}\NormalTok{ PR\_real, }\DecValTok{0}\NormalTok{),}
    \FunctionTok{round}\NormalTok{((}\DecValTok{88} \SpecialCharTok{{-}} \DecValTok{65}\NormalTok{) }\SpecialCharTok{*}\NormalTok{ PR\_real, }\DecValTok{0}\NormalTok{),}
    \FunctionTok{round}\NormalTok{((}\DecValTok{140000} \SpecialCharTok{/} \DecValTok{583} \SpecialCharTok{{-}}\NormalTok{ PR\_real) }\SpecialCharTok{*} \DecValTok{583}\NormalTok{, }\DecValTok{0}\NormalTok{)}
\NormalTok{  )}
\NormalTok{)}

\CommentTok{\# Apresentar resultados}
\FunctionTok{kable}\NormalTok{(dados, }\AttributeTok{caption =} \StringTok{"Dados Programados vs Realizados"}\NormalTok{) }\SpecialCharTok{\%\textgreater{}\%}
  \FunctionTok{kable\_styling}\NormalTok{(}\AttributeTok{bootstrap\_options =} \FunctionTok{c}\NormalTok{(}\StringTok{"striped"}\NormalTok{, }\StringTok{"hover"}\NormalTok{), }\AttributeTok{full\_width =} \ConstantTok{FALSE}\NormalTok{)}
\end{Highlighting}
\end{Shaded}

\begin{longtable}[t]{lrrl}
\caption{Dados Programados vs Realizados}\\
\toprule
Indicador & Programado & Realizado & Unidade\\
\midrule
Horas Calendário (HC) & 720 & 720 & h\\
Horas Manutenção (HM) & 72 & 95 & h\\
Horas Disponíveis (HD) & 648 & 625 & h\\
Horas Ociosas (HO) & 65 & 88 & h\\
Horas Trabalhadas (HT) & 583 & 537 & h\\
\addlinespace
Massa Transportada (MT) & 140000 & 125000 & t\\
\bottomrule
\end{longtable}

\begin{Shaded}
\begin{Highlighting}[]
\FunctionTok{kable}\NormalTok{(icds, }\AttributeTok{caption =} \StringTok{"Indicadores{-}Chave de Desempenho"}\NormalTok{) }\SpecialCharTok{\%\textgreater{}\%}
  \FunctionTok{kable\_styling}\NormalTok{(}\AttributeTok{bootstrap\_options =} \FunctionTok{c}\NormalTok{(}\StringTok{"striped"}\NormalTok{, }\StringTok{"hover"}\NormalTok{), }\AttributeTok{full\_width =} \ConstantTok{FALSE}\NormalTok{)}
\end{Highlighting}
\end{Shaded}

\begin{longtable}[t]{lrrl}
\caption{Indicadores-Chave de Desempenho}\\
\toprule
Indicador & Programado & Realizado & Unidade\\
\midrule
Disponibilidade Física (DF) & 90.0 & 86.8 & \%\\
Utilização Física (UF) & 90.0 & 85.9 & \%\\
Produtividade (PR) & 240.1 & 232.8 & t/h\\
\bottomrule
\end{longtable}

\begin{Shaded}
\begin{Highlighting}[]
\FunctionTok{kable}\NormalTok{(perdas,}
  \AttributeTok{caption =} \StringTok{"Análise de Perda de Produção"}\NormalTok{,}
  \AttributeTok{col.names =} \FunctionTok{c}\NormalTok{(}\StringTok{"Tipo de Perda"}\NormalTok{, }\StringTok{"Cálculo"}\NormalTok{, }\StringTok{"Perda (t)"}\NormalTok{)}
\NormalTok{) }\SpecialCharTok{\%\textgreater{}\%}
  \FunctionTok{kable\_styling}\NormalTok{(}\AttributeTok{bootstrap\_options =} \FunctionTok{c}\NormalTok{(}\StringTok{"striped"}\NormalTok{, }\StringTok{"hover"}\NormalTok{), }\AttributeTok{full\_width =} \ConstantTok{FALSE}\NormalTok{)}
\end{Highlighting}
\end{Shaded}

\begin{longtable}[t]{llr}
\caption{Análise de Perda de Produção}\\
\toprule
Tipo de Perda & Cálculo & Perda (t)\\
\midrule
Perda por DF & (648 - 625) × 232.8 & 5354\\
Perda por UF & (88 - 65) × 232.8 & 5354\\
Perda por PR & (240.1 - 232.8) × 583 & 4292\\
\bottomrule
\end{longtable}

Exemplo de cálculo de ICDs e perda de produção

\begin{tcolorbox}[enhanced jigsaw, title=\textcolor{quarto-callout-important-color}{\faExclamation}\hspace{0.5em}{Interpretação das Perdas}, bottomrule=.15mm, bottomtitle=1mm, titlerule=0mm, arc=.35mm, toptitle=1mm, rightrule=.15mm, colbacktitle=quarto-callout-important-color!10!white, opacitybacktitle=0.6, opacityback=0, colback=white, breakable, toprule=.15mm, leftrule=.75mm, left=2mm, colframe=quarto-callout-important-color-frame, coltitle=black]

A análise de perda de produção permite identificar qual indicador teve
maior impacto negativo na produção realizada, direcionando as ações de
melhoria:

\begin{itemize}
\tightlist
\item
  \textbf{Perda por DF}: Indica problemas de manutenção
\item
  \textbf{Perda por UF}: Indica problemas operacionais (ociosidade)
\item
  \textbf{Perda por PR}: Indica problemas de eficiência do ciclo
\end{itemize}

\end{tcolorbox}

\section{Resumo}\label{resumo}

🔔 \textbf{Pontos-chave:}

\begin{itemize}
\tightlist
\item
  O ciclo de carregamento e transporte é composto por etapas sequenciais
\item
  Status do equipamento define responsabilidades entre operação e
  manutenção
\item
  Elementos de suporte estruturam o cálculo dos ICDs através de
  categorias de horas
\item
  DF, UF e PR são os principais indicadores de desempenho
\item
  Perdas de produção podem ser mensuradas por indicador
\end{itemize}

\begin{center}\rule{0.5\linewidth}{0.5pt}\end{center}

\textbf{Próximo capítulo:} Fatores Moderadores da Manutenção

\chapter{Fatores Moderadores da
Manutenção}\label{sec-fatores-moderadores}

\section{Introdução}\label{introduuxe7uxe3o}

A efetividade das estratégias de manutenção não depende apenas da sua
execução técnica, mas também de diversos fatores moderadores que
influenciam os resultados. Este capítulo examina os principais fatores
que afetam o desempenho da manutenção em caminhões fora de estrada.

\section{Gestão da Manutenção}\label{sec-gestao-manutencao}

\subsection{Estrutura Organizacional}\label{estrutura-organizacional}

A organização da função de manutenção impacta diretamente a efetividade:

\subsubsection{Modelos de Gestão}\label{modelos-de-gestuxe3o}

\begin{enumerate}
\def\labelenumi{\arabic{enumi}.}
\tightlist
\item
  \textbf{Manutenção Centralizada}

  \begin{itemize}
  \tightlist
  \item
    Equipe única atende toda a operação
  \item
    Maior especialização técnica
  \item
    Possível aumento no tempo de resposta
  \end{itemize}
\item
  \textbf{Manutenção Descentralizada}

  \begin{itemize}
  \tightlist
  \item
    Equipes dedicadas por área/frota
  \item
    Resposta mais rápida
  \item
    Possível duplicação de recursos
  \end{itemize}
\item
  \textbf{Modelo Híbrido}

  \begin{itemize}
  \tightlist
  \item
    Manutenção de rotina descentralizada
  \item
    Reparos complexos centralizados
  \item
    Equilíbrio entre especialização e agilidade
  \end{itemize}
\end{enumerate}

\subsection{Sistemas de Informação}\label{sistemas-de-informauxe7uxe3o}

\subsubsection{CMMS (Computerized Maintenance Management
System)}\label{cmms-computerized-maintenance-management-system}

Funcionalidades essenciais:

\begin{itemize}
\tightlist
\item
  \textbf{Gestão de ordens de serviço}: Criação, priorização,
  atribuição, fechamento
\item
  \textbf{Controle de estoque}: Peças, ferramentas, consumíveis
\item
  \textbf{Histórico de equipamentos}: Registro de todas as intervenções
\item
  \textbf{Planejamento}: Programação de preventivas e paradas
\item
  \textbf{Indicadores}: Dashboards de performance
\end{itemize}

\subsection{Gestão de Estoque de
Peças}\label{gestuxe3o-de-estoque-de-peuxe7as}

\subsubsection{Classificação ABC}\label{classificauxe7uxe3o-abc}

Classificação de itens por valor e criticidade:

\begin{itemize}
\tightlist
\item
  \textbf{Classe A (20\% dos itens, 80\% do valor)}

  \begin{itemize}
  \tightlist
  \item
    Componentes de alto valor
  \item
    Gestão rigorosa de estoque
  \item
    Exemplo: motores, transmissões
  \end{itemize}
\item
  \textbf{Classe B (30\% dos itens, 15\% do valor)}

  \begin{itemize}
  \tightlist
  \item
    Componentes de valor intermediário
  \item
    Gestão moderada
  \end{itemize}
\item
  \textbf{Classe C (50\% dos itens, 5\% do valor)}

  \begin{itemize}
  \tightlist
  \item
    Componentes de baixo valor, alto volume
  \item
    Gestão simplificada
  \item
    Exemplo: filtros, porcas/parafusos
  \end{itemize}
\end{itemize}

\subsubsection{Estratégias de Estoque}\label{estratuxe9gias-de-estoque}

\begin{longtable}[]{@{}llll@{}}
\toprule\noalign{}
Item & Estratégia & Estoque & Lead Time \\
\midrule\noalign{}
\endhead
\bottomrule\noalign{}
\endlastfoot
\textbf{Crítico + Alto Valor} & Consignação & Mínimo & Reduzido \\
\textbf{Crítico + Baixo Valor} & Estoque Local & Alto & N/A \\
\textbf{Não-Crítico} & Just-in-Time & Mínimo & Aceitável \\
\end{longtable}

\subsection{Capacitação da Equipe}\label{capacitauxe7uxe3o-da-equipe}

Competências essenciais:

\begin{enumerate}
\def\labelenumi{\arabic{enumi}.}
\tightlist
\item
  \textbf{Técnicas}: Diagnóstico, reparo, uso de ferramentas
  especializadas
\item
  \textbf{Analíticas}: Interpretação de dados, análise de falhas
\item
  \textbf{Sistêmicas}: Compreensão de interdependências entre sistemas
\item
  \textbf{Digitais}: Uso de CMMS, sistemas de diagnóstico eletrônico
\end{enumerate}

\section{Análise de Dados
Recorrentes}\label{sec-analise-dados-recorrentes}

\subsection{Eventos Recorrentes em
Manutenção}\label{eventos-recorrentes-em-manutenuxe7uxe3o}

Diferente de análises de confiabilidade tradicionais que consideram
tempo até primeira falha, a manutenção de equipamentos envolve
\textbf{eventos recorrentes}:

\begin{itemize}
\tightlist
\item
  Múltiplas falhas do mesmo tipo
\item
  Múltiplas intervenções preventivas
\item
  Substituições de componentes reparáveis
\end{itemize}

\subsection{Modelagem de Processos
Recorrentes}\label{modelagem-de-processos-recorrentes}

\subsubsection{Processo de Poisson Homogêneo
(HPP)}\label{processo-de-poisson-homoguxeaneo-hpp}

Assume taxa de falha constante:

\[
\lambda(t) = \lambda
\]

Número esperado de eventos no intervalo \([0, t]\):

\[
\Lambda(t) = \lambda t
\]

\textbf{Limitação}: Não captura envelhecimento ou melhoria após reparos.

\subsubsection{Processo de Poisson Não-Homogêneo
(NHPP)}\label{processo-de-poisson-nuxe3o-homoguxeaneo-nhpp}

Permite taxa de falha variável no tempo:

\[
\Lambda(t) = \int_0^t \lambda(s) \, ds
\]

\paragraph{Modelo de Lei de Potência (Power Law
Process)}\label{modelo-de-lei-de-potuxeancia-power-law-process}

\[
\lambda(t) = \frac{\beta}{\eta} \left(\frac{t}{\eta}\right)^{\beta-1}
\]

Onde: - \(\beta > 1\): Deterioração (taxa crescente) - \(\beta < 1\):
Melhoria (taxa decrescente) - \(\beta = 1\): Taxa constante (HPP)

\begin{Shaded}
\begin{Highlighting}[]
\FunctionTok{library}\NormalTok{(ggplot2)}

\CommentTok{\# Parâmetros}
\NormalTok{eta }\OtherTok{\textless{}{-}} \DecValTok{1000}
\NormalTok{t }\OtherTok{\textless{}{-}} \FunctionTok{seq}\NormalTok{(}\DecValTok{0}\NormalTok{, }\DecValTok{3000}\NormalTok{, }\AttributeTok{by =} \DecValTok{10}\NormalTok{)}

\CommentTok{\# Calcular taxa para diferentes betas}
\NormalTok{dados\_taxa }\OtherTok{\textless{}{-}} \FunctionTok{data.frame}\NormalTok{(}
  \AttributeTok{t =} \FunctionTok{rep}\NormalTok{(t, }\DecValTok{3}\NormalTok{),}
  \AttributeTok{beta =} \FunctionTok{rep}\NormalTok{(}\FunctionTok{c}\NormalTok{(}\FloatTok{0.8}\NormalTok{, }\FloatTok{1.0}\NormalTok{, }\FloatTok{1.5}\NormalTok{), }\AttributeTok{each =} \FunctionTok{length}\NormalTok{(t)),}
  \AttributeTok{taxa =} \FunctionTok{c}\NormalTok{(}
\NormalTok{    (}\FloatTok{0.8}\SpecialCharTok{/}\NormalTok{eta) }\SpecialCharTok{*}\NormalTok{ (t}\SpecialCharTok{/}\NormalTok{eta)}\SpecialCharTok{\^{}}\NormalTok{(}\FloatTok{0.8}\DecValTok{{-}1}\NormalTok{),}
\NormalTok{    (}\FloatTok{1.0}\SpecialCharTok{/}\NormalTok{eta) }\SpecialCharTok{*}\NormalTok{ (t}\SpecialCharTok{/}\NormalTok{eta)}\SpecialCharTok{\^{}}\NormalTok{(}\FloatTok{1.0}\DecValTok{{-}1}\NormalTok{),}
\NormalTok{    (}\FloatTok{1.5}\SpecialCharTok{/}\NormalTok{eta) }\SpecialCharTok{*}\NormalTok{ (t}\SpecialCharTok{/}\NormalTok{eta)}\SpecialCharTok{\^{}}\NormalTok{(}\FloatTok{1.5}\DecValTok{{-}1}\NormalTok{)}
\NormalTok{  )}
\NormalTok{)}

\NormalTok{dados\_taxa}\SpecialCharTok{$}\NormalTok{beta\_label }\OtherTok{\textless{}{-}} \FunctionTok{factor}\NormalTok{(}
\NormalTok{  dados\_taxa}\SpecialCharTok{$}\NormalTok{beta,}
  \AttributeTok{levels =} \FunctionTok{c}\NormalTok{(}\FloatTok{0.8}\NormalTok{, }\FloatTok{1.0}\NormalTok{, }\FloatTok{1.5}\NormalTok{),}
  \AttributeTok{labels =} \FunctionTok{c}\NormalTok{(}\StringTok{"β = 0.8 (Melhoria)"}\NormalTok{, }\StringTok{"β = 1.0 (Constante)"}\NormalTok{, }\StringTok{"β = 1.5 (Deterioração)"}\NormalTok{)}
\NormalTok{)}

\FunctionTok{ggplot}\NormalTok{(dados\_taxa, }\FunctionTok{aes}\NormalTok{(}\AttributeTok{x =}\NormalTok{ t, }\AttributeTok{y =}\NormalTok{ taxa, }\AttributeTok{color =}\NormalTok{ beta\_label)) }\SpecialCharTok{+}
  \FunctionTok{geom\_line}\NormalTok{(}\AttributeTok{linewidth =} \FloatTok{1.2}\NormalTok{) }\SpecialCharTok{+}
  \FunctionTok{labs}\NormalTok{(}
    \AttributeTok{title =} \StringTok{"Taxa de Falha {-} Processo de Lei de Potência"}\NormalTok{,}
    \AttributeTok{x =} \StringTok{"Tempo (horas)"}\NormalTok{,}
    \AttributeTok{y =} \StringTok{"Taxa de Falha λ(t)"}\NormalTok{,}
    \AttributeTok{color =} \StringTok{"Parâmetro β"}
\NormalTok{  ) }\SpecialCharTok{+}
  \FunctionTok{scale\_color\_manual}\NormalTok{(}\AttributeTok{values =} \FunctionTok{c}\NormalTok{(}\StringTok{"\#27AE60"}\NormalTok{, }\StringTok{"\#95A5A6"}\NormalTok{, }\StringTok{"\#E74C3C"}\NormalTok{)) }\SpecialCharTok{+}
  \FunctionTok{theme\_minimal}\NormalTok{() }\SpecialCharTok{+}
  \FunctionTok{theme}\NormalTok{(}
    \AttributeTok{legend.position =} \StringTok{"bottom"}\NormalTok{,}
    \AttributeTok{plot.title =} \FunctionTok{element\_text}\NormalTok{(}\AttributeTok{face =} \StringTok{"bold"}\NormalTok{)}
\NormalTok{  )}
\end{Highlighting}
\end{Shaded}

\begin{figure}[H]

\centering{

\pandocbounded{\includegraphics[keepaspectratio]{chapters/03-fatores-moderadores_files/figure-pdf/fig-nhpp-taxa-1.pdf}}

}

\caption{\label{fig-nhpp-taxa}Função de taxa de falha do NHPP para
diferentes valores de β}

\end{figure}%

\subsection{Tipos de Reparo}\label{tipos-de-reparo}

\subsubsection{Reparo Mínimo (Minimal
Repair)}\label{reparo-muxednimo-minimal-repair}

\begin{itemize}
\tightlist
\item
  Restaura equipamento à condição imediatamente anterior à falha
\item
  Não altera a taxa de falha subsequente
\item
  Modelo: \textbf{NHPP sem reinício}
\end{itemize}

\subsubsection{Reparo Perfeito (Perfect
Repair)}\label{reparo-perfeito-perfect-repair}

\begin{itemize}
\tightlist
\item
  Restaura equipamento a condição ``como novo''
\item
  Reinicia o processo de envelhecimento
\item
  Modelo: \textbf{Processo de Renovação}
\end{itemize}

\subsubsection{Reparo Imperfeito (Imperfect
Repair)}\label{reparo-imperfeito-imperfect-repair}

\begin{itemize}
\tightlist
\item
  Melhora condição, mas não totalmente
\item
  Redução parcial da idade efetiva
\item
  Modelos: \textbf{Redução de idade virtual}, \textbf{Redução de
  intensidade}
\end{itemize}

\subsection{Estimação de
Parâmetros}\label{estimauxe7uxe3o-de-paruxe2metros}

\subsubsection{Método de Máxima
Verossimilhança}\label{muxe9todo-de-muxe1xima-verossimilhanuxe7a}

Para NHPP com Lei de Potência, dados \(n\) eventos nos tempos
\(t_1, t_2, \ldots, t_n\) em um período \([0, T]\):

Log-verossimilhança:

\[
\ell(\beta, \eta) = n \ln \beta - n \beta \ln \eta + (\beta - 1) \sum_{i=1}^{n} \ln t_i - \left(\frac{T}{\eta}\right)^\beta
\]

Estimadores:

\[
\hat{\beta} = \frac{n}{\sum_{i=1}^{n} \ln(T/t_i)}
\]

\[
\hat{\eta} = T \cdot n^{-1/\hat{\beta}}
\]

\begin{Shaded}
\begin{Highlighting}[]
\CommentTok{\# Função para estimar parâmetros do NHPP (Lei de Potência)}
\NormalTok{estimar\_nhpp }\OtherTok{\textless{}{-}} \ControlFlowTok{function}\NormalTok{(tempos\_falha, tempo\_observacao) \{}
\NormalTok{  n }\OtherTok{\textless{}{-}} \FunctionTok{length}\NormalTok{(tempos\_falha)}

  \CommentTok{\# Estimador de beta}
\NormalTok{  soma\_log }\OtherTok{\textless{}{-}} \FunctionTok{sum}\NormalTok{(}\FunctionTok{log}\NormalTok{(tempo\_observacao }\SpecialCharTok{/}\NormalTok{ tempos\_falha))}
\NormalTok{  beta\_hat }\OtherTok{\textless{}{-}}\NormalTok{ n }\SpecialCharTok{/}\NormalTok{ soma\_log}

  \CommentTok{\# Estimador de eta}
\NormalTok{  eta\_hat }\OtherTok{\textless{}{-}}\NormalTok{ tempo\_observacao }\SpecialCharTok{*}\NormalTok{ n}\SpecialCharTok{\^{}}\NormalTok{(}\SpecialCharTok{{-}}\DecValTok{1}\SpecialCharTok{/}\NormalTok{beta\_hat)}

  \CommentTok{\# Intervalo de confiança para beta (aproximado)}
\NormalTok{  se\_beta }\OtherTok{\textless{}{-}}\NormalTok{ beta\_hat }\SpecialCharTok{/} \FunctionTok{sqrt}\NormalTok{(n)}
\NormalTok{  ic\_beta }\OtherTok{\textless{}{-}}\NormalTok{ beta\_hat }\SpecialCharTok{+} \FunctionTok{c}\NormalTok{(}\SpecialCharTok{{-}}\FloatTok{1.96}\NormalTok{, }\FloatTok{1.96}\NormalTok{) }\SpecialCharTok{*}\NormalTok{ se\_beta}

  \FunctionTok{return}\NormalTok{(}\FunctionTok{list}\NormalTok{(}
    \AttributeTok{beta =}\NormalTok{ beta\_hat,}
    \AttributeTok{eta =}\NormalTok{ eta\_hat,}
    \AttributeTok{ic\_beta =}\NormalTok{ ic\_beta,}
    \AttributeTok{n\_eventos =}\NormalTok{ n,}
    \AttributeTok{tempo\_obs =}\NormalTok{ tempo\_observacao}
\NormalTok{  ))}
\NormalTok{\}}

\CommentTok{\# Exemplo: Falhas de um caminhão em 5000 horas}
\NormalTok{falhas }\OtherTok{\textless{}{-}} \FunctionTok{c}\NormalTok{(}\DecValTok{450}\NormalTok{, }\DecValTok{920}\NormalTok{, }\DecValTok{1580}\NormalTok{, }\DecValTok{2100}\NormalTok{, }\DecValTok{2890}\NormalTok{, }\DecValTok{3650}\NormalTok{, }\DecValTok{4200}\NormalTok{, }\DecValTok{4780}\NormalTok{)}

\NormalTok{resultado }\OtherTok{\textless{}{-}} \FunctionTok{estimar\_nhpp}\NormalTok{(falhas, }\DecValTok{5000}\NormalTok{)}

\FunctionTok{cat}\NormalTok{(}\StringTok{"Estimativas do NHPP:}\SpecialCharTok{\textbackslash{}n}\StringTok{"}\NormalTok{)}
\end{Highlighting}
\end{Shaded}

\begin{verbatim}
Estimativas do NHPP:
\end{verbatim}

\begin{Shaded}
\begin{Highlighting}[]
\FunctionTok{cat}\NormalTok{(}\StringTok{"β̂ ="}\NormalTok{, }\FunctionTok{round}\NormalTok{(resultado}\SpecialCharTok{$}\NormalTok{beta, }\DecValTok{3}\NormalTok{),}
    \StringTok{"(IC 95\%: ["}\NormalTok{, }\FunctionTok{round}\NormalTok{(resultado}\SpecialCharTok{$}\NormalTok{ic\_beta[}\DecValTok{1}\NormalTok{], }\DecValTok{3}\NormalTok{), }\StringTok{","}\NormalTok{,}
    \FunctionTok{round}\NormalTok{(resultado}\SpecialCharTok{$}\NormalTok{ic\_beta[}\DecValTok{2}\NormalTok{], }\DecValTok{3}\NormalTok{), }\StringTok{"])}\SpecialCharTok{\textbackslash{}n}\StringTok{"}\NormalTok{)}
\end{Highlighting}
\end{Shaded}

\begin{verbatim}
β̂ = 1.111 (IC 95%: [ 0.341 , 1.88 ])
\end{verbatim}

\begin{Shaded}
\begin{Highlighting}[]
\FunctionTok{cat}\NormalTok{(}\StringTok{"η̂ ="}\NormalTok{, }\FunctionTok{round}\NormalTok{(resultado}\SpecialCharTok{$}\NormalTok{eta, }\DecValTok{1}\NormalTok{), }\StringTok{"horas}\SpecialCharTok{\textbackslash{}n}\StringTok{"}\NormalTok{)}
\end{Highlighting}
\end{Shaded}

\begin{verbatim}
η̂ = 769 horas
\end{verbatim}

\begin{Shaded}
\begin{Highlighting}[]
\ControlFlowTok{if}\NormalTok{ (resultado}\SpecialCharTok{$}\NormalTok{beta }\SpecialCharTok{\textgreater{}} \DecValTok{1}\NormalTok{) \{}
  \FunctionTok{cat}\NormalTok{(}\StringTok{"}\SpecialCharTok{\textbackslash{}n}\StringTok{Interpretação: β \textgreater{} 1 indica DETERIORAÇÃO (taxa crescente)}\SpecialCharTok{\textbackslash{}n}\StringTok{"}\NormalTok{)}
\NormalTok{\} }\ControlFlowTok{else} \ControlFlowTok{if}\NormalTok{ (resultado}\SpecialCharTok{$}\NormalTok{beta }\SpecialCharTok{\textless{}} \DecValTok{1}\NormalTok{) \{}
  \FunctionTok{cat}\NormalTok{(}\StringTok{"}\SpecialCharTok{\textbackslash{}n}\StringTok{Interpretação: β \textless{} 1 indica MELHORIA (taxa decrescente)}\SpecialCharTok{\textbackslash{}n}\StringTok{"}\NormalTok{)}
\NormalTok{\} }\ControlFlowTok{else}\NormalTok{ \{}
  \FunctionTok{cat}\NormalTok{(}\StringTok{"}\SpecialCharTok{\textbackslash{}n}\StringTok{Interpretação: β ≈ 1 indica taxa CONSTANTE}\SpecialCharTok{\textbackslash{}n}\StringTok{"}\NormalTok{)}
\NormalTok{\}}
\end{Highlighting}
\end{Shaded}

\begin{verbatim}

Interpretação: β > 1 indica DETERIORAÇÃO (taxa crescente)
\end{verbatim}

\section{Política de Manutenção Preventiva
Periódica}\label{sec-politica-preventiva}

\subsection{Conceito}\label{conceito}

Manutenção preventiva periódica consiste em intervenções programadas em
intervalos fixos de tempo ou uso, independentemente da condição do
equipamento.

\textbf{Objetivos:} - Reduzir probabilidade de falhas - Aumentar vida
útil de componentes - Melhorar previsibilidade operacional

\subsection{Disponibilidade e
Mantenabilidade}\label{sec-disponibilidade-mantenabilidade}

\subsubsection{Disponibilidade}\label{disponibilidade}

A disponibilidade \(A\) é a fração do tempo em que o equipamento está
disponível para operar:

\[
A = \frac{MTBF}{MTBF + MTTR}
\]

Ou, considerando preventivas:

\[
A = \frac{Tempo_{Operando} + Tempo_{Espera}}{Tempo_{Total}}
\]

\subsubsection{Mantenabilidade}\label{mantenabilidade}

Mantenabilidade \(M(t)\) é a probabilidade de que um reparo seja
completado no tempo \(t\):

\[
M(t) = P(T_{reparo} \leq t)
\]

Para distribuição exponencial dos tempos de reparo:

\[
M(t) = 1 - e^{-\mu t}
\]

Onde \(\mu = 1/MTTR\) é a taxa de reparo.

\subsubsection{Relação com Manutenção
Preventiva}\label{relauxe7uxe3o-com-manutenuxe7uxe3o-preventiva}

Manutenções preventivas afetam:

\begin{itemize}
\tightlist
\item
  \textbf{MTBF}: Pode aumentar ao prevenir falhas
\item
  \textbf{MTTR}: Preventivas geralmente mais rápidas que corretivas
\item
  \textbf{Disponibilidade}: Efeito líquido depende do equilíbrio
\end{itemize}

\subsection{Confiabilidade e Periodicidade
Ótima}\label{sec-periodicidade-otima}

\subsubsection{Função Objetivo}\label{funuxe7uxe3o-objetivo}

A periodicidade ótima da preventiva minimiza custos ou maximiza
disponibilidade.

\textbf{Custo total por unidade de tempo:}

\[
C(T_p) = \frac{C_p \cdot R(T_p) + C_f \cdot [1 - R(T_p)]}{\int_0^{T_p} R(t) \, dt}
\]

Onde: - \(C_p\): Custo da preventiva - \(C_f\): Custo da falha
(corretiva) - \(R(T_p)\): Confiabilidade no intervalo \(T_p\) -
Denominador: Tempo médio de ciclo

\subsubsection{Condição de
Otimalidade}\label{condiuxe7uxe3o-de-otimalidade}

A derivada deve ser zero:

\[
\frac{d C(T_p)}{d T_p} = 0
\]

Simplificando (para Weibull):

\[
\frac{h(T_p^*)}{C_f/C_p - 1} = \frac{1 - R(T_p^*)}{\int_0^{T_p^*} R(t) \, dt}
\]

Onde \(h(t) = \lambda(t)\) é a taxa de falha.

\subsubsection{Algoritmo de Busca}\label{algoritmo-de-busca}

\begin{Shaded}
\begin{Highlighting}[]
\CommentTok{\# Função para calcular custo por unidade de tempo}
\NormalTok{custo\_preventiva }\OtherTok{\textless{}{-}} \ControlFlowTok{function}\NormalTok{(Tp, beta, eta, C\_prev, C\_falha) \{}
  \CommentTok{\# Confiabilidade em Tp}
\NormalTok{  R\_Tp }\OtherTok{\textless{}{-}} \FunctionTok{exp}\NormalTok{(}\SpecialCharTok{{-}}\NormalTok{(Tp}\SpecialCharTok{/}\NormalTok{eta)}\SpecialCharTok{\^{}}\NormalTok{beta)}

  \CommentTok{\# Integral da confiabilidade de 0 a Tp (tempo médio de ciclo)}
\NormalTok{  integrand }\OtherTok{\textless{}{-}} \ControlFlowTok{function}\NormalTok{(t) }\FunctionTok{exp}\NormalTok{(}\SpecialCharTok{{-}}\NormalTok{(t}\SpecialCharTok{/}\NormalTok{eta)}\SpecialCharTok{\^{}}\NormalTok{beta)}
\NormalTok{  integral\_R }\OtherTok{\textless{}{-}} \FunctionTok{integrate}\NormalTok{(integrand, }\DecValTok{0}\NormalTok{, Tp)}\SpecialCharTok{$}\NormalTok{value}

  \CommentTok{\# Custo por unidade de tempo}
\NormalTok{  custo }\OtherTok{\textless{}{-}}\NormalTok{ (C\_prev }\SpecialCharTok{*}\NormalTok{ R\_Tp }\SpecialCharTok{+}\NormalTok{ C\_falha }\SpecialCharTok{*}\NormalTok{ (}\DecValTok{1} \SpecialCharTok{{-}}\NormalTok{ R\_Tp)) }\SpecialCharTok{/}\NormalTok{ integral\_R}

  \FunctionTok{return}\NormalTok{(custo)}
\NormalTok{\}}

\CommentTok{\# Parâmetros exemplo}
\NormalTok{beta }\OtherTok{\textless{}{-}} \FloatTok{2.3}
\NormalTok{eta }\OtherTok{\textless{}{-}} \DecValTok{1500}
\NormalTok{C\_prev }\OtherTok{\textless{}{-}} \DecValTok{5000}
\NormalTok{C\_falha }\OtherTok{\textless{}{-}} \DecValTok{25000}

\CommentTok{\# Busca do intervalo ótimo}
\NormalTok{intervalos }\OtherTok{\textless{}{-}} \FunctionTok{seq}\NormalTok{(}\DecValTok{300}\NormalTok{, }\DecValTok{1800}\NormalTok{, }\AttributeTok{by =} \DecValTok{10}\NormalTok{)}
\NormalTok{custos }\OtherTok{\textless{}{-}} \FunctionTok{sapply}\NormalTok{(intervalos, custo\_preventiva,}
                 \AttributeTok{beta =}\NormalTok{ beta, }\AttributeTok{eta =}\NormalTok{ eta,}
                 \AttributeTok{C\_prev =}\NormalTok{ C\_prev, }\AttributeTok{C\_falha =}\NormalTok{ C\_falha)}

\NormalTok{Tp\_otimo }\OtherTok{\textless{}{-}}\NormalTok{ intervalos[}\FunctionTok{which.min}\NormalTok{(custos)]}
\NormalTok{custo\_otimo }\OtherTok{\textless{}{-}} \FunctionTok{min}\NormalTok{(custos)}

\FunctionTok{cat}\NormalTok{(}\StringTok{"Intervalo ótimo de preventiva:"}\NormalTok{, Tp\_otimo, }\StringTok{"horas}\SpecialCharTok{\textbackslash{}n}\StringTok{"}\NormalTok{)}
\end{Highlighting}
\end{Shaded}

\begin{verbatim}
Intervalo ótimo de preventiva: 740 horas
\end{verbatim}

\begin{Shaded}
\begin{Highlighting}[]
\FunctionTok{cat}\NormalTok{(}\StringTok{"Custo esperado:"}\NormalTok{, }\FunctionTok{round}\NormalTok{(custo\_otimo, }\DecValTok{2}\NormalTok{), }\StringTok{"$/hora}\SpecialCharTok{\textbackslash{}n}\StringTok{"}\NormalTok{)}
\end{Highlighting}
\end{Shaded}

\begin{verbatim}
Custo esperado: 12.28 $/hora
\end{verbatim}

\begin{Shaded}
\begin{Highlighting}[]
\CommentTok{\# Visualizar}
\FunctionTok{library}\NormalTok{(ggplot2)}
\FunctionTok{ggplot}\NormalTok{(}\FunctionTok{data.frame}\NormalTok{(}\AttributeTok{Tp =}\NormalTok{ intervalos, }\AttributeTok{Custo =}\NormalTok{ custos),}
       \FunctionTok{aes}\NormalTok{(}\AttributeTok{x =}\NormalTok{ Tp, }\AttributeTok{y =}\NormalTok{ Custo)) }\SpecialCharTok{+}
  \FunctionTok{geom\_line}\NormalTok{(}\AttributeTok{linewidth =} \FloatTok{1.2}\NormalTok{, }\AttributeTok{color =} \StringTok{"\#2C3E50"}\NormalTok{) }\SpecialCharTok{+}
  \FunctionTok{geom\_point}\NormalTok{(}\AttributeTok{x =}\NormalTok{ Tp\_otimo, }\AttributeTok{y =}\NormalTok{ custo\_otimo,}
             \AttributeTok{size =} \DecValTok{4}\NormalTok{, }\AttributeTok{color =} \StringTok{"\#E74C3C"}\NormalTok{) }\SpecialCharTok{+}
  \FunctionTok{geom\_vline}\NormalTok{(}\AttributeTok{xintercept =}\NormalTok{ Tp\_otimo, }\AttributeTok{linetype =} \StringTok{"dashed"}\NormalTok{, }\AttributeTok{alpha =} \FloatTok{0.5}\NormalTok{) }\SpecialCharTok{+}
  \FunctionTok{annotate}\NormalTok{(}\StringTok{"text"}\NormalTok{, }\AttributeTok{x =}\NormalTok{ Tp\_otimo }\SpecialCharTok{+} \DecValTok{150}\NormalTok{, }\AttributeTok{y =}\NormalTok{ custo\_otimo,}
           \AttributeTok{label =} \FunctionTok{paste}\NormalTok{(}\StringTok{"Ótimo:"}\NormalTok{, Tp\_otimo, }\StringTok{"h"}\NormalTok{), }\AttributeTok{hjust =} \DecValTok{0}\NormalTok{) }\SpecialCharTok{+}
  \FunctionTok{labs}\NormalTok{(}
    \AttributeTok{title =} \StringTok{"Otimização do Intervalo de Manutenção Preventiva"}\NormalTok{,}
    \AttributeTok{subtitle =} \FunctionTok{paste}\NormalTok{(}\StringTok{"β ="}\NormalTok{, beta, }\StringTok{", η ="}\NormalTok{, eta, }\StringTok{"| Cp ="}\NormalTok{, C\_prev, }\StringTok{", Cf ="}\NormalTok{, C\_falha),}
    \AttributeTok{x =} \StringTok{"Intervalo de Preventiva (horas)"}\NormalTok{,}
    \AttributeTok{y =} \StringTok{"Custo por Hora de Operação ($/h)"}
\NormalTok{  ) }\SpecialCharTok{+}
  \FunctionTok{theme\_minimal}\NormalTok{() }\SpecialCharTok{+}
  \FunctionTok{theme}\NormalTok{(}\AttributeTok{plot.title =} \FunctionTok{element\_text}\NormalTok{(}\AttributeTok{face =} \StringTok{"bold"}\NormalTok{))}
\end{Highlighting}
\end{Shaded}

\pandocbounded{\includegraphics[keepaspectratio]{chapters/03-fatores-moderadores_files/figure-pdf/otimizacao-periodicidade-1.pdf}}

\subsubsection{Sensibilidade aos
Parâmetros}\label{sensibilidade-aos-paruxe2metros}

O intervalo ótimo é sensível a:

\begin{enumerate}
\def\labelenumi{\arabic{enumi}.}
\tightlist
\item
  \textbf{Razão de custos \(C_f/C_p\)}: Maior razão → intervalos mais
  curtos
\item
  \textbf{Parâmetro de forma \(\beta\)}:

  \begin{itemize}
  \tightlist
  \item
    \(\beta > 1\): Preventiva é benéfica
  \item
    \(\beta < 1\): Preventiva pode ser contraproducente
  \end{itemize}
\item
  \textbf{Parâmetro de escala \(\eta\)}: Define escala de tempo dos
  intervalos
\end{enumerate}

\section{Medidas de Performance da
Manutenção}\label{sec-medidas-performance}

\subsection{Conceito de Impacto}\label{conceito-de-impacto}

Avaliar a efetividade da manutenção preventiva requer comparar cenários:

\begin{itemize}
\tightlist
\item
  \textbf{Com preventiva} vs.~\textbf{Sem preventiva}
\item
  \textbf{Diferentes intervalos} de preventiva
\item
  \textbf{Antes} vs.~\textbf{Depois} de mudança na política
\end{itemize}

\subsection{Mensuração de Impacto na Incidência de
Corretivas}\label{sec-impacto-corretivas}

\subsubsection{Taxa de Falhas Observada}\label{taxa-de-falhas-observada}

\[
\lambda_{obs} = \frac{Número_{Falhas}}{Tempo_{Exposição}}
\]

\subsubsection{Comparação de Taxas}\label{comparauxe7uxe3o-de-taxas}

\textbf{Razão de Taxas (Rate Ratio)}:

\[
RR = \frac{\lambda_{com\_prev}}{\lambda_{sem\_prev}}
\]

\begin{itemize}
\tightlist
\item
  \(RR < 1\): Preventiva reduz falhas
\item
  \(RR = 1\): Sem efeito
\item
  \(RR > 1\): Preventiva piora situação (indicando sobre-manutenção ou
  outros problemas)
\end{itemize}

\subsubsection{Modelo de Regressão de
Poisson}\label{modelo-de-regressuxe3o-de-poisson}

Para ajustar por covariáveis (idade do equipamento, condições
operacionais):

\[
\log(\lambda_i) = \beta_0 + \beta_1 \cdot Preventiva_i + \beta_2 \cdot Idade_i + \ldots
\]

\begin{Shaded}
\begin{Highlighting}[]
\FunctionTok{library}\NormalTok{(MASS)  }\CommentTok{\# para glm.nb (binomial negativa)}

\CommentTok{\# Dados exemplo (simulados)}
\FunctionTok{set.seed}\NormalTok{(}\DecValTok{123}\NormalTok{)}
\NormalTok{n\_equipamentos }\OtherTok{\textless{}{-}} \DecValTok{50}

\NormalTok{dados }\OtherTok{\textless{}{-}} \FunctionTok{data.frame}\NormalTok{(}
  \AttributeTok{equip\_id =} \DecValTok{1}\SpecialCharTok{:}\NormalTok{n\_equipamentos,}
  \AttributeTok{intensidade\_prev =} \FunctionTok{runif}\NormalTok{(n\_equipamentos, }\DecValTok{0}\NormalTok{, }\DecValTok{10}\NormalTok{),  }\CommentTok{\# preventivas/ano}
  \AttributeTok{idade\_anos =} \FunctionTok{runif}\NormalTok{(n\_equipamentos, }\DecValTok{0}\NormalTok{, }\DecValTok{10}\NormalTok{),}
  \AttributeTok{horas\_operacao =} \FunctionTok{runif}\NormalTok{(n\_equipamentos, }\DecValTok{3000}\NormalTok{, }\DecValTok{6000}\NormalTok{)}
\NormalTok{)}

\CommentTok{\# Simular número de falhas (menor com mais preventivas)}
\NormalTok{dados}\SpecialCharTok{$}\NormalTok{n\_falhas }\OtherTok{\textless{}{-}} \FunctionTok{rpois}\NormalTok{(n\_equipamentos,}
                        \AttributeTok{lambda =} \FunctionTok{exp}\NormalTok{(}\DecValTok{2} \SpecialCharTok{{-}} \FloatTok{0.15} \SpecialCharTok{*}\NormalTok{ dados}\SpecialCharTok{$}\NormalTok{intensidade\_prev }\SpecialCharTok{+}
                                     \FloatTok{0.08} \SpecialCharTok{*}\NormalTok{ dados}\SpecialCharTok{$}\NormalTok{idade\_anos))}

\CommentTok{\# Modelo de Poisson}
\NormalTok{modelo }\OtherTok{\textless{}{-}} \FunctionTok{glm}\NormalTok{(n\_falhas }\SpecialCharTok{\textasciitilde{}}\NormalTok{ intensidade\_prev }\SpecialCharTok{+}\NormalTok{ idade\_anos }\SpecialCharTok{+} \FunctionTok{offset}\NormalTok{(}\FunctionTok{log}\NormalTok{(horas\_operacao}\SpecialCharTok{/}\DecValTok{1000}\NormalTok{)),}
              \AttributeTok{data =}\NormalTok{ dados,}
              \AttributeTok{family =} \FunctionTok{poisson}\NormalTok{(}\AttributeTok{link =} \StringTok{"log"}\NormalTok{))}

\FunctionTok{summary}\NormalTok{(modelo)}

\CommentTok{\# Interpretação}
\FunctionTok{exp}\NormalTok{(}\FunctionTok{coef}\NormalTok{(modelo))  }\CommentTok{\# Rate Ratios}
\end{Highlighting}
\end{Shaded}

\subsection{Mensuração de Impacto na
Disponibilidade}\label{sec-impacto-disponibilidade}

\subsubsection{Diferença de
Disponibilidades}\label{diferenuxe7a-de-disponibilidades}

\[
\Delta A = A_{com\_prev} - A_{sem\_prev}
\]

Ou em termos percentuais:

\[
\Delta A_{\%} = \frac{A_{com\_prev} - A_{sem\_prev}}{A_{sem\_prev}} \times 100\%
\]

\subsubsection{Análise de Séries
Temporais}\label{anuxe1lise-de-suxe9ries-temporais}

Disponibilidade pode variar ao longo do tempo. Modelos de séries
temporais permitem:

\begin{itemize}
\tightlist
\item
  Detectar tendências
\item
  Avaliar sazonalidade
\item
  Comparar períodos com diferentes políticas
\end{itemize}

\subsubsection{Teste de Hipóteses}\label{teste-de-hipuxf3teses}

\textbf{H₀}: Não há diferença na disponibilidade média

\textbf{H₁}: Há diferença significativa

Teste t para comparar médias de períodos:

\begin{Shaded}
\begin{Highlighting}[]
\CommentTok{\# Dados exemplo: disponibilidade mensal antes e depois de mudança}
\NormalTok{antes }\OtherTok{\textless{}{-}} \FunctionTok{c}\NormalTok{(}\FloatTok{0.82}\NormalTok{, }\FloatTok{0.84}\NormalTok{, }\FloatTok{0.81}\NormalTok{, }\FloatTok{0.83}\NormalTok{, }\FloatTok{0.85}\NormalTok{, }\FloatTok{0.84}\NormalTok{, }\FloatTok{0.82}\NormalTok{, }\FloatTok{0.83}\NormalTok{, }\FloatTok{0.84}\NormalTok{, }\FloatTok{0.83}\NormalTok{, }\FloatTok{0.82}\NormalTok{, }\FloatTok{0.84}\NormalTok{)}
\NormalTok{depois }\OtherTok{\textless{}{-}} \FunctionTok{c}\NormalTok{(}\FloatTok{0.87}\NormalTok{, }\FloatTok{0.88}\NormalTok{, }\FloatTok{0.89}\NormalTok{, }\FloatTok{0.86}\NormalTok{, }\FloatTok{0.88}\NormalTok{, }\FloatTok{0.87}\NormalTok{, }\FloatTok{0.89}\NormalTok{, }\FloatTok{0.88}\NormalTok{, }\FloatTok{0.87}\NormalTok{, }\FloatTok{0.88}\NormalTok{, }\FloatTok{0.89}\NormalTok{, }\FloatTok{0.87}\NormalTok{)}

\CommentTok{\# Teste t}
\NormalTok{resultado\_teste }\OtherTok{\textless{}{-}} \FunctionTok{t.test}\NormalTok{(depois, antes, }\AttributeTok{alternative =} \StringTok{"greater"}\NormalTok{)}

\FunctionTok{cat}\NormalTok{(}\StringTok{"Disponibilidade média ANTES:"}\NormalTok{, }\FunctionTok{round}\NormalTok{(}\FunctionTok{mean}\NormalTok{(antes), }\DecValTok{3}\NormalTok{), }\StringTok{"}\SpecialCharTok{\textbackslash{}n}\StringTok{"}\NormalTok{)}
\end{Highlighting}
\end{Shaded}

\begin{verbatim}
Disponibilidade média ANTES: 0.831 
\end{verbatim}

\begin{Shaded}
\begin{Highlighting}[]
\FunctionTok{cat}\NormalTok{(}\StringTok{"Disponibilidade média DEPOIS:"}\NormalTok{, }\FunctionTok{round}\NormalTok{(}\FunctionTok{mean}\NormalTok{(depois), }\DecValTok{3}\NormalTok{), }\StringTok{"}\SpecialCharTok{\textbackslash{}n}\StringTok{"}\NormalTok{)}
\end{Highlighting}
\end{Shaded}

\begin{verbatim}
Disponibilidade média DEPOIS: 0.877 
\end{verbatim}

\begin{Shaded}
\begin{Highlighting}[]
\FunctionTok{cat}\NormalTok{(}\StringTok{"Diferença:"}\NormalTok{, }\FunctionTok{round}\NormalTok{(}\FunctionTok{mean}\NormalTok{(depois) }\SpecialCharTok{{-}} \FunctionTok{mean}\NormalTok{(antes), }\DecValTok{3}\NormalTok{), }\StringTok{"}\SpecialCharTok{\textbackslash{}n}\StringTok{"}\NormalTok{)}
\end{Highlighting}
\end{Shaded}

\begin{verbatim}
Diferença: 0.047 
\end{verbatim}

\begin{Shaded}
\begin{Highlighting}[]
\FunctionTok{cat}\NormalTok{(}\StringTok{"p{-}valor:"}\NormalTok{, }\FunctionTok{format.pval}\NormalTok{(resultado\_teste}\SpecialCharTok{$}\NormalTok{p.value), }\StringTok{"}\SpecialCharTok{\textbackslash{}n}\StringTok{"}\NormalTok{)}
\end{Highlighting}
\end{Shaded}

\begin{verbatim}
p-valor: 0.00000000025923 
\end{verbatim}

\begin{Shaded}
\begin{Highlighting}[]
\ControlFlowTok{if}\NormalTok{ (resultado\_teste}\SpecialCharTok{$}\NormalTok{p.value }\SpecialCharTok{\textless{}} \FloatTok{0.05}\NormalTok{) \{}
  \FunctionTok{cat}\NormalTok{(}\StringTok{"}\SpecialCharTok{\textbackslash{}n}\StringTok{✓ Diferença estatisticamente significativa (α = 0.05)}\SpecialCharTok{\textbackslash{}n}\StringTok{"}\NormalTok{)}
\NormalTok{\} }\ControlFlowTok{else}\NormalTok{ \{}
  \FunctionTok{cat}\NormalTok{(}\StringTok{"}\SpecialCharTok{\textbackslash{}n}\StringTok{✗ Diferença NÃO estatisticamente significativa (α = 0.05)}\SpecialCharTok{\textbackslash{}n}\StringTok{"}\NormalTok{)}
\NormalTok{\}}
\end{Highlighting}
\end{Shaded}

\begin{verbatim}

✓ Diferença estatisticamente significativa (α = 0.05)
\end{verbatim}

\subsubsection{Análise de Covariância
(ANCOVA)}\label{anuxe1lise-de-covariuxe2ncia-ancova}

Para controlar efeitos de variáveis confundidoras:

\[
Disponibilidade = \beta_0 + \beta_1 \cdot Política + \beta_2 \cdot Idade + \beta_3 \cdot Demanda + \epsilon
\]

\section{Resumo}\label{resumo-1}

🔔 \textbf{Pontos-chave:}

\begin{itemize}
\tightlist
\item
  Gestão eficaz requer organização, sistemas de informação e equipes
  capacitadas
\item
  Eventos recorrentes exigem modelos específicos (NHPP, processos de
  renovação)
\item
  Periodicidade ótima de preventiva equilibra custos e confiabilidade
\item
  Impacto da manutenção deve ser mensurado com métodos estatísticos
  robustos
\item
  Disponibilidade e incidência de falhas são métricas complementares
\end{itemize}

\begin{center}\rule{0.5\linewidth}{0.5pt}\end{center}

\textbf{Próximo capítulo:} Fundamentos de Confiabilidade

\part{Fundamentos e Teoria}

\chapter{Fundamentos de
Confiabilidade}\label{fundamentos-de-confiabilidade}

\section{Introdução}\label{introduuxe7uxe3o-1}

A \textbf{confiabilidade} é a probabilidade de um item desempenhar sua
função de forma adequada, sob condições estabelecidas, durante um
intervalo de tempo especificado. Este conceito fundamental é a base para
todas as análises de manutenção e gestão de ativos.

\section{Conceitos Básicos}\label{conceitos-buxe1sicos}

\subsection{\texorpdfstring{Confiabilidade
\(R(t)\)}{Confiabilidade R(t)}}\label{confiabilidade-rt}

A função confiabilidade representa a probabilidade de um sistema
sobreviver até o tempo \(t\):

\begin{equation}\phantomsection\label{eq-confiabilidade}{
R(t) = P(T > t)
}\end{equation}

onde \(T\) é a variável aleatória que representa o tempo até a falha.

\subsection{\texorpdfstring{Taxa de Falha
\(\lambda(t)\)}{Taxa de Falha \textbackslash lambda(t)}}\label{taxa-de-falha-lambdat}

A taxa de falha (ou função de risco) é definida como:

\begin{equation}\phantomsection\label{eq-taxa-falha}{
\lambda(t) = \frac{f(t)}{R(t)} = -\frac{d}{dt}\ln R(t)
}\end{equation}

\subsection{MTBF e MTTF}\label{mtbf-e-mttf}

\begin{itemize}
\tightlist
\item
  \textbf{MTBF} (Mean Time Between Failures): Tempo médio entre falhas
  para sistemas reparáveis
\item
  \textbf{MTTF} (Mean Time To Failure): Tempo médio até a primeira falha
  para sistemas não-reparáveis
\end{itemize}

\begin{equation}\phantomsection\label{eq-mttf}{
MTTF = \int_0^\infty R(t) \, dt
}\end{equation}

\subsection{Disponibilidade}\label{disponibilidade-1}

A disponibilidade é a probabilidade de um sistema estar operacional em
um dado momento:

\begin{equation}\phantomsection\label{eq-disponibilidade}{
A = \frac{MTBF}{MTBF + MTTR}
}\end{equation}

onde MTTR é o Mean Time To Repair (tempo médio de reparo).

\section{Tipos de Manutenção}\label{tipos-de-manutenuxe7uxe3o}

\subsection{Manutenção Corretiva}\label{manutenuxe7uxe3o-corretiva}

Intervenção realizada \textbf{após} a ocorrência de falha para restaurar
o funcionamento do equipamento.

\textbf{Vantagens:} - Menor custo inicial - Não interrompe operação
programada

\textbf{Desvantagens:} - Paradas não programadas - Custos imprevistos -
Possíveis danos secundários

\subsection{Manutenção Preventiva}\label{manutenuxe7uxe3o-preventiva}

Intervenção realizada em \textbf{intervalos predeterminados} para
reduzir a probabilidade de falha.

\textbf{Vantagens:} - Redução de falhas inesperadas - Planejamento de
paradas - Maior vida útil dos equipamentos

\textbf{Desvantagens:} - Custo fixo programado - Possível substituição
prematura de componentes

\subsection{Manutenção Preditiva}\label{manutenuxe7uxe3o-preditiva}

Intervenção baseada em \textbf{condição do equipamento}, monitorada por
sensores e análise de dados.

\textbf{Vantagens:} - Otimização de recursos - Manutenção just-in-time -
Maior disponibilidade

\textbf{Desvantagens:} - Investimento em sensores e sistemas - Requer
expertise em análise de dados

\section{Métricas Fundamentais}\label{muxe9tricas-fundamentais}

\subsection{Tabela de Métricas Principais}\label{tbl-metricas}

\begin{longtable}[]{@{}
  >{\raggedright\arraybackslash}p{(\linewidth - 6\tabcolsep) * \real{0.2143}}
  >{\raggedright\arraybackslash}p{(\linewidth - 6\tabcolsep) * \real{0.2143}}
  >{\raggedright\arraybackslash}p{(\linewidth - 6\tabcolsep) * \real{0.2143}}
  >{\raggedright\arraybackslash}p{(\linewidth - 6\tabcolsep) * \real{0.3571}}@{}}
\toprule\noalign{}
\begin{minipage}[b]{\linewidth}\raggedright
Métrica
\end{minipage} & \begin{minipage}[b]{\linewidth}\raggedright
Símbolo
\end{minipage} & \begin{minipage}[b]{\linewidth}\raggedright
Fórmula
\end{minipage} & \begin{minipage}[b]{\linewidth}\raggedright
Interpretação
\end{minipage} \\
\midrule\noalign{}
\endhead
\bottomrule\noalign{}
\endlastfoot
Confiabilidade & \(R(t)\) & \(P(T > t)\) & Probabilidade de
sobrevivência \\
Taxa de Falha & \(\lambda(t)\) & \(f(t)/R(t)\) & Risco instantâneo \\
MTBF & - & \(\int_0^\infty R(t) dt\) & Tempo médio entre falhas \\
Disponibilidade & \(A\) & \(MTBF/(MTBF+MTTR)\) & Percentual
operacional \\
\end{longtable}

\section{Exemplo Prático com R}\label{exemplo-pruxe1tico-com-r}

💡 \textbf{Exemplo}: Cálculo de métricas básicas de confiabilidade

\begin{Shaded}
\begin{Highlighting}[]
\FunctionTok{library}\NormalTok{(tidyverse)}

\CommentTok{\# Dados de tempo até falha (em horas)}
\NormalTok{tempos\_falha }\OtherTok{\textless{}{-}} \FunctionTok{c}\NormalTok{(}\DecValTok{120}\NormalTok{, }\DecValTok{145}\NormalTok{, }\DecValTok{180}\NormalTok{, }\DecValTok{200}\NormalTok{, }\DecValTok{250}\NormalTok{, }\DecValTok{310}\NormalTok{, }\DecValTok{420}\NormalTok{, }\DecValTok{500}\NormalTok{, }\DecValTok{650}\NormalTok{, }\DecValTok{800}\NormalTok{)}

\CommentTok{\# Calcular MTTF}
\NormalTok{mttf }\OtherTok{\textless{}{-}} \FunctionTok{mean}\NormalTok{(tempos\_falha)}

\CommentTok{\# Estimar taxa de falha (assumindo distribuição exponencial)}
\NormalTok{lambda\_est }\OtherTok{\textless{}{-}} \DecValTok{1} \SpecialCharTok{/}\NormalTok{ mttf}

\CommentTok{\# Calcular confiabilidade para diferentes tempos}
\NormalTok{tempos }\OtherTok{\textless{}{-}} \FunctionTok{seq}\NormalTok{(}\DecValTok{0}\NormalTok{, }\DecValTok{1000}\NormalTok{, }\AttributeTok{by =} \DecValTok{50}\NormalTok{)}
\NormalTok{confiabilidade }\OtherTok{\textless{}{-}} \FunctionTok{exp}\NormalTok{(}\SpecialCharTok{{-}}\NormalTok{lambda\_est }\SpecialCharTok{*}\NormalTok{ tempos)}

\CommentTok{\# Criar dataframe para visualização}
\NormalTok{df\_conf }\OtherTok{\textless{}{-}} \FunctionTok{data.frame}\NormalTok{(}
  \AttributeTok{tempo =}\NormalTok{ tempos,}
  \AttributeTok{confiabilidade =}\NormalTok{ confiabilidade}
\NormalTok{)}

\CommentTok{\# Visualizar}
\FunctionTok{ggplot}\NormalTok{(df\_conf, }\FunctionTok{aes}\NormalTok{(}\AttributeTok{x =}\NormalTok{ tempo, }\AttributeTok{y =}\NormalTok{ confiabilidade)) }\SpecialCharTok{+}
  \FunctionTok{geom\_line}\NormalTok{(}\AttributeTok{color =} \StringTok{"steelblue"}\NormalTok{, }\AttributeTok{linewidth =} \FloatTok{1.2}\NormalTok{) }\SpecialCharTok{+}
  \FunctionTok{geom\_hline}\NormalTok{(}\AttributeTok{yintercept =} \FloatTok{0.5}\NormalTok{, }\AttributeTok{linetype =} \StringTok{"dashed"}\NormalTok{, }\AttributeTok{color =} \StringTok{"red"}\NormalTok{) }\SpecialCharTok{+}
  \FunctionTok{labs}\NormalTok{(}
    \AttributeTok{title =} \StringTok{"Função Confiabilidade"}\NormalTok{,}
    \AttributeTok{x =} \StringTok{"Tempo (horas)"}\NormalTok{,}
    \AttributeTok{y =} \StringTok{"R(t)"}\NormalTok{,}
    \AttributeTok{caption =} \FunctionTok{paste}\NormalTok{(}\StringTok{"MTTF ="}\NormalTok{, }\FunctionTok{round}\NormalTok{(mttf, }\DecValTok{1}\NormalTok{), }\StringTok{"horas"}\NormalTok{)}
\NormalTok{  ) }\SpecialCharTok{+}
  \FunctionTok{theme\_minimal}\NormalTok{()}
\end{Highlighting}
\end{Shaded}

\pandocbounded{\includegraphics[keepaspectratio]{chapters/04-fundamentos_files/figure-pdf/exemplo-metricas-1.pdf}}

\subsection{Análise dos Resultados}\label{anuxe1lise-dos-resultados}

\begin{Shaded}
\begin{Highlighting}[]
\FunctionTok{cat}\NormalTok{(}\StringTok{"Métricas Calculadas:}\SpecialCharTok{\textbackslash{}n}\StringTok{"}\NormalTok{)}
\end{Highlighting}
\end{Shaded}

\begin{verbatim}
Métricas Calculadas:
\end{verbatim}

\begin{Shaded}
\begin{Highlighting}[]
\FunctionTok{cat}\NormalTok{(}\StringTok{"MTTF:"}\NormalTok{, }\FunctionTok{round}\NormalTok{(mttf, }\DecValTok{2}\NormalTok{), }\StringTok{"horas}\SpecialCharTok{\textbackslash{}n}\StringTok{"}\NormalTok{)}
\end{Highlighting}
\end{Shaded}

\begin{verbatim}
MTTF: 357.5 horas
\end{verbatim}

\begin{Shaded}
\begin{Highlighting}[]
\FunctionTok{cat}\NormalTok{(}\StringTok{"Taxa de falha (λ):"}\NormalTok{, }\FunctionTok{round}\NormalTok{(lambda\_est, }\DecValTok{6}\NormalTok{), }\StringTok{"falhas/hora}\SpecialCharTok{\textbackslash{}n}\StringTok{"}\NormalTok{)}
\end{Highlighting}
\end{Shaded}

\begin{verbatim}
Taxa de falha (λ): 0.002797 falhas/hora
\end{verbatim}

\begin{Shaded}
\begin{Highlighting}[]
\FunctionTok{cat}\NormalTok{(}\StringTok{"Confiabilidade em 300h:"}\NormalTok{, }\FunctionTok{round}\NormalTok{(}\FunctionTok{exp}\NormalTok{(}\SpecialCharTok{{-}}\NormalTok{lambda\_est }\SpecialCharTok{*} \DecValTok{300}\NormalTok{), }\DecValTok{4}\NormalTok{), }\StringTok{"}\SpecialCharTok{\textbackslash{}n}\StringTok{"}\NormalTok{)}
\end{Highlighting}
\end{Shaded}

\begin{verbatim}
Confiabilidade em 300h: 0.4321 
\end{verbatim}

\begin{Shaded}
\begin{Highlighting}[]
\FunctionTok{cat}\NormalTok{(}\StringTok{"Tempo para R(t) = 0.5:"}\NormalTok{, }\FunctionTok{round}\NormalTok{(}\FunctionTok{log}\NormalTok{(}\DecValTok{2}\NormalTok{) }\SpecialCharTok{/}\NormalTok{ lambda\_est, }\DecValTok{2}\NormalTok{), }\StringTok{"horas}\SpecialCharTok{\textbackslash{}n}\StringTok{"}\NormalTok{)}
\end{Highlighting}
\end{Shaded}

\begin{verbatim}
Tempo para R(t) = 0.5: 247.8 horas
\end{verbatim}

\section{Exercícios}\label{exercuxedcios}

📝 \textbf{Exercício 1}: Dados os tempos de falha
\texttt{c(100,\ 150,\ 200,\ 250,\ 300)} horas: - Calcule o MTTF - Estime
a taxa de falha - Determine a confiabilidade em 200 horas

📝 \textbf{Exercício 2}: Um sistema tem MTBF = 500h e MTTR = 10h: -
Calcule a disponibilidade - Quantas falhas são esperadas em 1 ano de
operação contínua?

📝 \textbf{Exercício 3}: Compare a confiabilidade de dois sistemas: -
Sistema A: taxa de falha constante λ = 0.001 falhas/hora - Sistema B:
taxa de falha crescente (Weibull com β = 2)

\section{Referências e Leitura
Adicional}\label{referuxeancias-e-leitura-adicional}

\begin{itemize}
\tightlist
\item
  Rausand, M., \& Høyland, A. (2004). \emph{System Reliability Theory:
  Models, Statistical Methods, and Applications}.
\item
  O'Connor, P., \& Kleyner, A. (2011). \emph{Practical Reliability
  Engineering}.
\item
  Meeker, W. Q., \& Escobar, L. A. (1998). \emph{Statistical Methods for
  Reliability Data}.
\end{itemize}

\section{Resumo do Capítulo}\label{resumo-do-capuxedtulo}

🔔 \textbf{Pontos-chave:}

\begin{itemize}
\tightlist
\item
  Confiabilidade é a base para análise de sistemas e equipamentos
\item
  Taxa de falha caracteriza o comportamento de falhas ao longo do tempo
\item
  MTBF e disponibilidade são métricas práticas de gestão
\item
  Tipos de manutenção devem ser escolhidos considerando custos e riscos
\item
  R fornece ferramentas poderosas para cálculo e visualização de
  métricas
\end{itemize}

\begin{center}\rule{0.5\linewidth}{0.5pt}\end{center}

\textbf{Próximo capítulo:} Distribuições de Probabilidade para Vida Útil

\chapter{Probabilidades e Distribuições para Vida
Útil}\label{probabilidades-e-distribuiuxe7uxf5es-para-vida-uxfatil}

\section{Introdução}\label{introduuxe7uxe3o-2}

Este capítulo apresenta as principais distribuições de probabilidade
utilizadas para modelar o tempo até a falha de componentes e sistemas.

\section{Distribuição Exponencial}\label{distribuiuxe7uxe3o-exponencial}

A distribuição exponencial é o modelo mais simples, caracterizada por
\textbf{taxa de falha constante}.

\subsection{Função Densidade de
Probabilidade}\label{funuxe7uxe3o-densidade-de-probabilidade}

\begin{equation}\phantomsection\label{eq-exp-pdf}{
f(t) = \lambda e^{-\lambda t}, \quad t \geq 0
}\end{equation}

\subsection{Função Confiabilidade}\label{funuxe7uxe3o-confiabilidade}

\begin{equation}\phantomsection\label{eq-exp-reliability}{
R(t) = e^{-\lambda t}
}\end{equation}

\subsection{Propriedades}\label{propriedades}

\begin{itemize}
\tightlist
\item
  Taxa de falha constante: \(\lambda(t) = \lambda\)
\item
  Sem memória: \(P(T > s+t | T > s) = P(T > t)\)
\item
  MTTF = \(1/\lambda\)
\end{itemize}

\subsection{Exemplo em R}\label{exemplo-em-r}

\begin{Shaded}
\begin{Highlighting}[]
\FunctionTok{library}\NormalTok{(tidyverse)}

\NormalTok{lambda }\OtherTok{\textless{}{-}} \FloatTok{0.01}  \CommentTok{\# taxa de falha}
\NormalTok{t }\OtherTok{\textless{}{-}} \FunctionTok{seq}\NormalTok{(}\DecValTok{0}\NormalTok{, }\DecValTok{500}\NormalTok{, }\AttributeTok{by =} \DecValTok{10}\NormalTok{)}

\CommentTok{\# Funções}
\NormalTok{pdf\_exp }\OtherTok{\textless{}{-}} \FunctionTok{dexp}\NormalTok{(t, }\AttributeTok{rate =}\NormalTok{ lambda)}
\NormalTok{cdf\_exp }\OtherTok{\textless{}{-}} \FunctionTok{pexp}\NormalTok{(t, }\AttributeTok{rate =}\NormalTok{ lambda)}
\NormalTok{reliability }\OtherTok{\textless{}{-}} \DecValTok{1} \SpecialCharTok{{-}}\NormalTok{ cdf\_exp}
\NormalTok{hazard }\OtherTok{\textless{}{-}} \FunctionTok{rep}\NormalTok{(lambda, }\FunctionTok{length}\NormalTok{(t))}

\CommentTok{\# Dataframe}
\NormalTok{df }\OtherTok{\textless{}{-}} \FunctionTok{data.frame}\NormalTok{(t, pdf\_exp, cdf\_exp, reliability, hazard)}

\CommentTok{\# Plot confiabilidade}
\FunctionTok{ggplot}\NormalTok{(df, }\FunctionTok{aes}\NormalTok{(}\AttributeTok{x =}\NormalTok{ t, }\AttributeTok{y =}\NormalTok{ reliability)) }\SpecialCharTok{+}
  \FunctionTok{geom\_line}\NormalTok{(}\AttributeTok{color =} \StringTok{"steelblue"}\NormalTok{, }\AttributeTok{linewidth =} \FloatTok{1.2}\NormalTok{) }\SpecialCharTok{+}
  \FunctionTok{labs}\NormalTok{(}\AttributeTok{title =} \StringTok{"Confiabilidade {-} Exponencial"}\NormalTok{,}
       \AttributeTok{x =} \StringTok{"Tempo"}\NormalTok{, }\AttributeTok{y =} \StringTok{"R(t)"}\NormalTok{) }\SpecialCharTok{+}
  \FunctionTok{theme\_minimal}\NormalTok{()}
\end{Highlighting}
\end{Shaded}

\pandocbounded{\includegraphics[keepaspectratio]{chapters/05-distribuicoes_files/figure-pdf/exp-example-1.pdf}}

\section{Distribuição Weibull}\label{distribuiuxe7uxe3o-weibull}

A distribuição Weibull é a \textbf{mais utilizada} em engenharia de
confiabilidade por sua flexibilidade.

\subsection{Função Densidade de
Probabilidade}\label{funuxe7uxe3o-densidade-de-probabilidade-1}

\begin{equation}\phantomsection\label{eq-weibull-pdf}{
f(t) = \frac{\beta}{\eta}\left(\frac{t}{\eta}\right)^{\beta-1} e^{-(t/\eta)^\beta}
}\end{equation}

\subsection{Parâmetros}\label{paruxe2metros}

\begin{itemize}
\tightlist
\item
  \textbf{β (beta)}: parâmetro de forma

  \begin{itemize}
  \tightlist
  \item
    β \textless{} 1: taxa de falha decrescente (mortalidade infantil)
  \item
    β = 1: taxa de falha constante (equivale à exponencial)
  \item
    β \textgreater{} 1: taxa de falha crescente (desgaste)
  \end{itemize}
\item
  \textbf{η (eta)}: parâmetro de escala (vida característica)
\end{itemize}

\subsection{Função Confiabilidade}\label{funuxe7uxe3o-confiabilidade-1}

\begin{equation}\phantomsection\label{eq-weibull-reliability}{
R(t) = e^{-(t/\eta)^\beta}
}\end{equation}

\subsection{Taxa de Falha}\label{taxa-de-falha}

\begin{equation}\phantomsection\label{eq-weibull-hazard}{
\lambda(t) = \frac{\beta}{\eta}\left(\frac{t}{\eta}\right)^{\beta-1}
}\end{equation}

\subsection{Exemplo em R}\label{exemplo-em-r-1}

\begin{Shaded}
\begin{Highlighting}[]
\CommentTok{\# Diferentes formas}
\NormalTok{beta\_values }\OtherTok{\textless{}{-}} \FunctionTok{c}\NormalTok{(}\FloatTok{0.5}\NormalTok{, }\DecValTok{1}\NormalTok{, }\FloatTok{1.5}\NormalTok{, }\FloatTok{2.5}\NormalTok{)}
\NormalTok{eta }\OtherTok{\textless{}{-}} \DecValTok{100}
\NormalTok{t }\OtherTok{\textless{}{-}} \FunctionTok{seq}\NormalTok{(}\DecValTok{0}\NormalTok{, }\DecValTok{200}\NormalTok{, }\AttributeTok{by =} \DecValTok{1}\NormalTok{)}

\NormalTok{df\_weibull }\OtherTok{\textless{}{-}} \FunctionTok{expand.grid}\NormalTok{(}\AttributeTok{t =}\NormalTok{ t, }\AttributeTok{beta =}\NormalTok{ beta\_values) }\SpecialCharTok{\%\textgreater{}\%}
  \FunctionTok{mutate}\NormalTok{(}
    \AttributeTok{reliability =} \FunctionTok{exp}\NormalTok{(}\SpecialCharTok{{-}}\NormalTok{(t}\SpecialCharTok{/}\NormalTok{eta)}\SpecialCharTok{\^{}}\NormalTok{beta),}
    \AttributeTok{hazard =}\NormalTok{ (beta}\SpecialCharTok{/}\NormalTok{eta) }\SpecialCharTok{*}\NormalTok{ (t}\SpecialCharTok{/}\NormalTok{eta)}\SpecialCharTok{\^{}}\NormalTok{(beta}\DecValTok{{-}1}\NormalTok{),}
    \AttributeTok{beta\_label =} \FunctionTok{paste}\NormalTok{(}\StringTok{"β ="}\NormalTok{, beta)}
\NormalTok{  )}

\CommentTok{\# Plot taxa de falha}
\FunctionTok{ggplot}\NormalTok{(df\_weibull, }\FunctionTok{aes}\NormalTok{(}\AttributeTok{x =}\NormalTok{ t, }\AttributeTok{y =}\NormalTok{ hazard, }\AttributeTok{color =}\NormalTok{ beta\_label)) }\SpecialCharTok{+}
  \FunctionTok{geom\_line}\NormalTok{(}\AttributeTok{linewidth =} \DecValTok{1}\NormalTok{) }\SpecialCharTok{+}
  \FunctionTok{labs}\NormalTok{(}\AttributeTok{title =} \StringTok{"Taxa de Falha {-} Weibull"}\NormalTok{,}
       \AttributeTok{x =} \StringTok{"Tempo"}\NormalTok{, }\AttributeTok{y =} \StringTok{"λ(t)"}\NormalTok{,}
       \AttributeTok{color =} \StringTok{"Parâmetro"}\NormalTok{) }\SpecialCharTok{+}
  \FunctionTok{theme\_minimal}\NormalTok{()}
\end{Highlighting}
\end{Shaded}

\pandocbounded{\includegraphics[keepaspectratio]{chapters/05-distribuicoes_files/figure-pdf/weibull-example-1.pdf}}

\section{Distribuição Lognormal}\label{distribuiuxe7uxe3o-lognormal}

A distribuição lognormal é adequada para processos onde o
\textbf{logaritmo do tempo} até a falha segue distribuição normal.

\subsection{Função Densidade}\label{funuxe7uxe3o-densidade}

\begin{equation}\phantomsection\label{eq-lognormal-pdf}{
f(t) = \frac{1}{t\sigma\sqrt{2\pi}} e^{-\frac{(\ln t - \mu)^2}{2\sigma^2}}
}\end{equation}

\subsection{Parâmetros}\label{paruxe2metros-1}

\begin{itemize}
\tightlist
\item
  \textbf{μ (mu)}: média do logaritmo
\item
  \textbf{σ (sigma)}: desvio padrão do logaritmo
\end{itemize}

\subsection{Quando Usar}\label{quando-usar}

\begin{itemize}
\tightlist
\item
  Falhas causadas por \textbf{fadiga}
\item
  Processos de \textbf{degradação} multiplicativos
\item
  Tempos de reparo
\end{itemize}

\subsection{Exemplo em R}\label{exemplo-em-r-2}

\begin{Shaded}
\begin{Highlighting}[]
\NormalTok{mu }\OtherTok{\textless{}{-}} \FunctionTok{log}\NormalTok{(}\DecValTok{100}\NormalTok{)}
\NormalTok{sigma }\OtherTok{\textless{}{-}} \FloatTok{0.5}
\NormalTok{t }\OtherTok{\textless{}{-}} \FunctionTok{seq}\NormalTok{(}\DecValTok{0}\NormalTok{, }\DecValTok{300}\NormalTok{, }\AttributeTok{by =} \DecValTok{1}\NormalTok{)}

\NormalTok{df\_lognormal }\OtherTok{\textless{}{-}} \FunctionTok{data.frame}\NormalTok{(}
  \AttributeTok{t =}\NormalTok{ t,}
  \AttributeTok{pdf =} \FunctionTok{dlnorm}\NormalTok{(t, }\AttributeTok{meanlog =}\NormalTok{ mu, }\AttributeTok{sdlog =}\NormalTok{ sigma),}
  \AttributeTok{reliability =} \DecValTok{1} \SpecialCharTok{{-}} \FunctionTok{plnorm}\NormalTok{(t, }\AttributeTok{meanlog =}\NormalTok{ mu, }\AttributeTok{sdlog =}\NormalTok{ sigma)}
\NormalTok{)}

\FunctionTok{ggplot}\NormalTok{(df\_lognormal, }\FunctionTok{aes}\NormalTok{(}\AttributeTok{x =}\NormalTok{ t, }\AttributeTok{y =}\NormalTok{ reliability)) }\SpecialCharTok{+}
  \FunctionTok{geom\_line}\NormalTok{(}\AttributeTok{color =} \StringTok{"darkgreen"}\NormalTok{, }\AttributeTok{linewidth =} \FloatTok{1.2}\NormalTok{) }\SpecialCharTok{+}
  \FunctionTok{labs}\NormalTok{(}\AttributeTok{title =} \StringTok{"Confiabilidade {-} Lognormal"}\NormalTok{,}
       \AttributeTok{x =} \StringTok{"Tempo"}\NormalTok{, }\AttributeTok{y =} \StringTok{"R(t)"}\NormalTok{) }\SpecialCharTok{+}
  \FunctionTok{theme\_minimal}\NormalTok{()}
\end{Highlighting}
\end{Shaded}

\pandocbounded{\includegraphics[keepaspectratio]{chapters/05-distribuicoes_files/figure-pdf/lognormal-example-1.pdf}}

\section{Comparação de
Distribuições}\label{comparauxe7uxe3o-de-distribuiuxe7uxf5es}

\begin{Shaded}
\begin{Highlighting}[]
\NormalTok{t }\OtherTok{\textless{}{-}} \FunctionTok{seq}\NormalTok{(}\DecValTok{0}\NormalTok{, }\DecValTok{200}\NormalTok{, }\AttributeTok{by =} \DecValTok{1}\NormalTok{)}

\NormalTok{df\_comp }\OtherTok{\textless{}{-}} \FunctionTok{data.frame}\NormalTok{(}
  \AttributeTok{t =}\NormalTok{ t,}
  \AttributeTok{Exponencial =} \FunctionTok{exp}\NormalTok{(}\SpecialCharTok{{-}}\FloatTok{0.01} \SpecialCharTok{*}\NormalTok{ t),}
  \AttributeTok{Weibull =} \FunctionTok{exp}\NormalTok{(}\SpecialCharTok{{-}}\NormalTok{(t}\SpecialCharTok{/}\DecValTok{100}\NormalTok{)}\SpecialCharTok{\^{}}\DecValTok{2}\NormalTok{),}
  \AttributeTok{Lognormal =} \DecValTok{1} \SpecialCharTok{{-}} \FunctionTok{plnorm}\NormalTok{(t, }\AttributeTok{meanlog =} \FunctionTok{log}\NormalTok{(}\DecValTok{100}\NormalTok{), }\AttributeTok{sdlog =} \FloatTok{0.5}\NormalTok{)}
\NormalTok{) }\SpecialCharTok{\%\textgreater{}\%}
  \FunctionTok{pivot\_longer}\NormalTok{(}\AttributeTok{cols =} \SpecialCharTok{{-}}\NormalTok{t, }\AttributeTok{names\_to =} \StringTok{"Distribuição"}\NormalTok{, }\AttributeTok{values\_to =} \StringTok{"R(t)"}\NormalTok{)}

\FunctionTok{ggplot}\NormalTok{(df\_comp, }\FunctionTok{aes}\NormalTok{(}\AttributeTok{x =}\NormalTok{ t, }\AttributeTok{y =} \StringTok{\textasciigrave{}}\AttributeTok{R(t)}\StringTok{\textasciigrave{}}\NormalTok{, }\AttributeTok{color =}\NormalTok{ Distribuição)) }\SpecialCharTok{+}
  \FunctionTok{geom\_line}\NormalTok{(}\AttributeTok{linewidth =} \FloatTok{1.2}\NormalTok{) }\SpecialCharTok{+}
  \FunctionTok{labs}\NormalTok{(}\AttributeTok{title =} \StringTok{"Comparação de Distribuições"}\NormalTok{,}
       \AttributeTok{x =} \StringTok{"Tempo"}\NormalTok{, }\AttributeTok{y =} \StringTok{"Confiabilidade"}\NormalTok{) }\SpecialCharTok{+}
  \FunctionTok{theme\_minimal}\NormalTok{() }\SpecialCharTok{+}
  \FunctionTok{theme}\NormalTok{(}\AttributeTok{legend.position =} \StringTok{"top"}\NormalTok{)}
\end{Highlighting}
\end{Shaded}

\pandocbounded{\includegraphics[keepaspectratio]{chapters/05-distribuicoes_files/figure-pdf/comparison-1.pdf}}

\section{Escolha do Modelo}\label{escolha-do-modelo}

\begin{longtable}[]{@{}
  >{\raggedright\arraybackslash}p{(\linewidth - 4\tabcolsep) * \real{0.3333}}
  >{\raggedright\arraybackslash}p{(\linewidth - 4\tabcolsep) * \real{0.3095}}
  >{\raggedright\arraybackslash}p{(\linewidth - 4\tabcolsep) * \real{0.3571}}@{}}
\toprule\noalign{}
\begin{minipage}[b]{\linewidth}\raggedright
Distribuição
\end{minipage} & \begin{minipage}[b]{\linewidth}\raggedright
Quando Usar
\end{minipage} & \begin{minipage}[b]{\linewidth}\raggedright
Taxa de Falha
\end{minipage} \\
\midrule\noalign{}
\endhead
\bottomrule\noalign{}
\endlastfoot
\textbf{Exponencial} & Falhas aleatórias, componentes eletrônicos &
Constante \\
\textbf{Weibull} & Maioria dos casos industriais, flexível & Variável \\
\textbf{Lognormal} & Fadiga, degradação multiplicativa & Não-monótona \\
\end{longtable}

\section{Exercícios}\label{exercuxedcios-1}

📝 \textbf{Exercício 1}: Ajuste uma distribuição Weibull a dados de
falha e estime β e η.

📝 \textbf{Exercício 2}: Compare o ajuste de exponencial vs.~Weibull
usando teste de qualidade de ajuste.

📝 \textbf{Exercício 3}: Simule 1000 tempos de falha de uma distribuição
Weibull(β=2, η=100) e verifique se a média empírica se aproxima da
teórica.

\section{Resumo}\label{resumo-2}

🔔 \textbf{Pontos-chave:}

\begin{itemize}
\tightlist
\item
  Exponencial: modelo simples, taxa constante
\item
  Weibull: mais flexível, amplamente usado
\item
  Lognormal: adequado para fadiga e degradação
\item
  Escolha do modelo depende do mecanismo de falha
\end{itemize}

\begin{center}\rule{0.5\linewidth}{0.5pt}\end{center}

\textbf{Próximo capítulo:} Estimativa de Parâmetros e Inferência

\chapter{Estimativa de Parâmetros e
Inferência}\label{estimativa-de-paruxe2metros-e-inferuxeancia}

\section{Introdução}\label{introduuxe7uxe3o-3}

Este capítulo aborda métodos para estimar parâmetros de distribuições de
vida útil a partir de dados observados.

\section{Método da Máxima Verossimilhança
(MLE)}\label{muxe9todo-da-muxe1xima-verossimilhanuxe7a-mle}

\subsection{Conceito}\label{conceito-1}

O método MLE busca os parâmetros que \textbf{maximizam a probabilidade}
de observar os dados coletados.

\subsection{Função de
Verossimilhança}\label{funuxe7uxe3o-de-verossimilhanuxe7a}

Para dados de falha \(t_1, t_2, \ldots, t_n\):

\[
L(\theta) = \prod_{i=1}^n f(t_i; \theta)
\]

Log-verossimilhança:

\[
\ell(\theta) = \sum_{i=1}^n \log f(t_i; \theta)
\]

\subsection{Exemplo: Exponencial}\label{exemplo-exponencial}

\begin{Shaded}
\begin{Highlighting}[]
\FunctionTok{library}\NormalTok{(tidyverse)}

\CommentTok{\# Dados simulados}
\FunctionTok{set.seed}\NormalTok{(}\DecValTok{123}\NormalTok{)}
\NormalTok{dados }\OtherTok{\textless{}{-}} \FunctionTok{rexp}\NormalTok{(}\DecValTok{50}\NormalTok{, }\AttributeTok{rate =} \FloatTok{0.01}\NormalTok{)}

\CommentTok{\# MLE analítico para exponencial}
\NormalTok{lambda\_mle }\OtherTok{\textless{}{-}} \DecValTok{1} \SpecialCharTok{/} \FunctionTok{mean}\NormalTok{(dados)}

\FunctionTok{cat}\NormalTok{(}\StringTok{"Lambda MLE:"}\NormalTok{, lambda\_mle, }\StringTok{"}\SpecialCharTok{\textbackslash{}n}\StringTok{"}\NormalTok{)}
\end{Highlighting}
\end{Shaded}

\begin{verbatim}
Lambda MLE: 0.008846654 
\end{verbatim}

\begin{Shaded}
\begin{Highlighting}[]
\FunctionTok{cat}\NormalTok{(}\StringTok{"MTTF estimado:"}\NormalTok{, }\DecValTok{1}\SpecialCharTok{/}\NormalTok{lambda\_mle, }\StringTok{"}\SpecialCharTok{\textbackslash{}n}\StringTok{"}\NormalTok{)}
\end{Highlighting}
\end{Shaded}

\begin{verbatim}
MTTF estimado: 113.0371 
\end{verbatim}

\subsection{\texorpdfstring{Exemplo: Weibull com
\texttt{fitdistrplus}}{Exemplo: Weibull com fitdistrplus}}\label{exemplo-weibull-com-fitdistrplus}

\begin{Shaded}
\begin{Highlighting}[]
\FunctionTok{library}\NormalTok{(fitdistrplus)}

\CommentTok{\# Ajuste Weibull}
\NormalTok{fit\_weibull }\OtherTok{\textless{}{-}} \FunctionTok{fitdist}\NormalTok{(dados, }\StringTok{"weibull"}\NormalTok{)}

\FunctionTok{summary}\NormalTok{(fit\_weibull)}
\end{Highlighting}
\end{Shaded}

\begin{verbatim}
Fitting of the distribution ' weibull ' by maximum likelihood 
Parameters : 
        estimate Std. Error
shape   1.050058  0.1112928
scale 115.332088 16.3678738
Loglikelihood:  -286.2823   AIC:  576.5646   BIC:  580.3886 
Correlation matrix:
          shape     scale
shape 1.0000000 0.3158044
scale 0.3158044 1.0000000
\end{verbatim}

\begin{Shaded}
\begin{Highlighting}[]
\CommentTok{\# Visualizar ajuste}
\FunctionTok{plot}\NormalTok{(fit\_weibull)}
\end{Highlighting}
\end{Shaded}

\pandocbounded{\includegraphics[keepaspectratio]{chapters/06-estimacao_files/figure-pdf/mle-weibull-1.pdf}}

\section{Intervalos de Confiança}\label{intervalos-de-confianuxe7a}

\subsection{Bootstrap}\label{bootstrap}

\begin{Shaded}
\begin{Highlighting}[]
\CommentTok{\# Bootstrap para IC}
\NormalTok{boot\_samples }\OtherTok{\textless{}{-}} \FunctionTok{replicate}\NormalTok{(}\DecValTok{1000}\NormalTok{, \{}
\NormalTok{  sample\_boot }\OtherTok{\textless{}{-}} \FunctionTok{sample}\NormalTok{(dados, }\AttributeTok{replace =} \ConstantTok{TRUE}\NormalTok{)}
  \DecValTok{1} \SpecialCharTok{/} \FunctionTok{mean}\NormalTok{(sample\_boot)}
\NormalTok{\})}

\CommentTok{\# IC 95\%}
\NormalTok{ic\_lower }\OtherTok{\textless{}{-}} \FunctionTok{quantile}\NormalTok{(boot\_samples, }\FloatTok{0.025}\NormalTok{)}
\NormalTok{ic\_upper }\OtherTok{\textless{}{-}} \FunctionTok{quantile}\NormalTok{(boot\_samples, }\FloatTok{0.975}\NormalTok{)}

\FunctionTok{cat}\NormalTok{(}\StringTok{"IC 95\% para lambda:"}\NormalTok{, ic\_lower, }\StringTok{"{-}"}\NormalTok{, ic\_upper, }\StringTok{"}\SpecialCharTok{\textbackslash{}n}\StringTok{"}\NormalTok{)}
\end{Highlighting}
\end{Shaded}

\begin{verbatim}
IC 95% para lambda: 0.006737779 - 0.01235932 
\end{verbatim}

\section{Testes de Qualidade de
Ajuste}\label{testes-de-qualidade-de-ajuste}

\subsection{Teste Kolmogorov-Smirnov}\label{teste-kolmogorov-smirnov}

\begin{Shaded}
\begin{Highlighting}[]
\CommentTok{\# Testar ajuste exponencial}
\NormalTok{ks\_result }\OtherTok{\textless{}{-}} \FunctionTok{ks.test}\NormalTok{(dados, }\StringTok{"pexp"}\NormalTok{, }\AttributeTok{rate =}\NormalTok{ lambda\_mle)}
\FunctionTok{print}\NormalTok{(ks\_result)}
\end{Highlighting}
\end{Shaded}

\begin{verbatim}

    Exact one-sample Kolmogorov-Smirnov test

data:  dados
D = 0.10156, p-value = 0.6436
alternative hypothesis: two-sided
\end{verbatim}

\section{Exercícios}\label{exercuxedcios-2}

📝 \textbf{Exercício 1}: Estime parâmetros de distribuição Weibull
usando MLE.

📝 \textbf{Exercício 2}: Compare IC paramétrico vs.~bootstrap.

📝 \textbf{Exercício 3}: Use AIC/BIC para comparar modelos.

\section{Resumo}\label{resumo-3}

🔔 \textbf{Pontos-chave:}

\begin{itemize}
\tightlist
\item
  MLE é o método padrão para estimação
\item
  Intervalos de confiança quantificam incerteza
\item
  Testes de qualidade validam o ajuste
\end{itemize}

\begin{center}\rule{0.5\linewidth}{0.5pt}\end{center}

\textbf{Próximo capítulo:} Análise de Dados de Falha com R

\part{Análise e Modelagem Prática}

\chapter{Análise de Dados de Falha com
R}\label{anuxe1lise-de-dados-de-falha-com-r}

\section{Introdução}\label{introduuxe7uxe3o-4}

Análise prática de dados de falha incluindo censura, tabelas de vida e
visualizações.

\section{Tipos de Censura}\label{tipos-de-censura}

\subsection{Censura à Direita}\label{censura-uxe0-direita}

Observação interrompida antes da falha (mais comum).

\subsection{Censura à Esquerda}\label{censura-uxe0-esquerda}

Falha ocorreu antes do início da observação.

\subsection{Censura Intervalar}\label{censura-intervalar}

Falha ocorreu em um intervalo de tempo.

\section{Preparação de Dados}\label{preparauxe7uxe3o-de-dados}

\begin{Shaded}
\begin{Highlighting}[]
\FunctionTok{library}\NormalTok{(survival)}
\FunctionTok{library}\NormalTok{(tidyverse)}

\CommentTok{\# Dados de exemplo}
\NormalTok{dados\_falha }\OtherTok{\textless{}{-}} \FunctionTok{data.frame}\NormalTok{(}
  \AttributeTok{tempo =} \FunctionTok{c}\NormalTok{(}\DecValTok{120}\NormalTok{, }\DecValTok{145}\NormalTok{, }\DecValTok{180}\NormalTok{, }\DecValTok{200}\NormalTok{, }\DecValTok{250}\NormalTok{, }\DecValTok{300}\NormalTok{, }\DecValTok{350}\NormalTok{, }\DecValTok{400}\NormalTok{, }\DecValTok{450}\NormalTok{, }\DecValTok{500}\NormalTok{),}
  \AttributeTok{status =} \FunctionTok{c}\NormalTok{(}\DecValTok{1}\NormalTok{, }\DecValTok{1}\NormalTok{, }\DecValTok{1}\NormalTok{, }\DecValTok{0}\NormalTok{, }\DecValTok{1}\NormalTok{, }\DecValTok{0}\NormalTok{, }\DecValTok{1}\NormalTok{, }\DecValTok{1}\NormalTok{, }\DecValTok{0}\NormalTok{, }\DecValTok{1}\NormalTok{),  }\CommentTok{\# 1=falha, 0=censura}
  \AttributeTok{equipamento =} \FunctionTok{paste0}\NormalTok{(}\StringTok{"Equip\_"}\NormalTok{, }\DecValTok{1}\SpecialCharTok{:}\DecValTok{10}\NormalTok{)}
\NormalTok{)}

\CommentTok{\# Criar objeto Surv}
\NormalTok{surv\_obj }\OtherTok{\textless{}{-}} \FunctionTok{Surv}\NormalTok{(}\AttributeTok{time =}\NormalTok{ dados\_falha}\SpecialCharTok{$}\NormalTok{tempo, }\AttributeTok{event =}\NormalTok{ dados\_falha}\SpecialCharTok{$}\NormalTok{status)}

\FunctionTok{print}\NormalTok{(surv\_obj)}
\end{Highlighting}
\end{Shaded}

\begin{verbatim}
 [1] 120  145  180  200+ 250  300+ 350  400  450+ 500 
\end{verbatim}

\section{Tabela de Vida}\label{tabela-de-vida}

\begin{Shaded}
\begin{Highlighting}[]
\FunctionTok{library}\NormalTok{(kableExtra)}

\CommentTok{\# Criar tabela de vida}
\NormalTok{life\_table }\OtherTok{\textless{}{-}}\NormalTok{ dados\_falha }\SpecialCharTok{\%\textgreater{}\%}
  \FunctionTok{arrange}\NormalTok{(tempo) }\SpecialCharTok{\%\textgreater{}\%}
  \FunctionTok{mutate}\NormalTok{(}
    \AttributeTok{n\_risk =} \FunctionTok{n}\NormalTok{()}\SpecialCharTok{:}\DecValTok{1}\NormalTok{,}
    \AttributeTok{n\_event =}\NormalTok{ status,}
    \AttributeTok{survival =} \FunctionTok{cumprod}\NormalTok{(}\DecValTok{1} \SpecialCharTok{{-}}\NormalTok{ n\_event}\SpecialCharTok{/}\NormalTok{n\_risk)}
\NormalTok{  )}

\FunctionTok{kable}\NormalTok{(life\_table, }\AttributeTok{digits =} \DecValTok{3}\NormalTok{) }\SpecialCharTok{\%\textgreater{}\%}
  \FunctionTok{kable\_styling}\NormalTok{(}\AttributeTok{bootstrap\_options =} \FunctionTok{c}\NormalTok{(}\StringTok{"striped"}\NormalTok{, }\StringTok{"hover"}\NormalTok{))}
\end{Highlighting}
\end{Shaded}

\begin{longtable*}[t]{rrlrrr}
\toprule
tempo & status & equipamento & n\_risk & n\_event & survival\\
\midrule
120 & 1 & Equip\_1 & 10 & 1 & 0.900\\
145 & 1 & Equip\_2 & 9 & 1 & 0.800\\
180 & 1 & Equip\_3 & 8 & 1 & 0.700\\
200 & 0 & Equip\_4 & 7 & 0 & 0.700\\
250 & 1 & Equip\_5 & 6 & 1 & 0.583\\
\addlinespace
300 & 0 & Equip\_6 & 5 & 0 & 0.583\\
350 & 1 & Equip\_7 & 4 & 1 & 0.438\\
400 & 1 & Equip\_8 & 3 & 1 & 0.292\\
450 & 0 & Equip\_9 & 2 & 0 & 0.292\\
500 & 1 & Equip\_10 & 1 & 1 & 0.000\\
\bottomrule
\end{longtable*}

\section{Curva de Kaplan-Meier}\label{curva-de-kaplan-meier}

\begin{Shaded}
\begin{Highlighting}[]
\FunctionTok{library}\NormalTok{(survminer)}

\CommentTok{\# Ajuste Kaplan{-}Meier}
\NormalTok{km\_fit }\OtherTok{\textless{}{-}} \FunctionTok{survfit}\NormalTok{(surv\_obj }\SpecialCharTok{\textasciitilde{}} \DecValTok{1}\NormalTok{)}

\CommentTok{\# Visualizar}
\FunctionTok{ggsurvplot}\NormalTok{(km\_fit,}
           \AttributeTok{data =}\NormalTok{ dados\_falha,}
           \AttributeTok{conf.int =} \ConstantTok{TRUE}\NormalTok{,}
           \AttributeTok{risk.table =} \ConstantTok{TRUE}\NormalTok{,}
           \AttributeTok{title =} \StringTok{"Curva de Sobrevivência {-} Kaplan{-}Meier"}\NormalTok{)}
\end{Highlighting}
\end{Shaded}

\pandocbounded{\includegraphics[keepaspectratio]{chapters/07-analise-falhas_files/figure-pdf/km-curve-1.pdf}}

\section{Exercícios}\label{exercuxedcios-3}

📝 \textbf{Exercício 1}: Analise dados com diferentes tipos de censura.

📝 \textbf{Exercício 2}: Compare curvas KM de grupos diferentes.

\section{Resumo}\label{resumo-4}

🔔 \textbf{Pontos-chave:}

\begin{itemize}
\tightlist
\item
  Censura é comum em dados de confiabilidade
\item
  Pacote \texttt{survival} é essencial para análise
\item
  Kaplan-Meier é método não-paramétrico robusto
\end{itemize}

\begin{center}\rule{0.5\linewidth}{0.5pt}\end{center}

\textbf{Próximo capítulo:} Modelagem Weibull na Prática

\chapter{Modelagem Weibull na
Prática}\label{modelagem-weibull-na-pruxe1tica}

\section{Introdução}\label{introduuxe7uxe3o-5}

Aplicação prática da distribuição Weibull em análise de confiabilidade
industrial.

\section{Ajuste de Modelo Weibull}\label{ajuste-de-modelo-weibull}

\begin{Shaded}
\begin{Highlighting}[]
\FunctionTok{library}\NormalTok{(WeibullR)}
\FunctionTok{library}\NormalTok{(survival)}
\FunctionTok{library}\NormalTok{(dplyr)}

\CommentTok{\# Dados de tempo até falha}
\NormalTok{tempos }\OtherTok{\textless{}{-}} \FunctionTok{c}\NormalTok{(}\DecValTok{120}\NormalTok{, }\DecValTok{145}\NormalTok{, }\DecValTok{180}\NormalTok{, }\DecValTok{200}\NormalTok{, }\DecValTok{250}\NormalTok{, }\DecValTok{310}\NormalTok{, }\DecValTok{420}\NormalTok{, }\DecValTok{500}\NormalTok{, }\DecValTok{650}\NormalTok{, }\DecValTok{800}\NormalTok{)}

\CommentTok{\# Ajustar Weibull}
\NormalTok{fit }\OtherTok{\textless{}{-}} \FunctionTok{wblr}\NormalTok{(tempos, }\AttributeTok{col=}\StringTok{"steelblue"}\NormalTok{)}
\NormalTok{fit }\OtherTok{\textless{}{-}} \FunctionTok{wblr.conf}\NormalTok{(fit)}

\CommentTok{\# Visualizar papel de Weibull}
\FunctionTok{plot}\NormalTok{(fit, }\AttributeTok{main=}\StringTok{"Análise Weibull {-} Bombas Industriais"}\NormalTok{)}
\end{Highlighting}
\end{Shaded}

\pandocbounded{\includegraphics[keepaspectratio]{chapters/08-weibull_files/figure-pdf/weibull-fit-1.pdf}}

\section{Interpretação de
Parâmetros}\label{interpretauxe7uxe3o-de-paruxe2metros}

\subsection{Parâmetro β (forma)}\label{paruxe2metro-ux3b2-forma}

\begin{itemize}
\tightlist
\item
  \textbf{β \textless{} 1}: Mortalidade infantil (defeitos iniciais)
\item
  \textbf{β ≈ 1}: Falhas aleatórias
\item
  \textbf{β \textgreater{} 1}: Desgaste (envelhecimento)
\end{itemize}

\subsection{Parâmetro η (escala)}\label{paruxe2metro-ux3b7-escala}

Vida característica: tempo no qual 63.2\% dos itens falharam.

\section{Diagnóstico do Modelo}\label{diagnuxf3stico-do-modelo}

\begin{Shaded}
\begin{Highlighting}[]
\FunctionTok{library}\NormalTok{(fitdistrplus)}

\NormalTok{fit\_distr }\OtherTok{\textless{}{-}} \FunctionTok{fitdist}\NormalTok{(tempos, }\StringTok{"weibull"}\NormalTok{)}

\CommentTok{\# Gráficos de diagnóstico}
\FunctionTok{plot}\NormalTok{(fit\_distr)}
\end{Highlighting}
\end{Shaded}

\pandocbounded{\includegraphics[keepaspectratio]{chapters/08-weibull_files/figure-pdf/diagnostic-1.pdf}}

\begin{Shaded}
\begin{Highlighting}[]
\CommentTok{\# Teste de qualidade}
\FunctionTok{gofstat}\NormalTok{(fit\_distr)}
\end{Highlighting}
\end{Shaded}

\begin{verbatim}
Goodness-of-fit statistics
                             1-mle-weibull
Kolmogorov-Smirnov statistic    0.15041211
Cramer-von Mises statistic      0.04302523
Anderson-Darling statistic      0.27952785

Goodness-of-fit criteria
                               1-mle-weibull
Akaike's Information Criterion      137.4790
Bayesian Information Criterion      138.0841
\end{verbatim}

\section{Estimativa de
Confiabilidade}\label{estimativa-de-confiabilidade}

\begin{Shaded}
\begin{Highlighting}[]
\NormalTok{beta }\OtherTok{\textless{}{-}}\NormalTok{ fit\_distr}\SpecialCharTok{$}\NormalTok{estimate[}\StringTok{"shape"}\NormalTok{]}
\NormalTok{eta }\OtherTok{\textless{}{-}}\NormalTok{ fit\_distr}\SpecialCharTok{$}\NormalTok{estimate[}\StringTok{"scale"}\NormalTok{]}

\CommentTok{\# Confiabilidade em diferentes tempos}
\NormalTok{tempos\_pred }\OtherTok{\textless{}{-}} \FunctionTok{c}\NormalTok{(}\DecValTok{100}\NormalTok{, }\DecValTok{200}\NormalTok{, }\DecValTok{300}\NormalTok{, }\DecValTok{500}\NormalTok{, }\DecValTok{1000}\NormalTok{)}
\NormalTok{R\_pred }\OtherTok{\textless{}{-}} \FunctionTok{exp}\NormalTok{(}\SpecialCharTok{{-}}\NormalTok{(tempos\_pred}\SpecialCharTok{/}\NormalTok{eta)}\SpecialCharTok{\^{}}\NormalTok{beta)}

\FunctionTok{data.frame}\NormalTok{(}
  \AttributeTok{Tempo =}\NormalTok{ tempos\_pred,}
  \AttributeTok{Confiabilidade =} \FunctionTok{round}\NormalTok{(R\_pred, }\DecValTok{4}\NormalTok{)}
\NormalTok{) }\SpecialCharTok{\%\textgreater{}\%}
\NormalTok{  knitr}\SpecialCharTok{::}\FunctionTok{kable}\NormalTok{()}
\end{Highlighting}
\end{Shaded}

\begin{longtable}[]{@{}rr@{}}
\toprule\noalign{}
Tempo & Confiabilidade \\
\midrule\noalign{}
\endhead
\bottomrule\noalign{}
\endlastfoot
100 & 0.9174 \\
200 & 0.7476 \\
300 & 0.5531 \\
500 & 0.2344 \\
1000 & 0.0075 \\
\end{longtable}

\section{Caso Prático: Bombas
Centrífugas}\label{caso-pruxe1tico-bombas-centruxedfugas}

\begin{Shaded}
\begin{Highlighting}[]
\CommentTok{\# Dados reais de bombas (exemplo)}
\NormalTok{bombas }\OtherTok{\textless{}{-}} \FunctionTok{read.csv}\NormalTok{(}\StringTok{"resources/data/falhas\_bombas.csv"}\NormalTok{)}

\CommentTok{\# Análise estratificada por tipo}
\NormalTok{fit\_bomba\_A }\OtherTok{\textless{}{-}} \FunctionTok{fitdist}\NormalTok{(bombas}\SpecialCharTok{$}\NormalTok{tempo[bombas}\SpecialCharTok{$}\NormalTok{tipo }\SpecialCharTok{==} \StringTok{"A"}\NormalTok{], }\StringTok{"weibull"}\NormalTok{)}
\NormalTok{fit\_bomba\_B }\OtherTok{\textless{}{-}} \FunctionTok{fitdist}\NormalTok{(bombas}\SpecialCharTok{$}\NormalTok{tempo[bombas}\SpecialCharTok{$}\NormalTok{tipo }\SpecialCharTok{==} \StringTok{"B"}\NormalTok{], }\StringTok{"weibull"}\NormalTok{)}

\CommentTok{\# Comparar parâmetros}
\FunctionTok{rbind}\NormalTok{(}
  \AttributeTok{A =}\NormalTok{ fit\_bomba\_A}\SpecialCharTok{$}\NormalTok{estimate,}
  \AttributeTok{B =}\NormalTok{ fit\_bomba\_B}\SpecialCharTok{$}\NormalTok{estimate}
\NormalTok{)}
\end{Highlighting}
\end{Shaded}

\section{Exercícios}\label{exercuxedcios-4}

📝 \textbf{Exercício 1}: Ajuste modelo Weibull e interprete β.

📝 \textbf{Exercício 2}: Calcule o tempo para R(t) = 0.90.

📝 \textbf{Exercício 3}: Compare Weibull vs.~Lognormal para seus dados.

\section{Resumo}\label{resumo-5}

🔔 \textbf{Pontos-chave:}

\begin{itemize}
\tightlist
\item
  Weibull é versátil e amplamente aplicável
\item
  Parâmetro β indica modo de falha
\item
  Papel de Weibull facilita visualização
\end{itemize}

\begin{center}\rule{0.5\linewidth}{0.5pt}\end{center}

\textbf{Próximo capítulo:} Análise de Sobrevivência

\chapter{Análise de Sobrevivência e Modelos
Semiparamétricos}\label{anuxe1lise-de-sobrevivuxeancia-e-modelos-semiparamuxe9tricos}

\section{Introdução}\label{introduuxe7uxe3o-6}

Análise de sobrevivência com modelos de regressão, incluindo
Kaplan-Meier e Cox.

\section{Modelo de Cox (Regressão de Riscos
Proporcionais)}\label{modelo-de-cox-regressuxe3o-de-riscos-proporcionais}

\subsection{Formulação}\label{formulauxe7uxe3o}

\[
\lambda(t|X) = \lambda_0(t) \exp(\beta_1 X_1 + \beta_2 X_2 + \ldots + \beta_p X_p)
\]

onde \(\lambda_0(t)\) é a taxa de risco basal.

\section{Exemplo com Covariáveis}\label{exemplo-com-covariuxe1veis}

\begin{Shaded}
\begin{Highlighting}[]
\FunctionTok{library}\NormalTok{(survival)}
\FunctionTok{library}\NormalTok{(survminer)}

\CommentTok{\# Dados de exemplo com covariáveis}
\NormalTok{dados }\OtherTok{\textless{}{-}} \FunctionTok{data.frame}\NormalTok{(}
  \AttributeTok{tempo =} \FunctionTok{c}\NormalTok{(}\DecValTok{120}\NormalTok{, }\DecValTok{145}\NormalTok{, }\DecValTok{180}\NormalTok{, }\DecValTok{200}\NormalTok{, }\DecValTok{250}\NormalTok{, }\DecValTok{300}\NormalTok{, }\DecValTok{350}\NormalTok{, }\DecValTok{400}\NormalTok{, }\DecValTok{450}\NormalTok{, }\DecValTok{500}\NormalTok{,}
            \DecValTok{100}\NormalTok{, }\DecValTok{130}\NormalTok{, }\DecValTok{160}\NormalTok{, }\DecValTok{190}\NormalTok{, }\DecValTok{220}\NormalTok{, }\DecValTok{280}\NormalTok{, }\DecValTok{320}\NormalTok{, }\DecValTok{380}\NormalTok{, }\DecValTok{420}\NormalTok{, }\DecValTok{480}\NormalTok{),}
  \AttributeTok{status =} \FunctionTok{c}\NormalTok{(}\DecValTok{1}\NormalTok{, }\DecValTok{1}\NormalTok{, }\DecValTok{1}\NormalTok{, }\DecValTok{0}\NormalTok{, }\DecValTok{1}\NormalTok{, }\DecValTok{0}\NormalTok{, }\DecValTok{1}\NormalTok{, }\DecValTok{1}\NormalTok{, }\DecValTok{0}\NormalTok{, }\DecValTok{1}\NormalTok{,}
             \DecValTok{1}\NormalTok{, }\DecValTok{1}\NormalTok{, }\DecValTok{1}\NormalTok{, }\DecValTok{1}\NormalTok{, }\DecValTok{0}\NormalTok{, }\DecValTok{1}\NormalTok{, }\DecValTok{0}\NormalTok{, }\DecValTok{1}\NormalTok{, }\DecValTok{1}\NormalTok{, }\DecValTok{1}\NormalTok{),}
  \AttributeTok{temperatura =} \FunctionTok{c}\NormalTok{(}\FunctionTok{rep}\NormalTok{(}\DecValTok{70}\NormalTok{, }\DecValTok{10}\NormalTok{), }\FunctionTok{rep}\NormalTok{(}\DecValTok{90}\NormalTok{, }\DecValTok{10}\NormalTok{)),  }\CommentTok{\# °C}
  \AttributeTok{carga =} \FunctionTok{c}\NormalTok{(}\FunctionTok{rep}\NormalTok{(}\StringTok{"Normal"}\NormalTok{, }\DecValTok{10}\NormalTok{), }\FunctionTok{rep}\NormalTok{(}\StringTok{"Alta"}\NormalTok{, }\DecValTok{10}\NormalTok{))}
\NormalTok{)}

\CommentTok{\# Ajustar modelo Cox}
\NormalTok{cox\_model }\OtherTok{\textless{}{-}} \FunctionTok{coxph}\NormalTok{(}\FunctionTok{Surv}\NormalTok{(tempo, status) }\SpecialCharTok{\textasciitilde{}}\NormalTok{ temperatura }\SpecialCharTok{+}\NormalTok{ carga, }\AttributeTok{data =}\NormalTok{ dados)}

\FunctionTok{summary}\NormalTok{(cox\_model)}
\end{Highlighting}
\end{Shaded}

\begin{verbatim}
Call:
coxph(formula = Surv(tempo, status) ~ temperatura + carga, data = dados)

  n= 20, number of events= 15 

               coef exp(coef) se(coef)     z Pr(>|z|)
temperatura 0.01883   1.01901  0.02705 0.696    0.486
cargaNormal      NA        NA  0.00000    NA       NA

            exp(coef) exp(-coef) lower .95 upper .95
temperatura     1.019     0.9813    0.9664     1.075
cargaNormal        NA         NA        NA        NA

Concordance= 0.537  (se = 0.078 )
Likelihood ratio test= 0.49  on 1 df,   p=0.5
Wald test            = 0.48  on 1 df,   p=0.5
Score (logrank) test = 0.49  on 1 df,   p=0.5
\end{verbatim}

\section{Interpretação dos
Coeficientes}\label{interpretauxe7uxe3o-dos-coeficientes}

\begin{Shaded}
\begin{Highlighting}[]
\CommentTok{\# Hazard ratios}
\FunctionTok{exp}\NormalTok{(}\FunctionTok{coef}\NormalTok{(cox\_model))}
\end{Highlighting}
\end{Shaded}

\begin{verbatim}
temperatura cargaNormal 
   1.019008          NA 
\end{verbatim}

\begin{Shaded}
\begin{Highlighting}[]
\CommentTok{\# Intervalos de confiança}
\FunctionTok{exp}\NormalTok{(}\FunctionTok{confint}\NormalTok{(cox\_model))}
\end{Highlighting}
\end{Shaded}

\begin{verbatim}
                2.5 % 97.5 %
temperatura 0.9663812 1.0745
cargaNormal        NA     NA
\end{verbatim}

\section{Verificação de
Pressupostos}\label{verificauxe7uxe3o-de-pressupostos}

\begin{Shaded}
\begin{Highlighting}[]
\CommentTok{\# Teste de proporcionalidade}
\FunctionTok{cox.zph}\NormalTok{(cox\_model)}
\end{Highlighting}
\end{Shaded}

\begin{verbatim}
            chisq df    p
temperatura 0.291  1 0.59
GLOBAL      0.291  1 0.59
\end{verbatim}

\begin{Shaded}
\begin{Highlighting}[]
\CommentTok{\# Gráfico de resíduos}
\FunctionTok{ggcoxzph}\NormalTok{(}\FunctionTok{cox.zph}\NormalTok{(cox\_model))}
\end{Highlighting}
\end{Shaded}

\pandocbounded{\includegraphics[keepaspectratio]{chapters/09-sobrevivencia_files/figure-pdf/cox-assumptions-1.pdf}}

\section{Curvas de Sobrevivência por
Grupo}\label{curvas-de-sobrevivuxeancia-por-grupo}

\begin{Shaded}
\begin{Highlighting}[]
\CommentTok{\# Curvas ajustadas}
\NormalTok{fit\_km }\OtherTok{\textless{}{-}} \FunctionTok{survfit}\NormalTok{(}\FunctionTok{Surv}\NormalTok{(tempo, status) }\SpecialCharTok{\textasciitilde{}}\NormalTok{ carga, }\AttributeTok{data =}\NormalTok{ dados)}

\FunctionTok{ggsurvplot}\NormalTok{(fit\_km,}
           \AttributeTok{data =}\NormalTok{ dados,}
           \AttributeTok{pval =} \ConstantTok{TRUE}\NormalTok{,}
           \AttributeTok{conf.int =} \ConstantTok{TRUE}\NormalTok{,}
           \AttributeTok{risk.table =} \ConstantTok{TRUE}\NormalTok{,}
           \AttributeTok{palette =} \FunctionTok{c}\NormalTok{(}\StringTok{"steelblue"}\NormalTok{, }\StringTok{"coral"}\NormalTok{),}
           \AttributeTok{title =} \StringTok{"Sobrevivência por Nível de Carga"}\NormalTok{)}
\end{Highlighting}
\end{Shaded}

\pandocbounded{\includegraphics[keepaspectratio]{chapters/09-sobrevivencia_files/figure-pdf/survival-curves-1.pdf}}

\section{Modelos Paramétricos AFT}\label{modelos-paramuxe9tricos-aft}

\begin{Shaded}
\begin{Highlighting}[]
\FunctionTok{library}\NormalTok{(flexsurv)}

\CommentTok{\# Modelo AFT Weibull}
\NormalTok{aft\_weibull }\OtherTok{\textless{}{-}} \FunctionTok{flexsurvreg}\NormalTok{(}\FunctionTok{Surv}\NormalTok{(tempo, status) }\SpecialCharTok{\textasciitilde{}}\NormalTok{ temperatura }\SpecialCharTok{+}\NormalTok{ carga,}
                            \AttributeTok{data =}\NormalTok{ dados,}
                            \AttributeTok{dist =} \StringTok{"weibull"}\NormalTok{)}

\FunctionTok{summary}\NormalTok{(aft\_weibull)}
\end{Highlighting}
\end{Shaded}

\section{Exercícios}\label{exercuxedcios-5}

📝 \textbf{Exercício 1}: Ajuste modelo Cox com 3 covariáveis.

📝 \textbf{Exercício 2}: Compare modelo Cox vs.~AFT.

📝 \textbf{Exercício 3}: Interprete hazard ratios e seus IC.

\section{Resumo}\label{resumo-6}

🔔 \textbf{Pontos-chave:}

\begin{itemize}
\tightlist
\item
  Modelo Cox não requer especificação da distribuição
\item
  Hazard ratios quantificam efeito das covariáveis
\item
  Verificar pressupostos é essencial
\end{itemize}

\begin{center}\rule{0.5\linewidth}{0.5pt}\end{center}

\textbf{Próximo capítulo:} Manutenção Preventiva

\part{Estratégias de Manutenção}

\chapter{Manutenção Preventiva e Otimização de
Políticas}\label{manutenuxe7uxe3o-preventiva-e-otimizauxe7uxe3o-de-poluxedticas}

\section{Introdução}\label{introduuxe7uxe3o-7}

Otimização de intervalos de manutenção preventiva baseada em modelos de
confiabilidade.

\section{Políticas de
Manutenção}\label{poluxedticas-de-manutenuxe7uxe3o}

\subsection{Manutenção Baseada em Idade
(Age-Based)}\label{manutenuxe7uxe3o-baseada-em-idade-age-based}

Substituir componente quando atinge idade \(T_p\).

\subsection{Manutenção Baseada em Uso
(Usage-Based)}\label{manutenuxe7uxe3o-baseada-em-uso-usage-based}

Substituir após \(N\) ciclos de operação.

\subsection{Manutenção Baseada em
Condição}\label{manutenuxe7uxe3o-baseada-em-condiuxe7uxe3o}

Monitorar e intervir quando indicadores ultrapassam limites.

\section{Otimização de Intervalos}\label{otimizauxe7uxe3o-de-intervalos}

\subsection{Função Custo}\label{funuxe7uxe3o-custo}

\[
C(T_p) = \frac{C_p R(T_p) + C_f [1 - R(T_p)]}{\int_0^{T_p} R(t) dt}
\]

onde: - \(C_p\) = custo de manutenção preventiva - \(C_f\) = custo de
falha corretiva - \(T_p\) = intervalo preventivo

\subsection{Exemplo: Otimização
Numérica}\label{exemplo-otimizauxe7uxe3o-numuxe9rica}

\begin{Shaded}
\begin{Highlighting}[]
\FunctionTok{library}\NormalTok{(tidyverse)}

\CommentTok{\# Parâmetros Weibull}
\NormalTok{beta }\OtherTok{\textless{}{-}} \FloatTok{2.5}
\NormalTok{eta }\OtherTok{\textless{}{-}} \DecValTok{1000}

\CommentTok{\# Custos}
\NormalTok{C\_p }\OtherTok{\textless{}{-}} \DecValTok{1000}   \CommentTok{\# Custo preventiva}
\NormalTok{C\_f }\OtherTok{\textless{}{-}} \DecValTok{5000}   \CommentTok{\# Custo falha}

\CommentTok{\# Função custo esperado}
\NormalTok{custo\_esperado }\OtherTok{\textless{}{-}} \ControlFlowTok{function}\NormalTok{(Tp) \{}
\NormalTok{  R\_Tp }\OtherTok{\textless{}{-}} \FunctionTok{exp}\NormalTok{(}\SpecialCharTok{{-}}\NormalTok{(Tp}\SpecialCharTok{/}\NormalTok{eta)}\SpecialCharTok{\^{}}\NormalTok{beta)}
\NormalTok{  integral }\OtherTok{\textless{}{-}} \FunctionTok{integrate}\NormalTok{(}\ControlFlowTok{function}\NormalTok{(t) }\FunctionTok{exp}\NormalTok{(}\SpecialCharTok{{-}}\NormalTok{(t}\SpecialCharTok{/}\NormalTok{eta)}\SpecialCharTok{\^{}}\NormalTok{beta), }\DecValTok{0}\NormalTok{, Tp)}\SpecialCharTok{$}\NormalTok{value}

\NormalTok{  (C\_p }\SpecialCharTok{*}\NormalTok{ R\_Tp }\SpecialCharTok{+}\NormalTok{ C\_f }\SpecialCharTok{*}\NormalTok{ (}\DecValTok{1} \SpecialCharTok{{-}}\NormalTok{ R\_Tp)) }\SpecialCharTok{/}\NormalTok{ integral}
\NormalTok{\}}

\CommentTok{\# Avaliar diferentes intervalos}
\NormalTok{intervalos }\OtherTok{\textless{}{-}} \FunctionTok{seq}\NormalTok{(}\DecValTok{100}\NormalTok{, }\DecValTok{1500}\NormalTok{, }\AttributeTok{by =} \DecValTok{50}\NormalTok{)}
\NormalTok{custos }\OtherTok{\textless{}{-}} \FunctionTok{sapply}\NormalTok{(intervalos, custo\_esperado)}

\NormalTok{df\_opt }\OtherTok{\textless{}{-}} \FunctionTok{data.frame}\NormalTok{(}\AttributeTok{intervalo =}\NormalTok{ intervalos, }\AttributeTok{custo =}\NormalTok{ custos)}

\CommentTok{\# Intervalo ótimo}
\NormalTok{intervalo\_otimo }\OtherTok{\textless{}{-}}\NormalTok{ intervalos[}\FunctionTok{which.min}\NormalTok{(custos)]}

\CommentTok{\# Visualizar}
\FunctionTok{ggplot}\NormalTok{(df\_opt, }\FunctionTok{aes}\NormalTok{(}\AttributeTok{x =}\NormalTok{ intervalo, }\AttributeTok{y =}\NormalTok{ custo)) }\SpecialCharTok{+}
  \FunctionTok{geom\_line}\NormalTok{(}\AttributeTok{color =} \StringTok{"steelblue"}\NormalTok{, }\AttributeTok{linewidth =} \FloatTok{1.2}\NormalTok{) }\SpecialCharTok{+}
  \FunctionTok{geom\_vline}\NormalTok{(}\AttributeTok{xintercept =}\NormalTok{ intervalo\_otimo,}
             \AttributeTok{linetype =} \StringTok{"dashed"}\NormalTok{, }\AttributeTok{color =} \StringTok{"red"}\NormalTok{) }\SpecialCharTok{+}
  \FunctionTok{geom\_point}\NormalTok{(}\AttributeTok{data =}\NormalTok{ df\_opt[}\FunctionTok{which.min}\NormalTok{(custos), ],}
             \AttributeTok{color =} \StringTok{"red"}\NormalTok{, }\AttributeTok{size =} \DecValTok{3}\NormalTok{) }\SpecialCharTok{+}
  \FunctionTok{annotate}\NormalTok{(}\StringTok{"text"}\NormalTok{, }\AttributeTok{x =}\NormalTok{ intervalo\_otimo }\SpecialCharTok{+} \DecValTok{100}\NormalTok{,}
           \AttributeTok{y =} \FunctionTok{min}\NormalTok{(custos) }\SpecialCharTok{+} \FloatTok{0.2}\NormalTok{,}
           \AttributeTok{label =} \FunctionTok{paste}\NormalTok{(}\StringTok{"Ótimo:"}\NormalTok{, intervalo\_otimo, }\StringTok{"h"}\NormalTok{),}
           \AttributeTok{color =} \StringTok{"red"}\NormalTok{) }\SpecialCharTok{+}
  \FunctionTok{labs}\NormalTok{(}\AttributeTok{title =} \StringTok{"Otimização de Intervalo Preventivo"}\NormalTok{,}
       \AttributeTok{x =} \StringTok{"Intervalo Tp (horas)"}\NormalTok{,}
       \AttributeTok{y =} \StringTok{"Custo Esperado por Hora"}\NormalTok{) }\SpecialCharTok{+}
  \FunctionTok{theme\_minimal}\NormalTok{()}
\end{Highlighting}
\end{Shaded}

\pandocbounded{\includegraphics[keepaspectratio]{chapters/10-manutencao-preventiva_files/figure-pdf/otimizacao-intervalo-1.pdf}}

\section{Análise de Sensibilidade}\label{anuxe1lise-de-sensibilidade}

\begin{Shaded}
\begin{Highlighting}[]
\CommentTok{\# Variar relação de custos}
\NormalTok{ratios }\OtherTok{\textless{}{-}} \FunctionTok{c}\NormalTok{(}\DecValTok{2}\NormalTok{, }\DecValTok{5}\NormalTok{, }\DecValTok{10}\NormalTok{, }\DecValTok{20}\NormalTok{)}

\NormalTok{df\_sensibilidade }\OtherTok{\textless{}{-}} \FunctionTok{expand.grid}\NormalTok{(}
  \AttributeTok{intervalo =} \FunctionTok{seq}\NormalTok{(}\DecValTok{100}\NormalTok{, }\DecValTok{1500}\NormalTok{, }\AttributeTok{by =} \DecValTok{50}\NormalTok{),}
  \AttributeTok{ratio =}\NormalTok{ ratios}
\NormalTok{) }\SpecialCharTok{\%\textgreater{}\%}
  \FunctionTok{mutate}\NormalTok{(}
    \AttributeTok{C\_p =} \DecValTok{1000}\NormalTok{,}
    \AttributeTok{C\_f =}\NormalTok{ C\_p }\SpecialCharTok{*}\NormalTok{ ratio,}
    \AttributeTok{R\_Tp =} \FunctionTok{exp}\NormalTok{(}\SpecialCharTok{{-}}\NormalTok{(intervalo}\SpecialCharTok{/}\NormalTok{eta)}\SpecialCharTok{\^{}}\NormalTok{beta),}
    \AttributeTok{integral =} \FunctionTok{sapply}\NormalTok{(intervalo, }\ControlFlowTok{function}\NormalTok{(Tp) \{}
      \FunctionTok{integrate}\NormalTok{(}\ControlFlowTok{function}\NormalTok{(t) }\FunctionTok{exp}\NormalTok{(}\SpecialCharTok{{-}}\NormalTok{(t}\SpecialCharTok{/}\NormalTok{eta)}\SpecialCharTok{\^{}}\NormalTok{beta), }\DecValTok{0}\NormalTok{, Tp)}\SpecialCharTok{$}\NormalTok{value}
\NormalTok{    \}),}
    \AttributeTok{custo =}\NormalTok{ (C\_p }\SpecialCharTok{*}\NormalTok{ R\_Tp }\SpecialCharTok{+}\NormalTok{ C\_f }\SpecialCharTok{*}\NormalTok{ (}\DecValTok{1} \SpecialCharTok{{-}}\NormalTok{ R\_Tp)) }\SpecialCharTok{/}\NormalTok{ integral,}
    \AttributeTok{ratio\_label =} \FunctionTok{paste}\NormalTok{(}\StringTok{"Cf/Cp ="}\NormalTok{, ratio)}
\NormalTok{  )}

\FunctionTok{ggplot}\NormalTok{(df\_sensibilidade, }\FunctionTok{aes}\NormalTok{(}\AttributeTok{x =}\NormalTok{ intervalo, }\AttributeTok{y =}\NormalTok{ custo, }\AttributeTok{color =}\NormalTok{ ratio\_label)) }\SpecialCharTok{+}
  \FunctionTok{geom\_line}\NormalTok{(}\AttributeTok{linewidth =} \DecValTok{1}\NormalTok{) }\SpecialCharTok{+}
  \FunctionTok{labs}\NormalTok{(}\AttributeTok{title =} \StringTok{"Sensibilidade ao Custo de Falha"}\NormalTok{,}
       \AttributeTok{x =} \StringTok{"Intervalo Preventivo (h)"}\NormalTok{,}
       \AttributeTok{y =} \StringTok{"Custo Esperado"}\NormalTok{,}
       \AttributeTok{color =} \StringTok{"Relação de Custos"}\NormalTok{) }\SpecialCharTok{+}
  \FunctionTok{theme\_minimal}\NormalTok{() }\SpecialCharTok{+}
  \FunctionTok{theme}\NormalTok{(}\AttributeTok{legend.position =} \StringTok{"top"}\NormalTok{)}
\end{Highlighting}
\end{Shaded}

\pandocbounded{\includegraphics[keepaspectratio]{chapters/10-manutencao-preventiva_files/figure-pdf/sensibilidade-1.pdf}}

\section{Política Block
vs.~Age-Based}\label{poluxedtica-block-vs.-age-based}

\subsection{Block Replacement}\label{block-replacement}

Todos os componentes substituídos em intervalos fixos.

\subsection{Age-Based Replacement}\label{age-based-replacement}

Cada componente substituído individualmente.

\begin{Shaded}
\begin{Highlighting}[]
\CommentTok{\# Simulação comparativa}
\NormalTok{simular\_manutencao }\OtherTok{\textless{}{-}} \ControlFlowTok{function}\NormalTok{(n\_componentes, tempo\_max, politica, ...) \{}
  \CommentTok{\# Implementação de simulação}
  \CommentTok{\# ...}
\NormalTok{\}}

\CommentTok{\# Comparar políticas}
\NormalTok{resultado\_block }\OtherTok{\textless{}{-}} \FunctionTok{simular\_manutencao}\NormalTok{(}\DecValTok{100}\NormalTok{, }\DecValTok{10000}\NormalTok{, }\StringTok{"block"}\NormalTok{, }\AttributeTok{Tp =} \DecValTok{500}\NormalTok{)}
\NormalTok{resultado\_age }\OtherTok{\textless{}{-}} \FunctionTok{simular\_manutencao}\NormalTok{(}\DecValTok{100}\NormalTok{, }\DecValTok{10000}\NormalTok{, }\StringTok{"age"}\NormalTok{, }\AttributeTok{Tp =} \DecValTok{500}\NormalTok{)}
\end{Highlighting}
\end{Shaded}

\section{Exercícios}\label{exercuxedcios-6}

📝 \textbf{Exercício 1}: Encontre o intervalo ótimo para seus dados.

📝 \textbf{Exercício 2}: Faça análise de sensibilidade variando β.

📝 \textbf{Exercício 3}: Compare políticas block vs.~age por simulação.

\section{Resumo}\label{resumo-7}

🔔 \textbf{Pontos-chave:}

\begin{itemize}
\tightlist
\item
  Intervalo ótimo equilibra custos preventivo e corretivo
\item
  Análise de sensibilidade é crucial
\item
  Diferentes políticas para diferentes contextos
\end{itemize}

\begin{center}\rule{0.5\linewidth}{0.5pt}\end{center}

\textbf{Próximo capítulo:} Manutenção Preditiva com Dados Sensor

\chapter{Manutenção Preditiva com Dados
Sensor}\label{manutenuxe7uxe3o-preditiva-com-dados-sensor}

\section{Introdução}\label{introduuxe7uxe3o-8}

Uso de dados de sensores e machine learning para predição de falhas.

\section{Preparação de Dados de
Sensores}\label{preparauxe7uxe3o-de-dados-de-sensores}

\begin{Shaded}
\begin{Highlighting}[]
\FunctionTok{library}\NormalTok{(tidyverse)}
\FunctionTok{library}\NormalTok{(lubridate)}

\CommentTok{\# Carregar dados de sensores}
\NormalTok{sensores }\OtherTok{\textless{}{-}} \FunctionTok{read\_csv}\NormalTok{(}\StringTok{"resources/data/sensores\_compressor.csv"}\NormalTok{)}

\CommentTok{\# Pré{-}processamento}
\NormalTok{sensores\_clean }\OtherTok{\textless{}{-}}\NormalTok{ sensores }\SpecialCharTok{\%\textgreater{}\%}
  \FunctionTok{mutate}\NormalTok{(}
    \AttributeTok{timestamp =} \FunctionTok{ymd\_hms}\NormalTok{(timestamp),}
    \AttributeTok{temp\_norm =}\NormalTok{ (temperatura }\SpecialCharTok{{-}} \FunctionTok{mean}\NormalTok{(temperatura)) }\SpecialCharTok{/} \FunctionTok{sd}\NormalTok{(temperatura),}
    \AttributeTok{vibr\_norm =}\NormalTok{ (vibracao }\SpecialCharTok{{-}} \FunctionTok{mean}\NormalTok{(vibracao)) }\SpecialCharTok{/} \FunctionTok{sd}\NormalTok{(vibracao)}
\NormalTok{  ) }\SpecialCharTok{\%\textgreater{}\%}
  \FunctionTok{arrange}\NormalTok{(equipamento\_id, timestamp)}

\FunctionTok{head}\NormalTok{(sensores\_clean)}
\end{Highlighting}
\end{Shaded}

\section{Feature Engineering}\label{feature-engineering}

\subsection{Estatísticas em Janela
Deslizante}\label{estatuxedsticas-em-janela-deslizante}

\begin{Shaded}
\begin{Highlighting}[]
\FunctionTok{library}\NormalTok{(zoo)}

\CommentTok{\# Criar features}
\NormalTok{sensores\_features }\OtherTok{\textless{}{-}}\NormalTok{ sensores\_clean }\SpecialCharTok{\%\textgreater{}\%}
  \FunctionTok{group\_by}\NormalTok{(equipamento\_id) }\SpecialCharTok{\%\textgreater{}\%}
  \FunctionTok{arrange}\NormalTok{(timestamp) }\SpecialCharTok{\%\textgreater{}\%}
  \FunctionTok{mutate}\NormalTok{(}
    \AttributeTok{temp\_mean\_7d =} \FunctionTok{rollmean}\NormalTok{(temperatura, }\AttributeTok{k =} \DecValTok{7}\SpecialCharTok{*}\DecValTok{24}\NormalTok{, }\AttributeTok{fill =} \ConstantTok{NA}\NormalTok{, }\AttributeTok{align =} \StringTok{"right"}\NormalTok{),}
    \AttributeTok{temp\_max\_7d =} \FunctionTok{rollmax}\NormalTok{(temperatura, }\AttributeTok{k =} \DecValTok{7}\SpecialCharTok{*}\DecValTok{24}\NormalTok{, }\AttributeTok{fill =} \ConstantTok{NA}\NormalTok{, }\AttributeTok{align =} \StringTok{"right"}\NormalTok{),}
    \AttributeTok{temp\_std\_7d =} \FunctionTok{rollapply}\NormalTok{(temperatura, }\AttributeTok{width =} \DecValTok{7}\SpecialCharTok{*}\DecValTok{24}\NormalTok{, }\AttributeTok{FUN =}\NormalTok{ sd,}
                            \AttributeTok{fill =} \ConstantTok{NA}\NormalTok{, }\AttributeTok{align =} \StringTok{"right"}\NormalTok{),}
    \AttributeTok{vibr\_trend =} \FunctionTok{c}\NormalTok{(}\ConstantTok{NA}\NormalTok{, }\FunctionTok{diff}\NormalTok{(vibracao))}
\NormalTok{  ) }\SpecialCharTok{\%\textgreater{}\%}
  \FunctionTok{ungroup}\NormalTok{()}
\end{Highlighting}
\end{Shaded}

\section{Detecção de Anomalias}\label{detecuxe7uxe3o-de-anomalias}

\begin{Shaded}
\begin{Highlighting}[]
\FunctionTok{library}\NormalTok{(anomalize)}

\CommentTok{\# Decomposição e detecção}
\NormalTok{anomalias }\OtherTok{\textless{}{-}}\NormalTok{ sensores\_clean }\SpecialCharTok{\%\textgreater{}\%}
  \FunctionTok{time\_decompose}\NormalTok{(temperatura, }\AttributeTok{method =} \StringTok{"stl"}\NormalTok{) }\SpecialCharTok{\%\textgreater{}\%}
  \FunctionTok{anomalize}\NormalTok{(remainder, }\AttributeTok{method =} \StringTok{"iqr"}\NormalTok{) }\SpecialCharTok{\%\textgreater{}\%}
  \FunctionTok{time\_recompose}\NormalTok{()}

\CommentTok{\# Visualizar}
\NormalTok{anomalias }\SpecialCharTok{\%\textgreater{}\%}
  \FunctionTok{plot\_anomalies}\NormalTok{(}\AttributeTok{time\_recomposed =} \ConstantTok{TRUE}\NormalTok{) }\SpecialCharTok{+}
  \FunctionTok{labs}\NormalTok{(}\AttributeTok{title =} \StringTok{"Detecção de Anomalias {-} Temperatura"}\NormalTok{)}
\end{Highlighting}
\end{Shaded}

\section{Modelo Preditivo com Machine
Learning}\label{modelo-preditivo-com-machine-learning}

\begin{Shaded}
\begin{Highlighting}[]
\FunctionTok{library}\NormalTok{(caret)}
\FunctionTok{library}\NormalTok{(randomForest)}

\CommentTok{\# Preparar dataset}
\CommentTok{\# Assumir que temos features + label \textquotesingle{}falha\_proxima\textquotesingle{}}
\NormalTok{dados\_treino }\OtherTok{\textless{}{-}}\NormalTok{ sensores\_features }\SpecialCharTok{\%\textgreater{}\%}
  \FunctionTok{filter}\NormalTok{(}\SpecialCharTok{!}\FunctionTok{is.na}\NormalTok{(temp\_mean\_7d)) }\SpecialCharTok{\%\textgreater{}\%}
  \FunctionTok{select}\NormalTok{(temp\_mean\_7d, temp\_std\_7d, vibr\_trend, falha\_proxima)}

\CommentTok{\# Dividir treino/teste}
\FunctionTok{set.seed}\NormalTok{(}\DecValTok{123}\NormalTok{)}
\NormalTok{indices\_treino }\OtherTok{\textless{}{-}} \FunctionTok{createDataPartition}\NormalTok{(dados\_treino}\SpecialCharTok{$}\NormalTok{falha\_proxima,}
                                      \AttributeTok{p =} \FloatTok{0.7}\NormalTok{, }\AttributeTok{list =} \ConstantTok{FALSE}\NormalTok{)}
\NormalTok{treino }\OtherTok{\textless{}{-}}\NormalTok{ dados\_treino[indices\_treino, ]}
\NormalTok{teste }\OtherTok{\textless{}{-}}\NormalTok{ dados\_treino[}\SpecialCharTok{{-}}\NormalTok{indices\_treino, ]}

\CommentTok{\# Treinar Random Forest}
\NormalTok{modelo\_rf }\OtherTok{\textless{}{-}} \FunctionTok{randomForest}\NormalTok{(falha\_proxima }\SpecialCharTok{\textasciitilde{}}\NormalTok{ .,}
                          \AttributeTok{data =}\NormalTok{ treino,}
                          \AttributeTok{ntree =} \DecValTok{100}\NormalTok{,}
                          \AttributeTok{importance =} \ConstantTok{TRUE}\NormalTok{)}

\CommentTok{\# Predição}
\NormalTok{pred\_teste }\OtherTok{\textless{}{-}} \FunctionTok{predict}\NormalTok{(modelo\_rf, teste)}

\CommentTok{\# Matriz de confusão}
\FunctionTok{confusionMatrix}\NormalTok{(pred\_teste, teste}\SpecialCharTok{$}\NormalTok{falha\_proxima)}
\end{Highlighting}
\end{Shaded}

\section{Importância de Features}\label{importuxe2ncia-de-features}

\begin{Shaded}
\begin{Highlighting}[]
\CommentTok{\# Plot importância}
\FunctionTok{varImpPlot}\NormalTok{(modelo\_rf, }\AttributeTok{main =} \StringTok{"Importância das Variáveis"}\NormalTok{)}
\end{Highlighting}
\end{Shaded}

\section{Séries Temporais com
Prophet}\label{suxe9ries-temporais-com-prophet}

\begin{Shaded}
\begin{Highlighting}[]
\FunctionTok{library}\NormalTok{(prophet)}

\CommentTok{\# Preparar dados}
\NormalTok{df\_prophet }\OtherTok{\textless{}{-}}\NormalTok{ sensores\_clean }\SpecialCharTok{\%\textgreater{}\%}
  \FunctionTok{select}\NormalTok{(}\AttributeTok{ds =}\NormalTok{ timestamp, }\AttributeTok{y =}\NormalTok{ temperatura)}

\CommentTok{\# Ajustar modelo}
\NormalTok{m }\OtherTok{\textless{}{-}} \FunctionTok{prophet}\NormalTok{(df\_prophet, }\AttributeTok{changepoint.prior.scale =} \FloatTok{0.05}\NormalTok{)}

\CommentTok{\# Previsão 30 dias}
\NormalTok{future }\OtherTok{\textless{}{-}} \FunctionTok{make\_future\_dataframe}\NormalTok{(m, }\AttributeTok{periods =} \DecValTok{30}\NormalTok{, }\AttributeTok{freq =} \StringTok{"day"}\NormalTok{)}
\NormalTok{forecast }\OtherTok{\textless{}{-}} \FunctionTok{predict}\NormalTok{(m, future)}

\CommentTok{\# Visualizar}
\FunctionTok{plot}\NormalTok{(m, forecast)}
\FunctionTok{prophet\_plot\_components}\NormalTok{(m, forecast)}
\end{Highlighting}
\end{Shaded}

\section{Threshold Dinâmico}\label{threshold-dinuxe2mico}

\begin{Shaded}
\begin{Highlighting}[]
\CommentTok{\# Calcular threshold baseado em percentil histórico}
\NormalTok{threshold\_dinamico }\OtherTok{\textless{}{-}}\NormalTok{ sensores\_clean }\SpecialCharTok{\%\textgreater{}\%}
  \FunctionTok{group\_by}\NormalTok{(equipamento\_id) }\SpecialCharTok{\%\textgreater{}\%}
  \FunctionTok{summarize}\NormalTok{(}
    \AttributeTok{temp\_p95 =} \FunctionTok{quantile}\NormalTok{(temperatura, }\FloatTok{0.95}\NormalTok{),}
    \AttributeTok{vibr\_p95 =} \FunctionTok{quantile}\NormalTok{(vibracao, }\FloatTok{0.95}\NormalTok{)}
\NormalTok{  )}

\CommentTok{\# Aplicar alertas}
\NormalTok{alertas }\OtherTok{\textless{}{-}}\NormalTok{ sensores\_clean }\SpecialCharTok{\%\textgreater{}\%}
  \FunctionTok{left\_join}\NormalTok{(threshold\_dinamico, }\AttributeTok{by =} \StringTok{"equipamento\_id"}\NormalTok{) }\SpecialCharTok{\%\textgreater{}\%}
  \FunctionTok{filter}\NormalTok{(temperatura }\SpecialCharTok{\textgreater{}}\NormalTok{ temp\_p95 }\SpecialCharTok{|}\NormalTok{ vibracao }\SpecialCharTok{\textgreater{}}\NormalTok{ vibr\_p95)}
\end{Highlighting}
\end{Shaded}

\section{Exercícios}\label{exercuxedcios-7}

📝 \textbf{Exercício 1}: Implemente detecção de anomalias com IQR.

📝 \textbf{Exercício 2}: Treine modelo XGBoost e compare com Random
Forest.

📝 \textbf{Exercício 3}: Crie dashboard Shiny para monitoramento em
tempo real.

\section{Resumo}\label{resumo-8}

🔔 \textbf{Pontos-chave:}

\begin{itemize}
\tightlist
\item
  Feature engineering é crucial
\item
  Detecção de anomalias identifica comportamento anormal
\item
  Machine learning prevê falhas antes da ocorrência
\item
  Manutenção preditiva reduz custos e downtime
\end{itemize}

\begin{center}\rule{0.5\linewidth}{0.5pt}\end{center}

\textbf{Próximo capítulo:} Modelos de Confiabilidade de Sistema

\chapter{Modelos de Confiabilidade de
Sistema}\label{modelos-de-confiabilidade-de-sistema}

\section{Introdução}\label{introduuxe7uxe3o-9}

Análise de confiabilidade de sistemas complexos com múltiplos
componentes.

\section{Sistemas em Série}\label{sistemas-em-suxe9rie}

Todos os componentes devem funcionar para o sistema operar.

\subsection{Confiabilidade}\label{confiabilidade}

\[
R_{\text{série}}(t) = \prod_{i=1}^n R_i(t)
\]

\subsection{Exemplo}\label{exemplo}

\begin{Shaded}
\begin{Highlighting}[]
\FunctionTok{library}\NormalTok{(tidyverse)}

\CommentTok{\# Confiabilidades individuais}
\NormalTok{t }\OtherTok{\textless{}{-}} \FunctionTok{seq}\NormalTok{(}\DecValTok{0}\NormalTok{, }\DecValTok{1000}\NormalTok{, }\AttributeTok{by =} \DecValTok{10}\NormalTok{)}
\NormalTok{R1 }\OtherTok{\textless{}{-}} \FunctionTok{exp}\NormalTok{(}\SpecialCharTok{{-}}\FloatTok{0.001} \SpecialCharTok{*}\NormalTok{ t)}
\NormalTok{R2 }\OtherTok{\textless{}{-}} \FunctionTok{exp}\NormalTok{(}\SpecialCharTok{{-}}\FloatTok{0.0015} \SpecialCharTok{*}\NormalTok{ t)}
\NormalTok{R3 }\OtherTok{\textless{}{-}} \FunctionTok{exp}\NormalTok{(}\SpecialCharTok{{-}}\FloatTok{0.0008} \SpecialCharTok{*}\NormalTok{ t)}

\CommentTok{\# Sistema em série}
\NormalTok{R\_serie }\OtherTok{\textless{}{-}}\NormalTok{ R1 }\SpecialCharTok{*}\NormalTok{ R2 }\SpecialCharTok{*}\NormalTok{ R3}

\NormalTok{df }\OtherTok{\textless{}{-}} \FunctionTok{data.frame}\NormalTok{(t, R1, R2, R3, R\_serie)}

\FunctionTok{ggplot}\NormalTok{(df) }\SpecialCharTok{+}
  \FunctionTok{geom\_line}\NormalTok{(}\FunctionTok{aes}\NormalTok{(t, R1, }\AttributeTok{color =} \StringTok{"Componente 1"}\NormalTok{)) }\SpecialCharTok{+}
  \FunctionTok{geom\_line}\NormalTok{(}\FunctionTok{aes}\NormalTok{(t, R2, }\AttributeTok{color =} \StringTok{"Componente 2"}\NormalTok{)) }\SpecialCharTok{+}
  \FunctionTok{geom\_line}\NormalTok{(}\FunctionTok{aes}\NormalTok{(t, R3, }\AttributeTok{color =} \StringTok{"Componente 3"}\NormalTok{)) }\SpecialCharTok{+}
  \FunctionTok{geom\_line}\NormalTok{(}\FunctionTok{aes}\NormalTok{(t, R\_serie, }\AttributeTok{color =} \StringTok{"Sistema"}\NormalTok{), }\AttributeTok{linewidth =} \FloatTok{1.2}\NormalTok{, }\AttributeTok{linetype =} \StringTok{"dashed"}\NormalTok{) }\SpecialCharTok{+}
  \FunctionTok{labs}\NormalTok{(}\AttributeTok{title =} \StringTok{"Sistema em Série"}\NormalTok{,}
       \AttributeTok{x =} \StringTok{"Tempo (h)"}\NormalTok{, }\AttributeTok{y =} \StringTok{"Confiabilidade"}\NormalTok{,}
       \AttributeTok{color =} \StringTok{""}\NormalTok{) }\SpecialCharTok{+}
  \FunctionTok{theme\_minimal}\NormalTok{()}
\end{Highlighting}
\end{Shaded}

\pandocbounded{\includegraphics[keepaspectratio]{chapters/12-sistemas_files/figure-pdf/sistema-serie-1.pdf}}

\section{Sistemas em Paralelo}\label{sistemas-em-paralelo}

Sistema funciona se \textbf{pelo menos um} componente funcionar
(redundância).

\subsection{Confiabilidade}\label{confiabilidade-1}

\[
R_{\text{paralelo}}(t) = 1 - \prod_{i=1}^n [1 - R_i(t)]
\]

\subsection{Exemplo}\label{exemplo-1}

\begin{Shaded}
\begin{Highlighting}[]
\CommentTok{\# Sistema em paralelo (redundância)}
\NormalTok{R\_paralelo }\OtherTok{\textless{}{-}} \DecValTok{1} \SpecialCharTok{{-}}\NormalTok{ (}\DecValTok{1} \SpecialCharTok{{-}}\NormalTok{ R1) }\SpecialCharTok{*}\NormalTok{ (}\DecValTok{1} \SpecialCharTok{{-}}\NormalTok{ R2) }\SpecialCharTok{*}\NormalTok{ (}\DecValTok{1} \SpecialCharTok{{-}}\NormalTok{ R3)}

\NormalTok{df}\SpecialCharTok{$}\NormalTok{R\_paralelo }\OtherTok{\textless{}{-}}\NormalTok{ R\_paralelo}

\FunctionTok{ggplot}\NormalTok{(df) }\SpecialCharTok{+}
  \FunctionTok{geom\_line}\NormalTok{(}\FunctionTok{aes}\NormalTok{(t, R1, }\AttributeTok{color =} \StringTok{"Componente 1"}\NormalTok{)) }\SpecialCharTok{+}
  \FunctionTok{geom\_line}\NormalTok{(}\FunctionTok{aes}\NormalTok{(t, R\_paralelo, }\AttributeTok{color =} \StringTok{"Sistema Paralelo"}\NormalTok{),}
            \AttributeTok{linewidth =} \FloatTok{1.2}\NormalTok{, }\AttributeTok{linetype =} \StringTok{"dashed"}\NormalTok{) }\SpecialCharTok{+}
  \FunctionTok{labs}\NormalTok{(}\AttributeTok{title =} \StringTok{"Sistema em Paralelo (Redundância)"}\NormalTok{,}
       \AttributeTok{x =} \StringTok{"Tempo (h)"}\NormalTok{, }\AttributeTok{y =} \StringTok{"Confiabilidade"}\NormalTok{) }\SpecialCharTok{+}
  \FunctionTok{theme\_minimal}\NormalTok{()}
\end{Highlighting}
\end{Shaded}

\pandocbounded{\includegraphics[keepaspectratio]{chapters/12-sistemas_files/figure-pdf/sistema-paralelo-1.pdf}}

\section{Sistema k-out-of-n}\label{sistema-k-out-of-n}

Sistema funciona se \textbf{pelo menos k de n} componentes funcionarem.

\subsection{Confiabilidade}\label{confiabilidade-2}

\[
R_{k/n}(t) = \sum_{i=k}^n \binom{n}{i} R(t)^i [1-R(t)]^{n-i}
\]

\subsection{Exemplo: 2-out-of-3}\label{exemplo-2-out-of-3}

\begin{Shaded}
\begin{Highlighting}[]
\CommentTok{\# Assumindo componentes idênticos}
\NormalTok{R }\OtherTok{\textless{}{-}} \FunctionTok{exp}\NormalTok{(}\SpecialCharTok{{-}}\FloatTok{0.001} \SpecialCharTok{*}\NormalTok{ t)}
\NormalTok{n }\OtherTok{\textless{}{-}} \DecValTok{3}
\NormalTok{k }\OtherTok{\textless{}{-}} \DecValTok{2}

\NormalTok{R\_2of3 }\OtherTok{\textless{}{-}} \FunctionTok{dbinom}\NormalTok{(}\DecValTok{2}\NormalTok{, n, R) }\SpecialCharTok{+} \FunctionTok{dbinom}\NormalTok{(}\DecValTok{3}\NormalTok{, n, R)}

\NormalTok{df}\SpecialCharTok{$}\NormalTok{R\_2of3 }\OtherTok{\textless{}{-}}\NormalTok{ R\_2of3}

\FunctionTok{ggplot}\NormalTok{(df) }\SpecialCharTok{+}
  \FunctionTok{geom\_line}\NormalTok{(}\FunctionTok{aes}\NormalTok{(t, R, }\AttributeTok{color =} \StringTok{"Componente Individual"}\NormalTok{)) }\SpecialCharTok{+}
  \FunctionTok{geom\_line}\NormalTok{(}\FunctionTok{aes}\NormalTok{(t, R\_serie, }\AttributeTok{color =} \StringTok{"3 em Série"}\NormalTok{)) }\SpecialCharTok{+}
  \FunctionTok{geom\_line}\NormalTok{(}\FunctionTok{aes}\NormalTok{(t, R\_2of3, }\AttributeTok{color =} \StringTok{"2{-}out{-}of{-}3"}\NormalTok{), }\AttributeTok{linewidth =} \FloatTok{1.2}\NormalTok{) }\SpecialCharTok{+}
  \FunctionTok{labs}\NormalTok{(}\AttributeTok{title =} \StringTok{"Comparação de Arquiteturas"}\NormalTok{,}
       \AttributeTok{x =} \StringTok{"Tempo (h)"}\NormalTok{, }\AttributeTok{y =} \StringTok{"Confiabilidade"}\NormalTok{) }\SpecialCharTok{+}
  \FunctionTok{theme\_minimal}\NormalTok{()}
\end{Highlighting}
\end{Shaded}

\pandocbounded{\includegraphics[keepaspectratio]{chapters/12-sistemas_files/figure-pdf/k-out-of-n-1.pdf}}

\section{Análise de Árvore de Falhas
(FTA)}\label{anuxe1lise-de-uxe1rvore-de-falhas-fta}

\subsection{Estrutura}\label{estrutura-1}

\begin{verbatim}
                 TOP EVENT
                     |
              +------+------+
              |             |
          Gate OR       Gate AND
              |             |
         +----+----+   +----+----+
         |         |   |         |
      Event1   Event2 Event3  Event4
\end{verbatim}

\subsection{Cálculo com R}\label{cuxe1lculo-com-r}

\begin{Shaded}
\begin{Highlighting}[]
\CommentTok{\# Probabilidades de eventos básicos}
\NormalTok{P\_E1 }\OtherTok{\textless{}{-}} \FloatTok{0.01}
\NormalTok{P\_E2 }\OtherTok{\textless{}{-}} \FloatTok{0.02}
\NormalTok{P\_E3 }\OtherTok{\textless{}{-}} \FloatTok{0.015}
\NormalTok{P\_E4 }\OtherTok{\textless{}{-}} \FloatTok{0.01}

\CommentTok{\# Gate OR}
\NormalTok{P\_OR }\OtherTok{\textless{}{-}} \DecValTok{1} \SpecialCharTok{{-}}\NormalTok{ (}\DecValTok{1} \SpecialCharTok{{-}}\NormalTok{ P\_E1) }\SpecialCharTok{*}\NormalTok{ (}\DecValTok{1} \SpecialCharTok{{-}}\NormalTok{ P\_E2)}

\CommentTok{\# Gate AND}
\NormalTok{P\_AND }\OtherTok{\textless{}{-}}\NormalTok{ P\_E3 }\SpecialCharTok{*}\NormalTok{ P\_E4}

\CommentTok{\# TOP EVENT (OR dos gates)}
\NormalTok{P\_TOP }\OtherTok{\textless{}{-}} \DecValTok{1} \SpecialCharTok{{-}}\NormalTok{ (}\DecValTok{1} \SpecialCharTok{{-}}\NormalTok{ P\_OR) }\SpecialCharTok{*}\NormalTok{ (}\DecValTok{1} \SpecialCharTok{{-}}\NormalTok{ P\_AND)}

\FunctionTok{cat}\NormalTok{(}\StringTok{"Probabilidade do TOP EVENT:"}\NormalTok{, P\_TOP, }\StringTok{"}\SpecialCharTok{\textbackslash{}n}\StringTok{"}\NormalTok{)}
\end{Highlighting}
\end{Shaded}

\begin{verbatim}
Probabilidade do TOP EVENT: 0.02994553 
\end{verbatim}

\section{FMEA (Análise de Modos e Efeitos de
Falha)}\label{fmea-anuxe1lise-de-modos-e-efeitos-de-falha}

\subsection{Template}\label{template}

\begin{Shaded}
\begin{Highlighting}[]
\FunctionTok{library}\NormalTok{(kableExtra)}

\NormalTok{fmea }\OtherTok{\textless{}{-}} \FunctionTok{data.frame}\NormalTok{(}
  \AttributeTok{Componente =} \FunctionTok{c}\NormalTok{(}\StringTok{"Motor"}\NormalTok{, }\StringTok{"Rolamento"}\NormalTok{, }\StringTok{"Selo"}\NormalTok{),}
  \AttributeTok{Modo\_Falha =} \FunctionTok{c}\NormalTok{(}\StringTok{"Superaquecimento"}\NormalTok{, }\StringTok{"Desgaste"}\NormalTok{, }\StringTok{"Vazamento"}\NormalTok{),}
  \AttributeTok{Efeito =} \FunctionTok{c}\NormalTok{(}\StringTok{"Parada"}\NormalTok{, }\StringTok{"Vibração"}\NormalTok{, }\StringTok{"Contaminação"}\NormalTok{),}
  \AttributeTok{Severidade =} \FunctionTok{c}\NormalTok{(}\DecValTok{10}\NormalTok{, }\DecValTok{7}\NormalTok{, }\DecValTok{5}\NormalTok{),}
  \AttributeTok{Ocorrencia =} \FunctionTok{c}\NormalTok{(}\DecValTok{3}\NormalTok{, }\DecValTok{6}\NormalTok{, }\DecValTok{8}\NormalTok{),}
  \AttributeTok{Deteccao =} \FunctionTok{c}\NormalTok{(}\DecValTok{5}\NormalTok{, }\DecValTok{4}\NormalTok{, }\DecValTok{6}\NormalTok{),}
  \AttributeTok{RPN =} \FunctionTok{c}\NormalTok{(}\DecValTok{150}\NormalTok{, }\DecValTok{168}\NormalTok{, }\DecValTok{240}\NormalTok{)}
\NormalTok{) }\SpecialCharTok{\%\textgreater{}\%}
  \FunctionTok{arrange}\NormalTok{(}\FunctionTok{desc}\NormalTok{(RPN))}

\FunctionTok{kable}\NormalTok{(fmea, }\AttributeTok{caption =} \StringTok{"Análise FMEA"}\NormalTok{) }\SpecialCharTok{\%\textgreater{}\%}
  \FunctionTok{kable\_styling}\NormalTok{(}\AttributeTok{bootstrap\_options =} \FunctionTok{c}\NormalTok{(}\StringTok{"striped"}\NormalTok{, }\StringTok{"hover"}\NormalTok{)) }\SpecialCharTok{\%\textgreater{}\%}
  \FunctionTok{row\_spec}\NormalTok{(}\FunctionTok{which.max}\NormalTok{(fmea}\SpecialCharTok{$}\NormalTok{RPN), }\AttributeTok{bold =} \ConstantTok{TRUE}\NormalTok{, }\AttributeTok{color =} \StringTok{"white"}\NormalTok{, }\AttributeTok{background =} \StringTok{"red"}\NormalTok{)}
\end{Highlighting}
\end{Shaded}

\begin{longtable}[t]{lllrrrr}
\caption{Análise FMEA}\\
\toprule
Componente & Modo\_Falha & Efeito & Severidade & Ocorrencia & Deteccao & RPN\\
\midrule
\cellcolor{red}{\textcolor{white}{\textbf{Selo}}} & \cellcolor{red}{\textcolor{white}{\textbf{Vazamento}}} & \cellcolor{red}{\textcolor{white}{\textbf{Contaminação}}} & \cellcolor{red}{\textcolor{white}{\textbf{5}}} & \cellcolor{red}{\textcolor{white}{\textbf{8}}} & \cellcolor{red}{\textcolor{white}{\textbf{6}}} & \cellcolor{red}{\textcolor{white}{\textbf{240}}}\\
Rolamento & Desgaste & Vibração & 7 & 6 & 4 & 168\\
Motor & Superaquecimento & Parada & 10 & 3 & 5 & 150\\
\bottomrule
\end{longtable}

\section{Diagrama de Blocos de Confiabilidade
(RBD)}\label{diagrama-de-blocos-de-confiabilidade-rbd}

\begin{Shaded}
\begin{Highlighting}[]
\FunctionTok{library}\NormalTok{(DiagrammeR)}

\CommentTok{\# Criar RBD}
\FunctionTok{grViz}\NormalTok{(}\StringTok{"}
\StringTok{digraph RBD \{}
\StringTok{  rankdir=LR}
\StringTok{  node [shape=box]}

\StringTok{  Input {-}\textgreater{} A}
\StringTok{  A {-}\textgreater{} B}
\StringTok{  B {-}\textgreater{} C}
\StringTok{  C {-}\textgreater{} Output}

\StringTok{  A {-}\textgreater{} D [style=dashed]}
\StringTok{  D {-}\textgreater{} C [style=dashed]}

\StringTok{  A [label=\textquotesingle{}Componente A\textquotesingle{}]}
\StringTok{  B [label=\textquotesingle{}Componente B\textquotesingle{}]}
\StringTok{  C [label=\textquotesingle{}Componente C\textquotesingle{}]}
\StringTok{  D [label=\textquotesingle{}Redundância D\textquotesingle{}]}
\StringTok{\}}
\StringTok{"}\NormalTok{)}
\end{Highlighting}
\end{Shaded}

\section{Exercícios}\label{exercuxedcios-8}

📝 \textbf{Exercício 1}: Calcule confiabilidade de sistema
série-paralelo.

📝 \textbf{Exercício 2}: Construa FTA para um sistema real.

📝 \textbf{Exercício 3}: Elabore FMEA completa de um equipamento.

\section{Resumo}\label{resumo-9}

🔔 \textbf{Pontos-chave:}

\begin{itemize}
\tightlist
\item
  Sistemas série: menos confiáveis que componentes
\item
  Sistemas paralelo: redundância aumenta confiabilidade
\item
  FTA e FMEA são ferramentas essenciais
\item
  RBD visualiza arquitetura de confiabilidade
\end{itemize}

\begin{center}\rule{0.5\linewidth}{0.5pt}\end{center}

\textbf{Próximo capítulo:} Simulação Monte Carlo

\part{Métodos Avançados e Aplicações}

\chapter{Simulação Monte Carlo Aplicada à
Confiabilidade}\label{simulauxe7uxe3o-monte-carlo-aplicada-uxe0-confiabilidade}

\section{Introdução}\label{introduuxe7uxe3o-10}

Uso de simulação Monte Carlo para avaliar confiabilidade e políticas de
manutenção sob incerteza.

\section{Conceitos Básicos}\label{conceitos-buxe1sicos-1}

\subsection{Método Monte Carlo}\label{muxe9todo-monte-carlo}

\begin{enumerate}
\def\labelenumi{\arabic{enumi}.}
\tightlist
\item
  Definir modelo matemático
\item
  Identificar variáveis aleatórias
\item
  Gerar amostras aleatórias
\item
  Calcular resultados para cada amostra
\item
  Analisar distribuição dos resultados
\end{enumerate}

\section{Simulação de Tempos de
Vida}\label{simulauxe7uxe3o-de-tempos-de-vida}

\subsection{Distribuição Weibull}\label{distribuiuxe7uxe3o-weibull-1}

\begin{Shaded}
\begin{Highlighting}[]
\FunctionTok{library}\NormalTok{(tidyverse)}

\FunctionTok{set.seed}\NormalTok{(}\DecValTok{123}\NormalTok{)}

\CommentTok{\# Parâmetros}
\NormalTok{beta }\OtherTok{\textless{}{-}} \DecValTok{2}
\NormalTok{eta }\OtherTok{\textless{}{-}} \DecValTok{1000}
\NormalTok{n\_sim }\OtherTok{\textless{}{-}} \DecValTok{10000}

\CommentTok{\# Simular tempos de falha}
\NormalTok{tempos\_falha }\OtherTok{\textless{}{-}} \FunctionTok{rweibull}\NormalTok{(n\_sim, }\AttributeTok{shape =}\NormalTok{ beta, }\AttributeTok{scale =}\NormalTok{ eta)}

\CommentTok{\# Visualizar}
\NormalTok{df\_sim }\OtherTok{\textless{}{-}} \FunctionTok{data.frame}\NormalTok{(}\AttributeTok{tempo =}\NormalTok{ tempos\_falha)}

\FunctionTok{ggplot}\NormalTok{(df\_sim, }\FunctionTok{aes}\NormalTok{(}\AttributeTok{x =}\NormalTok{ tempo)) }\SpecialCharTok{+}
  \FunctionTok{geom\_histogram}\NormalTok{(}\FunctionTok{aes}\NormalTok{(}\AttributeTok{y =} \FunctionTok{after\_stat}\NormalTok{(density)),}
                 \AttributeTok{bins =} \DecValTok{50}\NormalTok{, }\AttributeTok{fill =} \StringTok{"steelblue"}\NormalTok{, }\AttributeTok{alpha =} \FloatTok{0.7}\NormalTok{) }\SpecialCharTok{+}
  \FunctionTok{stat\_function}\NormalTok{(}\AttributeTok{fun =}\NormalTok{ dweibull,}
                \AttributeTok{args =} \FunctionTok{list}\NormalTok{(}\AttributeTok{shape =}\NormalTok{ beta, }\AttributeTok{scale =}\NormalTok{ eta),}
                \AttributeTok{color =} \StringTok{"red"}\NormalTok{, }\AttributeTok{linewidth =} \FloatTok{1.2}\NormalTok{) }\SpecialCharTok{+}
  \FunctionTok{labs}\NormalTok{(}\AttributeTok{title =} \StringTok{"Simulação Monte Carlo {-} Weibull"}\NormalTok{,}
       \AttributeTok{x =} \StringTok{"Tempo até Falha"}\NormalTok{, }\AttributeTok{y =} \StringTok{"Densidade"}\NormalTok{) }\SpecialCharTok{+}
  \FunctionTok{theme\_minimal}\NormalTok{()}
\end{Highlighting}
\end{Shaded}

\pandocbounded{\includegraphics[keepaspectratio]{chapters/13-simulacao_files/figure-pdf/simulacao-weibull-1.pdf}}

\subsection{Estatísticas Simuladas}\label{estatuxedsticas-simuladas}

\begin{Shaded}
\begin{Highlighting}[]
\CommentTok{\# Comparar com valores teóricos}
\NormalTok{MTTF\_teorico }\OtherTok{\textless{}{-}}\NormalTok{ eta }\SpecialCharTok{*} \FunctionTok{gamma}\NormalTok{(}\DecValTok{1} \SpecialCharTok{+} \DecValTok{1}\SpecialCharTok{/}\NormalTok{beta)}
\NormalTok{MTTF\_simulado }\OtherTok{\textless{}{-}} \FunctionTok{mean}\NormalTok{(tempos\_falha)}

\FunctionTok{cat}\NormalTok{(}\StringTok{"MTTF Teórico:"}\NormalTok{, }\FunctionTok{round}\NormalTok{(MTTF\_teorico, }\DecValTok{2}\NormalTok{), }\StringTok{"}\SpecialCharTok{\textbackslash{}n}\StringTok{"}\NormalTok{)}
\end{Highlighting}
\end{Shaded}

\begin{verbatim}
MTTF Teórico: 886.23 
\end{verbatim}

\begin{Shaded}
\begin{Highlighting}[]
\FunctionTok{cat}\NormalTok{(}\StringTok{"MTTF Simulado:"}\NormalTok{, }\FunctionTok{round}\NormalTok{(MTTF\_simulado, }\DecValTok{2}\NormalTok{), }\StringTok{"}\SpecialCharTok{\textbackslash{}n}\StringTok{"}\NormalTok{)}
\end{Highlighting}
\end{Shaded}

\begin{verbatim}
MTTF Simulado: 889.75 
\end{verbatim}

\begin{Shaded}
\begin{Highlighting}[]
\FunctionTok{cat}\NormalTok{(}\StringTok{"Erro relativo:"}\NormalTok{, }\FunctionTok{round}\NormalTok{(}\FunctionTok{abs}\NormalTok{(MTTF\_teorico }\SpecialCharTok{{-}}\NormalTok{ MTTF\_simulado)}\SpecialCharTok{/}\NormalTok{MTTF\_teorico }\SpecialCharTok{*} \DecValTok{100}\NormalTok{, }\DecValTok{2}\NormalTok{), }\StringTok{"\%}\SpecialCharTok{\textbackslash{}n}\StringTok{"}\NormalTok{)}
\end{Highlighting}
\end{Shaded}

\begin{verbatim}
Erro relativo: 0.4 %
\end{verbatim}

\section{Simulação de Sistemas
Complexos}\label{simulauxe7uxe3o-de-sistemas-complexos}

\subsection{Sistema Série}\label{sistema-suxe9rie}

\begin{Shaded}
\begin{Highlighting}[]
\CommentTok{\# Simular sistema com 3 componentes em série}
\NormalTok{n\_comp }\OtherTok{\textless{}{-}} \DecValTok{3}
\NormalTok{n\_sim }\OtherTok{\textless{}{-}} \DecValTok{5000}

\CommentTok{\# Parâmetros de cada componente}
\NormalTok{params }\OtherTok{\textless{}{-}} \FunctionTok{data.frame}\NormalTok{(}
  \AttributeTok{comp =} \DecValTok{1}\SpecialCharTok{:}\NormalTok{n\_comp,}
  \AttributeTok{beta =} \FunctionTok{c}\NormalTok{(}\DecValTok{2}\NormalTok{, }\FloatTok{2.5}\NormalTok{, }\FloatTok{1.8}\NormalTok{),}
  \AttributeTok{eta =} \FunctionTok{c}\NormalTok{(}\DecValTok{1000}\NormalTok{, }\DecValTok{1200}\NormalTok{, }\DecValTok{900}\NormalTok{)}
\NormalTok{)}

\CommentTok{\# Simular}
\FunctionTok{set.seed}\NormalTok{(}\DecValTok{456}\NormalTok{)}
\NormalTok{falhas\_sistema }\OtherTok{\textless{}{-}} \FunctionTok{replicate}\NormalTok{(n\_sim, \{}
\NormalTok{  tempos\_comp }\OtherTok{\textless{}{-}} \FunctionTok{mapply}\NormalTok{(rweibull,}
                        \AttributeTok{n =} \DecValTok{1}\NormalTok{,}
                        \AttributeTok{shape =}\NormalTok{ params}\SpecialCharTok{$}\NormalTok{beta,}
                        \AttributeTok{scale =}\NormalTok{ params}\SpecialCharTok{$}\NormalTok{eta)}
  \FunctionTok{min}\NormalTok{(tempos\_comp)  }\CommentTok{\# Sistema série: falha no primeiro componente}
\NormalTok{\})}

\CommentTok{\# MTTF do sistema}
\FunctionTok{cat}\NormalTok{(}\StringTok{"MTTF Sistema Série:"}\NormalTok{, }\FunctionTok{round}\NormalTok{(}\FunctionTok{mean}\NormalTok{(falhas\_sistema), }\DecValTok{2}\NormalTok{), }\StringTok{"h}\SpecialCharTok{\textbackslash{}n}\StringTok{"}\NormalTok{)}
\end{Highlighting}
\end{Shaded}

\begin{verbatim}
MTTF Sistema Série: 525.87 h
\end{verbatim}

\begin{Shaded}
\begin{Highlighting}[]
\CommentTok{\# Visualizar}
\FunctionTok{data.frame}\NormalTok{(}\AttributeTok{tempo =}\NormalTok{ falhas\_sistema) }\SpecialCharTok{\%\textgreater{}\%}
  \FunctionTok{ggplot}\NormalTok{(}\FunctionTok{aes}\NormalTok{(}\AttributeTok{x =}\NormalTok{ tempo)) }\SpecialCharTok{+}
  \FunctionTok{geom\_histogram}\NormalTok{(}\AttributeTok{bins =} \DecValTok{50}\NormalTok{, }\AttributeTok{fill =} \StringTok{"coral"}\NormalTok{, }\AttributeTok{alpha =} \FloatTok{0.7}\NormalTok{) }\SpecialCharTok{+}
  \FunctionTok{labs}\NormalTok{(}\AttributeTok{title =} \StringTok{"Distribuição de Tempos de Falha {-} Sistema Série"}\NormalTok{,}
       \AttributeTok{x =} \StringTok{"Tempo até Falha do Sistema"}\NormalTok{, }\AttributeTok{y =} \StringTok{"Frequência"}\NormalTok{) }\SpecialCharTok{+}
  \FunctionTok{theme\_minimal}\NormalTok{()}
\end{Highlighting}
\end{Shaded}

\pandocbounded{\includegraphics[keepaspectratio]{chapters/13-simulacao_files/figure-pdf/sim-sistema-serie-1.pdf}}

\section{Avaliação de Políticas de
Manutenção}\label{avaliauxe7uxe3o-de-poluxedticas-de-manutenuxe7uxe3o}

\subsection{Simulação de Política
Preventiva}\label{simulauxe7uxe3o-de-poluxedtica-preventiva}

\begin{Shaded}
\begin{Highlighting}[]
\CommentTok{\# Função para simular política de manutenção}
\NormalTok{simular\_manutencao }\OtherTok{\textless{}{-}} \ControlFlowTok{function}\NormalTok{(T\_preventivo, tempo\_max,}
\NormalTok{                               beta, eta, C\_prev, C\_falha) \{}
\NormalTok{  tempo\_atual }\OtherTok{\textless{}{-}} \DecValTok{0}
\NormalTok{  idade\_componente }\OtherTok{\textless{}{-}} \DecValTok{0}
\NormalTok{  custo\_total }\OtherTok{\textless{}{-}} \DecValTok{0}
\NormalTok{  n\_prev }\OtherTok{\textless{}{-}} \DecValTok{0}
\NormalTok{  n\_falhas }\OtherTok{\textless{}{-}} \DecValTok{0}

  \ControlFlowTok{while}\NormalTok{ (tempo\_atual }\SpecialCharTok{\textless{}}\NormalTok{ tempo\_max) \{}
    \CommentTok{\# Tempo até próxima falha}
\NormalTok{    tempo\_falha }\OtherTok{\textless{}{-}} \FunctionTok{rweibull}\NormalTok{(}\DecValTok{1}\NormalTok{, }\AttributeTok{shape =}\NormalTok{ beta, }\AttributeTok{scale =}\NormalTok{ eta)}

    \ControlFlowTok{if}\NormalTok{ (idade\_componente }\SpecialCharTok{+}\NormalTok{ tempo\_falha }\SpecialCharTok{\textless{}}\NormalTok{ T\_preventivo) \{}
      \CommentTok{\# Falha antes da preventiva}
\NormalTok{      tempo\_atual }\OtherTok{\textless{}{-}}\NormalTok{ tempo\_atual }\SpecialCharTok{+}\NormalTok{ tempo\_falha }\SpecialCharTok{+}\NormalTok{ idade\_componente}
\NormalTok{      custo\_total }\OtherTok{\textless{}{-}}\NormalTok{ custo\_total }\SpecialCharTok{+}\NormalTok{ C\_falha}
\NormalTok{      idade\_componente }\OtherTok{\textless{}{-}} \DecValTok{0}
\NormalTok{      n\_falhas }\OtherTok{\textless{}{-}}\NormalTok{ n\_falhas }\SpecialCharTok{+} \DecValTok{1}
\NormalTok{    \} }\ControlFlowTok{else}\NormalTok{ \{}
      \CommentTok{\# Preventiva executada}
\NormalTok{      tempo\_atual }\OtherTok{\textless{}{-}}\NormalTok{ tempo\_atual }\SpecialCharTok{+}\NormalTok{ T\_preventivo}
\NormalTok{      custo\_total }\OtherTok{\textless{}{-}}\NormalTok{ custo\_total }\SpecialCharTok{+}\NormalTok{ C\_prev}
\NormalTok{      idade\_componente }\OtherTok{\textless{}{-}} \DecValTok{0}
\NormalTok{      n\_prev }\OtherTok{\textless{}{-}}\NormalTok{ n\_prev }\SpecialCharTok{+} \DecValTok{1}
\NormalTok{    \}}
\NormalTok{  \}}

  \FunctionTok{list}\NormalTok{(}
    \AttributeTok{custo\_medio =}\NormalTok{ custo\_total }\SpecialCharTok{/}\NormalTok{ tempo\_max,}
    \AttributeTok{n\_preventivas =}\NormalTok{ n\_prev,}
    \AttributeTok{n\_falhas =}\NormalTok{ n\_falhas}
\NormalTok{  )}
\NormalTok{\}}

\CommentTok{\# Parâmetros}
\NormalTok{beta }\OtherTok{\textless{}{-}} \FloatTok{2.5}
\NormalTok{eta }\OtherTok{\textless{}{-}} \DecValTok{1000}
\NormalTok{C\_prev }\OtherTok{\textless{}{-}} \DecValTok{1000}
\NormalTok{C\_falha }\OtherTok{\textless{}{-}} \DecValTok{5000}
\NormalTok{tempo\_max }\OtherTok{\textless{}{-}} \DecValTok{50000}

\CommentTok{\# Testar diferentes intervalos}
\NormalTok{intervalos }\OtherTok{\textless{}{-}} \FunctionTok{seq}\NormalTok{(}\DecValTok{300}\NormalTok{, }\DecValTok{1200}\NormalTok{, }\AttributeTok{by =} \DecValTok{50}\NormalTok{)}

\NormalTok{resultados }\OtherTok{\textless{}{-}} \FunctionTok{lapply}\NormalTok{(intervalos, }\ControlFlowTok{function}\NormalTok{(Tp) \{}
  \FunctionTok{replicate}\NormalTok{(}\DecValTok{100}\NormalTok{, \{}
\NormalTok{    sim }\OtherTok{\textless{}{-}} \FunctionTok{simular\_manutencao}\NormalTok{(Tp, tempo\_max, beta, eta, C\_prev, C\_falha)}
\NormalTok{    sim}\SpecialCharTok{$}\NormalTok{custo\_medio}
\NormalTok{  \}) }\SpecialCharTok{\%\textgreater{}\%} \FunctionTok{mean}\NormalTok{()}
\NormalTok{\})}

\NormalTok{df\_politicas }\OtherTok{\textless{}{-}} \FunctionTok{data.frame}\NormalTok{(}
  \AttributeTok{intervalo =}\NormalTok{ intervalos,}
  \AttributeTok{custo\_medio =} \FunctionTok{unlist}\NormalTok{(resultados)}
\NormalTok{)}

\CommentTok{\# Visualizar}
\FunctionTok{ggplot}\NormalTok{(df\_politicas, }\FunctionTok{aes}\NormalTok{(}\AttributeTok{x =}\NormalTok{ intervalo, }\AttributeTok{y =}\NormalTok{ custo\_medio)) }\SpecialCharTok{+}
  \FunctionTok{geom\_line}\NormalTok{(}\AttributeTok{color =} \StringTok{"steelblue"}\NormalTok{, }\AttributeTok{linewidth =} \FloatTok{1.2}\NormalTok{) }\SpecialCharTok{+}
  \FunctionTok{geom\_point}\NormalTok{(}\AttributeTok{data =}\NormalTok{ df\_politicas[}\FunctionTok{which.min}\NormalTok{(df\_politicas}\SpecialCharTok{$}\NormalTok{custo\_medio), ],}
             \AttributeTok{color =} \StringTok{"red"}\NormalTok{, }\AttributeTok{size =} \DecValTok{4}\NormalTok{) }\SpecialCharTok{+}
  \FunctionTok{labs}\NormalTok{(}\AttributeTok{title =} \StringTok{"Otimização por Simulação Monte Carlo"}\NormalTok{,}
       \AttributeTok{x =} \StringTok{"Intervalo Preventivo (h)"}\NormalTok{,}
       \AttributeTok{y =} \StringTok{"Custo Médio por Hora"}\NormalTok{) }\SpecialCharTok{+}
  \FunctionTok{theme\_minimal}\NormalTok{()}
\end{Highlighting}
\end{Shaded}

\pandocbounded{\includegraphics[keepaspectratio]{chapters/13-simulacao_files/figure-pdf/sim-politica-preventiva-1.pdf}}

\section{Análise de Incerteza}\label{anuxe1lise-de-incerteza}

\subsection{Propagação de
Incerteza}\label{propagauxe7uxe3o-de-incerteza}

\begin{Shaded}
\begin{Highlighting}[]
\CommentTok{\# Incerteza nos parâmetros}
\NormalTok{n\_sim }\OtherTok{\textless{}{-}} \DecValTok{1000}

\CommentTok{\# Distribuições dos parâmetros (baseadas em IC)}
\NormalTok{beta\_sim }\OtherTok{\textless{}{-}} \FunctionTok{rnorm}\NormalTok{(n\_sim, }\AttributeTok{mean =} \DecValTok{2}\NormalTok{, }\AttributeTok{sd =} \FloatTok{0.2}\NormalTok{)}
\NormalTok{eta\_sim }\OtherTok{\textless{}{-}} \FunctionTok{rnorm}\NormalTok{(n\_sim, }\AttributeTok{mean =} \DecValTok{1000}\NormalTok{, }\AttributeTok{sd =} \DecValTok{100}\NormalTok{)}

\CommentTok{\# Calcular MTTF para cada combinação}
\NormalTok{MTTF\_sim }\OtherTok{\textless{}{-}}\NormalTok{ eta\_sim }\SpecialCharTok{*} \FunctionTok{gamma}\NormalTok{(}\DecValTok{1} \SpecialCharTok{+} \DecValTok{1}\SpecialCharTok{/}\NormalTok{beta\_sim)}

\CommentTok{\# Visualizar incerteza}
\FunctionTok{data.frame}\NormalTok{(}\AttributeTok{MTTF =}\NormalTok{ MTTF\_sim) }\SpecialCharTok{\%\textgreater{}\%}
  \FunctionTok{ggplot}\NormalTok{(}\FunctionTok{aes}\NormalTok{(}\AttributeTok{x =}\NormalTok{ MTTF)) }\SpecialCharTok{+}
  \FunctionTok{geom\_histogram}\NormalTok{(}\AttributeTok{bins =} \DecValTok{50}\NormalTok{, }\AttributeTok{fill =} \StringTok{"darkgreen"}\NormalTok{, }\AttributeTok{alpha =} \FloatTok{0.7}\NormalTok{) }\SpecialCharTok{+}
  \FunctionTok{geom\_vline}\NormalTok{(}\AttributeTok{xintercept =} \FunctionTok{median}\NormalTok{(MTTF\_sim),}
             \AttributeTok{color =} \StringTok{"red"}\NormalTok{, }\AttributeTok{linetype =} \StringTok{"dashed"}\NormalTok{, }\AttributeTok{linewidth =} \DecValTok{1}\NormalTok{) }\SpecialCharTok{+}
  \FunctionTok{annotate}\NormalTok{(}\StringTok{"text"}\NormalTok{, }\AttributeTok{x =} \FunctionTok{median}\NormalTok{(MTTF\_sim) }\SpecialCharTok{+} \DecValTok{50}\NormalTok{, }\AttributeTok{y =} \DecValTok{50}\NormalTok{,}
           \AttributeTok{label =} \FunctionTok{paste}\NormalTok{(}\StringTok{"Mediana:"}\NormalTok{, }\FunctionTok{round}\NormalTok{(}\FunctionTok{median}\NormalTok{(MTTF\_sim), }\DecValTok{1}\NormalTok{)),}
           \AttributeTok{color =} \StringTok{"red"}\NormalTok{) }\SpecialCharTok{+}
  \FunctionTok{labs}\NormalTok{(}\AttributeTok{title =} \StringTok{"Incerteza no MTTF"}\NormalTok{,}
       \AttributeTok{x =} \StringTok{"MTTF (h)"}\NormalTok{, }\AttributeTok{y =} \StringTok{"Frequência"}\NormalTok{) }\SpecialCharTok{+}
  \FunctionTok{theme\_minimal}\NormalTok{()}
\end{Highlighting}
\end{Shaded}

\pandocbounded{\includegraphics[keepaspectratio]{chapters/13-simulacao_files/figure-pdf/incerteza-parametros-1.pdf}}

\begin{Shaded}
\begin{Highlighting}[]
\CommentTok{\# IC 95\%}
\NormalTok{ic }\OtherTok{\textless{}{-}} \FunctionTok{quantile}\NormalTok{(MTTF\_sim, }\FunctionTok{c}\NormalTok{(}\FloatTok{0.025}\NormalTok{, }\FloatTok{0.975}\NormalTok{))}
\FunctionTok{cat}\NormalTok{(}\StringTok{"IC 95\% para MTTF:"}\NormalTok{, }\FunctionTok{round}\NormalTok{(ic[}\DecValTok{1}\NormalTok{], }\DecValTok{1}\NormalTok{), }\StringTok{"{-}"}\NormalTok{, }\FunctionTok{round}\NormalTok{(ic[}\DecValTok{2}\NormalTok{], }\DecValTok{1}\NormalTok{), }\StringTok{"}\SpecialCharTok{\textbackslash{}n}\StringTok{"}\NormalTok{)}
\end{Highlighting}
\end{Shaded}

\begin{verbatim}
IC 95% para MTTF: 718.1 - 1063.7 
\end{verbatim}

\section{Exercícios}\label{exercuxedcios-9}

📝 \textbf{Exercício 1}: Simule sistema paralelo (redundância).

📝 \textbf{Exercício 2}: Compare políticas age-based vs.~block por
simulação.

📝 \textbf{Exercício 3}: Analise impacto da incerteza paramétrica na
decisão de manutenção.

\section{Resumo}\label{resumo-10}

🔔 \textbf{Pontos-chave:}

\begin{itemize}
\tightlist
\item
  Monte Carlo é poderoso para sistemas complexos
\item
  Permite avaliar políticas sem fórmulas analíticas
\item
  Quantifica incerteza em decisões
\item
  Adequado para otimização de parâmetros
\end{itemize}

\begin{center}\rule{0.5\linewidth}{0.5pt}\end{center}

\textbf{Próximo capítulo:} Estudos de Caso Industriais

\chapter{Estudos de Caso Industriais}\label{estudos-de-caso-industriais}

\section{Introdução}\label{introduuxe7uxe3o-11}

Este capítulo apresenta estudos de caso completos aplicando as técnicas
de confiabilidade e manutenção em contextos industriais reais.

\section{Caso 1: Bombas Centrífugas em Planta
Química}\label{caso-1-bombas-centruxedfugas-em-planta-quuxedmica}

\subsection{Contexto}\label{contexto}

Planta química com 50 bombas centrífugas críticas. Histórico de 3 anos
de falhas disponível.

\subsection{Objetivos}\label{objetivos-1}

\begin{enumerate}
\def\labelenumi{\arabic{enumi}.}
\tightlist
\item
  Modelar distribuição de tempos de falha
\item
  Estimar confiabilidade
\item
  Otimizar política de manutenção preventiva
\end{enumerate}

\subsection{Análise de Dados}\label{anuxe1lise-de-dados}

\begin{Shaded}
\begin{Highlighting}[]
\FunctionTok{library}\NormalTok{(tidyverse)}
\FunctionTok{library}\NormalTok{(survival)}
\FunctionTok{library}\NormalTok{(flexsurv)}

\CommentTok{\# Carregar dados}
\NormalTok{bombas }\OtherTok{\textless{}{-}} \FunctionTok{read\_csv}\NormalTok{(}\StringTok{"resources/data/falhas\_bombas.csv"}\NormalTok{)}

\CommentTok{\# Estrutura esperada:}
\CommentTok{\# bomba\_id, tempo\_operacao, status (1=falha, 0=censura), tipo, fluido, temperatura}

\FunctionTok{head}\NormalTok{(bombas)}

\CommentTok{\# Estatísticas descritivas}
\NormalTok{bombas }\SpecialCharTok{\%\textgreater{}\%}
  \FunctionTok{group\_by}\NormalTok{(tipo) }\SpecialCharTok{\%\textgreater{}\%}
  \FunctionTok{summarise}\NormalTok{(}
    \AttributeTok{n =} \FunctionTok{n}\NormalTok{(),}
    \AttributeTok{tempo\_medio =} \FunctionTok{mean}\NormalTok{(tempo\_operacao),}
    \AttributeTok{falhas =} \FunctionTok{sum}\NormalTok{(status),}
    \AttributeTok{taxa\_falha =} \FunctionTok{sum}\NormalTok{(status) }\SpecialCharTok{/} \FunctionTok{n}\NormalTok{()}
\NormalTok{  )}
\end{Highlighting}
\end{Shaded}

\subsection{Modelagem Weibull}\label{modelagem-weibull}

\begin{Shaded}
\begin{Highlighting}[]
\CommentTok{\# Ajustar Weibull por tipo de bomba}
\NormalTok{fit\_tipo\_A }\OtherTok{\textless{}{-}} \FunctionTok{flexsurvreg}\NormalTok{(}\FunctionTok{Surv}\NormalTok{(tempo\_operacao, status) }\SpecialCharTok{\textasciitilde{}} \DecValTok{1}\NormalTok{,}
                          \AttributeTok{data =} \FunctionTok{filter}\NormalTok{(bombas, tipo }\SpecialCharTok{==} \StringTok{"A"}\NormalTok{),}
                          \AttributeTok{dist =} \StringTok{"weibull"}\NormalTok{)}

\NormalTok{fit\_tipo\_B }\OtherTok{\textless{}{-}} \FunctionTok{flexsurvreg}\NormalTok{(}\FunctionTok{Surv}\NormalTok{(tempo\_operacao, status) }\SpecialCharTok{\textasciitilde{}} \DecValTok{1}\NormalTok{,}
                          \AttributeTok{data =} \FunctionTok{filter}\NormalTok{(bombas, tipo }\SpecialCharTok{==} \StringTok{"B"}\NormalTok{),}
                          \AttributeTok{dist =} \StringTok{"weibull"}\NormalTok{)}

\CommentTok{\# Comparar parâmetros}
\FunctionTok{tibble}\NormalTok{(}
  \AttributeTok{Tipo =} \FunctionTok{c}\NormalTok{(}\StringTok{"A"}\NormalTok{, }\StringTok{"B"}\NormalTok{),}
  \AttributeTok{Beta =} \FunctionTok{c}\NormalTok{(fit\_tipo\_A}\SpecialCharTok{$}\NormalTok{res[}\StringTok{"shape"}\NormalTok{, }\StringTok{"est"}\NormalTok{],}
\NormalTok{           fit\_tipo\_B}\SpecialCharTok{$}\NormalTok{res[}\StringTok{"shape"}\NormalTok{, }\StringTok{"est"}\NormalTok{]),}
  \AttributeTok{Eta =} \FunctionTok{c}\NormalTok{(fit\_tipo\_A}\SpecialCharTok{$}\NormalTok{res[}\StringTok{"scale"}\NormalTok{, }\StringTok{"est"}\NormalTok{],}
\NormalTok{          fit\_tipo\_B}\SpecialCharTok{$}\NormalTok{res[}\StringTok{"scale"}\NormalTok{, }\StringTok{"est"}\NormalTok{]),}
  \AttributeTok{MTTF =} \FunctionTok{c}\NormalTok{(fit\_tipo\_A}\SpecialCharTok{$}\NormalTok{res[}\StringTok{"scale"}\NormalTok{, }\StringTok{"est"}\NormalTok{] }\SpecialCharTok{*} \FunctionTok{gamma}\NormalTok{(}\DecValTok{1} \SpecialCharTok{+} \DecValTok{1}\SpecialCharTok{/}\NormalTok{fit\_tipo\_A}\SpecialCharTok{$}\NormalTok{res[}\StringTok{"shape"}\NormalTok{, }\StringTok{"est"}\NormalTok{]),}
\NormalTok{           fit\_tipo\_B}\SpecialCharTok{$}\NormalTok{res[}\StringTok{"scale"}\NormalTok{, }\StringTok{"est"}\NormalTok{] }\SpecialCharTok{*} \FunctionTok{gamma}\NormalTok{(}\DecValTok{1} \SpecialCharTok{+} \DecValTok{1}\SpecialCharTok{/}\NormalTok{fit\_tipo\_B}\SpecialCharTok{$}\NormalTok{res[}\StringTok{"shape"}\NormalTok{, }\StringTok{"est"}\NormalTok{]))}
\NormalTok{)}
\end{Highlighting}
\end{Shaded}

\subsection{Curvas de Confiabilidade}\label{curvas-de-confiabilidade}

\begin{Shaded}
\begin{Highlighting}[]
\CommentTok{\# Visualizar}
\NormalTok{t }\OtherTok{\textless{}{-}} \FunctionTok{seq}\NormalTok{(}\DecValTok{0}\NormalTok{, }\DecValTok{3000}\NormalTok{, }\AttributeTok{by =} \DecValTok{50}\NormalTok{)}

\NormalTok{df\_conf }\OtherTok{\textless{}{-}} \FunctionTok{data.frame}\NormalTok{(}
  \AttributeTok{tempo =} \FunctionTok{rep}\NormalTok{(t, }\DecValTok{2}\NormalTok{),}
  \AttributeTok{tipo =} \FunctionTok{rep}\NormalTok{(}\FunctionTok{c}\NormalTok{(}\StringTok{"A"}\NormalTok{, }\StringTok{"B"}\NormalTok{), }\AttributeTok{each =} \FunctionTok{length}\NormalTok{(t)),}
  \AttributeTok{confiabilidade =} \FunctionTok{c}\NormalTok{(}
    \FunctionTok{exp}\NormalTok{(}\SpecialCharTok{{-}}\NormalTok{(t}\SpecialCharTok{/}\NormalTok{fit\_tipo\_A}\SpecialCharTok{$}\NormalTok{res[}\StringTok{"scale"}\NormalTok{, }\StringTok{"est"}\NormalTok{])}\SpecialCharTok{\^{}}\NormalTok{fit\_tipo\_A}\SpecialCharTok{$}\NormalTok{res[}\StringTok{"shape"}\NormalTok{, }\StringTok{"est"}\NormalTok{]),}
    \FunctionTok{exp}\NormalTok{(}\SpecialCharTok{{-}}\NormalTok{(t}\SpecialCharTok{/}\NormalTok{fit\_tipo\_B}\SpecialCharTok{$}\NormalTok{res[}\StringTok{"scale"}\NormalTok{, }\StringTok{"est"}\NormalTok{])}\SpecialCharTok{\^{}}\NormalTok{fit\_tipo\_B}\SpecialCharTok{$}\NormalTok{res[}\StringTok{"shape"}\NormalTok{, }\StringTok{"est"}\NormalTok{])}
\NormalTok{  )}
\NormalTok{)}

\FunctionTok{ggplot}\NormalTok{(df\_conf, }\FunctionTok{aes}\NormalTok{(}\AttributeTok{x =}\NormalTok{ tempo, }\AttributeTok{y =}\NormalTok{ confiabilidade, }\AttributeTok{color =}\NormalTok{ tipo)) }\SpecialCharTok{+}
  \FunctionTok{geom\_line}\NormalTok{(}\AttributeTok{linewidth =} \FloatTok{1.2}\NormalTok{) }\SpecialCharTok{+}
  \FunctionTok{geom\_hline}\NormalTok{(}\AttributeTok{yintercept =} \FloatTok{0.9}\NormalTok{, }\AttributeTok{linetype =} \StringTok{"dashed"}\NormalTok{) }\SpecialCharTok{+}
  \FunctionTok{labs}\NormalTok{(}\AttributeTok{title =} \StringTok{"Confiabilidade por Tipo de Bomba"}\NormalTok{,}
       \AttributeTok{x =} \StringTok{"Tempo de Operação (h)"}\NormalTok{, }\AttributeTok{y =} \StringTok{"R(t)"}\NormalTok{,}
       \AttributeTok{color =} \StringTok{"Tipo"}\NormalTok{) }\SpecialCharTok{+}
  \FunctionTok{theme\_minimal}\NormalTok{()}
\end{Highlighting}
\end{Shaded}

\subsection{Otimização de
Manutenção}\label{otimizauxe7uxe3o-de-manutenuxe7uxe3o}

\begin{Shaded}
\begin{Highlighting}[]
\CommentTok{\# Custos estimados}
\NormalTok{C\_preventiva }\OtherTok{\textless{}{-}} \DecValTok{2000}
\NormalTok{C\_falha }\OtherTok{\textless{}{-}} \DecValTok{15000}

\CommentTok{\# Função custo para Tipo A}
\NormalTok{beta\_A }\OtherTok{\textless{}{-}}\NormalTok{ fit\_tipo\_A}\SpecialCharTok{$}\NormalTok{res[}\StringTok{"shape"}\NormalTok{, }\StringTok{"est"}\NormalTok{]}
\NormalTok{eta\_A }\OtherTok{\textless{}{-}}\NormalTok{ fit\_tipo\_A}\SpecialCharTok{$}\NormalTok{res[}\StringTok{"scale"}\NormalTok{, }\StringTok{"est"}\NormalTok{]}

\NormalTok{custo\_tipo\_A }\OtherTok{\textless{}{-}} \ControlFlowTok{function}\NormalTok{(Tp) \{}
\NormalTok{  R\_Tp }\OtherTok{\textless{}{-}} \FunctionTok{exp}\NormalTok{(}\SpecialCharTok{{-}}\NormalTok{(Tp}\SpecialCharTok{/}\NormalTok{eta\_A)}\SpecialCharTok{\^{}}\NormalTok{beta\_A)}
\NormalTok{  integral }\OtherTok{\textless{}{-}} \FunctionTok{integrate}\NormalTok{(}\ControlFlowTok{function}\NormalTok{(t) }\FunctionTok{exp}\NormalTok{(}\SpecialCharTok{{-}}\NormalTok{(t}\SpecialCharTok{/}\NormalTok{eta\_A)}\SpecialCharTok{\^{}}\NormalTok{beta\_A), }\DecValTok{0}\NormalTok{, Tp)}\SpecialCharTok{$}\NormalTok{value}
\NormalTok{  (C\_preventiva }\SpecialCharTok{*}\NormalTok{ R\_Tp }\SpecialCharTok{+}\NormalTok{ C\_falha }\SpecialCharTok{*}\NormalTok{ (}\DecValTok{1} \SpecialCharTok{{-}}\NormalTok{ R\_Tp)) }\SpecialCharTok{/}\NormalTok{ integral}
\NormalTok{\}}

\CommentTok{\# Otimizar}
\NormalTok{intervalos }\OtherTok{\textless{}{-}} \FunctionTok{seq}\NormalTok{(}\DecValTok{500}\NormalTok{, }\DecValTok{2500}\NormalTok{, }\AttributeTok{by =} \DecValTok{50}\NormalTok{)}
\NormalTok{custos }\OtherTok{\textless{}{-}} \FunctionTok{sapply}\NormalTok{(intervalos, custo\_tipo\_A)}

\NormalTok{intervalo\_otimo }\OtherTok{\textless{}{-}}\NormalTok{ intervalos[}\FunctionTok{which.min}\NormalTok{(custos)]}

\FunctionTok{cat}\NormalTok{(}\StringTok{"Intervalo ótimo Tipo A:"}\NormalTok{, intervalo\_otimo, }\StringTok{"horas}\SpecialCharTok{\textbackslash{}n}\StringTok{"}\NormalTok{)}
\end{Highlighting}
\end{Shaded}

\subsection{Resultados e
Recomendações}\label{resultados-e-recomendauxe7uxf5es}

\textbf{Para Tipo A:} - β = 2.3 → desgaste progressivo - Intervalo
ótimo: 1200 horas - Economia estimada: 30\% vs.~política atual

\textbf{Para Tipo B:} - β = 1.1 → falhas quase aleatórias - Manutenção
baseada em condição recomendada

\begin{center}\rule{0.5\linewidth}{0.5pt}\end{center}

\section{Caso 2: Motores Elétricos em
Mineração}\label{caso-2-motores-eluxe9tricos-em-minerauxe7uxe3o}

\subsection{Contexto}\label{contexto-1}

Frota de 80 motores elétricos de alta potência. Dados de sensores
disponíveis.

\subsection{Abordagem}\label{abordagem}

\begin{enumerate}
\def\labelenumi{\arabic{enumi}.}
\tightlist
\item
  Análise de sobrevivência tradicional
\item
  Manutenção preditiva com sensores
\item
  Comparação de estratégias
\end{enumerate}

\subsection{Análise com
Covariáveis}\label{anuxe1lise-com-covariuxe1veis}

\begin{Shaded}
\begin{Highlighting}[]
\NormalTok{motores }\OtherTok{\textless{}{-}} \FunctionTok{read\_csv}\NormalTok{(}\StringTok{"resources/data/historico\_motores.rds"}\NormalTok{)}

\CommentTok{\# Modelo Cox}
\NormalTok{cox\_model }\OtherTok{\textless{}{-}} \FunctionTok{coxph}\NormalTok{(}\FunctionTok{Surv}\NormalTok{(tempo\_operacao, falha) }\SpecialCharTok{\textasciitilde{}}
\NormalTok{                   potencia }\SpecialCharTok{+}\NormalTok{ temperatura\_media }\SpecialCharTok{+}
\NormalTok{                   vibracao\_rms }\SpecialCharTok{+}\NormalTok{ carga\_media,}
                   \AttributeTok{data =}\NormalTok{ motores)}

\FunctionTok{summary}\NormalTok{(cox\_model)}
\end{Highlighting}
\end{Shaded}

\subsection{Predição com Dados de
Sensores}\label{prediuxe7uxe3o-com-dados-de-sensores}

\begin{Shaded}
\begin{Highlighting}[]
\FunctionTok{library}\NormalTok{(caret)}

\CommentTok{\# Features extraídas}
\NormalTok{features }\OtherTok{\textless{}{-}}\NormalTok{ motores }\SpecialCharTok{\%\textgreater{}\%}
  \FunctionTok{group\_by}\NormalTok{(motor\_id) }\SpecialCharTok{\%\textgreater{}\%}
  \FunctionTok{summarise}\NormalTok{(}
    \AttributeTok{temp\_max =} \FunctionTok{max}\NormalTok{(temperatura),}
    \AttributeTok{temp\_std =} \FunctionTok{sd}\NormalTok{(temperatura),}
    \AttributeTok{vibr\_trend =} \FunctionTok{cor}\NormalTok{(}\FunctionTok{seq\_along}\NormalTok{(vibracao), vibracao),}
    \AttributeTok{corrente\_p95 =} \FunctionTok{quantile}\NormalTok{(corrente, }\FloatTok{0.95}\NormalTok{),}
    \AttributeTok{falha\_30d =} \FunctionTok{any}\NormalTok{(falha }\SpecialCharTok{\&}\NormalTok{ tempo\_ate\_falha }\SpecialCharTok{\textless{}} \DecValTok{30}\NormalTok{)}
\NormalTok{  )}

\CommentTok{\# Modelo Random Forest}
\NormalTok{modelo }\OtherTok{\textless{}{-}} \FunctionTok{train}\NormalTok{(falha\_30d }\SpecialCharTok{\textasciitilde{}}\NormalTok{ .,}
                \AttributeTok{data =}\NormalTok{ features,}
                \AttributeTok{method =} \StringTok{"rf"}\NormalTok{,}
                \AttributeTok{trControl =} \FunctionTok{trainControl}\NormalTok{(}\AttributeTok{method =} \StringTok{"cv"}\NormalTok{, }\AttributeTok{number =} \DecValTok{5}\NormalTok{))}

\FunctionTok{varImp}\NormalTok{(modelo)}
\end{Highlighting}
\end{Shaded}

\subsection{Impacto Econômico}\label{impacto-econuxf4mico}

\begin{longtable}[]{@{}llll@{}}
\toprule\noalign{}
Estratégia & Custo Anual & Disponibilidade & Economia vs.~Baseline \\
\midrule\noalign{}
\endhead
\bottomrule\noalign{}
\endlastfoot
Corretiva & R\$ 450.000 & 87\% & - \\
Preventiva (fixa) & R\$ 320.000 & 94\% & 29\% \\
Preditiva (sensores) & R\$ 280.000 & 97\% & 38\% \\
\end{longtable}

\begin{center}\rule{0.5\linewidth}{0.5pt}\end{center}

\section{Caso 3: Compressores em Indústria de
Gás}\label{caso-3-compressores-em-induxfastria-de-guxe1s}

\subsection{Contexto}\label{contexto-2}

Compressores críticos com alto custo de parada. Implementação de gêmeo
digital.

\subsection{Metodologia}\label{metodologia}

\begin{Shaded}
\begin{Highlighting}[]
\CommentTok{\# Simulação de gêmeo digital}
\NormalTok{simular\_degradacao }\OtherTok{\textless{}{-}} \ControlFlowTok{function}\NormalTok{(tempo\_max, params) \{}
  \CommentTok{\# Modelo físico + ML}
  \CommentTok{\# Predizer RUL (Remaining Useful Life)}
\NormalTok{\}}

\CommentTok{\# Integração com IoT}
\NormalTok{rul\_atual }\OtherTok{\textless{}{-}} \FunctionTok{predizer\_rul}\NormalTok{(dados\_tempo\_real)}

\ControlFlowTok{if}\NormalTok{ (rul\_atual }\SpecialCharTok{\textless{}}\NormalTok{ threshold\_critico) \{}
  \FunctionTok{enviar\_alerta}\NormalTok{()}
  \FunctionTok{agendar\_manutencao}\NormalTok{()}
\NormalTok{\}}
\end{Highlighting}
\end{Shaded}

\subsection{Resultados}\label{resultados}

\begin{itemize}
\tightlist
\item
  Redução de 45\% em paradas não programadas
\item
  Aumento de 12\% na vida útil média
\item
  ROI positivo em 18 meses
\end{itemize}

\begin{center}\rule{0.5\linewidth}{0.5pt}\end{center}

\section{Caso 4: Transformadores em
Subestação}\label{caso-4-transformadores-em-subestauxe7uxe3o}

\subsection{Desafio}\label{desafio}

Componentes de alta criticidade, falhas catastróficas.

\subsection{Solução}\label{soluuxe7uxe3o}

\begin{itemize}
\tightlist
\item
  Análise de óleo (DGA)
\item
  Termografia
\item
  Monitoramento contínuo
\item
  Modelo de degradação
\end{itemize}

\begin{Shaded}
\begin{Highlighting}[]
\CommentTok{\# Modelo de degradação baseado em DGA}
\NormalTok{modelo\_degradacao }\OtherTok{\textless{}{-}} \FunctionTok{lm}\NormalTok{(indice\_saude }\SpecialCharTok{\textasciitilde{}} \FunctionTok{log}\NormalTok{(h2) }\SpecialCharTok{+} \FunctionTok{log}\NormalTok{(ch4) }\SpecialCharTok{+}
                        \FunctionTok{log}\NormalTok{(c2h2) }\SpecialCharTok{+}\NormalTok{ temperatura,}
                        \AttributeTok{data =}\NormalTok{ dados\_dga)}

\CommentTok{\# Predição de vida residual}
\NormalTok{vida\_residual }\OtherTok{\textless{}{-}} \FunctionTok{estimar\_vida\_residual}\NormalTok{(modelo\_degradacao,}
\NormalTok{                                       novos\_dados)}
\end{Highlighting}
\end{Shaded}

\begin{center}\rule{0.5\linewidth}{0.5pt}\end{center}

\section{Lições Aprendidas}\label{liuxe7uxf5es-aprendidas}

\subsection{Fatores Críticos de
Sucesso}\label{fatores-cruxedticos-de-sucesso}

\begin{enumerate}
\def\labelenumi{\arabic{enumi}.}
\tightlist
\item
  \textbf{Qualidade dos dados} --- Fundamental para qualquer análise
\item
  \textbf{Envolvimento operacional} --- Equipes de campo devem validar
  modelos
\item
  \textbf{Integração de sistemas} --- CMMS + IoT + Analytics
\item
  \textbf{Capacitação} --- Treinar equipes em interpretação de modelos
\end{enumerate}

\subsection{Armadilhas Comuns}\label{armadilhas-comuns}

\begin{itemize}
\tightlist
\item
  Overfitting em modelos preditivos
\item
  Ignorar contexto operacional
\item
  Subestimar custos de implementação
\item
  Falta de validação contínua
\end{itemize}

\begin{center}\rule{0.5\linewidth}{0.5pt}\end{center}

\section{Exercícios}\label{exercuxedcios-10}

📝 \textbf{Exercício 1}: Aplique análise Weibull aos seus dados reais.

📝 \textbf{Exercício 2}: Desenvolva análise custo-benefício completa.

📝 \textbf{Exercício 3}: Crie dashboard de monitoramento com Shiny.

\section{Resumo}\label{resumo-11}

🔔 \textbf{Pontos-chave:}

\begin{itemize}
\tightlist
\item
  Estudos de caso demonstram aplicabilidade real
\item
  Análise deve integrar estatística + contexto operacional
\item
  ROI mensurável justifica investimentos
\item
  Manutenção preditiva oferece melhor custo-benefício
\end{itemize}

\begin{center}\rule{0.5\linewidth}{0.5pt}\end{center}

\textbf{Próximo capítulo:} Boas Práticas e Governança de Dados

\chapter{Boas Práticas e Governança de Dados de
Manutenção}\label{boas-pruxe1ticas-e-governanuxe7a-de-dados-de-manutenuxe7uxe3o}

\section{Introdução}\label{introduuxe7uxe3o-12}

Estratégias para gestão eficaz de dados de manutenção, qualidade,
versionamento e integração.

\section{Arquitetura de Dados de
Manutenção}\label{arquitetura-de-dados-de-manutenuxe7uxe3o}

\subsection{Camadas de Dados}\label{camadas-de-dados}

\begin{verbatim}
┌─────────────────────────────────────┐
│     Camada de Visualização          │  ← Dashboards, Relatórios
├─────────────────────────────────────┤
│     Camada de Analytics             │  ← Modelos ML, Estatística
├─────────────────────────────────────┤
│     Camada de Integração (ETL)      │  ← Limpeza, Transformação
├─────────────────────────────────────┤
│     Camada de Armazenamento         │  ← Data Lake, Data Warehouse
├─────────────────────────────────────┤
│     Camada de Coleta                │  ← CMMS, IoT, Sensores
└─────────────────────────────────────┘
\end{verbatim}

\subsection{Fontes de Dados}\label{fontes-de-dados}

\begin{longtable}[]{@{}llll@{}}
\toprule\noalign{}
Fonte & Tipo & Frequência & Exemplos \\
\midrule\noalign{}
\endhead
\bottomrule\noalign{}
\endlastfoot
\textbf{CMMS} & Estruturado & Contínua & Ordens de serviço, histórico \\
\textbf{Sensores IoT} & Time-series & Tempo real & Temperatura,
vibração \\
\textbf{Inspeções} & Semi-estruturado & Periódica & Checklists, fotos \\
\textbf{Manuais} & Não estruturado & Estática & PDFs, especificações \\
\end{longtable}

\section{Qualidade de Dados}\label{qualidade-de-dados}

\subsection{Dimensões de Qualidade}\label{dimensuxf5es-de-qualidade}

\begin{enumerate}
\def\labelenumi{\arabic{enumi}.}
\tightlist
\item
  \textbf{Completude} --- Dados ausentes
\item
  \textbf{Acurácia} --- Dados corretos
\item
  \textbf{Consistência} --- Dados coerentes entre sistemas
\item
  \textbf{Temporalidade} --- Dados atualizados
\item
  \textbf{Unicidade} --- Sem duplicação
\end{enumerate}

\subsection{Verificação em R}\label{verificauxe7uxe3o-em-r}

\begin{Shaded}
\begin{Highlighting}[]
\FunctionTok{library}\NormalTok{(tidyverse)}
\FunctionTok{library}\NormalTok{(naniar)}

\CommentTok{\# Carregar dados}
\NormalTok{manutencao }\OtherTok{\textless{}{-}} \FunctionTok{read\_csv}\NormalTok{(}\StringTok{"data/ordens\_servico.csv"}\NormalTok{)}

\CommentTok{\# Verificar missing values}
\FunctionTok{vis\_miss}\NormalTok{(manutencao)}

\CommentTok{\# Estatísticas de completude}
\NormalTok{miss\_summary }\OtherTok{\textless{}{-}} \FunctionTok{miss\_var\_summary}\NormalTok{(manutencao)}
\FunctionTok{print}\NormalTok{(miss\_summary)}

\CommentTok{\# Detectar duplicatas}
\NormalTok{duplicatas }\OtherTok{\textless{}{-}}\NormalTok{ manutencao }\SpecialCharTok{\%\textgreater{}\%}
  \FunctionTok{group\_by}\NormalTok{(ordem\_id) }\SpecialCharTok{\%\textgreater{}\%}
  \FunctionTok{filter}\NormalTok{(}\FunctionTok{n}\NormalTok{() }\SpecialCharTok{\textgreater{}} \DecValTok{1}\NormalTok{)}

\FunctionTok{cat}\NormalTok{(}\StringTok{"Registros duplicados:"}\NormalTok{, }\FunctionTok{nrow}\NormalTok{(duplicatas), }\StringTok{"}\SpecialCharTok{\textbackslash{}n}\StringTok{"}\NormalTok{)}
\end{Highlighting}
\end{Shaded}

\subsection{Limpeza de Dados}\label{limpeza-de-dados}

\begin{Shaded}
\begin{Highlighting}[]
\CommentTok{\# Pipeline de limpeza}
\NormalTok{manutencao\_clean }\OtherTok{\textless{}{-}}\NormalTok{ manutencao }\SpecialCharTok{\%\textgreater{}\%}
  \CommentTok{\# Remover duplicatas}
  \FunctionTok{distinct}\NormalTok{(ordem\_id, }\AttributeTok{.keep\_all =} \ConstantTok{TRUE}\NormalTok{) }\SpecialCharTok{\%\textgreater{}\%}
  \CommentTok{\# Tratar missings}
  \FunctionTok{mutate}\NormalTok{(}
    \AttributeTok{tempo\_reparo =} \FunctionTok{ifelse}\NormalTok{(}\FunctionTok{is.na}\NormalTok{(tempo\_reparo),}
                         \FunctionTok{median}\NormalTok{(tempo\_reparo, }\AttributeTok{na.rm =} \ConstantTok{TRUE}\NormalTok{),}
\NormalTok{                         tempo\_reparo)}
\NormalTok{  ) }\SpecialCharTok{\%\textgreater{}\%}
  \CommentTok{\# Validar datas}
  \FunctionTok{filter}\NormalTok{(data\_fim }\SpecialCharTok{\textgreater{}=}\NormalTok{ data\_inicio) }\SpecialCharTok{\%\textgreater{}\%}
  \CommentTok{\# Padronizar categorias}
  \FunctionTok{mutate}\NormalTok{(}
    \AttributeTok{tipo\_manutencao =} \FunctionTok{str\_to\_upper}\NormalTok{(tipo\_manutencao),}
    \AttributeTok{tipo\_manutencao =} \FunctionTok{case\_when}\NormalTok{(}
\NormalTok{      tipo\_manutencao }\SpecialCharTok{\%in\%} \FunctionTok{c}\NormalTok{(}\StringTok{"PREV"}\NormalTok{, }\StringTok{"PREVENTIVA"}\NormalTok{) }\SpecialCharTok{\textasciitilde{}} \StringTok{"PREVENTIVA"}\NormalTok{,}
\NormalTok{      tipo\_manutencao }\SpecialCharTok{\%in\%} \FunctionTok{c}\NormalTok{(}\StringTok{"CORR"}\NormalTok{, }\StringTok{"CORRETIVA"}\NormalTok{) }\SpecialCharTok{\textasciitilde{}} \StringTok{"CORRETIVA"}\NormalTok{,}
      \ConstantTok{TRUE} \SpecialCharTok{\textasciitilde{}} \StringTok{"OUTRAS"}
\NormalTok{    )}
\NormalTok{  )}
\end{Highlighting}
\end{Shaded}

\section{Integração de Dados}\label{integrauxe7uxe3o-de-dados}

\subsection{ETL com R}\label{etl-com-r}

\begin{Shaded}
\begin{Highlighting}[]
\FunctionTok{library}\NormalTok{(DBI)}
\FunctionTok{library}\NormalTok{(RSQLite)}

\CommentTok{\# Conexão com banco}
\NormalTok{con }\OtherTok{\textless{}{-}} \FunctionTok{dbConnect}\NormalTok{(}\FunctionTok{SQLite}\NormalTok{(), }\StringTok{"data/manutencao.db"}\NormalTok{)}

\CommentTok{\# Extract {-} Ler de múltiplas fontes}
\NormalTok{cmms\_data }\OtherTok{\textless{}{-}} \FunctionTok{read\_csv}\NormalTok{(}\StringTok{"sources/cmms\_export.csv"}\NormalTok{)}
\NormalTok{sensor\_data }\OtherTok{\textless{}{-}} \FunctionTok{read\_csv}\NormalTok{(}\StringTok{"sources/sensores.csv"}\NormalTok{)}
\NormalTok{inspection\_data }\OtherTok{\textless{}{-}} \FunctionTok{read\_excel}\NormalTok{(}\StringTok{"sources/inspecoes.xlsx"}\NormalTok{)}

\CommentTok{\# Transform {-} Limpar e integrar}
\NormalTok{dados\_integrados }\OtherTok{\textless{}{-}}\NormalTok{ cmms\_data }\SpecialCharTok{\%\textgreater{}\%}
  \FunctionTok{left\_join}\NormalTok{(sensor\_data, }\AttributeTok{by =} \FunctionTok{c}\NormalTok{(}\StringTok{"equipamento\_id"}\NormalTok{, }\StringTok{"data"}\NormalTok{)) }\SpecialCharTok{\%\textgreater{}\%}
  \FunctionTok{left\_join}\NormalTok{(inspection\_data, }\AttributeTok{by =} \StringTok{"equipamento\_id"}\NormalTok{) }\SpecialCharTok{\%\textgreater{}\%}
  \CommentTok{\# Engenharia de features}
  \FunctionTok{mutate}\NormalTok{(}
    \AttributeTok{tempo\_desde\_ultima\_manutencao =} \FunctionTok{as.numeric}\NormalTok{(data }\SpecialCharTok{{-}} \FunctionTok{lag}\NormalTok{(data)),}
    \AttributeTok{custo\_acumulado =} \FunctionTok{cumsum}\NormalTok{(custo)}
\NormalTok{  )}

\CommentTok{\# Load {-} Carregar no banco}
\FunctionTok{dbWriteTable}\NormalTok{(con, }\StringTok{"manutencao\_integrada"}\NormalTok{, dados\_integrados, }\AttributeTok{overwrite =} \ConstantTok{TRUE}\NormalTok{)}

\FunctionTok{dbDisconnect}\NormalTok{(con)}
\end{Highlighting}
\end{Shaded}

\section{Versionamento de Dados}\label{versionamento-de-dados}

\subsection{DVC (Data Version Control)}\label{dvc-data-version-control}

\begin{Shaded}
\begin{Highlighting}[]
\CommentTok{\# Inicializar DVC}
\ExtensionTok{dvc}\NormalTok{ init}

\CommentTok{\# Adicionar dados ao DVC}
\ExtensionTok{dvc}\NormalTok{ add data/falhas\_historicas.csv}

\CommentTok{\# Versionar com Git}
\FunctionTok{git}\NormalTok{ add data/falhas\_historicas.csv.dvc .gitignore}
\FunctionTok{git}\NormalTok{ commit }\AttributeTok{{-}m} \StringTok{"Adiciona dados históricos v1.0"}

\CommentTok{\# Push para remote storage (S3, Azure, etc.)}
\ExtensionTok{dvc}\NormalTok{ push}
\end{Highlighting}
\end{Shaded}

\subsection{Versionamento de Modelos}\label{versionamento-de-modelos}

\begin{Shaded}
\begin{Highlighting}[]
\FunctionTok{library}\NormalTok{(pins)}

\CommentTok{\# Configurar board (local ou remoto)}
\NormalTok{board }\OtherTok{\textless{}{-}} \FunctionTok{board\_folder}\NormalTok{(}\StringTok{"models/"}\NormalTok{)}

\CommentTok{\# Salvar modelo com versão}
\FunctionTok{pin\_write}\NormalTok{(board, modelo\_rf,}
          \AttributeTok{name =} \StringTok{"modelo\_predicao\_falhas"}\NormalTok{,}
          \AttributeTok{description =} \StringTok{"Random Forest v2.1 {-} Acurácia 0.89"}\NormalTok{)}

\CommentTok{\# Carregar versão específica}
\NormalTok{modelo\_prod }\OtherTok{\textless{}{-}} \FunctionTok{pin\_read}\NormalTok{(board, }\StringTok{"modelo\_predicao\_falhas"}\NormalTok{,}
                        \AttributeTok{version =} \StringTok{"20250115T103045Z"}\NormalTok{)}
\end{Highlighting}
\end{Shaded}

\section{Pipelines Automatizados}\label{pipelines-automatizados}

\subsection{Exemplo com \{targets\}}\label{exemplo-com-targets}

\begin{Shaded}
\begin{Highlighting}[]
\FunctionTok{library}\NormalTok{(targets)}

\CommentTok{\# \_targets.R}
\FunctionTok{list}\NormalTok{(}
  \FunctionTok{tar\_target}\NormalTok{(dados\_raw, }\FunctionTok{read\_csv}\NormalTok{(}\StringTok{"data/raw/ordens.csv"}\NormalTok{)),}
  \FunctionTok{tar\_target}\NormalTok{(dados\_clean, }\FunctionTok{limpar\_dados}\NormalTok{(dados\_raw)),}
  \FunctionTok{tar\_target}\NormalTok{(features, }\FunctionTok{engenharia\_features}\NormalTok{(dados\_clean)),}
  \FunctionTok{tar\_target}\NormalTok{(modelo, }\FunctionTok{treinar\_modelo}\NormalTok{(features)),}
  \FunctionTok{tar\_target}\NormalTok{(metricas, }\FunctionTok{avaliar\_modelo}\NormalTok{(modelo, features)),}
  \FunctionTok{tar\_target}\NormalTok{(relatorio, }\FunctionTok{gerar\_relatorio}\NormalTok{(metricas))}
\NormalTok{)}

\CommentTok{\# Executar pipeline}
\FunctionTok{tar\_make}\NormalTok{()}

\CommentTok{\# Visualizar dependências}
\FunctionTok{tar\_visnetwork}\NormalTok{()}
\end{Highlighting}
\end{Shaded}

\section{Integração com CMMS}\label{integrauxe7uxe3o-com-cmms}

\subsection{API REST}\label{api-rest}

\begin{Shaded}
\begin{Highlighting}[]
\FunctionTok{library}\NormalTok{(httr)}

\CommentTok{\# Função para buscar ordens de serviço}
\NormalTok{buscar\_os }\OtherTok{\textless{}{-}} \ControlFlowTok{function}\NormalTok{(api\_url, token, data\_inicio, data\_fim) \{}
\NormalTok{  response }\OtherTok{\textless{}{-}} \FunctionTok{GET}\NormalTok{(}
    \AttributeTok{url =} \FunctionTok{paste0}\NormalTok{(api\_url, }\StringTok{"/workorders"}\NormalTok{),}
    \AttributeTok{query =} \FunctionTok{list}\NormalTok{(}
      \AttributeTok{start\_date =}\NormalTok{ data\_inicio,}
      \AttributeTok{end\_date =}\NormalTok{ data\_fim}
\NormalTok{    ),}
    \FunctionTok{add\_headers}\NormalTok{(}\AttributeTok{Authorization =} \FunctionTok{paste}\NormalTok{(}\StringTok{"Bearer"}\NormalTok{, token))}
\NormalTok{  )}

  \ControlFlowTok{if}\NormalTok{ (}\FunctionTok{status\_code}\NormalTok{(response) }\SpecialCharTok{==} \DecValTok{200}\NormalTok{) \{}
    \FunctionTok{content}\NormalTok{(response, }\AttributeTok{as =} \StringTok{"parsed"}\NormalTok{)}
\NormalTok{  \} }\ControlFlowTok{else}\NormalTok{ \{}
    \FunctionTok{stop}\NormalTok{(}\StringTok{"Erro na API: "}\NormalTok{, }\FunctionTok{status\_code}\NormalTok{(response))}
\NormalTok{  \}}
\NormalTok{\}}

\CommentTok{\# Usar}
\NormalTok{os\_data }\OtherTok{\textless{}{-}} \FunctionTok{buscar\_os}\NormalTok{(}
  \AttributeTok{api\_url =} \StringTok{"https://cmms.empresa.com/api"}\NormalTok{,}
  \AttributeTok{token =} \FunctionTok{Sys.getenv}\NormalTok{(}\StringTok{"CMMS\_TOKEN"}\NormalTok{),}
  \AttributeTok{data\_inicio =} \StringTok{"2024{-}01{-}01"}\NormalTok{,}
  \AttributeTok{data\_fim =} \StringTok{"2024{-}12{-}31"}
\NormalTok{)}
\end{Highlighting}
\end{Shaded}

\section{Integração com IoT}\label{integrauxe7uxe3o-com-iot}

\subsection{MQTT para Sensores}\label{mqtt-para-sensores}

\begin{Shaded}
\begin{Highlighting}[]
\FunctionTok{library}\NormalTok{(mqtt)}

\CommentTok{\# Conectar ao broker MQTT}
\NormalTok{client }\OtherTok{\textless{}{-}} \FunctionTok{mqtt\_connect}\NormalTok{(}\StringTok{"mqtt.empresa.com"}\NormalTok{, }\AttributeTok{port =} \DecValTok{1883}\NormalTok{)}

\CommentTok{\# Subscrever tópico}
\FunctionTok{mqtt\_subscribe}\NormalTok{(client, }\StringTok{"sensores/temperatura/\#"}\NormalTok{)}

\CommentTok{\# Callback para mensagens}
\FunctionTok{mqtt\_on\_message}\NormalTok{(client, }\ControlFlowTok{function}\NormalTok{(topic, payload) \{}
\NormalTok{  dados\_sensor }\OtherTok{\textless{}{-}}\NormalTok{ jsonlite}\SpecialCharTok{::}\FunctionTok{fromJSON}\NormalTok{(payload)}

  \CommentTok{\# Processar e armazenar}
  \FunctionTok{processar\_leitura\_sensor}\NormalTok{(dados\_sensor)}
\NormalTok{\})}
\end{Highlighting}
\end{Shaded}

\section{Segurança e Compliance}\label{seguranuxe7a-e-compliance}

\subsection{Boas Práticas}\label{boas-pruxe1ticas}

\begin{enumerate}
\def\labelenumi{\arabic{enumi}.}
\tightlist
\item
  \textbf{Autenticação e autorização}

  \begin{itemize}
  \tightlist
  \item
    OAuth2 para APIs
  \item
    RBAC (Role-Based Access Control)
  \end{itemize}
\item
  \textbf{Criptografia}

  \begin{itemize}
  \tightlist
  \item
    Dados em trânsito (TLS)
  \item
    Dados em repouso (AES-256)
  \end{itemize}
\item
  \textbf{Auditoria}

  \begin{itemize}
  \tightlist
  \item
    Logs de acesso
  \item
    Rastreabilidade de mudanças
  \end{itemize}
\item
  \textbf{Backup e recuperação}

  \begin{itemize}
  \tightlist
  \item
    Backups automáticos diários
  \item
    Testes de recuperação trimestrais
  \end{itemize}
\end{enumerate}

\subsection{Exemplo de Auditoria}\label{exemplo-de-auditoria}

\begin{Shaded}
\begin{Highlighting}[]
\NormalTok{registrar\_auditoria }\OtherTok{\textless{}{-}} \ControlFlowTok{function}\NormalTok{(usuario, acao, tabela, registro\_id) \{}
\NormalTok{  log\_entry }\OtherTok{\textless{}{-}} \FunctionTok{data.frame}\NormalTok{(}
    \AttributeTok{timestamp =} \FunctionTok{Sys.time}\NormalTok{(),}
    \AttributeTok{usuario =}\NormalTok{ usuario,}
    \AttributeTok{acao =}\NormalTok{ acao,}
    \AttributeTok{tabela =}\NormalTok{ tabela,}
    \AttributeTok{registro\_id =}\NormalTok{ registro\_id,}
    \AttributeTok{ip\_address =} \FunctionTok{Sys.getenv}\NormalTok{(}\StringTok{"REMOTE\_ADDR"}\NormalTok{)}
\NormalTok{  )}

  \CommentTok{\# Append ao log}
  \FunctionTok{write\_csv}\NormalTok{(log\_entry, }\StringTok{"logs/audit.csv"}\NormalTok{, }\AttributeTok{append =} \ConstantTok{TRUE}\NormalTok{)}
\NormalTok{\}}
\end{Highlighting}
\end{Shaded}

\section{Reprodutibilidade}\label{reprodutibilidade-1}

\subsection{Ambiente com \{renv\}}\label{ambiente-com-renv}

\begin{Shaded}
\begin{Highlighting}[]
\CommentTok{\# Inicializar renv}
\NormalTok{renv}\SpecialCharTok{::}\FunctionTok{init}\NormalTok{()}

\CommentTok{\# Salvar snapshot de pacotes}
\NormalTok{renv}\SpecialCharTok{::}\FunctionTok{snapshot}\NormalTok{()}

\CommentTok{\# Restaurar ambiente}
\NormalTok{renv}\SpecialCharTok{::}\FunctionTok{restore}\NormalTok{()}
\end{Highlighting}
\end{Shaded}

\subsection{Containerização com
Docker}\label{containerizauxe7uxe3o-com-docker}

\begin{Shaded}
\begin{Highlighting}[]
\CommentTok{\# Dockerfile}
\KeywordTok{FROM}\NormalTok{ rocker/r{-}ver:4.5.1}

\CommentTok{\# Instalar dependências do sistema}
\KeywordTok{RUN} \ExtensionTok{apt{-}get}\NormalTok{ update }\KeywordTok{\&\&} \ExtensionTok{apt{-}get}\NormalTok{ install }\AttributeTok{{-}y} \DataTypeTok{\textbackslash{}}
\NormalTok{    libcurl4{-}openssl{-}dev }\DataTypeTok{\textbackslash{}}
\NormalTok{    libssl{-}dev }\DataTypeTok{\textbackslash{}}
\NormalTok{    libxml2{-}dev}

\CommentTok{\# Copiar projeto}
\KeywordTok{COPY}\NormalTok{ . /app}
\KeywordTok{WORKDIR}\NormalTok{ /app}

\CommentTok{\# Instalar pacotes R}
\KeywordTok{RUN} \ExtensionTok{Rscript} \AttributeTok{{-}e} \StringTok{"renv::restore()"}

\CommentTok{\# Executar análise}
\KeywordTok{CMD}\NormalTok{ [}\StringTok{"Rscript"}\NormalTok{, }\StringTok{"run\_analysis.R"}\NormalTok{]}
\end{Highlighting}
\end{Shaded}

\section{Documentação}\label{documentauxe7uxe3o}

\subsection{Metadados}\label{metadados}

\begin{Shaded}
\begin{Highlighting}[]
\CommentTok{\# metadata.yml}
\FunctionTok{dataset}\KeywordTok{:}
\AttributeTok{  }\FunctionTok{name}\KeywordTok{:}\AttributeTok{ }\StringTok{"Histórico de Falhas {-} Bombas"}
\AttributeTok{  }\FunctionTok{version}\KeywordTok{:}\AttributeTok{ }\StringTok{"2.1"}
\AttributeTok{  }\FunctionTok{date}\KeywordTok{:}\AttributeTok{ }\StringTok{"2025{-}01{-}15"}
\AttributeTok{  }\FunctionTok{owner}\KeywordTok{:}\AttributeTok{ }\StringTok{"Engenharia de Confiabilidade"}
\FunctionTok{  description}\KeywordTok{: }\CharTok{|}
\NormalTok{    Dados de falhas de bombas centrífugas coletados}
\NormalTok{    entre 2021{-}2024.}

\FunctionTok{fields}\KeywordTok{:}
\AttributeTok{  }\KeywordTok{{-}}\AttributeTok{ }\FunctionTok{name}\KeywordTok{:}\AttributeTok{ bomba\_id}
\AttributeTok{    }\FunctionTok{type}\KeywordTok{:}\AttributeTok{ character}
\AttributeTok{    }\FunctionTok{description}\KeywordTok{:}\AttributeTok{ Identificador único da bomba}

\AttributeTok{  }\KeywordTok{{-}}\AttributeTok{ }\FunctionTok{name}\KeywordTok{:}\AttributeTok{ tempo\_operacao}
\AttributeTok{    }\FunctionTok{type}\KeywordTok{:}\AttributeTok{ numeric}
\AttributeTok{    }\FunctionTok{unit}\KeywordTok{:}\AttributeTok{ horas}
\AttributeTok{    }\FunctionTok{description}\KeywordTok{:}\AttributeTok{ Tempo acumulado de operação até falha ou censura}
\end{Highlighting}
\end{Shaded}

\section{Exercícios}\label{exercuxedcios-11}

📝 \textbf{Exercício 1}: Implemente pipeline ETL para seus dados.

📝 \textbf{Exercício 2}: Configure versionamento com DVC.

📝 \textbf{Exercício 3}: Crie dashboard de qualidade de dados.

\section{Resumo}\label{resumo-12}

🔔 \textbf{Pontos-chave:}

\begin{itemize}
\tightlist
\item
  Governança de dados é fundamental para análises confiáveis
\item
  ETL automatizado reduz erros e retrabalho
\item
  Versionamento garante reprodutibilidade
\item
  Integração CMMS + IoT habilita manutenção inteligente
\item
  Segurança e compliance são inegociáveis
\end{itemize}

\begin{center}\rule{0.5\linewidth}{0.5pt}\end{center}

\textbf{Próximo:} Apêndices

\part{Apêndices}

\chapter*{Apêndice A: Pacotes e Funções Úteis em
R}\label{apuxeandice-a-pacotes-e-funuxe7uxf5es-uxfateis-em-r}
\addcontentsline{toc}{chapter}{Apêndice A: Pacotes e Funções Úteis em R}

\markboth{Apêndice A: Pacotes e Funções Úteis em R}{Apêndice A: Pacotes
e Funções Úteis em R}

\section*{Pacotes Essenciais para
Confiabilidade}\label{pacotes-essenciais-para-confiabilidade}
\addcontentsline{toc}{section}{Pacotes Essenciais para Confiabilidade}

\markright{Pacotes Essenciais para Confiabilidade}

\subsection*{Manipulação de Dados}\label{manipulauxe7uxe3o-de-dados}
\addcontentsline{toc}{subsection}{Manipulação de Dados}

\subsubsection*{\texorpdfstring{\texttt{tidyverse}}{tidyverse}}\label{tidyverse}
\addcontentsline{toc}{subsubsection}{\texttt{tidyverse}}

Coleção de pacotes para ciência de dados.

\begin{Shaded}
\begin{Highlighting}[]
\FunctionTok{library}\NormalTok{(tidyverse)}

\CommentTok{\# dplyr {-} manipulação}
\NormalTok{dados }\SpecialCharTok{\%\textgreater{}\%}
  \FunctionTok{filter}\NormalTok{(status }\SpecialCharTok{==} \DecValTok{1}\NormalTok{) }\SpecialCharTok{\%\textgreater{}\%}
  \FunctionTok{mutate}\NormalTok{(}\AttributeTok{tempo\_dias =}\NormalTok{ tempo }\SpecialCharTok{/} \DecValTok{24}\NormalTok{) }\SpecialCharTok{\%\textgreater{}\%}
  \FunctionTok{group\_by}\NormalTok{(tipo) }\SpecialCharTok{\%\textgreater{}\%}
  \FunctionTok{summarise}\NormalTok{(}\AttributeTok{mtbf =} \FunctionTok{mean}\NormalTok{(tempo\_dias))}

\CommentTok{\# ggplot2 {-} visualização}
\FunctionTok{ggplot}\NormalTok{(dados, }\FunctionTok{aes}\NormalTok{(}\AttributeTok{x =}\NormalTok{ tempo, }\AttributeTok{y =}\NormalTok{ confiabilidade)) }\SpecialCharTok{+}
  \FunctionTok{geom\_line}\NormalTok{() }\SpecialCharTok{+}
  \FunctionTok{theme\_minimal}\NormalTok{()}
\end{Highlighting}
\end{Shaded}

\subsection*{Análise de
Sobrevivência}\label{anuxe1lise-de-sobrevivuxeancia}
\addcontentsline{toc}{subsection}{Análise de Sobrevivência}

\subsubsection*{\texorpdfstring{\texttt{survival}}{survival}}\label{survival}
\addcontentsline{toc}{subsubsection}{\texttt{survival}}

Pacote padrão para análise de sobrevivência.

\begin{Shaded}
\begin{Highlighting}[]
\FunctionTok{library}\NormalTok{(survival)}

\CommentTok{\# Criar objeto Surv}
\NormalTok{surv\_obj }\OtherTok{\textless{}{-}} \FunctionTok{Surv}\NormalTok{(}\AttributeTok{time =}\NormalTok{ tempo, }\AttributeTok{event =}\NormalTok{ status)}

\CommentTok{\# Kaplan{-}Meier}
\NormalTok{km\_fit }\OtherTok{\textless{}{-}} \FunctionTok{survfit}\NormalTok{(surv\_obj }\SpecialCharTok{\textasciitilde{}}\NormalTok{ grupo)}

\CommentTok{\# Modelo Cox}
\NormalTok{cox\_model }\OtherTok{\textless{}{-}} \FunctionTok{coxph}\NormalTok{(surv\_obj }\SpecialCharTok{\textasciitilde{}}\NormalTok{ idade }\SpecialCharTok{+}\NormalTok{ temperatura, }\AttributeTok{data =}\NormalTok{ dados)}
\end{Highlighting}
\end{Shaded}

\textbf{Funções principais:} - \texttt{Surv()} - Criar objeto de
sobrevivência - \texttt{survfit()} - Ajustar Kaplan-Meier -
\texttt{coxph()} - Regressão de Cox - \texttt{survdiff()} - Teste
log-rank

\subsubsection*{\texorpdfstring{\texttt{flexsurv}}{flexsurv}}\label{flexsurv}
\addcontentsline{toc}{subsubsection}{\texttt{flexsurv}}

Modelos paramétricos flexíveis.

\begin{Shaded}
\begin{Highlighting}[]
\FunctionTok{library}\NormalTok{(flexsurv)}

\CommentTok{\# Modelo Weibull AFT}
\NormalTok{fit }\OtherTok{\textless{}{-}} \FunctionTok{flexsurvreg}\NormalTok{(}\FunctionTok{Surv}\NormalTok{(tempo, status) }\SpecialCharTok{\textasciitilde{}}\NormalTok{ temperatura,}
                   \AttributeTok{dist =} \StringTok{"weibull"}\NormalTok{)}

\CommentTok{\# Distribuições disponíveis:}
\CommentTok{\# "weibull", "exp", "lnorm", "gamma", "gompertz", etc.}
\end{Highlighting}
\end{Shaded}

\subsubsection*{\texorpdfstring{\texttt{WeibullR}}{WeibullR}}\label{weibullr}
\addcontentsline{toc}{subsubsection}{\texttt{WeibullR}}

Especializado em análise Weibull.

\begin{Shaded}
\begin{Highlighting}[]
\FunctionTok{library}\NormalTok{(WeibullR)}

\CommentTok{\# Criar objeto Weibull}
\NormalTok{fit }\OtherTok{\textless{}{-}} \FunctionTok{wblr}\NormalTok{(tempos, }\AttributeTok{suspensoes =}\NormalTok{ censuras)}

\CommentTok{\# Adicionar intervalos de confiança}
\NormalTok{fit }\OtherTok{\textless{}{-}} \FunctionTok{wblr.conf}\NormalTok{(fit)}

\CommentTok{\# Plotar papel de Weibull}
\FunctionTok{wblr.plot}\NormalTok{(fit)}
\end{Highlighting}
\end{Shaded}

\subsection*{Ajuste de
Distribuições}\label{ajuste-de-distribuiuxe7uxf5es}
\addcontentsline{toc}{subsection}{Ajuste de Distribuições}

\subsubsection*{\texorpdfstring{\texttt{fitdistrplus}}{fitdistrplus}}\label{fitdistrplus}
\addcontentsline{toc}{subsubsection}{\texttt{fitdistrplus}}

Ajuste e diagnóstico de distribuições.

\begin{Shaded}
\begin{Highlighting}[]
\FunctionTok{library}\NormalTok{(fitdistrplus)}

\CommentTok{\# Ajustar distribuição}
\NormalTok{fit\_weibull }\OtherTok{\textless{}{-}} \FunctionTok{fitdist}\NormalTok{(dados, }\StringTok{"weibull"}\NormalTok{)}
\NormalTok{fit\_lnorm }\OtherTok{\textless{}{-}} \FunctionTok{fitdist}\NormalTok{(dados, }\StringTok{"lnorm"}\NormalTok{)}

\CommentTok{\# Comparar ajustes}
\FunctionTok{gofstat}\NormalTok{(}\FunctionTok{list}\NormalTok{(fit\_weibull, fit\_lnorm))}

\CommentTok{\# Plotar diagnósticos}
\FunctionTok{plot}\NormalTok{(fit\_weibull)}
\end{Highlighting}
\end{Shaded}

\subsection*{Machine Learning}\label{machine-learning}
\addcontentsline{toc}{subsection}{Machine Learning}

\subsubsection*{\texorpdfstring{\texttt{caret}}{caret}}\label{caret}
\addcontentsline{toc}{subsubsection}{\texttt{caret}}

Framework unificado para ML.

\begin{Shaded}
\begin{Highlighting}[]
\FunctionTok{library}\NormalTok{(caret)}

\CommentTok{\# Treinar modelo}
\NormalTok{modelo }\OtherTok{\textless{}{-}} \FunctionTok{train}\NormalTok{(falha }\SpecialCharTok{\textasciitilde{}}\NormalTok{ .,}
                \AttributeTok{data =}\NormalTok{ treino,}
                \AttributeTok{method =} \StringTok{"rf"}\NormalTok{,}
                \AttributeTok{trControl =} \FunctionTok{trainControl}\NormalTok{(}\AttributeTok{method =} \StringTok{"cv"}\NormalTok{, }\AttributeTok{number =} \DecValTok{5}\NormalTok{))}

\CommentTok{\# Importância de variáveis}
\FunctionTok{varImp}\NormalTok{(modelo)}
\end{Highlighting}
\end{Shaded}

\textbf{Métodos disponíveis:} - \texttt{"rf"} - Random Forest -
\texttt{"xgbTree"} - XGBoost - \texttt{"glm"} - Regressão Logística -
\texttt{"svmRadial"} - SVM

\subsubsection*{\texorpdfstring{\texttt{randomForest}}{randomForest}}\label{randomforest}
\addcontentsline{toc}{subsubsection}{\texttt{randomForest}}

Implementação específica de Random Forest.

\begin{Shaded}
\begin{Highlighting}[]
\FunctionTok{library}\NormalTok{(randomForest)}

\NormalTok{modelo }\OtherTok{\textless{}{-}} \FunctionTok{randomForest}\NormalTok{(falha }\SpecialCharTok{\textasciitilde{}}\NormalTok{ .,}
                       \AttributeTok{data =}\NormalTok{ treino,}
                       \AttributeTok{ntree =} \DecValTok{500}\NormalTok{,}
                       \AttributeTok{importance =} \ConstantTok{TRUE}\NormalTok{)}

\CommentTok{\# Matriz de confusão}
\FunctionTok{predict}\NormalTok{(modelo, teste)}
\end{Highlighting}
\end{Shaded}

\subsection*{Séries Temporais}\label{suxe9ries-temporais}
\addcontentsline{toc}{subsection}{Séries Temporais}

\subsubsection*{\texorpdfstring{\texttt{prophet}}{prophet}}\label{prophet}
\addcontentsline{toc}{subsubsection}{\texttt{prophet}}

Previsão de séries temporais (Facebook).

\begin{Shaded}
\begin{Highlighting}[]
\FunctionTok{library}\NormalTok{(prophet)}

\CommentTok{\# Preparar dados (requer colunas \textquotesingle{}ds\textquotesingle{} e \textquotesingle{}y\textquotesingle{})}
\NormalTok{df }\OtherTok{\textless{}{-}} \FunctionTok{data.frame}\NormalTok{(}\AttributeTok{ds =}\NormalTok{ datas, }\AttributeTok{y =}\NormalTok{ valores)}

\CommentTok{\# Ajustar modelo}
\NormalTok{m }\OtherTok{\textless{}{-}} \FunctionTok{prophet}\NormalTok{(df)}

\CommentTok{\# Previsão}
\NormalTok{future }\OtherTok{\textless{}{-}} \FunctionTok{make\_future\_dataframe}\NormalTok{(m, }\AttributeTok{periods =} \DecValTok{30}\NormalTok{)}
\NormalTok{forecast }\OtherTok{\textless{}{-}} \FunctionTok{predict}\NormalTok{(m, future)}
\end{Highlighting}
\end{Shaded}

\subsubsection*{\texorpdfstring{\texttt{anomalize}}{anomalize}}\label{anomalize}
\addcontentsline{toc}{subsubsection}{\texttt{anomalize}}

Detecção de anomalias em séries temporais.

\begin{Shaded}
\begin{Highlighting}[]
\FunctionTok{library}\NormalTok{(anomalize)}

\NormalTok{dados }\SpecialCharTok{\%\textgreater{}\%}
  \FunctionTok{time\_decompose}\NormalTok{(temperatura) }\SpecialCharTok{\%\textgreater{}\%}
  \FunctionTok{anomalize}\NormalTok{(remainder) }\SpecialCharTok{\%\textgreater{}\%}
  \FunctionTok{time\_recompose}\NormalTok{() }\SpecialCharTok{\%\textgreater{}\%}
  \FunctionTok{plot\_anomalies}\NormalTok{()}
\end{Highlighting}
\end{Shaded}

\subsection*{Visualização}\label{visualizauxe7uxe3o}
\addcontentsline{toc}{subsection}{Visualização}

\subsubsection*{\texorpdfstring{\texttt{survminer}}{survminer}}\label{survminer}
\addcontentsline{toc}{subsubsection}{\texttt{survminer}}

Visualizações para análise de sobrevivência.

\begin{Shaded}
\begin{Highlighting}[]
\FunctionTok{library}\NormalTok{(survminer)}

\CommentTok{\# Curva KM estilizada}
\FunctionTok{ggsurvplot}\NormalTok{(km\_fit,}
           \AttributeTok{conf.int =} \ConstantTok{TRUE}\NormalTok{,}
           \AttributeTok{risk.table =} \ConstantTok{TRUE}\NormalTok{,}
           \AttributeTok{pval =} \ConstantTok{TRUE}\NormalTok{,}
           \AttributeTok{palette =} \StringTok{"jco"}\NormalTok{)}
\end{Highlighting}
\end{Shaded}

\subsubsection*{\texorpdfstring{\texttt{plotly}}{plotly}}\label{plotly}
\addcontentsline{toc}{subsubsection}{\texttt{plotly}}

Gráficos interativos.

\begin{Shaded}
\begin{Highlighting}[]
\FunctionTok{library}\NormalTok{(plotly)}

\CommentTok{\# Converter ggplot}
\NormalTok{p }\OtherTok{\textless{}{-}} \FunctionTok{ggplot}\NormalTok{(dados, }\FunctionTok{aes}\NormalTok{(x, y)) }\SpecialCharTok{+} \FunctionTok{geom\_line}\NormalTok{()}
\FunctionTok{ggplotly}\NormalTok{(p)}

\CommentTok{\# Plot direto}
\FunctionTok{plot\_ly}\NormalTok{(dados, }\AttributeTok{x =} \SpecialCharTok{\textasciitilde{}}\NormalTok{tempo, }\AttributeTok{y =} \SpecialCharTok{\textasciitilde{}}\NormalTok{confiabilidade, }\AttributeTok{type =} \StringTok{"scatter"}\NormalTok{, }\AttributeTok{mode =} \StringTok{"lines"}\NormalTok{)}
\end{Highlighting}
\end{Shaded}

\subsection*{Tabelas}\label{tabelas}
\addcontentsline{toc}{subsection}{Tabelas}

\subsubsection*{\texorpdfstring{\texttt{kableExtra}}{kableExtra}}\label{kableextra}
\addcontentsline{toc}{subsubsection}{\texttt{kableExtra}}

Tabelas estilizadas para relatórios.

\begin{Shaded}
\begin{Highlighting}[]
\FunctionTok{library}\NormalTok{(kableExtra)}

\NormalTok{dados }\SpecialCharTok{\%\textgreater{}\%}
  \FunctionTok{kable}\NormalTok{() }\SpecialCharTok{\%\textgreater{}\%}
  \FunctionTok{kable\_styling}\NormalTok{(}\AttributeTok{bootstrap\_options =} \FunctionTok{c}\NormalTok{(}\StringTok{"striped"}\NormalTok{, }\StringTok{"hover"}\NormalTok{)) }\SpecialCharTok{\%\textgreater{}\%}
  \FunctionTok{row\_spec}\NormalTok{(}\DecValTok{0}\NormalTok{, }\AttributeTok{bold =} \ConstantTok{TRUE}\NormalTok{, }\AttributeTok{color =} \StringTok{"white"}\NormalTok{, }\AttributeTok{background =} \StringTok{"steelblue"}\NormalTok{)}
\end{Highlighting}
\end{Shaded}

\subsubsection*{\texorpdfstring{\texttt{DT}}{DT}}\label{dt}
\addcontentsline{toc}{subsubsection}{\texttt{DT}}

Tabelas interativas.

\begin{Shaded}
\begin{Highlighting}[]
\FunctionTok{library}\NormalTok{(DT)}

\FunctionTok{datatable}\NormalTok{(dados,}
          \AttributeTok{filter =} \StringTok{"top"}\NormalTok{,}
          \AttributeTok{options =} \FunctionTok{list}\NormalTok{(}\AttributeTok{pageLength =} \DecValTok{25}\NormalTok{))}
\end{Highlighting}
\end{Shaded}

\subsection*{Simulação}\label{simulauxe7uxe3o}
\addcontentsline{toc}{subsection}{Simulação}

\subsubsection*{\texorpdfstring{\texttt{simmer}}{simmer}}\label{simmer}
\addcontentsline{toc}{subsubsection}{\texttt{simmer}}

Simulação de eventos discretos.

\begin{Shaded}
\begin{Highlighting}[]
\FunctionTok{library}\NormalTok{(simmer)}

\CommentTok{\# Criar ambiente de simulação}
\NormalTok{env }\OtherTok{\textless{}{-}} \FunctionTok{simmer}\NormalTok{()}

\CommentTok{\# Definir trajetórias e recursos}
\CommentTok{\# ...}

\CommentTok{\# Executar simulação}
\FunctionTok{run}\NormalTok{(env, }\AttributeTok{until =} \DecValTok{1000}\NormalTok{)}
\end{Highlighting}
\end{Shaded}

\section*{Snippets Úteis}\label{snippets-uxfateis}
\addcontentsline{toc}{section}{Snippets Úteis}

\markright{Snippets Úteis}

\subsection*{Calcular MTTF de Weibull}\label{calcular-mttf-de-weibull}
\addcontentsline{toc}{subsection}{Calcular MTTF de Weibull}

\begin{Shaded}
\begin{Highlighting}[]
\NormalTok{mttf\_weibull }\OtherTok{\textless{}{-}} \ControlFlowTok{function}\NormalTok{(beta, eta) \{}
\NormalTok{  eta }\SpecialCharTok{*} \FunctionTok{gamma}\NormalTok{(}\DecValTok{1} \SpecialCharTok{+} \DecValTok{1}\SpecialCharTok{/}\NormalTok{beta)}
\NormalTok{\}}
\end{Highlighting}
\end{Shaded}

\subsection*{Confiabilidade em Tempo t}\label{confiabilidade-em-tempo-t}
\addcontentsline{toc}{subsection}{Confiabilidade em Tempo t}

\begin{Shaded}
\begin{Highlighting}[]
\NormalTok{confiabilidade\_weibull }\OtherTok{\textless{}{-}} \ControlFlowTok{function}\NormalTok{(t, beta, eta) \{}
  \FunctionTok{exp}\NormalTok{(}\SpecialCharTok{{-}}\NormalTok{(t}\SpecialCharTok{/}\NormalTok{eta)}\SpecialCharTok{\^{}}\NormalTok{beta)}
\NormalTok{\}}
\end{Highlighting}
\end{Shaded}

\subsection*{Bootstrap IC}\label{bootstrap-ic}
\addcontentsline{toc}{subsection}{Bootstrap IC}

\begin{Shaded}
\begin{Highlighting}[]
\NormalTok{bootstrap\_ci }\OtherTok{\textless{}{-}} \ControlFlowTok{function}\NormalTok{(data, stat\_fun, }\AttributeTok{n\_boot =} \DecValTok{1000}\NormalTok{, }\AttributeTok{conf =} \FloatTok{0.95}\NormalTok{) \{}
\NormalTok{  boot\_stats }\OtherTok{\textless{}{-}} \FunctionTok{replicate}\NormalTok{(n\_boot, \{}
\NormalTok{    sample\_boot }\OtherTok{\textless{}{-}} \FunctionTok{sample}\NormalTok{(data, }\AttributeTok{replace =} \ConstantTok{TRUE}\NormalTok{)}
    \FunctionTok{stat\_fun}\NormalTok{(sample\_boot)}
\NormalTok{  \})}

\NormalTok{  alpha }\OtherTok{\textless{}{-}} \DecValTok{1} \SpecialCharTok{{-}}\NormalTok{ conf}
  \FunctionTok{quantile}\NormalTok{(boot\_stats, }\FunctionTok{c}\NormalTok{(alpha}\SpecialCharTok{/}\DecValTok{2}\NormalTok{, }\DecValTok{1} \SpecialCharTok{{-}}\NormalTok{ alpha}\SpecialCharTok{/}\DecValTok{2}\NormalTok{))}
\NormalTok{\}}
\end{Highlighting}
\end{Shaded}

\subsection*{Plotar Taxa de Falha}\label{plotar-taxa-de-falha}
\addcontentsline{toc}{subsection}{Plotar Taxa de Falha}

\begin{Shaded}
\begin{Highlighting}[]
\NormalTok{plot\_hazard }\OtherTok{\textless{}{-}} \ControlFlowTok{function}\NormalTok{(beta, eta, }\AttributeTok{t\_max =} \DecValTok{1000}\NormalTok{) \{}
\NormalTok{  t }\OtherTok{\textless{}{-}} \FunctionTok{seq}\NormalTok{(}\DecValTok{0}\NormalTok{, t\_max, }\AttributeTok{length.out =} \DecValTok{200}\NormalTok{)}
\NormalTok{  hazard }\OtherTok{\textless{}{-}}\NormalTok{ (beta}\SpecialCharTok{/}\NormalTok{eta) }\SpecialCharTok{*}\NormalTok{ (t}\SpecialCharTok{/}\NormalTok{eta)}\SpecialCharTok{\^{}}\NormalTok{(beta }\SpecialCharTok{{-}} \DecValTok{1}\NormalTok{)}

  \FunctionTok{ggplot}\NormalTok{(}\FunctionTok{data.frame}\NormalTok{(t, hazard), }\FunctionTok{aes}\NormalTok{(t, hazard)) }\SpecialCharTok{+}
    \FunctionTok{geom\_line}\NormalTok{(}\AttributeTok{color =} \StringTok{"steelblue"}\NormalTok{, }\AttributeTok{linewidth =} \FloatTok{1.2}\NormalTok{) }\SpecialCharTok{+}
    \FunctionTok{labs}\NormalTok{(}\AttributeTok{title =} \FunctionTok{paste}\NormalTok{(}\StringTok{"Taxa de Falha (β ="}\NormalTok{, beta, }\StringTok{")"}\NormalTok{),}
         \AttributeTok{x =} \StringTok{"Tempo"}\NormalTok{, }\AttributeTok{y =} \StringTok{"λ(t)"}\NormalTok{) }\SpecialCharTok{+}
    \FunctionTok{theme\_minimal}\NormalTok{()}
\NormalTok{\}}
\end{Highlighting}
\end{Shaded}

\section*{Instalação Rápida}\label{instalauxe7uxe3o-ruxe1pida}
\addcontentsline{toc}{section}{Instalação Rápida}

\markright{Instalação Rápida}

\begin{Shaded}
\begin{Highlighting}[]
\CommentTok{\# Todos os pacotes essenciais}
\FunctionTok{install.packages}\NormalTok{(}\FunctionTok{c}\NormalTok{(}
  \CommentTok{\# Manipulação}
  \StringTok{"tidyverse"}\NormalTok{, }\StringTok{"data.table"}\NormalTok{, }\StringTok{"lubridate"}\NormalTok{,}

  \CommentTok{\# Confiabilidade}
  \StringTok{"survival"}\NormalTok{, }\StringTok{"flexsurv"}\NormalTok{, }\StringTok{"WeibullR"}\NormalTok{, }\StringTok{"reliability"}\NormalTok{,}
  \StringTok{"fitdistrplus"}\NormalTok{,}

  \CommentTok{\# Machine Learning}
  \StringTok{"caret"}\NormalTok{, }\StringTok{"randomForest"}\NormalTok{, }\StringTok{"xgboost"}\NormalTok{,}

  \CommentTok{\# Séries Temporais}
  \StringTok{"prophet"}\NormalTok{, }\StringTok{"anomalize"}\NormalTok{, }\StringTok{"forecast"}\NormalTok{,}

  \CommentTok{\# Visualização}
  \StringTok{"survminer"}\NormalTok{, }\StringTok{"plotly"}\NormalTok{, }\StringTok{"patchwork"}\NormalTok{,}

  \CommentTok{\# Tabelas}
  \StringTok{"kableExtra"}\NormalTok{, }\StringTok{"DT"}\NormalTok{, }\StringTok{"gt"}\NormalTok{,}

  \CommentTok{\# Relatórios}
  \StringTok{"knitr"}\NormalTok{, }\StringTok{"rmarkdown"}\NormalTok{, }\StringTok{"quarto"}\NormalTok{,}

  \CommentTok{\# Outros}
  \StringTok{"broom"}\NormalTok{, }\StringTok{"here"}\NormalTok{, }\StringTok{"pins"}
\NormalTok{))}
\end{Highlighting}
\end{Shaded}

\section*{Recursos Online}\label{recursos-online}
\addcontentsline{toc}{section}{Recursos Online}

\markright{Recursos Online}

\begin{itemize}
\tightlist
\item
  \href{https://cran.r-project.org/web/views/Survival.html}{CRAN Task
  View: Survival Analysis}
\item
  \href{https://cran.r-project.org/web/packages/survival/vignettes/survival.pdf}{survival
  package vignette}
\item
  \href{https://r4ds.hadley.nz/}{R for Data Science}
\item
  \href{https://posit.co/resources/cheatsheets/}{Posit Cheatsheets}
\end{itemize}

\chapter*{Apêndice B: Templates Quarto e
Automação}\label{apuxeandice-b-templates-quarto-e-automauxe7uxe3o}
\addcontentsline{toc}{chapter}{Apêndice B: Templates Quarto e Automação}

\markboth{Apêndice B: Templates Quarto e Automação}{Apêndice B:
Templates Quarto e Automação}

\section*{Estrutura de Projeto Quarto
Book}\label{estrutura-de-projeto-quarto-book}
\addcontentsline{toc}{section}{Estrutura de Projeto Quarto Book}

\markright{Estrutura de Projeto Quarto Book}

\subsection*{\texorpdfstring{Arquivo
\texttt{\_quarto.yml}}{Arquivo \_quarto.yml}}\label{arquivo-_quarto.yml}
\addcontentsline{toc}{subsection}{Arquivo \texttt{\_quarto.yml}}

\begin{Shaded}
\begin{Highlighting}[]
\FunctionTok{project}\KeywordTok{:}
\AttributeTok{  }\FunctionTok{type}\KeywordTok{:}\AttributeTok{ book}
\AttributeTok{  }\FunctionTok{output{-}dir}\KeywordTok{:}\AttributeTok{ docs}

\FunctionTok{book}\KeywordTok{:}
\AttributeTok{  }\FunctionTok{title}\KeywordTok{:}\AttributeTok{ }\StringTok{"Confiabilidade e Manutenção"}
\AttributeTok{  }\FunctionTok{subtitle}\KeywordTok{:}\AttributeTok{ }\StringTok{"Aplicações em R"}
\AttributeTok{  }\FunctionTok{author}\KeywordTok{:}\AttributeTok{ }\StringTok{"Seu Nome"}
\AttributeTok{  }\FunctionTok{date}\KeywordTok{:}\AttributeTok{ today}

\AttributeTok{  }\FunctionTok{chapters}\KeywordTok{:}
\AttributeTok{    }\KeywordTok{{-}}\AttributeTok{ index.qmd}
\AttributeTok{    }\KeywordTok{{-}}\AttributeTok{ preface.qmd}

\AttributeTok{    }\KeywordTok{{-}}\AttributeTok{ }\FunctionTok{part}\KeywordTok{:}\AttributeTok{ }\StringTok{"Parte I"}
\AttributeTok{      }\FunctionTok{chapters}\KeywordTok{:}
\AttributeTok{        }\KeywordTok{{-}}\AttributeTok{ chapters/01{-}intro.qmd}
\AttributeTok{        }\KeywordTok{{-}}\AttributeTok{ chapters/02{-}analise.qmd}

\AttributeTok{    }\KeywordTok{{-}}\AttributeTok{ references.qmd}

\FunctionTok{bibliography}\KeywordTok{:}\AttributeTok{ references.bib}

\FunctionTok{format}\KeywordTok{:}
\AttributeTok{  }\FunctionTok{html}\KeywordTok{:}
\AttributeTok{    }\FunctionTok{theme}\KeywordTok{:}
\AttributeTok{      }\FunctionTok{light}\KeywordTok{:}\AttributeTok{ flatly}
\AttributeTok{      }\FunctionTok{dark}\KeywordTok{:}\AttributeTok{ darkly}
\AttributeTok{    }\FunctionTok{code{-}fold}\KeywordTok{:}\AttributeTok{ show}
\AttributeTok{    }\FunctionTok{code{-}tools}\KeywordTok{:}\AttributeTok{ }\CharTok{true}
\AttributeTok{    }\FunctionTok{toc}\KeywordTok{:}\AttributeTok{ }\CharTok{true}
\AttributeTok{    }\FunctionTok{toc{-}depth}\KeywordTok{:}\AttributeTok{ }\DecValTok{3}
\AttributeTok{    }\FunctionTok{number{-}sections}\KeywordTok{:}\AttributeTok{ }\CharTok{true}

\AttributeTok{  }\FunctionTok{pdf}\KeywordTok{:}
\AttributeTok{    }\FunctionTok{documentclass}\KeywordTok{:}\AttributeTok{ scrbook}
\AttributeTok{    }\FunctionTok{toc}\KeywordTok{:}\AttributeTok{ }\CharTok{true}
\AttributeTok{    }\FunctionTok{number{-}sections}\KeywordTok{:}\AttributeTok{ }\CharTok{true}

\FunctionTok{execute}\KeywordTok{:}
\AttributeTok{  }\FunctionTok{echo}\KeywordTok{:}\AttributeTok{ }\CharTok{true}
\AttributeTok{  }\FunctionTok{warning}\KeywordTok{:}\AttributeTok{ }\CharTok{false}
\AttributeTok{  }\FunctionTok{message}\KeywordTok{:}\AttributeTok{ }\CharTok{false}
\AttributeTok{  }\FunctionTok{cache}\KeywordTok{:}\AttributeTok{ }\CharTok{true}
\AttributeTok{  }\FunctionTok{freeze}\KeywordTok{:}\AttributeTok{ auto}
\end{Highlighting}
\end{Shaded}

\section*{Template de Capítulo}\label{template-de-capuxedtulo}
\addcontentsline{toc}{section}{Template de Capítulo}

\markright{Template de Capítulo}

\begin{Shaded}
\begin{Highlighting}[]
\FunctionTok{\# Título do Capítulo}

\FunctionTok{\#\# Introdução}

\NormalTok{Breve introdução ao tema...}

\FunctionTok{\#\# Conceitos}

\FunctionTok{\#\#\# Subseção 1}

\NormalTok{Explicação com equação:}

\NormalTok{$$}
\NormalTok{R(t) = e\^{}\{{-}\textbackslash{}lambda t\}}
\NormalTok{$$ \{\#eq{-}confiabilidade\}}

\FunctionTok{\#\#\# Exemplo Prático}

\NormalTok{💡 **Exemplo**: Descrição do exemplo}

\NormalTok{::: \{.cell\}}

\InformationTok{\textasciigrave{}\textasciigrave{}\textasciigrave{}\{.r .cell{-}code\}}
\InformationTok{\# Exemplo ilustrativo (não executado)}
\InformationTok{\# library(tidyverse)}
\InformationTok{\# dados \textless{}{-} data.frame(...)}
\InformationTok{\# ggplot(dados, aes(x, y)) + geom\_line()}
\end{Highlighting}
\end{Shaded}

:::

\section*{Exercícios}\label{exercuxedcios-12}
\addcontentsline{toc}{section}{Exercícios}

\markright{Exercícios}

📝 \textbf{Exercício 1}: Descrição do exercício.

\section*{Resumo}\label{resumo-13}
\addcontentsline{toc}{section}{Resumo}

\markright{Resumo}

🔔 \textbf{Pontos-chave:}

\begin{itemize}
\tightlist
\item
  Ponto 1
\item
  Ponto 2
\end{itemize}

\begin{center}\rule{0.5\linewidth}{0.5pt}\end{center}

\textbf{Próximo capítulo:} Nome do próximo

\begin{verbatim}

## Opções de Chunks

### Controle de Execução

::: {.cell}

:::

### Figuras








::: {.cell layout-align="center"}

```{.r .cell-code}
# Código do gráfico
\end{verbatim}

:::

\subsection*{Tabelas}\label{tabelas-1}
\addcontentsline{toc}{subsection}{Tabelas}

\begin{table}

\caption{\label{tbl-exemplo}Legenda da tabela}

\centering{

\begin{Shaded}
\begin{Highlighting}[]
\CommentTok{\# Código da tabela}
\end{Highlighting}
\end{Shaded}

}

\end{table}%

\section*{Referências Cruzadas}\label{referuxeancias-cruzadas-1}
\addcontentsline{toc}{section}{Referências Cruzadas}

\markright{Referências Cruzadas}

\subsection*{Figuras}\label{figuras}
\addcontentsline{toc}{subsection}{Figuras}

\begin{Shaded}
\begin{Highlighting}[]
\NormalTok{Veja @fig{-}exemplo para detalhes.}

\NormalTok{::: \{.cell\}}

\InformationTok{\textasciigrave{}\textasciigrave{}\textasciigrave{}\{.r .cell{-}code\}}
\InformationTok{plot(1:10)}
\end{Highlighting}
\end{Shaded}

\begin{figure}[H]

\centering{

\pandocbounded{\includegraphics[keepaspectratio]{appendices/B-templates_files/figure-pdf/fig-exemplo-2-1.pdf}}

}

\caption{\label{fig-exemplo-2}Minha figura}

\end{figure}%

:::

\begin{verbatim}

### Tabelas

```markdown
A @tbl-resultados apresenta os dados.

::: {#tbl-resultados .cell tbl-cap='Resultados'}

```{.r .cell-code}
# knitr::kable(dados)
\end{verbatim}

:::

\begin{verbatim}

### Equações

```markdown
Como mostra a @eq-weibull:

$$
f(t) = \frac{\beta}{\eta}\left(\frac{t}{\eta}\right)^{\beta-1} e^{-(t/\eta)^\beta}
$$ {#eq-weibull}
\end{verbatim}

\subsection*{Capítulos e Seções}\label{capuxedtulos-e-seuxe7uxf5es}
\addcontentsline{toc}{subsection}{Capítulos e Seções}

\begin{Shaded}
\begin{Highlighting}[]
\NormalTok{Como discutido no @sec{-}introducao...}

\FunctionTok{\#\# Introdução \{\#sec{-}introducao\}}
\end{Highlighting}
\end{Shaded}

\section*{CI/CD com GitHub Actions}\label{cicd-com-github-actions}
\addcontentsline{toc}{section}{CI/CD com GitHub Actions}

\markright{CI/CD com GitHub Actions}

\subsection*{\texorpdfstring{\texttt{.github/workflows/quarto-publish.yml}}{.github/workflows/quarto-publish.yml}}\label{githubworkflowsquarto-publish.yml}
\addcontentsline{toc}{subsection}{\texttt{.github/workflows/quarto-publish.yml}}

\begin{Shaded}
\begin{Highlighting}[]
\FunctionTok{name}\KeywordTok{:}\AttributeTok{ Render and Publish Quarto Book}

\FunctionTok{on}\KeywordTok{:}
\AttributeTok{  }\FunctionTok{push}\KeywordTok{:}
\AttributeTok{    }\FunctionTok{branches}\KeywordTok{:}\AttributeTok{ }\KeywordTok{[}\AttributeTok{main}\KeywordTok{]}
\AttributeTok{  }\FunctionTok{pull\_request}\KeywordTok{:}
\AttributeTok{    }\FunctionTok{branches}\KeywordTok{:}\AttributeTok{ }\KeywordTok{[}\AttributeTok{main}\KeywordTok{]}

\FunctionTok{jobs}\KeywordTok{:}
\AttributeTok{  }\FunctionTok{build{-}deploy}\KeywordTok{:}
\AttributeTok{    }\FunctionTok{runs{-}on}\KeywordTok{:}\AttributeTok{ ubuntu{-}latest}

\AttributeTok{    }\FunctionTok{steps}\KeywordTok{:}
\AttributeTok{      }\KeywordTok{{-}}\AttributeTok{ }\FunctionTok{name}\KeywordTok{:}\AttributeTok{ Checkout repository}
\AttributeTok{        }\FunctionTok{uses}\KeywordTok{:}\AttributeTok{ actions/checkout@v3}

\AttributeTok{      }\KeywordTok{{-}}\AttributeTok{ }\FunctionTok{name}\KeywordTok{:}\AttributeTok{ Setup R}
\AttributeTok{        }\FunctionTok{uses}\KeywordTok{:}\AttributeTok{ r{-}lib/actions/setup{-}r@v2}
\AttributeTok{        }\FunctionTok{with}\KeywordTok{:}
\AttributeTok{          }\FunctionTok{r{-}version}\KeywordTok{:}\AttributeTok{ }\StringTok{\textquotesingle{}4.5.1\textquotesingle{}}

\AttributeTok{      }\KeywordTok{{-}}\AttributeTok{ }\FunctionTok{name}\KeywordTok{:}\AttributeTok{ Install R packages}
\FunctionTok{        run}\KeywordTok{: }\CharTok{|}
\NormalTok{          install.packages(c("tidyverse", "survival", "ggplot2"))}
\AttributeTok{        }\FunctionTok{shell}\KeywordTok{:}\AttributeTok{ Rscript \{0\}}

\AttributeTok{      }\KeywordTok{{-}}\AttributeTok{ }\FunctionTok{name}\KeywordTok{:}\AttributeTok{ Setup Quarto}
\AttributeTok{        }\FunctionTok{uses}\KeywordTok{:}\AttributeTok{ quarto{-}dev/quarto{-}actions/setup@v2}
\AttributeTok{        }\FunctionTok{with}\KeywordTok{:}
\AttributeTok{          }\FunctionTok{version}\KeywordTok{:}\AttributeTok{ }\StringTok{\textquotesingle{}1.6.39\textquotesingle{}}

\AttributeTok{      }\KeywordTok{{-}}\AttributeTok{ }\FunctionTok{name}\KeywordTok{:}\AttributeTok{ Render Quarto Book}
\AttributeTok{        }\FunctionTok{run}\KeywordTok{:}\AttributeTok{ quarto render}

\AttributeTok{      }\KeywordTok{{-}}\AttributeTok{ }\FunctionTok{name}\KeywordTok{:}\AttributeTok{ Deploy to GitHub Pages}
\AttributeTok{        }\FunctionTok{if}\KeywordTok{:}\AttributeTok{ github.ref == \textquotesingle{}refs/heads/main\textquotesingle{}}
\AttributeTok{        }\FunctionTok{uses}\KeywordTok{:}\AttributeTok{ peaceiris/actions{-}gh{-}pages@v3}
\AttributeTok{        }\FunctionTok{with}\KeywordTok{:}
\AttributeTok{          }\FunctionTok{github\_token}\KeywordTok{:}\AttributeTok{ $\{\{ secrets.GITHUB\_TOKEN \}\}}
\AttributeTok{          }\FunctionTok{publish\_dir}\KeywordTok{:}\AttributeTok{ ./docs}
\end{Highlighting}
\end{Shaded}

\section*{Deploy para Quarto Pub}\label{deploy-para-quarto-pub}
\addcontentsline{toc}{section}{Deploy para Quarto Pub}

\markright{Deploy para Quarto Pub}

\subsection*{Comandos}\label{comandos}
\addcontentsline{toc}{subsection}{Comandos}

\begin{Shaded}
\begin{Highlighting}[]
\CommentTok{\# Publicar pela primeira vez}
\ExtensionTok{quarto}\NormalTok{ publish quarto{-}pub}

\CommentTok{\# Atualizar}
\ExtensionTok{quarto}\NormalTok{ publish quarto{-}pub }\AttributeTok{{-}{-}update}
\end{Highlighting}
\end{Shaded}

\subsection*{\texorpdfstring{Configuração no
\texttt{\_quarto.yml}}{Configuração no \_quarto.yml}}\label{configurauxe7uxe3o-no-_quarto.yml}
\addcontentsline{toc}{subsection}{Configuração no \texttt{\_quarto.yml}}

\begin{Shaded}
\begin{Highlighting}[]
\FunctionTok{project}\KeywordTok{:}
\AttributeTok{  }\FunctionTok{type}\KeywordTok{:}\AttributeTok{ book}

\FunctionTok{quarto{-}pub}\KeywordTok{:}
\AttributeTok{  }\FunctionTok{site{-}id}\KeywordTok{:}\AttributeTok{ }\StringTok{"seu{-}site{-}id"}
\end{Highlighting}
\end{Shaded}

\section*{Deploy para GitHub Pages}\label{deploy-para-github-pages}
\addcontentsline{toc}{section}{Deploy para GitHub Pages}

\markright{Deploy para GitHub Pages}

\subsection*{Passo a Passo}\label{passo-a-passo}
\addcontentsline{toc}{subsection}{Passo a Passo}

\begin{enumerate}
\def\labelenumi{\arabic{enumi}.}
\tightlist
\item
  \textbf{Configurar output para \texttt{docs/}}
\end{enumerate}

\begin{Shaded}
\begin{Highlighting}[]
\FunctionTok{project}\KeywordTok{:}
\AttributeTok{  }\FunctionTok{output{-}dir}\KeywordTok{:}\AttributeTok{ docs}
\end{Highlighting}
\end{Shaded}

\begin{enumerate}
\def\labelenumi{\arabic{enumi}.}
\setcounter{enumi}{1}
\tightlist
\item
  \textbf{Criar \texttt{.nojekyll}}
\end{enumerate}

\begin{Shaded}
\begin{Highlighting}[]
\FunctionTok{touch}\NormalTok{ docs/.nojekyll}
\end{Highlighting}
\end{Shaded}

\begin{enumerate}
\def\labelenumi{\arabic{enumi}.}
\setcounter{enumi}{2}
\tightlist
\item
  \textbf{Configurar GitHub Pages}

  \begin{itemize}
  \tightlist
  \item
    Settings → Pages
  \item
    Source: Deploy from branch
  \item
    Branch: \texttt{main} / \texttt{docs}
  \end{itemize}
\item
  \textbf{Push e esperar}
\end{enumerate}

\begin{Shaded}
\begin{Highlighting}[]
\FunctionTok{git}\NormalTok{ add .}
\FunctionTok{git}\NormalTok{ commit }\AttributeTok{{-}m} \StringTok{"Update book"}
\FunctionTok{git}\NormalTok{ push}
\end{Highlighting}
\end{Shaded}

\section*{Deploy para Netlify}\label{deploy-para-netlify}
\addcontentsline{toc}{section}{Deploy para Netlify}

\markright{Deploy para Netlify}

\subsection*{\texorpdfstring{\texttt{netlify.toml}}{netlify.toml}}\label{netlify.toml}
\addcontentsline{toc}{subsection}{\texttt{netlify.toml}}

\begin{Shaded}
\begin{Highlighting}[]
\KeywordTok{[build]}
  \DataTypeTok{command} \OperatorTok{=} \StringTok{"quarto render"}
  \DataTypeTok{publish} \OperatorTok{=} \StringTok{"docs"}

\KeywordTok{[build.environment]}
  \DataTypeTok{QUARTO\_VERSION} \OperatorTok{=} \StringTok{"1.6.39"}
  \DataTypeTok{R\_VERSION} \OperatorTok{=} \StringTok{"4.5.1"}

\KeywordTok{[[redirects]]}
  \DataTypeTok{from} \OperatorTok{=} \StringTok{"/*"}
  \DataTypeTok{to} \OperatorTok{=} \StringTok{"/index.html"}
  \DataTypeTok{status} \OperatorTok{=} \DecValTok{200}
\end{Highlighting}
\end{Shaded}

\section*{Makefile para Automação}\label{makefile-para-automauxe7uxe3o}
\addcontentsline{toc}{section}{Makefile para Automação}

\markright{Makefile para Automação}

\subsection*{\texorpdfstring{\texttt{Makefile}}{Makefile}}\label{makefile}
\addcontentsline{toc}{subsection}{\texttt{Makefile}}

\begin{Shaded}
\begin{Highlighting}[]
\OtherTok{.PHONY:}\DataTypeTok{ preview render clean deploy}

\DecValTok{preview:}
\ErrorTok{    }\NormalTok{quarto preview}

\DecValTok{render:}
\ErrorTok{    }\NormalTok{quarto render}

\DecValTok{clean:}
\ErrorTok{    }\NormalTok{rm {-}rf docs/ \_book/ .quarto/ *\_cache/ *\_files/}

\DecValTok{deploy:}\DataTypeTok{ render}
\ErrorTok{    }\NormalTok{git add docs/}
\NormalTok{    git commit {-}m }\StringTok{"Update rendered book"}
\NormalTok{    git push}

\DecValTok{install{-}deps:}
\ErrorTok{    }\NormalTok{Rscript {-}e }\StringTok{"install.packages(c(\textquotesingle{}tidyverse\textquotesingle{}, \textquotesingle{}survival\textquotesingle{}, \textquotesingle{}ggplot2\textquotesingle{}))"}
\end{Highlighting}
\end{Shaded}

\textbf{Uso:}

\begin{Shaded}
\begin{Highlighting}[]
\FunctionTok{make}\NormalTok{ preview   }\CommentTok{\# Visualizar}
\FunctionTok{make}\NormalTok{ render    }\CommentTok{\# Renderizar}
\FunctionTok{make}\NormalTok{ clean     }\CommentTok{\# Limpar}
\FunctionTok{make}\NormalTok{ deploy    }\CommentTok{\# Deploy}
\end{Highlighting}
\end{Shaded}

\section*{Docker para
Reprodutibilidade}\label{docker-para-reprodutibilidade}
\addcontentsline{toc}{section}{Docker para Reprodutibilidade}

\markright{Docker para Reprodutibilidade}

\subsection*{\texorpdfstring{\texttt{Dockerfile}}{Dockerfile}}\label{dockerfile}
\addcontentsline{toc}{subsection}{\texttt{Dockerfile}}

\begin{Shaded}
\begin{Highlighting}[]
\KeywordTok{FROM}\NormalTok{ rocker/r{-}ver:4.5.1}

\CommentTok{\# Instalar Quarto}
\KeywordTok{RUN} \FunctionTok{wget}\NormalTok{ https://github.com/quarto{-}dev/quarto{-}cli/releases/download/v1.6.39/quarto{-}1.6.39{-}linux{-}amd64.deb}
\KeywordTok{RUN} \ExtensionTok{dpkg} \AttributeTok{{-}i}\NormalTok{ quarto{-}1.6.39{-}linux{-}amd64.deb}

\CommentTok{\# Instalar dependências R}
\KeywordTok{RUN} \ExtensionTok{R} \AttributeTok{{-}e} \StringTok{"install.packages(c(\textquotesingle{}tidyverse\textquotesingle{}, \textquotesingle{}survival\textquotesingle{}, \textquotesingle{}ggplot2\textquotesingle{}))"}

\CommentTok{\# Copiar projeto}
\KeywordTok{COPY}\NormalTok{ . /workspace}
\KeywordTok{WORKDIR}\NormalTok{ /workspace}

\CommentTok{\# Renderizar}
\KeywordTok{CMD}\NormalTok{ [}\StringTok{"quarto"}\NormalTok{, }\StringTok{"render"}\NormalTok{]}
\end{Highlighting}
\end{Shaded}

\textbf{Uso:}

\begin{Shaded}
\begin{Highlighting}[]
\ExtensionTok{docker}\NormalTok{ build }\AttributeTok{{-}t}\NormalTok{ meu{-}livro .}
\ExtensionTok{docker}\NormalTok{ run }\AttributeTok{{-}v} \VariableTok{$(}\BuiltInTok{pwd}\VariableTok{)}\NormalTok{:/workspace meu{-}livro}
\end{Highlighting}
\end{Shaded}

\section*{Scripts Auxiliares}\label{scripts-auxiliares}
\addcontentsline{toc}{section}{Scripts Auxiliares}

\markright{Scripts Auxiliares}

\subsection*{\texorpdfstring{\texttt{render\_and\_open.R}}{render\_and\_open.R}}\label{render_and_open.r}
\addcontentsline{toc}{subsection}{\texttt{render\_and\_open.R}}

\begin{Shaded}
\begin{Highlighting}[]
\CommentTok{\#!/usr/bin/env Rscript}

\CommentTok{\# Renderizar e abrir no navegador}
\FunctionTok{system}\NormalTok{(}\StringTok{"quarto render"}\NormalTok{)}
\FunctionTok{system}\NormalTok{(}\StringTok{"open docs/index.html"}\NormalTok{)  }\CommentTok{\# macOS}
\CommentTok{\# system("xdg{-}open docs/index.html")  \# Linux}
\CommentTok{\# system("start docs/index.html")  \# Windows}
\end{Highlighting}
\end{Shaded}

\subsection*{\texorpdfstring{\texttt{check\_links.sh}}{check\_links.sh}}\label{check_links.sh}
\addcontentsline{toc}{subsection}{\texttt{check\_links.sh}}

\begin{Shaded}
\begin{Highlighting}[]
\CommentTok{\#!/bin/bash}
\CommentTok{\# Verificar links quebrados}

\ExtensionTok{quarto}\NormalTok{ render}
\BuiltInTok{cd}\NormalTok{ docs}
\ExtensionTok{python} \AttributeTok{{-}m}\NormalTok{ http.server 8000 }\KeywordTok{\&}
\VariableTok{SERVER\_PID}\OperatorTok{=}\VariableTok{$!}
\FunctionTok{sleep}\NormalTok{ 2}

\CommentTok{\# Usar linkchecker ou similar}
\ExtensionTok{linkchecker}\NormalTok{ http://localhost:8000}

\BuiltInTok{kill} \VariableTok{$SERVER\_PID}
\end{Highlighting}
\end{Shaded}

\section*{Boas Práticas}\label{boas-pruxe1ticas-1}
\addcontentsline{toc}{section}{Boas Práticas}

\markright{Boas Práticas}

✅ \textbf{Fazer:} - Usar \texttt{freeze:\ auto} para cache inteligente
- Versionar dados pequenos com Git LFS - Documentar requisitos em
\texttt{README.md} - Testar localmente antes de fazer push - Usar CI/CD
para deploy automático

❌ \textbf{Evitar:} - Commitar arquivos gerados (\texttt{\_book/},
\texttt{docs/}) - Dados grandes sem Git LFS - Executar código demorado
sem cache - Dependências não documentadas

\section*{Recursos}\label{recursos}
\addcontentsline{toc}{section}{Recursos}

\markright{Recursos}

\begin{itemize}
\tightlist
\item
  \href{https://quarto.org/docs/books/}{Quarto Books Guide}
\item
  \href{https://docs.github.com/actions}{GitHub Actions Documentation}
\item
  \href{https://quartopub.com/}{Quarto Pub}
\item
  \href{https://docs.netlify.com/}{Netlify Docs}
\end{itemize}

\chapter*{Apêndice C: Cheatsheets e
Fórmulas}\label{apuxeandice-c-cheatsheets-e-fuxf3rmulas}
\addcontentsline{toc}{chapter}{Apêndice C: Cheatsheets e Fórmulas}

\markboth{Apêndice C: Cheatsheets e Fórmulas}{Apêndice C: Cheatsheets e
Fórmulas}

\section*{Distribuições de Vida
Útil}\label{distribuiuxe7uxf5es-de-vida-uxfatil}
\addcontentsline{toc}{section}{Distribuições de Vida Útil}

\markright{Distribuições de Vida Útil}

\subsection*{Distribuição
Exponencial}\label{distribuiuxe7uxe3o-exponencial-1}
\addcontentsline{toc}{subsection}{Distribuição Exponencial}

\begin{longtable}[]{@{}ll@{}}
\toprule\noalign{}
Propriedade & Fórmula \\
\midrule\noalign{}
\endhead
\bottomrule\noalign{}
\endlastfoot
\textbf{PDF} & \(f(t) = \lambda e^{-\lambda t}\) \\
\textbf{CDF} & \(F(t) = 1 - e^{-\lambda t}\) \\
\textbf{Confiabilidade} & \(R(t) = e^{-\lambda t}\) \\
\textbf{Taxa de falha} & \(\lambda(t) = \lambda\) (constante) \\
\textbf{MTTF} & \(1/\lambda\) \\
\textbf{Variância} & \(1/\lambda^2\) \\
\end{longtable}

\textbf{Quando usar:} Falhas aleatórias, componentes eletrônicos, fase
útil da vida.

\begin{center}\rule{0.5\linewidth}{0.5pt}\end{center}

\subsection*{Distribuição Weibull}\label{distribuiuxe7uxe3o-weibull-2}
\addcontentsline{toc}{subsection}{Distribuição Weibull}

\begin{longtable}[]{@{}
  >{\raggedright\arraybackslash}p{(\linewidth - 2\tabcolsep) * \real{0.5714}}
  >{\raggedright\arraybackslash}p{(\linewidth - 2\tabcolsep) * \real{0.4286}}@{}}
\toprule\noalign{}
\begin{minipage}[b]{\linewidth}\raggedright
Propriedade
\end{minipage} & \begin{minipage}[b]{\linewidth}\raggedright
Fórmula
\end{minipage} \\
\midrule\noalign{}
\endhead
\bottomrule\noalign{}
\endlastfoot
\textbf{PDF} &
\(f(t) = \frac{\beta}{\eta}\left(\frac{t}{\eta}\right)^{\beta-1} e^{-(t/\eta)^\beta}\) \\
\textbf{CDF} & \(F(t) = 1 - e^{-(t/\eta)^\beta}\) \\
\textbf{Confiabilidade} & \(R(t) = e^{-(t/\eta)^\beta}\) \\
\textbf{Taxa de falha} &
\(\lambda(t) = \frac{\beta}{\eta}\left(\frac{t}{\eta}\right)^{\beta-1}\) \\
\textbf{MTTF} & \(\eta \cdot \Gamma\left(1 + \frac{1}{\beta}\right)\) \\
\end{longtable}

\textbf{Parâmetros:} - \textbf{β (beta)}: forma - β \textless{} 1:
mortalidade infantil - β = 1: taxa constante (exponencial) - β
\textgreater{} 1: desgaste/envelhecimento - \textbf{η (eta)}: escala
(vida característica)

\textbf{Quando usar:} Maioria dos casos industriais, muito flexível.

\begin{center}\rule{0.5\linewidth}{0.5pt}\end{center}

\subsection*{Distribuição
Lognormal}\label{distribuiuxe7uxe3o-lognormal-1}
\addcontentsline{toc}{subsection}{Distribuição Lognormal}

\begin{longtable}[]{@{}
  >{\raggedright\arraybackslash}p{(\linewidth - 2\tabcolsep) * \real{0.5714}}
  >{\raggedright\arraybackslash}p{(\linewidth - 2\tabcolsep) * \real{0.4286}}@{}}
\toprule\noalign{}
\begin{minipage}[b]{\linewidth}\raggedright
Propriedade
\end{minipage} & \begin{minipage}[b]{\linewidth}\raggedright
Fórmula
\end{minipage} \\
\midrule\noalign{}
\endhead
\bottomrule\noalign{}
\endlastfoot
\textbf{PDF} &
\(f(t) = \frac{1}{t\sigma\sqrt{2\pi}} e^{-\frac{(\ln t - \mu)^2}{2\sigma^2}}\) \\
\textbf{CDF} & \(F(t) = \Phi\left(\frac{\ln t - \mu}{\sigma}\right)\) \\
\textbf{Confiabilidade} &
\(R(t) = 1 - \Phi\left(\frac{\ln t - \mu}{\sigma}\right)\) \\
\textbf{Mediana} & \(e^\mu\) \\
\textbf{MTTF} & \(e^{\mu + \sigma^2/2}\) \\
\end{longtable}

\textbf{Quando usar:} Fadiga, degradação multiplicativa, tempos de
reparo.

\begin{center}\rule{0.5\linewidth}{0.5pt}\end{center}

\subsection*{Distribuição Gamma}\label{distribuiuxe7uxe3o-gamma}
\addcontentsline{toc}{subsection}{Distribuição Gamma}

\begin{longtable}[]{@{}
  >{\raggedright\arraybackslash}p{(\linewidth - 2\tabcolsep) * \real{0.5714}}
  >{\raggedright\arraybackslash}p{(\linewidth - 2\tabcolsep) * \real{0.4286}}@{}}
\toprule\noalign{}
\begin{minipage}[b]{\linewidth}\raggedright
Propriedade
\end{minipage} & \begin{minipage}[b]{\linewidth}\raggedright
Fórmula
\end{minipage} \\
\midrule\noalign{}
\endhead
\bottomrule\noalign{}
\endlastfoot
\textbf{PDF} &
\(f(t) = \frac{\lambda^\alpha}{\Gamma(\alpha)} t^{\alpha-1} e^{-\lambda t}\) \\
\textbf{MTTF} & \(\alpha / \lambda\) \\
\end{longtable}

\textbf{Quando usar:} Processos de Poisson, soma de exponenciais.

\begin{center}\rule{0.5\linewidth}{0.5pt}\end{center}

\section*{Métricas de
Confiabilidade}\label{muxe9tricas-de-confiabilidade}
\addcontentsline{toc}{section}{Métricas de Confiabilidade}

\markright{Métricas de Confiabilidade}

\begin{longtable}[]{@{}
  >{\raggedright\arraybackslash}p{(\linewidth - 6\tabcolsep) * \real{0.2500}}
  >{\raggedright\arraybackslash}p{(\linewidth - 6\tabcolsep) * \real{0.2500}}
  >{\raggedright\arraybackslash}p{(\linewidth - 6\tabcolsep) * \real{0.2500}}
  >{\raggedright\arraybackslash}p{(\linewidth - 6\tabcolsep) * \real{0.2500}}@{}}
\toprule\noalign{}
\begin{minipage}[b]{\linewidth}\raggedright
Métrica
\end{minipage} & \begin{minipage}[b]{\linewidth}\raggedright
Símbolo
\end{minipage} & \begin{minipage}[b]{\linewidth}\raggedright
Fórmula
\end{minipage} & \begin{minipage}[b]{\linewidth}\raggedright
Unidade
\end{minipage} \\
\midrule\noalign{}
\endhead
\bottomrule\noalign{}
\endlastfoot
\textbf{Confiabilidade} & \(R(t)\) & \(P(T > t)\) & - \\
\textbf{Taxa de falha} & \(\lambda(t)\) & \(\frac{f(t)}{R(t)}\) &
falhas/tempo \\
\textbf{MTTF} & - & \(\int_0^\infty R(t) dt\) & tempo \\
\textbf{MTBF} & - & Tempo médio entre falhas & tempo \\
\textbf{MTTR} & - & Tempo médio de reparo & tempo \\
\textbf{Disponibilidade} & \(A\) & \(\frac{MTBF}{MTBF + MTTR}\) & - \\
\end{longtable}

\begin{center}\rule{0.5\linewidth}{0.5pt}\end{center}

\section*{Sistemas}\label{sistemas}
\addcontentsline{toc}{section}{Sistemas}

\markright{Sistemas}

\subsection*{Sistema Série}\label{sistema-suxe9rie-1}
\addcontentsline{toc}{subsection}{Sistema Série}

\[R_{\text{série}}(t) = \prod_{i=1}^n R_i(t)\]

\textbf{MTTF:} Não há fórmula simples (depende das distribuições).

\textbf{Nota:} Confiabilidade do sistema é \textbf{menor} que a de
qualquer componente.

\begin{center}\rule{0.5\linewidth}{0.5pt}\end{center}

\subsection*{Sistema Paralelo
(Redundância)}\label{sistema-paralelo-redunduxe2ncia}
\addcontentsline{toc}{subsection}{Sistema Paralelo (Redundância)}

\[R_{\text{paralelo}}(t) = 1 - \prod_{i=1}^n [1 - R_i(t)]\]

\textbf{Componentes idênticos:}

\[R_{\text{paralelo}}(t) = 1 - [1 - R(t)]^n\]

\begin{center}\rule{0.5\linewidth}{0.5pt}\end{center}

\subsection*{Sistema k-out-of-n}\label{sistema-k-out-of-n-1}
\addcontentsline{toc}{subsection}{Sistema k-out-of-n}

\[R_{k/n}(t) = \sum_{i=k}^n \binom{n}{i} R(t)^i [1-R(t)]^{n-i}\]

\textbf{Casos especiais:} - n-out-of-n: sistema série - 1-out-of-n:
sistema paralelo

\begin{center}\rule{0.5\linewidth}{0.5pt}\end{center}

\section*{Testes Estatísticos}\label{testes-estatuxedsticos}
\addcontentsline{toc}{section}{Testes Estatísticos}

\markright{Testes Estatísticos}

\subsection*{Teste de
Kolmogorov-Smirnov}\label{teste-de-kolmogorov-smirnov}
\addcontentsline{toc}{subsection}{Teste de Kolmogorov-Smirnov}

\textbf{Hipóteses:} - \(H_0\): Dados seguem distribuição especificada -
\(H_1\): Dados não seguem

\textbf{Estatística:}

\[D_n = \sup_t |F_n(t) - F(t)|\]

\textbf{Decisão:} Rejeitar \(H_0\) se \(D_n > D_{\alpha, n}\) (valor
crítico).

\begin{center}\rule{0.5\linewidth}{0.5pt}\end{center}

\subsection*{Teste Log-Rank (Comparação de
Curvas)}\label{teste-log-rank-comparauxe7uxe3o-de-curvas}
\addcontentsline{toc}{subsection}{Teste Log-Rank (Comparação de Curvas)}

\textbf{Hipóteses:} - \(H_0\): Curvas de sobrevivência são iguais -
\(H_1\): Curvas diferem

\textbf{Estatística:}

\[\chi^2 = \sum_{i=1}^g \frac{(O_i - E_i)^2}{E_i}\]

\begin{center}\rule{0.5\linewidth}{0.5pt}\end{center}

\section*{Intervalos de Confiança}\label{intervalos-de-confianuxe7a-1}
\addcontentsline{toc}{section}{Intervalos de Confiança}

\markright{Intervalos de Confiança}

\subsection*{IC para Taxa de Falha
(Exponencial)}\label{ic-para-taxa-de-falha-exponencial}
\addcontentsline{toc}{subsection}{IC para Taxa de Falha (Exponencial)}

\[\left[\frac{2n}{{\chi^2_{1-\alpha/2, 2n}}}, \frac{2n}{{\chi^2_{\alpha/2, 2n}}}\right] \cdot \frac{1}{T}\]

onde \(n\) = número de falhas, \(T\) = tempo total.

\begin{center}\rule{0.5\linewidth}{0.5pt}\end{center}

\subsection*{IC Bootstrap (Genérico)}\label{ic-bootstrap-genuxe9rico}
\addcontentsline{toc}{subsection}{IC Bootstrap (Genérico)}

\begin{enumerate}
\def\labelenumi{\arabic{enumi}.}
\tightlist
\item
  Reamostrar dados com reposição (B vezes)
\item
  Calcular estatística para cada amostra
\item
  IC: percentis \(\alpha/2\) e \(1-\alpha/2\)
\end{enumerate}

\begin{center}\rule{0.5\linewidth}{0.5pt}\end{center}

\section*{Códigos R Rápidos}\label{cuxf3digos-r-ruxe1pidos}
\addcontentsline{toc}{section}{Códigos R Rápidos}

\markright{Códigos R Rápidos}

\subsection*{Ajustar Weibull}\label{ajustar-weibull}
\addcontentsline{toc}{subsection}{Ajustar Weibull}

\begin{Shaded}
\begin{Highlighting}[]
\FunctionTok{library}\NormalTok{(fitdistrplus)}
\NormalTok{fit }\OtherTok{\textless{}{-}} \FunctionTok{fitdist}\NormalTok{(dados, }\StringTok{"weibull"}\NormalTok{)}
\FunctionTok{summary}\NormalTok{(fit)}
\end{Highlighting}
\end{Shaded}

\subsection*{Kaplan-Meier}\label{kaplan-meier}
\addcontentsline{toc}{subsection}{Kaplan-Meier}

\begin{Shaded}
\begin{Highlighting}[]
\FunctionTok{library}\NormalTok{(survival)}
\NormalTok{km }\OtherTok{\textless{}{-}} \FunctionTok{survfit}\NormalTok{(}\FunctionTok{Surv}\NormalTok{(tempo, status) }\SpecialCharTok{\textasciitilde{}} \DecValTok{1}\NormalTok{)}
\FunctionTok{plot}\NormalTok{(km, }\AttributeTok{xlab =} \StringTok{"Tempo"}\NormalTok{, }\AttributeTok{ylab =} \StringTok{"R(t)"}\NormalTok{)}
\end{Highlighting}
\end{Shaded}

\subsection*{Modelo Cox}\label{modelo-cox}
\addcontentsline{toc}{subsection}{Modelo Cox}

\begin{Shaded}
\begin{Highlighting}[]
\NormalTok{cox }\OtherTok{\textless{}{-}} \FunctionTok{coxph}\NormalTok{(}\FunctionTok{Surv}\NormalTok{(tempo, status) }\SpecialCharTok{\textasciitilde{}}\NormalTok{ covariate, }\AttributeTok{data =}\NormalTok{ df)}
\FunctionTok{summary}\NormalTok{(cox)}
\end{Highlighting}
\end{Shaded}

\subsection*{Simular Weibull}\label{simular-weibull}
\addcontentsline{toc}{subsection}{Simular Weibull}

\begin{Shaded}
\begin{Highlighting}[]
\NormalTok{dados }\OtherTok{\textless{}{-}} \FunctionTok{rweibull}\NormalTok{(}\AttributeTok{n =} \DecValTok{100}\NormalTok{, }\AttributeTok{shape =} \DecValTok{2}\NormalTok{, }\AttributeTok{scale =} \DecValTok{1000}\NormalTok{)}
\end{Highlighting}
\end{Shaded}

\subsection*{Calcular Confiabilidade}\label{calcular-confiabilidade}
\addcontentsline{toc}{subsection}{Calcular Confiabilidade}

\begin{Shaded}
\begin{Highlighting}[]
\NormalTok{R }\OtherTok{\textless{}{-}} \ControlFlowTok{function}\NormalTok{(t, beta, eta) }\FunctionTok{exp}\NormalTok{(}\SpecialCharTok{{-}}\NormalTok{(t}\SpecialCharTok{/}\NormalTok{eta)}\SpecialCharTok{\^{}}\NormalTok{beta)}
\FunctionTok{R}\NormalTok{(}\DecValTok{500}\NormalTok{, }\AttributeTok{beta =} \DecValTok{2}\NormalTok{, }\AttributeTok{eta =} \DecValTok{1000}\NormalTok{)}
\end{Highlighting}
\end{Shaded}

\begin{center}\rule{0.5\linewidth}{0.5pt}\end{center}

\section*{Fórmulas de Manutenção}\label{fuxf3rmulas-de-manutenuxe7uxe3o}
\addcontentsline{toc}{section}{Fórmulas de Manutenção}

\markright{Fórmulas de Manutenção}

\subsection*{Custo por Hora
(Preventiva)}\label{custo-por-hora-preventiva}
\addcontentsline{toc}{subsection}{Custo por Hora (Preventiva)}

\[C(T_p) = \frac{C_p R(T_p) + C_f [1 - R(T_p)]}{\int_0^{T_p} R(t) dt}\]

\begin{itemize}
\tightlist
\item
  \(C_p\): custo preventiva
\item
  \(C_f\): custo falha
\item
  \(T_p\): intervalo preventivo
\end{itemize}

\begin{center}\rule{0.5\linewidth}{0.5pt}\end{center}

\subsection*{Disponibilidade com
Manutenção}\label{disponibilidade-com-manutenuxe7uxe3o}
\addcontentsline{toc}{subsection}{Disponibilidade com Manutenção}

\[A = \frac{MTBF}{MTBF + MTTR}\]

\begin{center}\rule{0.5\linewidth}{0.5pt}\end{center}

\subsection*{Taxa de Falha Média}\label{taxa-de-falha-muxe9dia}
\addcontentsline{toc}{subsection}{Taxa de Falha Média}

\[\overline{\lambda} = \frac{1}{MTBF}\]

\begin{center}\rule{0.5\linewidth}{0.5pt}\end{center}

\section*{Interpretação Visual}\label{interpretauxe7uxe3o-visual}
\addcontentsline{toc}{section}{Interpretação Visual}

\markright{Interpretação Visual}

\subsection*{Papel de Weibull}\label{papel-de-weibull}
\addcontentsline{toc}{subsection}{Papel de Weibull}

\textbf{Eixos:} - X: \(\ln(t)\) - Y: \(\ln(-\ln(R(t)))\)

\textbf{Reta indica Weibull:} - Inclinação = \(\beta\) - Intercepto
relacionado a \(\eta\)

\begin{center}\rule{0.5\linewidth}{0.5pt}\end{center}

\subsection*{Curva Bathtub (Banheira)}\label{curva-bathtub-banheira}
\addcontentsline{toc}{subsection}{Curva Bathtub (Banheira)}

\begin{verbatim}
Taxa de Falha λ(t)
  │
  │ \___________________/─────
  │   ↑       ↑        ↑
  │   │       │        │
  │ Infant  Useful   Wear-out
  │ Mortality Life     Phase
  └─────────────────────────→ Tempo
\end{verbatim}

\begin{itemize}
\tightlist
\item
  \textbf{Fase 1:} β \textless{} 1 (mortalidade infantil)
\item
  \textbf{Fase 2:} β ≈ 1 (vida útil)
\item
  \textbf{Fase 3:} β \textgreater{} 1 (desgaste)
\end{itemize}

\begin{center}\rule{0.5\linewidth}{0.5pt}\end{center}

\section*{Conversões Úteis}\label{conversuxf5es-uxfateis}
\addcontentsline{toc}{section}{Conversões Úteis}

\markright{Conversões Úteis}

\subsection*{Parâmetros Weibull}\label{paruxe2metros-weibull}
\addcontentsline{toc}{subsection}{Parâmetros Weibull}

\textbf{De β e η para λ (exponencial) quando β = 1:}

\[\lambda = \frac{1}{\eta}\]

\textbf{MTTF para Weibull:}

\[MTTF = \eta \cdot \Gamma\left(1 + \frac{1}{\beta}\right)\]

\textbf{Valores de Γ:} - Γ(1) = 1 - Γ(1.5) ≈ 0.886 - Γ(2) = 1

\begin{center}\rule{0.5\linewidth}{0.5pt}\end{center}

\subsection*{Confiabilidade para Falha}\label{confiabilidade-para-falha}
\addcontentsline{toc}{subsection}{Confiabilidade para Falha}

\[F(t) = 1 - R(t)\]

\begin{center}\rule{0.5\linewidth}{0.5pt}\end{center}

\subsection*{Hazard Rate para PDF}\label{hazard-rate-para-pdf}
\addcontentsline{toc}{subsection}{Hazard Rate para PDF}

\[f(t) = \lambda(t) \cdot R(t)\]

\begin{center}\rule{0.5\linewidth}{0.5pt}\end{center}

\section*{Atalhos de
Interpretação}\label{atalhos-de-interpretauxe7uxe3o}
\addcontentsline{toc}{section}{Atalhos de Interpretação}

\markright{Atalhos de Interpretação}

\begin{longtable}[]{@{}lll@{}}
\toprule\noalign{}
Situação & Distribuição Provável & β Típico \\
\midrule\noalign{}
\endhead
\bottomrule\noalign{}
\endlastfoot
Componente eletrônico novo & Exponencial & 1.0 \\
Rolamento industrial & Weibull & 1.5 - 3.0 \\
Fadiga mecânica & Lognormal & - \\
Software bugs & Exponencial ou Weibull (β \textless{} 1) & 0.5 - 1.0 \\
Equipamento sem manutenção & Weibull & 2.5 - 4.0 \\
\end{longtable}

\begin{center}\rule{0.5\linewidth}{0.5pt}\end{center}

\section*{Referências Rápidas}\label{referuxeancias-ruxe1pidas}
\addcontentsline{toc}{section}{Referências Rápidas}

\markright{Referências Rápidas}

\subsection*{Pacotes R}\label{pacotes-r}
\addcontentsline{toc}{subsection}{Pacotes R}

\begin{itemize}
\tightlist
\item
  \texttt{survival} - Análise de sobrevivência
\item
  \texttt{flexsurv} - Modelos paramétricos
\item
  \texttt{WeibullR} - Análise Weibull
\item
  \texttt{fitdistrplus} - Ajuste de distribuições
\item
  \texttt{caret} - Machine learning
\end{itemize}

\subsection*{Websites}\label{websites}
\addcontentsline{toc}{subsection}{Websites}

\begin{itemize}
\tightlist
\item
  \href{http://reliawiki.org/}{reliawiki.org}
\item
  \href{https://www.weibull.com/}{Weibull.com}
\item
  \href{https://www.itl.nist.gov/div898/handbook/}{NIST Engineering
  Statistics Handbook}
\end{itemize}

\begin{center}\rule{0.5\linewidth}{0.5pt}\end{center}

\textbf{💡 Dica:} Imprima este apêndice para referência rápida durante
análises!

\bookmarksetup{startatroot}

\chapter*{Referências}\label{referuxeancias}
\addcontentsline{toc}{chapter}{Referências}

\markboth{Referências}{Referências}

\phantomsection\label{refs}
\begin{CSLReferences}{0}{0}
\bibitem[\citeproctext]{ref-correa2012}
\CSLLeftMargin{{[}1{]} }%
\CSLRightInline{H. L. Corrêa e C. A. Corrêa, \emph{Administração de
produção e operações: manufatura e serviços; uma abordagem estratégica}.
Atlas, 2012. Disponível em:
\url{https://books.google.com.br/books?id=eptZlQEACAAJ}}

\bibitem[\citeproctext]{ref-fernandes2021}
\CSLLeftMargin{{[}2{]} }%
\CSLRightInline{R. de S. Fernandes, L. R. C. de Sousa, e T. L. Santos,
{``Analysis, investigation and evaluation of quality management in the
mining process: a case study on divergencies in iron ore stock
deviations''}, \emph{Revista Produção Online}, vol. 21, nº 3, p.
770--793, 2021, doi:
\href{https://doi.org/10.14488/1676-1901.v21i3.4252}{10.14488/1676-1901.v21i3.4252}.}

\bibitem[\citeproctext]{ref-fernandes2023}
\CSLLeftMargin{{[}3{]} }%
\CSLRightInline{R. de S. Fernandes, T. R. Rocha, J. M. Coelho, e D. F.
Andrade, {``Development of a measurement instrument to evaluate
integrated management systems and differences in perception: an approach
to item response theory and the quality management process''},
\emph{Production}, vol. 33, 2023, doi:
\href{https://doi.org/10.1590/0103-6513.20220069}{10.1590/0103-6513.20220069}.}

\bibitem[\citeproctext]{ref-gackowiec2020}
\CSLLeftMargin{{[}4{]} }%
\CSLRightInline{P. Gackowiec, M. Podobińska-Staniec, E. Brzychczy, C.
Kúhlbach, e T. Özver, {``Review of key performance indicators for
process monitoring in the mining industry''}, \emph{Energies}, vol. 13,
nº 19, p. 5169, 2020, doi:
\href{https://doi.org/10.3390/en13195169}{10.3390/en13195169}.}

\bibitem[\citeproctext]{ref-carvalho2014}
\CSLLeftMargin{{[}5{]} }%
\CSLRightInline{P. S. L. Carvalho, M. M. Silva, M. A. R. Rocio, e J.
Moszkowicz, {``Minério de ferro''}. 2014.}

\bibitem[\citeproctext]{ref-luz2018}
\CSLLeftMargin{{[}6{]} }%
\CSLRightInline{A. B. da Luz e F. A. F. Lins, \emph{Introdução ao
tratamento de minérios}. CETEM/MCTIC, 2018.}

\bibitem[\citeproctext]{ref-rocha2021}
\CSLLeftMargin{{[}7{]} }%
\CSLRightInline{C. S. Rocha, E. L. de S. S. Cardoso, R. de S. Fernandes,
N. C. N. M. Branco, e R. M. de N. Luz, {``Quality assessment of
divergences causes in weighing copper ore at a mining company in the
amazon region''}, \emph{International Journal of Development Research},
vol. 11, nº 8, p. 49633--49639, 2021, doi:
\href{https://doi.org/10.37118/ijdr.22578.08.2021}{10.37118/ijdr.22578.08.2021}.}

\bibitem[\citeproctext]{ref-fernandes2023b}
\CSLLeftMargin{{[}8{]} }%
\CSLRightInline{R. de S. Fernandes, J. C. de C. Sousa, e R. M. de N.
Luz, {``Integrated management analysis of production performance,
maintenance, and PDCA cycle: a case study of the IPCC mining process''},
\emph{Revista Produção Online}, vol. 22, nº 2, p. 2731--2762, 2023, doi:
\href{https://doi.org/10.14488/1676-1901.v22i2.4661}{10.14488/1676-1901.v22i2.4661}.}

\bibitem[\citeproctext]{ref-osanloo2020}
\CSLLeftMargin{{[}9{]} }%
\CSLRightInline{M. Osanloo e M. Paricheh, {``In-pit crushing and
conveying technology in open-pit mining operations: a literature review
and research agenda''}, \emph{International Journal of Mining,
Reclamation and Environment}, vol. 34, nº 6, p. 430--457, 2020, doi:
\href{https://doi.org/10.1080/17480930.2019.1565054}{10.1080/17480930.2019.1565054}.}

\bibitem[\citeproctext]{ref-mining2015}
\CSLLeftMargin{{[}10{]} }%
\CSLRightInline{International Council on Mining and Metals, {``Critical
Control Management: Implementation Guide''}. 2015. Disponível em:
\url{https://books.google.com.br/books?id=kroRDAEACAAJ}}

\bibitem[\citeproctext]{ref-iso55000}
\CSLLeftMargin{{[}11{]} }%
\CSLRightInline{ABNT, {``ISO-55000:2014 Gestão de ativos: visão geral,
princípios e terminologia''}. 2014.}

\bibitem[\citeproctext]{ref-iso5462}
\CSLLeftMargin{{[}12{]} }%
\CSLRightInline{ABNT, {``ISO-5462:1994 Reliability and maintainability
-- Terminology''}. 1994.}

\bibitem[\citeproctext]{ref-pascual2016}
\CSLLeftMargin{{[}13{]} }%
\CSLRightInline{R. Pascual, R. Madariaga, G. Santelices, D. Godoy, e E.
L. Droguett, {``A structured methodology to optimise throughput of
production lines''}, \emph{International Journal of Mining, Reclamation
and Environment}, vol. 30, nº 1, p. 25--36, 2016, doi:
\href{https://doi.org/10.1080/17480930.2014.962235}{10.1080/17480930.2014.962235}.}

\bibitem[\citeproctext]{ref-qiao2017}
\CSLLeftMargin{{[}14{]} }%
\CSLRightInline{G. Qiao, C. Schlenoff, e B. A. Weiss, {``Quick
positional health assessment for industrial robot prognostics and health
management (PHM)''}, em \emph{2017 IEEE International Conference on
Robotics and Automation (ICRA)}, 2017.}

\bibitem[\citeproctext]{ref-slack2013}
\CSLLeftMargin{{[}15{]} }%
\CSLRightInline{N. Slack, S. Chambers, R. Johnston, e A. Betts,
\emph{Gerenciamento de Operações e de Processos: Princípios e práticas
de impacto estratégico}, 2nd ed. Bookman Editora, 2013. Disponível em:
\url{https://books.google.com.br/books?id=wLo3AgAAQBAJ}}

\bibitem[\citeproctext]{ref-zampolli2019}
\CSLLeftMargin{{[}16{]} }%
\CSLRightInline{M. Zampolli, \emph{Gestão de ativos: guia para a
aplicação da norma ABNT NBR ISO 55001 considerando as diretrizes da ISO
55002:2018}, 2nd ed. 2019. Disponível em:
\url{https://abcobre.org.br/wp-content/uploads/2021/04/gestao-de-ativos-guia-para-aplicacao-da-norma-abnt-nbr-iso-55001.pdf}}

\bibitem[\citeproctext]{ref-doyen2004}
\CSLLeftMargin{{[}17{]} }%
\CSLRightInline{L. Doyen e O. Gaudoin, {``Classes of imperfect repair
models based on reduction of failure intensity or virtual age''},
\emph{Reliability Engineering \& System Safety}, vol. 84, nº 1, p.
45--56, 2004, doi:
\href{https://doi.org/10.1016/S0951-8320(03)00173-X}{10.1016/S0951-8320(03)00173-X}.}

\bibitem[\citeproctext]{ref-gilardoni2007}
\CSLLeftMargin{{[}18{]} }%
\CSLRightInline{G. L. Gilardoni e E. A. Colosimo, {``Optimal maintenance
time for repairable systems''}, \emph{Journal of Quality Technology},
vol. 39, nº 1, p. 48--53, 2007, doi:
\href{https://doi.org/10.1080/00224065.2007.11917672}{10.1080/00224065.2007.11917672}.}

\bibitem[\citeproctext]{ref-toledo2016}
\CSLLeftMargin{{[}19{]} }%
\CSLRightInline{M. L. G. de Toledo, M. A. Freitas, E. A. Colosimo, e G.
L. Gilardoni, {``Optimal periodic maintenance policy under imperfect
repair: A case study on the engines of off-road vehicles''}, \emph{IIE
Transactions}, vol. 48, nº 8, p. 747--758, 2016, doi:
\href{https://doi.org/10.1080/0740817X.2016.1147663}{10.1080/0740817X.2016.1147663}.}

\bibitem[\citeproctext]{ref-gilardoni2016}
\CSLLeftMargin{{[}20{]} }%
\CSLRightInline{G. L. Gilardoni, M. L. G. de Toledo, M. A. Freitas, e E.
A. Colosimo, {``Dynamics of an optimal maintenance policy for imperfect
repair models''}, \emph{European Journal of Operational Research}, vol.
248, nº 3, p. 1104--1112, 2016, doi:
\href{https://doi.org/10.1016/j.ejor.2015.07.056}{10.1016/j.ejor.2015.07.056}.}

\bibitem[\citeproctext]{ref-doyen2011}
\CSLLeftMargin{{[}21{]} }%
\CSLRightInline{L. Doyen e O. Gaudoin, {``Modeling and Assessment of
Aging and Efficiency of Corrective and Planned Preventive
Maintenance''}, \emph{IEEE Transactions on Reliability}, vol. 60, nº 4,
p. 759--769, 2011, doi:
\href{https://doi.org/10.1109/TR.2011.2171115}{10.1109/TR.2011.2171115}.}

\bibitem[\citeproctext]{ref-kang2016}
\CSLLeftMargin{{[}22{]} }%
\CSLRightInline{N. Kang, C. Zhao, J. Li, e J. A. Horst, {``A
Hierarchical structure of key performance indicators for operation
management and continuous improvement in production systems''},
\emph{International Journal of Production Research}, vol. 54, nº 21, p.
6333--6350, 2016, doi:
\href{https://doi.org/10.1080/00207543.2015.1136082}{10.1080/00207543.2015.1136082}.}

\end{CSLReferences}

\section*{Recursos Adicionais}\label{recursos-adicionais-1}
\addcontentsline{toc}{section}{Recursos Adicionais}

\markright{Recursos Adicionais}

\begin{itemize}
\tightlist
\item
  \href{https://quarto.org/}{Documentação Quarto}
\item
  \href{https://r4ds.hadley.nz/}{R for Data Science}
\item
  \href{https://shiny.posit.co/}{Shiny Documentation}
\item
  \href{https://supabase.com/docs}{Supabase Documentation}
\end{itemize}

\bookmarksetup{startatroot}

\chapter*{Créditos e Agradecimentos}\label{cruxe9ditos-e-agradecimentos}
\addcontentsline{toc}{chapter}{Créditos e Agradecimentos}

\markboth{Créditos e Agradecimentos}{Créditos e Agradecimentos}

\section*{Autor}\label{autor}
\addcontentsline{toc}{section}{Autor}

\markright{Autor}

\textbf{Rafael da Silva Fernandes} Engenheiro, Analista de Dados e
Entusiasta de R

\begin{itemize}
\tightlist
\item
  📧 Email: rafael@example.com
\item
  💼 LinkedIn:
  \href{https://linkedin.com/in/rafaelfernandes}{linkedin.com/in/rafaelfernandes}
\item
  🐙 GitHub:
  \href{https://github.com/rafaelfernandes}{github.com/rafaelfernandes}
\end{itemize}

\section*{Sobre o Autor}\label{sobre-o-autor}
\addcontentsline{toc}{section}{Sobre o Autor}

\markright{Sobre o Autor}

Rafael da Silva Fernandes é engenheiro com experiência em análise de
confiabilidade, manutenção industrial e ciência de dados. Este livro é
fruto de anos de prática aplicando métodos estatísticos para otimização
de estratégias de manutenção em ambientes industriais.

\section*{Agradecimentos}\label{agradecimentos-1}
\addcontentsline{toc}{section}{Agradecimentos}

\markright{Agradecimentos}

Este livro não seria possível sem o apoio e contribuição de diversas
pessoas e comunidades:

\subsection*{Comunidade R}\label{comunidade-r}
\addcontentsline{toc}{subsection}{Comunidade R}

Agradeço à \textbf{comunidade R} e aos desenvolvedores dos pacotes
open-source que tornaram este trabalho possível:

\begin{itemize}
\tightlist
\item
  \textbf{Hadley Wickham} e equipe do \texttt{tidyverse}
\item
  \textbf{Terry Therneau} pelo pacote \texttt{survival}
\item
  \textbf{Christopher Jackson} pelo \texttt{flexsurv}
\item
  \textbf{David Gohel} pelo \texttt{officer} e \texttt{flextable}
\item
  Desenvolvedores do \textbf{Quarto} e \textbf{Posit}
\end{itemize}

\subsection*{Revisores e Colaboradores}\label{revisores-e-colaboradores}
\addcontentsline{toc}{subsection}{Revisores e Colaboradores}

Agradecimentos especiais a:

\begin{itemize}
\tightlist
\item
  Colegas de trabalho que forneceram feedback valioso
\item
  Beta readers que testaram os exemplos
\item
  Comunidade GitHub que reportou issues e sugeriu melhorias
\end{itemize}

\subsection*{Instituições}\label{instituiuxe7uxf5es}
\addcontentsline{toc}{subsection}{Instituições}

\begin{itemize}
\tightlist
\item
  \textbf{Empresas} que forneceram dados anonimizados para estudos de
  caso
\item
  \textbf{Universidades} que promovem pesquisa em confiabilidade
\item
  \textbf{Comunidades online} (Stack Overflow, RStudio Community, Cross
  Validated)
\end{itemize}

\subsection*{Família}\label{famuxedlia}
\addcontentsline{toc}{subsection}{Família}

Gratidão à minha família pelo apoio durante o desenvolvimento deste
livro.

\section*{Ferramentas Utilizadas}\label{ferramentas-utilizadas}
\addcontentsline{toc}{section}{Ferramentas Utilizadas}

\markright{Ferramentas Utilizadas}

Este livro foi desenvolvido com:

\begin{itemize}
\tightlist
\item
  \textbf{Quarto} --- Sistema de publicação científica
  (\href{https://quarto.org}{quarto.org})
\item
  \textbf{R} --- Linguagem de programação estatística
  (\href{https://www.r-project.org}{r-project.org})
\item
  \textbf{RStudio / VS Code} --- IDEs para desenvolvimento
\item
  \textbf{GitHub} --- Versionamento e colaboração
\item
  \textbf{Git} --- Controle de versão
\end{itemize}

\subsection*{Pacotes R Principais}\label{pacotes-r-principais}
\addcontentsline{toc}{subsection}{Pacotes R Principais}

\begin{Shaded}
\begin{Highlighting}[]
\NormalTok{tidyverse      }\CommentTok{\# Manipulação e visualização de dados}
\NormalTok{survival       }\CommentTok{\# Análise de sobrevivência}
\NormalTok{flexsurv       }\CommentTok{\# Modelos paramétricos flexíveis}
\NormalTok{WeibullR       }\CommentTok{\# Análise Weibull}
\NormalTok{fitdistrplus   }\CommentTok{\# Ajuste de distribuições}
\NormalTok{ggplot2        }\CommentTok{\# Visualizações}
\NormalTok{knitr          }\CommentTok{\# Relatórios reproduzíveis}
\end{Highlighting}
\end{Shaded}

\section*{Licença e Uso}\label{licenuxe7a-e-uso}
\addcontentsline{toc}{section}{Licença e Uso}

\markright{Licença e Uso}

\subsection*{Conteúdo}\label{conteuxfado}
\addcontentsline{toc}{subsection}{Conteúdo}

Este livro está licenciado sob
\href{https://creativecommons.org/licenses/by-nc-sa/4.0/}{Creative
Commons BY-NC-SA 4.0}.

Você pode: - ✅ Compartilhar --- copiar e redistribuir - ✅ Adaptar ---
remixar e transformar

Desde que: - 📝 Atribua crédito apropriado - 🚫 Não use para fins
comerciais - 🔄 Compartilhe adaptações sob a mesma licença

\subsection*{Código}\label{cuxf3digo}
\addcontentsline{toc}{subsection}{Código}

Todo o código R apresentado neste livro está licenciado sob
\href{https://opensource.org/licenses/MIT}{MIT License} e pode ser usado
livremente em projetos pessoais e comerciais.

\section*{Como Citar este Livro}\label{como-citar-este-livro}
\addcontentsline{toc}{section}{Como Citar este Livro}

\markright{Como Citar este Livro}

\subsection*{Formato ABNT}\label{formato-abnt}
\addcontentsline{toc}{subsection}{Formato ABNT}

\begin{verbatim}
FERNANDES, Rafael da Silva. Confiabilidade e Manutenção: Aplicações em R.
1. ed. [S.l.]: Publicação Independente, 2025. Disponível em:
https://seu-site.com/livro. Acesso em: DD mmm. AAAA.
\end{verbatim}

\subsection*{Formato BibTeX}\label{formato-bibtex}
\addcontentsline{toc}{subsection}{Formato BibTeX}

\begin{Shaded}
\begin{Highlighting}[]
\VariableTok{@book}\NormalTok{\{}\OtherTok{fernandes2025confiabilidade}\NormalTok{,}
  \DataTypeTok{title}\NormalTok{     = \{Confiabilidade e Manutenção: Aplicações em R\},}
  \DataTypeTok{author}\NormalTok{    = \{Fernandes, Rafael da Silva\},}
  \DataTypeTok{year}\NormalTok{      = \{2025\},}
  \DataTypeTok{publisher}\NormalTok{ = \{Publicação Independente\},}
  \DataTypeTok{url}\NormalTok{       = \{https://seu{-}site.com/livro\},}
  \DataTypeTok{note}\NormalTok{      = \{Acesso em: DD/MM/AAAA\}}
\NormalTok{\}}
\end{Highlighting}
\end{Shaded}

\section*{Contribuições}\label{contribuiuxe7uxf5es}
\addcontentsline{toc}{section}{Contribuições}

\markright{Contribuições}

Contribuições são bem-vindas! Se você encontrou erros, tem sugestões ou
quer contribuir com exemplos:

\begin{enumerate}
\def\labelenumi{\arabic{enumi}.}
\tightlist
\item
  \textbf{Reporte issues:}
  \href{https://github.com/seu-usuario/QuartoBook/issues}{GitHub Issues}
\item
  \textbf{Envie pull requests:}
  \href{https://github.com/seu-usuario/QuartoBook}{GitHub}
\item
  \textbf{Entre em contato:} rafael@example.com
\end{enumerate}

\section*{Changelog}\label{changelog}
\addcontentsline{toc}{section}{Changelog}

\markright{Changelog}

\subsection*{Versão 1.0 (Novembro
2025)}\label{versuxe3o-1.0-novembro-2025}
\addcontentsline{toc}{subsection}{Versão 1.0 (Novembro 2025)}

\begin{itemize}
\tightlist
\item
  ✨ Lançamento inicial
\item
  📚 12 capítulos principais
\item
  📖 3 apêndices
\item
  💻 Exemplos reproduzíveis em R
\item
  📊 Estudos de caso industriais
\end{itemize}

\subsection*{Planejado para Versões
Futuras}\label{planejado-para-versuxf5es-futuras}
\addcontentsline{toc}{subsection}{Planejado para Versões Futuras}

\begin{itemize}
\tightlist
\item
  🔄 Capítulo sobre análise bayesiana
\item
  🤖 Machine learning avançado para manutenção preditiva
\item
  📱 Aplicações Shiny interativas
\item
  🌐 Tradução para inglês
\end{itemize}

\section*{Feedback}\label{feedback}
\addcontentsline{toc}{section}{Feedback}

\markright{Feedback}

Sua opinião é importante! Se você gostou do livro, encontrou algo útil
ou tem sugestões de melhoria:

\begin{itemize}
\tightlist
\item
  ⭐ Deixe uma estrela no
  \href{https://github.com/seu-usuario/QuartoBook}{GitHub}
\item
  💬 Compartilhe nas redes sociais
\item
  📧 Envie feedback para rafael@example.com
\item
  📝 Escreva uma review
\end{itemize}

\section*{Recursos Adicionais}\label{recursos-adicionais-2}
\addcontentsline{toc}{section}{Recursos Adicionais}

\markright{Recursos Adicionais}

\subsection*{Links Úteis}\label{links-uxfateis}
\addcontentsline{toc}{subsection}{Links Úteis}

\begin{itemize}
\tightlist
\item
  \href{https://seu-site.com/livro}{Site do projeto}
\item
  \href{https://github.com/seu-usuario/QuartoBook}{Repositório GitHub}
\item
  \href{https://github.com/seu-usuario/QuartoBook/tree/main/resources/data}{Dados
  de exemplo}
\item
  \href{https://seu-site.com/livro/faq}{FAQ}
\end{itemize}

\subsection*{Redes Sociais}\label{redes-sociais}
\addcontentsline{toc}{subsection}{Redes Sociais}

\begin{itemize}
\tightlist
\item
  Twitter/X:
  \href{https://twitter.com/rafaelfernandes}{@rafaelfernandes}
\item
  LinkedIn: \href{https://linkedin.com/in/rafaelfernandes}{Rafael
  Fernandes}
\end{itemize}

\begin{center}\rule{0.5\linewidth}{0.5pt}\end{center}

\textbf{Obrigado por ler!} 🙏

Se este livro foi útil para você, considere: - ⭐ Dar uma estrela no
GitHub - 📢 Compartilhar com colegas - 💬 Deixar feedback - 🤝
Contribuir com o projeto

\begin{center}\rule{0.5\linewidth}{0.5pt}\end{center}

\emph{Última atualização: Novembro de 2025}


\backmatter


\end{document}
